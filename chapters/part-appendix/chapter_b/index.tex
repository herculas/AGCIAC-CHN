\chapter{基本概率论}\label{chap:B}

包括对统计距离的描述。

\section{生日悖论}\label{sec:B-1}

\begin{theorem}\label{theo:B-1}
令 $\mathcal{M}$ 是一个大小为 $n$ 的集合,令 $X_1,\dots,X_k$ 是均匀分布在 $\mathcal{M}$ 上的 $k$ 个独立随机变量。令 $C$ 是以下事件:对于某两个互不相同的 $i,j\in\{1,\dots,k\}$,我们有 $X_i=X_j$。那么:
\begin{enumerate}[(i)]
	\item $\displaystyle \Pr[C]\geq 1-e^{-k(k-1)/2n}\geq\min\bigg\{\frac{k(k-1)}{4n},\,0.63\bigg\}$,并且
	\item 当 $k<n/2$ 时,有 $\Pr[C]\leq 1-e^{-k(k-1)/n}$。
\end{enumerate}
\end{theorem}

\begin{proof}
上述两个不等式都很容易从不等式:
\[
1-x\leq e^{-x}\leq 1-x/2
\]
得出。该不等式对所有的 $x\in[0,1]$ 都成立。
\end{proof}

在大多数情况下,我们会使用上面的下界,并且表明,碰撞\emph{至少}会以一定的概率发生。但是,偶尔我们也会使用上界来论证碰撞不会发生。

有资料显示,人们的生日并不是均匀分布在一年之中的。例如,在美国,$9$ 月份的出生率比其他月份都要高。接下来我们表明,这种非均匀性只会增加碰撞的概率。

我们提出一个更强大的生日悖论版本,它适用于不一定均匀分布在 $\mathcal{M}$ 上的独立随机变量,但是我们要求,所有随机变量的分布都相同。这样的随机变量被称为是独立同分布 (independent and identically distributed, i.i.d.) 的。这个版本的生日悖论由 Blom 提出 \cite{bloom1973birthday}。

\begin{figure}
  \centering
  \begin{tikzpicture}

\begin{axis}[
width=5in,
height=2.8in,
scale only axis,
xmin=0,
xmax=5000,
ymin=0,
ymax=1,
xlabel={采样数量 ($k$)},
ylabel={碰撞概率},
label style={font=\small},
tick label style={font=\footnotesize}
]
\addplot [color=red!60, line width=2.5pt]
  table[row sep=crcr]{
0	0\\
10	5.49984875277687e-05\\
20	0.000209977951543427\\
30	0.000464891904255471\\
40	0.000819663891875821\\
50	0.00127418753283526\\
60	0.00182832657094734\\
70	0.00248191494348426\\
80	0.00323475686411534\\
90	0.00408662692065243\\
100	0.00503727018753242\\
110	0.0060864023529581\\
120	0.00723370986061012\\
130	0.00847885006582905\\
140	0.00982145140616097\\
150	0.0112611135861457\\
160	0.01279740777622\\
170	0.0144298768255975\\
180	0.0161580354889771\\
190	0.0179813706669224\\
200	0.0198993416597454\\
210	0.0219113804347192\\
220	0.0240168919064343\\
230	0.0262152542301048\\
240	0.0285058191076223\\
250	0.0308879121061463\\
260	0.0333608329890133\\
270	0.0359238560587347\\
280	0.0385762305118517\\
290	0.0413171808054003\\
300	0.0441459070347411\\
310	0.0470615853224909\\
320	0.0500633682182956\\
330	0.0531503851091705\\
340	0.0563217426401317\\
350	0.0595765251448321\\
360	0.0629137950859142\\
370	0.0663325935047812\\
380	0.0698319404804848\\
390	0.0734108355974249\\
400	0.0770682584215454\\
410	0.0808031689847116\\
420	0.0846145082769474\\
430	0.088501198746205\\
440	0.09246214480534\\
450	0.0964962333459556\\
460	0.100602334258781\\
470	0.104779300960245\\
480	0.109025970924899\\
490	0.113341166223343\\
500	0.117723694065321\\
510	0.122172347347608\\
520	0.126685905206371\\
530	0.131263133573625\\
540	0.135902785737443\\
550	0.140603602905564\\
560	0.14536431477204\\
570	0.150183640086574\\
580	0.155060287226183\\
590	0.159992954768842\\
600	0.16498033206875\\
610	0.170021099832861\\
620	0.175113930698334\\
630	0.180257489810547\\
640	0.185450435401329\\
650	0.190691419367059\\
660	0.195979087846291\\
670	0.201312081796561\\
680	0.20668903757003\\
690	0.212108587487644\\
700	0.217569360411456\\
710	0.223069982314792\\
720	0.228609076849939\\
730	0.234185265913017\\
740	0.239797170205736\\
750	0.245443409793713\\
760	0.251122604661033\\
770	0.256833375260777\\
780	0.262574343061184\\
790	0.268344131087177\\
800	0.27414136445696\\
810	0.279964670913395\\
820	0.285812681349892\\
830	0.291684030330533\\
840	0.29757735660418\\
850	0.303491303612284\\
860	0.30942451999017\\
870	0.315375660061534\\
880	0.321343384325925\\
890	0.327326359938973\\
900	0.333323261185138\\
910	0.339332769942779\\
920	0.345353576141305\\
930	0.351384378210229\\
940	0.357423883519926\\
950	0.363470808813891\\
960	0.36952388063234\\
970	0.375581835726966\\
980	0.381643421466692\\
990	0.387707396234263\\
1000	0.393772529813525\\
1010	0.399837603767257\\
1020	0.405901411805418\\
1030	0.411962760143688\\
1040	0.418020467852177\\
1050	0.424073367194204\\
1060	0.430120303955044\\
1070	0.436160137760542\\
1080	0.442191742385514\\
1090	0.448214006051861\\
1100	0.454225831716325\\
1110	0.460226137347823\\
1120	0.466213856194307\\
1130	0.472187937039116\\
1140	0.478147344446759\\
1150	0.484091058998125\\
1160	0.490018077515083\\
1170	0.495927413274458\\
1180	0.501818096211385\\
1190	0.507689173112034\\
1200	0.513539707795719\\
1210	0.519368781286398\\
1220	0.525175491973602\\
1230	0.530958955762807\\
1240	0.536718306215295\\
1250	0.542452694677541\\
1260	0.54816129040019\\
1270	0.55384328064666\\
1280	0.559497870791454\\
1290	0.565124284408236\\
1300	0.570721763347764\\
1310	0.576289567805738\\
1320	0.581826976380678\\
1330	0.58733328612191\\
1340	0.592807812567753\\
1350	0.598249889774041\\
1360	0.603658870333052\\
1370	0.609034125382992\\
1380	0.614375044608139\\
1390	0.619681036229774\\
1400	0.624951526988029\\
1410	0.630185962114797\\
1420	0.635383805297825\\
1430	0.640544538636153\\
1440	0.645667662587027\\
1450	0.650752695904456\\
1460	0.655799175569556\\
1470	0.66080665671284\\
1480	0.665774712528625\\
1490	0.670702934181707\\
1500	0.675590930706485\\
1510	0.680438328898698\\
1520	0.685244773199947\\
1530	0.690009925575185\\
1540	0.694733465383346\\
1550	0.699415089241292\\
1560	0.704054510881275\\
1570	0.708651461002071\\
1580	0.713205687113999\\
1590	0.717716953378\\
1600	0.722185040438953\\
1610	0.726609745253441\\
1620	0.730990880912135\\
1630	0.735328276457001\\
1640	0.739621776693513\\
1650	0.743871241998071\\
1660	0.748076548120805\\
1670	0.752237585983973\\
1680	0.756354261476125\\
1690	0.760426495242249\\
1700	0.764454222470063\\
1710	0.768437392672671\\
1720	0.772375969467751\\
1730	0.776269930353477\\
1740	0.780119266481366\\
1750	0.783923982426221\\
1760	0.787684095953377\\
1770	0.791399637783419\\
1780	0.79507065135457\\
1790	0.798697192582908\\
1800	0.802279329620627\\
1810	0.805817142612483\\
1820	0.80931072345063\\
1830	0.812760175528006\\
1840	0.816165613490445\\
1850	0.819527162987689\\
1860	0.822844960423462\\
1870	0.82611915270477\\
1880	0.829349896990608\\
1890	0.832537360440206\\
1900	0.835681719960992\\
1910	0.838783161956429\\
1920	0.841841882073858\\
1930	0.844858084952523\\
1940	0.847831983971901\\
1950	0.850763801000496\\
1960	0.853653766145236\\
1970	0.856502117501594\\
1980	0.859309100904596\\
1990	0.862074969680821\\
2000	0.864799984401532\\
2010	0.867484412637068\\
2020	0.8701285287126\\
2030	0.872732613465393\\
2040	0.875296954003669\\
2050	0.877821843467193\\
2060	0.880307580789679\\
2070	0.882754470463137\\
2080	0.885162822304249\\
2090	0.887532951222868\\
2100	0.889865176992748\\
2110	0.892159824024588\\
2120	0.894417221141471\\
2130	0.896637701356791\\
2140	0.898821601654749\\
2150	0.900969262773479\\
2160	0.903081028990901\\
2170	0.905157247913348\\
2180	0.907198270267049\\
2190	0.909204449692527\\
2200	0.911176142541961\\
2210	0.913113707679588\\
2220	0.915017506285183\\
2230	0.916887901660668\\
2240	0.918725259039904\\
2250	0.920529945401703\\
2260	0.922302329286103\\
2270	0.924042780613934\\
2280	0.925751670509733\\
2290	0.927429371128\\
2300	0.929076255482867\\
2310	0.930692697281164\\
2320	0.932279070758936\\
2330	0.933835750521408\\
2340	0.935363111386428\\
2350	0.936861528231391\\
2360	0.938331375843664\\
2370	0.93977302877451\\
2380	0.941186861196532\\
2390	0.942573246764616\\
2400	0.9439325584804\\
2410	0.945265168560244\\
2420	0.946571448306715\\
2430	0.947851767983562\\
2440	0.949106496694191\\
2450	0.950336002263616\\
2460	0.951540651123872\\
2470	0.952720808202883\\
2480	0.953876836816761\\
2490	0.955009098565523\\
2500	0.95611795323219\\
2510	0.957203758685267\\
2520	0.958266870784558\\
2530	0.959307643290304\\
2540	0.960326427775607\\
2550	0.961323573542117\\
2560	0.962299427538942\\
2570	0.963254334284757\\
2580	0.964188635793075\\
2590	0.965102671500643\\
2600	0.965996778198919\\
2610	0.966871289968618\\
2620	0.967726538117248\\
2630	0.968562851119636\\
2640	0.969380554561374\\
2650	0.970179971085156\\
2660	0.970961420339963\\
2670	0.971725218933044\\
2680	0.97247168038466\\
2690	0.973201115085531\\
2700	0.973913830256958\\
2710	0.974610129913555\\
2720	0.975290314828559\\
2730	0.975954682501652\\
2740	0.976603527129275\\
2750	0.977237139577351\\
2760	0.977855807356396\\
2770	0.978459814598953\\
2780	0.979049442039307\\
2790	0.979624966995423\\
2800	0.98018666335307\\
2810	0.980734801552062\\
2820	0.981269648574589\\
2830	0.981791467935562\\
2840	0.982300519674949\\
2850	0.982797060352022\\
2860	0.983281343041493\\
2870	0.983753617331468\\
2880	0.984214129323182\\
2890	0.984663121632457\\
2900	0.985100833392834\\
2910	0.985527500260344\\
2920	0.985943354419839\\
2930	0.986348624592867\\
2940	0.986743536047021\\
2950	0.987128310606725\\
2960	0.987503166665398\\
2970	0.987868319198963\\
2980	0.988223979780642\\
2990	0.988570356597002\\
3000	0.98890765446519\\
3010	0.989236074851331\\
3020	0.989555815890028\\
3030	0.989867072404931\\
3040	0.990170035930324\\
3050	0.990464894733689\\
3060	0.990751833839209\\
3070	0.991031035052162\\
3080	0.991302676984165\\
3090	0.99156693507923\\
3100	0.991823981640601\\
3110	0.992073985858308\\
3120	0.992317113837425\\
3130	0.992553528626984\\
3140	0.992783390249503\\
3150	0.993006855731101\\
3160	0.993224079132157\\
3170	0.993435211578483\\
3180	0.993640401292972\\
3190	0.993839793627693\\
3200	0.994033531096395\\
3210	0.99422175340739\\
3220	0.994404597496787\\
3230	0.994582197562038\\
3240	0.99475468509578\\
3250	0.994922188919923\\
3260	0.99508483521998\\
3270	0.995242747579595\\
3280	0.995396047015244\\
3290	0.995544852011088\\
3300	0.995689278553951\\
3310	0.995829440168399\\
3320	0.995965447951888\\
3330	0.996097410609971\\
3340	0.996225434491532\\
3350	0.996349623624031\\
3360	0.996470079748729\\
3370	0.99658690235589\\
3380	0.996700188719921\\
3390	0.996810033934448\\
3400	0.996916530947291\\
3410	0.997019770595345\\
3420	0.997119841639324\\
3430	0.997216830798369\\
3440	0.997310822784502\\
3450	0.997401900336902\\
3460	0.997490144256\\
3470	0.997575633437368\\
3480	0.997658444905401\\
3490	0.997738653846768\\
3500	0.997816333643631\\
3510	0.997891555906605\\
3520	0.997964390507467\\
3530	0.998034905611589\\
3540	0.998103167710095\\
3550	0.998169241651719\\
3560	0.998233190674378\\
3570	0.998295076436426\\
3580	0.998354959047602\\
3590	0.99841289709965\\
3600	0.998468947696615\\
3610	0.998523166484802\\
3620	0.998575607682394\\
3630	0.998626324108727\\
3640	0.99867536721321\\
3650	0.998722787103889\\
3660	0.998768632575656\\
3670	0.998812951138092\\
3680	0.99885578904294\\
3690	0.998897191311209\\
3700	0.998937201759912\\
3710	0.998975863028418\\
3720	0.999013216604439\\
3730	0.999049302849625\\
3740	0.999084161024791\\
3750	0.99911782931476\\
3760	0.999150344852818\\
3770	0.999181743744796\\
3780	0.999212061092764\\
3790	0.999241331018345\\
3800	0.999269586685648\\
3810	0.99929686032382\\
3820	0.999323183249215\\
3830	0.999348585887187\\
3840	0.999373097793506\\
3850	0.999396747675396\\
3860	0.999419563412199\\
3870	0.999441572075663\\
3880	0.999462799949871\\
3890	0.999483272550787\\
3900	0.999503014645446\\
3910	0.99952205027078\\
3920	0.999540402752081\\
3930	0.999558094721107\\
3940	0.999575148133836\\
3950	0.999591584287866\\
3960	0.999607423839471\\
3970	0.999622686820306\\
3980	0.999637392653779\\
3990	0.999651560171079\\
4000	0.999665207626875\\
4010	0.99967835271468\\
4020	0.999691012581894\\
4030	0.999703203844523\\
4040	0.999714942601578\\
4050	0.999726244449161\\
4060	0.999737124494243\\
4070	0.999747597368136\\
4080	0.999757677239658\\
4090	0.999767377828005\\
4100	0.999776712415332\\
4110	0.999785693859042\\
4120	0.999794334603789\\
4130	0.999802646693204\\
4140	0.999810641781347\\
4150	0.999818331143879\\
4160	0.999825725688979\\
4170	0.999832835967988\\
4180	0.999839672185802\\
4190	0.999846244211007\\
4200	0.999852561585767\\
4210	0.999858633535466\\
4220	0.999864468978109\\
4230	0.999870076533487\\
4240	0.999875464532113\\
4250	0.999880641023924\\
4260	0.999885613786769\\
4270	0.999890390334668\\
4280	0.99989497792586\\
4290	0.999899383570649\\
4300	0.999903614039026\\
4310	0.999907675868107\\
4320	0.999911575369362\\
4330	0.99991531863565\\
4340	0.999918911548074\\
4350	0.999922359782634\\
4360	0.999925668816714\\
4370	0.999928843935383\\
4380	0.999931890237519\\
4390	0.999934812641769\\
4400	0.999937615892342\\
4410	0.999940304564633\\
4420	0.999942883070692\\
4430	0.999945355664543\\
4440	0.999947726447336\\
4450	0.999949999372369\\
4460	0.99995217824995\\
4470	0.999954266752127\\
4480	0.999956268417272\\
4490	0.99995818665454\\
4500	0.999960024748187\\
4510	0.999961785861766\\
4520	0.999963473042196\\
4530	0.999965089223703\\
4540	0.999966637231657\\
4550	0.999968119786274\\
4560	0.999969539506224\\
4570	0.999970898912111\\
4580	0.999972200429861\\
4590	0.999973446393992\\
4600	0.999974639050798\\
4610	0.999975780561417\\
4620	0.999976873004815\\
4630	0.99997791838067\\
4640	0.999978918612167\\
4650	0.999979875548707\\
4660	0.999980790968521\\
4670	0.999981666581207\\
4680	0.999982504030188\\
4690	0.999983304895079\\
4700	0.999984070693991\\
4710	0.999984802885748\\
4720	0.999985502872042\\
4730	0.999986171999503\\
4740	0.999986811561719\\
4750	0.99998742280117\\
4760	0.999988006911111\\
4770	0.999988565037387\\
4780	0.999989098280185\\
4790	0.999989607695732\\
4800	0.99999009429793\\
4810	0.999990559059939\\
4820	0.999991002915702\\
4830	0.999991426761425\\
4840	0.999991831456997\\
4850	0.999992217827367\\
4860	0.99999258666387\\
4870	0.999992938725509\\
4880	0.999993274740192\\
4890	0.999993595405919\\
4900	0.999993901391942\\
4910	0.999994193339865\\
4920	0.99999447186472\\
4930	0.999994737555998\\
4940	0.999994990978644\\
4950	0.999995232674013\\
4960	0.999995463160801\\
4970	0.999995682935929\\
4980	0.999995892475408\\
4990	0.999996092235162\\
5000	0.999996282651825\\
};
\end{axis}

\end{tikzpicture}
  \begin{quote}
  {\small 该图展示了当元素总数量 $n=10^6$,采样数量 $k$ 从 $1$ 增长到 $5000$ 时的碰撞概率。它揭示了平方根附近的门限现象。在平方根处,$\sqrt{n}=1000$,碰撞概率大约是 $0.4$。当 $4\sqrt{n}=4000$ 时,碰撞概率已经极接近 $1$。而当 $0.4\sqrt{n}=500$ 时,碰撞概率相当小。}
  \end{quote}
  \caption{生日悖论}
  \label{fig:B-1}
\end{figure}

\begin{corollary}\label{cor:B-2}
令 $\mathcal{M}$ 是一个大小为 $n$ 的集合,令 $X_1,\dots,X_k$ 是 $\mathcal{M}$ 上的 $k$ 个独立同分布的随机变量,其中 $k\geq 2$。令 $C$ 是以下事件:对于某两个互不相同的 $i,j\in\{1,\dots,k\}$,我们有 $X_i=X_j$。那么:
\[
\Pr[C]\geq 1-e^{-k(k-1)/2n}\geq\min\bigg\{\frac{k(k-1)}{4n},\,0.63\bigg\}
\]
\end{corollary}

\begin{proof}
令 $X$ 是一个与 $X_1$ 分布相同的随机变量。令 $\mathcal{M}=\{a_1,\dots,a_n\}$,$p_i=\Pr[X=a_i]$。令 $I$ 是 $\mathcal{M}$ 上所有元素各不相同的 $k$ 元组构成的集合。那么 $I$ 中包含 $\tbinom{n}{k}k!$ 个元组。由于变量相互独立,我们有:
\begin{equation}\label{eq:B-1}
\Pr[\lnot C]
=\sum_{(b_1,\dots,b_k)\in I}\Pr[X_1=b_1\land\dots\land X_k=b_k]
=\sum_{(b_1,\dots,b_k)\in I}\prod_{j=1}^k p_{b_j}
\end{equation}
我们表明,当 $p_1=p_2=\dots=p_n=1/n$ 时,这个和是最大的。这将意味着,当所有变量都均匀的时候,碰撞的概率是最小的。然后,我们就可以从定理 \ref{theo:B-1} 中得出该推论。

假设有些 $p_i$ 不是 $1/n$,比如说 $p_i<1/n$。由于 $\sum^n_{j=1}p_i=1$,必然有另一个 $p_j$ 满足 $p_j>1/n$。令 $\epsilon=\min((1/n)-p_i,\,p_j-1/n)$,并且,注意到 $p_j-p_i>\epsilon$。我们表明,用 $p_i+\epsilon$ 替换 $p_i$,并用 $p_j-\epsilon$ 替换 $p_j$,会增加式 \ref{eq:B-1} 中的和。显然,所得到的 $p_1,\dots,p_n$ 的和仍为 $1$。因此,所产生的 $p_1,\dots,p_n$ 形成一个 $\mathcal{M}$ 上的分布,其中少了一个不是 $1/n$ 的值。此外,在这个分布中,不发生碰撞的概率要比在未修改的分布中更大。重复这个替换过程最多 $n$ 次,我们将会看到,当所有的 $p_i$ 都等于 $1/n$ 时,和才是最大的。同样,这意味着当变量是均匀的时候,不发生碰撞的概率是最大的。

现在,考虑式 \ref{eq:B-1} 中的和。注意到,存在四种不同类型的项。首先,有一些项不包含 $p_i$ 和 $p_j$。这些项不受 $p_i$ 和 $p_j$ 的变化的影响。第二,有的项恰好含有 $p_i$ 或 $p_j$ 中的一个。这些项会成对出现。对于每个包含 $i$ 但不包含 $j$ 的 $k$ 元组,都存在一个对应的包含 $j$ 但不包含 $i$ 的元组。于是,在式 \ref{eq:B-1} 中,对应两项的和看起来就形如 $A(p_i+\epsilon)+A(p_j-\epsilon)$,其中 $A\in[0,1]$。由于这就等于 $Ap_i+Ap_j$,这两项的和也不受 $p_i$ 和 $p_j$ 的变化的影响。最后,式 \ref{eq:B-1} 中有些项同时包含 $p_i$ 和 $p_j$。这些项的变化是:
\[
B(p_i+\epsilon)(p_j-\epsilon)-Bp_ip_j=B[\epsilon(p_j-p_i)-\epsilon^2]
\]
其中 $B\in[0,1]$。根据 $\epsilon$ 的定义,我们有 $p_j-p_i>\epsilon$,因此 $\epsilon(p_j-p_i)-\epsilon^2>0$。这样,修改了 $p_i$ 和 $p_j$ 之后,总和将大于修改前的和。

总的来说,我们证明了对 $p_i$ 和 $p_j$ 的修改增大了式 \ref{eq:B-1} 中的和,正如所推论所要求的。这就完成了对推论的证明。
\end{proof}

\subsection{更多的碰撞约束}\label{subsec:B-1-1}

对于一个随机函数 $f:\mathcal{X}\to\mathcal{X}$ 考虑序列 $x_i\leftarrow f(x_{i-1})$。分析以下游程的周期时间(用于 Pollard)。现在,对于一个置换 $\pi:\mathcal{X}\to\mathcal{X}$,考虑相同的序列。分析周期时间(用于对 SecurID 身份识别的分析)。

\subsection{一种简单的区分器}\label{subsec:B-1-2}

下面,我们描述一种简单的算法,它可以区分 $\{0,1\}^n$ 上字符串的两种分布。令 $X_1,\dots,X_n$ 和 $Y_1,\dots,Y_n$ 是在 $\{0,1\}$ 中取值的独立随机变量。那么:
\[
X:=(X_1,\dots,X_n)
\qquad\text{和}\qquad
Y:=(Y_1,\dots,Y_n)
\]
是 $\{0,1\}^n$ 上的两个元素。假设对于 $i=1,\dots,n$,我们有:
\[
\Pr[X_i=1]=p
\qquad\text{并且}\qquad
\Pr[Y_i=1]=(1+2\epsilon)\cdot p
\]
对于某个 $p\in[0,1]$ 和某个满足 $(1+2\epsilon)\cdot p\leq 1$ 的 $\epsilon >0$ 成立。那么,$X$ 和 $Y$ 就导出了 $\{0,1\}^n$ 上的两个不同分布。

现在,我们被给定一个 $n$ 比特的字符串 $T$,并被告知,它是根据分布 $X$ 或分布 $Y$ 采样得到的,所以 $p$ 和 $\epsilon$ 都是已知的。我们的目标是确定 $T$ 是根据哪个分布采样的。考虑下面的一个简单算法 $A$:

\vspace*{10pt}

\hspace*{5pt} 输入:$T:=(t_1,\dots,t_n)\in\{0,1\}^n$\\
\hspace*{26pt} 输出:如果 $T$ 是从 $X$ 种采样的,则输出 $1$,否则输出 $0$

\vspace*{5pt}

\hspace*{5pt} 令 $s\leftarrow(1/n)\cdot\sum_{i=1}^nt_i$\\
\hspace*{26pt} 如果 $s>p\cdot(1+\epsilon)$,则输出 $0$,否则输出 $1$

\vspace*{10pt}

\noindent
我们主要关注的是这个值:
\[
\Delta:=\big\lvert
\Pr[\mathcal{A}(T_x)=1]-\Pr[\mathcal{A}(T_y)=1]
\big\rvert
\quad\in[0,1]
\]
其中 $T_x\overset{\rm R}\leftarrow X$,$T_y\overset{\rm R}\leftarrow Y$。这个值反映了 $\mathcal{A}$ 对分布 $X$ 和 $Y$ 的区分程度。对于一个好的区分器,$\Delta$ 将接近于 $1$。下面的定理表明,当 $n$ 约为 $1/(p\epsilon^2)$ 时,$\Delta$ 约为 $1/2$。

\begin{theorem}\label{theo:B-3}
对于所有的 $p\in[0,1]$ 和 $\epsilon<0.3$,如果 $n=4\lceil 1/(p\epsilon^2)\rceil$,则有 $\Delta>0.5$。
\end{theorem}

\begin{proof}
对该定理的证明直接来自 Chernoff 约束。如果 $T$ 是从 $X$ 中采样得到的,Chernoff 约束就意味着:
\[
\Pr[\mathcal{A}(T_x)=1]
=\Pr[s>p(1+\epsilon)]
\leq e^{-n\cdot(p\epsilon^2/2)}
\leq e^{-2}
\leq 0.135
\]
如果 $T$ 是从 $Y$ 中采样得到的,Chernoff 约束就意味着:
\[
\Pr[\mathcal{A}(T_y)=0]
=\Pr[s<p(1+\epsilon)]
\leq e^{-n\cdot(p\epsilon^2/4)}
\leq e^{-1}
\leq 0.368
\]
因此,$\Delta>|(1-0.368)-0.135|=0.503$,这就得到了约束。
\end{proof}