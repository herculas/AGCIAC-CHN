\section{计算性零知识性及其应用}

事实证明,对于某些关系,我们需要更宽松的零知识的概念,以便得到更有效的 Sigma 协议。我们下面先用一个例子来建立直观上的感受。

\subsection{例子:范围证明}\label{subsec:20-4-1}

如 \ref{sec:20-2} 节中那样,我们还是使用乘性 ElGamal 加密方案。现在,假设 Bob 的公钥是 $u=g^\alpha\in\mathbb{G}$,私钥是 $\alpha\in\mathbb{Z}_q$。像往常一样,$\mathbb G$ 是一个素阶 $q$ 的循环群,生成元为 $g\in\mathbb{G}$。

我们现在进一步推广例 \ref{exmp:20-3},假如 Alice 现在加密的不再是一个比特 $b$,而是一个 $d$ 比特的数字 $x$,因此有 $x\in\{0,\dots,2^{d-1}\}$。为了进行加密,Alice 将 $x$ 编码为一个群元素 $g^x$,然后用 Bob 的公钥 $u$ 对这个群元素进行加密。由此产生的密文形如 $(v,e)$,其中 $v=g^\beta$,$e=u^\beta g^x$。我们假设 $2^d<q$,那么 $x$ 的编码方案是一个双射。像之前一样,Alice 想让 Charlie 相信 $(v,e)$ 确实来自于 Bob 的公钥加密的一个 $d$ 比特数字,但又不透露任何其他信息。

因此,我们希望有一个关系:
\begin{equation}
\mathcal{R}=
\bigg\lbrace
\Big((\beta,\gamma,x),(u,v,e)\Big)\;:\;
v=g^\beta,\;
e=u^\beta\cdot g^x,\;
x\in\{0,\dots,2^{d-1}\}\;
\bigg\rbrace
\end{equation}
上的 Sigma 协议。这里我们假设 $d$ 是一个固定的公共值。

一个直接的方法就是使用我们在例 \ref{exmp:20-3} 中所使用的 OR 证明构造。也就是说,Alice 可以分别证明 $x=0$,$x=1$,$\cdots$,或者 $x=2^{d-1}$。虽然这个想法可行,但由此产生的 Sigma 协议的通信和计算开销将与 $2^d$ 成正比。事实上我们可以做得更好,我们可以构建一个复杂度与 $d$ 成\emph{线性}关系的 Sigma 协议,而不是成\emph{指数}关系。

下面是具体的方法。Alice 首先用二进制表示 $x$,即令 $x=\sum_{i=0}^d2^ib_i$,其中$b_i\in\{0,1\}$对于$i=0,\dots,d-1$成立。接下来 Alice 分别加密每一比特。为了得到一个更简单有效的协议,她使用我们在练习 \ref{exer:11-8} 中讨论的 ElGamal 加密方案的变体。具体地,Alice 会生成一个随机公钥 $(u_0,\dots,u_{d-1})\in\mathbb{G}^d$,然后随机选择 $\beta_0\in\mathbb{Z}_q$ 并计算 $v_0\leftarrow g^{\beta_0}$;最后,她对 $i=0,\dots,d-1$ 计算 $e_i\leftarrow u_i^{\beta_0}g^{b_i}$。因此,$(v_0,e_0,e_1,\dots,e_{d-1})$ 就是用公钥 $(u_0,\dots,u_{d-1})$ 加密 $(b_0,\dots,b_{d-1})$ 的结果。然后 Alice 将 $v_0$,$(u_0,\dots,u_{d-1})$ 和 $(e_0,\dots,e_{d-1})$ 发送给 Charlie,并向他证明(i)每个被加密的 $b_i$ 都是一个比特;(ii) $\sum_{i}2^ib_i=x$。为了证明(i),Alice 会使用一个类似于例 \ref{exmp:20-5} 中所使用的技术,利用 $b_i\in\{0,1\}\iff b_i^2=b_i$ 这一事实。

为了证明(i)和(ii),Alice 和 Charlie 可以使用 \ref{subsec:19-5-3} 小节中介绍的通用线性协议。因此,Alice 向 Charlie 证明存在:
\[
\beta,x,
\quad
\beta_0,
\quad
b_0,\dots,b_{d-1},
\quad
\tau_0,\dots,\tau_{d-1}
\]
使得:
\begin{equation}\label{eq:20-11}
\left.
\begin{aligned}
& v=g^\beta,
\quad
e=u^\beta g^x\\
& v_0=g^{\beta_0}\\
& e_i=u_i^{\beta_0}g^{b_i},
\quad
v_0^{b_i}=g^{\tau_i},
\quad
e_i^{b_i}=u_i^{\tau_i}g^{b_i}\;\;(i=0,\dots,d-1)\\
& x=b_0+2b_1+\cdots+2^{d-1}b_{d-1}
\end{aligned}
\quad\quad
\right\}
\end{equation}
式 \ref{eq:20-11} 中第一行表明 $(v,e)$ 来自用 $u$ 加密的 $g^x$。第二行表明每个被加密的 $b_i$ 都是一个比特,使用例 \ref{exmp:20-5} 中所展示的技术的一个变体,其中 $\tau_i=\beta_0b_i$,第三行表明这些比特就是 $x$ 的二进制表示。

所以,该协议的整体结构如下:
\begin{enumerate}
	\item Alice 生成 $v_0$,$(u_0,\dots,u_{d-1})$ 和 $(e_0,\dots,e_{d-1})$,并将这些辅助群元素发送给 Charlie;
	\item Alice 和 Charlie 共同运行式 \ref{eq:20-11} 所描述的通用线性 Sigma 协议。
\end{enumerate}

我们首先就可以观察到,通过让 Alice 在通用线性 Sigma 协议的承诺信息之上``捎带"一些辅助群元素,这个协议在整体上仍然具有一般 Sigma 协议的基本结构。

读者可以自行证明该协议能够提供存在健全性。

在这里,我们感兴趣的问题是,这个协议在何种意义上是零知识的?问题在于,尽管通用线性协议是特殊 HVZK 的,但是整个协议并非如此,因为对 $x$ 的比特的加密仍然会向 Charlie 透露 $x$ 的一些信息。直观地说,在 DDH 假设下,这些加密是不应该透露任何信息的。因此该协议尽管仍然是零知识的,但只是\emph{计算上}的零知识。为了进一步增强零知识的概念,我们下面会定义特殊\emph{计算性} HVZK 的概念。

\subsection{特殊计算性 HVZK}

我们在定义 \ref{def:19-5} 中定义了 Sigma 协议的特殊 HVZK 的概念。现在,我们放松定义 \ref{def:19-5},以定义一个较弱的特殊\emph{计算性} HVZK 的概念,简称为特殊 cHVZK。我们的想法是,我们不要求真实定义和模拟定义的分布是\emph{完全相同}的,而只要求它们在\emph{计算上不可区分}。

令 $\Pi=(P,V)$ 是一个 $\mathcal{R}\subseteq\mathcal{X}\times\mathcal{Y}$ 上的 Sigma 协议,其挑战空间为 $\mathcal C$。如定义 \ref{def:19-5},$\Pi$ 的\emph{模拟器}是一个有效概率性算法 $Sim$,它接受 $(y,c)\in\mathcal{Y}\times\mathcal{C}$ 作为输入,并且总是输出数对 $(t,z)$,使得 $(t,c,z)$ 是 $y$ 的一个接受对话。

\begin{game}[特殊 cHVZK]\label{game:20-4}
令 $\Pi=(P,V)$ 是 $\mathcal{R}\subseteq\mathcal{X}\times\mathcal{Y}$ 上的一个 Sigma 协议,其挑战空间为 $\mathcal C$。如上所述,令 $Sim$ 是 $\Pi$ 的一个模拟器。对于一个给定的对手 $\mathcal A$,我们定义两个实验:实验 $0$ 和实验 $1$。在这两个实验中,$\mathcal A$ 一开始就计算 $(x,y)\in\mathcal{R}$,并将 $(x,y)$ 提交给挑战者。
\begin{itemize}
	\item 在实验 $0$ 中,挑战者运行 $P(x,y)$ 和 $V(y)$ 之间的协议,并将得到的对话 $(t,c,z)$ 交给 $\mathcal A$。
	\item 在实验 $1$ 中,挑战者计算:
	\[
    c\overset{\rm R}\leftarrow\mathcal C,
    \quad
    (t,z)\overset{\rm R}\leftarrow Sim(y,c)
    \]
    并将模拟的对话 $(t,c,z)$ 交给 $\mathcal A$。
\end{itemize}
在游戏结束时,$\mathcal A$ 计算并输出一个比特 $\hat b\in\{0,1\}$。

对于 $b=0,1$,令 $W_b$ 是 $\mathcal A$ 在实验 $b$ 中输出 $1$ 的事件。 我们定义 $\mathcal A$ 对于 $\Pi$ 和 $Sim$ 的优势为:
\[
{\rm cHVZK\mathsf{adv}}[\mathcal{A},\Pi,Sim]:=\big\lvert\Pr[W_0]-\Pr[W_1]\big\rvert
\]
\end{game}

\begin{definition}
如果存在一个 $\Pi$ 的模拟器 $Sim$,使得对于每个有效对手 $\mathcal A$ 来说,${\rm cHVZK\mathsf{adv}}[\mathcal{A},\Pi,Sim]$ 的值都可忽略不计,我们就称 $\Pi$ 是\textbf{特殊计算性 HVZK} 的,简称为\textbf{特殊 cHZVK} 的。
\end{definition}

许多对特殊 HVZK 的 Sigma 协议成立的结论也对特殊 cHVZK 的 Sigma 协议成立:
\begin{itemize}
	\item 如果使用一个 cHVZK 的协议代替 HVZK 的协议,定理 \ref{theo:19-15} 仍然成立,但具体的安全边界变成了:
	\[
    {\rm ID2\mathsf{adv}}[\mathcal A,\mathcal I]\leq{\rm ID1\mathsf{adv}}[\mathcal B,\mathcal I]+Q\cdot{\rm cHVZK\mathsf{adv}}[\mathcal{B}',\Pi,Sim]
    \]
    其中 $Q$ 是在窃听攻击中获得的交互记录数量的上界。这个 $Q$ 的系数来自于应用一个标准混合论证,它允许我们用 $Q$ 次模拟对话代替 $Q$ 次真实对话。
    \item 引理 \ref{theo:19-17} 也可以被改编以适用于 cHVZK 的协议,此时,式 \ref{eq:19-20} 中的安全边界变成了:
    \[
    \epsilon\leq\frac{r}{N}+\sqrt{r\epsilon'}+Q\cdot{\rm cHVZK\mathsf{adv}}[\mathcal{B}',\Pi,Sim]
    \]
    其中 $Q$ 仍是在窃听攻击中获得的交互记录数量的上界。
    \item 定理 \ref{theo:20-3} 对于 cHVZK 的 Sigma 协议也仍然成立。同样地,具体的安全上界新增了一项 $Q_{\rm p}\cdot{\rm cHVZK\mathsf{adv}}[\mathcal{B},\Pi,Sim_1]$,其中 $Q_{\rm p}$ 是证明查询的次数。
\end{itemize}

然而,我们注意到,定理 \ref{theo:19-21}(关于见证独立性)对于 cHVZK \emph{并不}成立。

\begin{snote}[范围证明.]
读者可以自行证明,我们在 \ref{subsec:20-4-1} 小节中介绍的证明加密数值在区间 $[0,2^d)$ 中的 Sigma 协议是特殊 cHVZK 的。
\end{snote}

\subsection{一种用于非线性关系的无约束通用协议}\label{subsec:20-4-3}

我们在 \ref{subsec:20-4-1} 小节中所使用的技术可以被推广,以允许我们将形如 $x_i=x_j\cdot x_k$ 的非线性关系添加到通用线性协议所处理的线性方程组中,就像我们在 \ref{subsec:20-2-1} 小节中所做的那样。然而,与 \ref{subsec:20-2-1} 小节不同,我们不需要引入任何辅助方程。我们需要为这种推广所付出的代价是,我们现在只能实现特殊 cHVZK,而不是 HVZK。

同样地,令 $\mathbb G$ 是一个由 $g\in\mathbb{G}$ 生成的素阶 $q$ 的循环群,令 $\phi$ 是一个如式 \ref{eq:19-13} 的方程,但其中包含形如 $x_i=x_j\cdot x_k$ 的非线性方程。假设证明者和验证者都得到了 $\phi$,并且证明者还得到满足 $\phi$ 的变量 $(x_1,\dots,x_n)$ 的一组赋值 $(\alpha_1,\dots,\alpha_n)$。证明者采用如下方法构造一个新的方程 $\phi'$。证明者首先随机选择一个 $\beta\in\mathbb{Z}_q$ 并计算 $v\leftarrow g^\beta$,然后将等式 $v=g^y$ 添加到 $\phi$ 中,其中 $y$ 是一个新的变量。随后,对于 $\phi$ 中的每个非线性方程 $x_i=x_j\cdot x_k$,证明者随机选择 $u\in\mathbb{G}$ 并计算 $e\leftarrow u^\beta g^{\alpha_j}$,然后将方程:
\begin{equation}\label{eq:20-12}
e=u^yg^{x_j},
\quad
v^{x_k}=g^t,
\quad
e^{x_k}=u^tg^{x_i}
\end{equation}
添加到 $\phi$ 中,其中 $t$ 是另一个新的变量。这就产生了一个新的可由通用线性协议处理的公式 $\phi'$。然后证明者将由 $v$ 组成的辅助群元素集合以及每个非线性方程所对应的群元素 $u$ 和 $e$ 发送给验证者。

基于这些辅助群元素,验证者就可以重建公式 $\phi'$,这样证明者和验证者都可以对 $\phi'$ 运行通用线性协议。验证者为变量 $y$ 赋值 $\beta$,为每个非线性方程 $x_i=x_j\cdot x_k$ 所产生的变量 $t$ 赋值 $\tau:=\beta\alpha_k$。此外,验证者可以在通用线性协议的承诺信息之上``捎带"辅助群元素,从而使产生的协议具有正确的通信模式。

读者可以自行证明,上述转换所产生的 Sigma 协议是特殊 cHVZK 的(在 DDH 假设下,参见练习 \ref{exer:11-8}),并且能为关系 \ref{eq:19-14} 提供知识健全性,其中的公式 $\phi$ 现在可以是上面介绍的非线性形式。

在上述转换过程中,有几个地方还存在改善空间。比如 $u$ 和式 \ref{eq:20-12} 的第一个方程可以在所有 $x_j$ 是第一个乘数的非线性方程中重复使用。同样地,$t$ 和式 \ref{eq:20-12} 的第二个方程也可以在所有 $x_k$ 是第二个乘数的非线性方程中重复使用。

\begin{snote}[范围证明2.]
不难看出,我们之前介绍的范围证明协议可以由上面的转换导出。Alice 想向 Charlie 证明存在:
\[
\beta,x,
\quad
\beta_0,
\quad
b_0,\dots,b_{d-1}
\]
使得:
\[
v=g^\beta,
\quad
e=u^\beta g^x,
\quad
x=\sum_{i=0}^{d-1}2^ib_i,
\quad
b_i=b^2_i
\;\;
(i=0,\dots,d-1)
\]
成立。读者可以自行证明,应用上面介绍的非线性到线性的转换,我们可以直接得到 \ref{subsec:20-4-1} 小节中介绍的协议(在转换中,$v_0$和$\beta_0$扮演$v$和$\beta$的角色)。
\end{snote}