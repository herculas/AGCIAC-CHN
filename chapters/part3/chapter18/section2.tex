\section{身份认证协议的定义}


我们首先定义图 \ref{fig:18—1} 所示的构成身份识别协议的各个算法。

\begin{figure}
    \centering
    \tikzset{every picture/.style={line width=0.75pt}}   

\begin{tikzpicture}[x=0.75pt,y=0.75pt,yscale=-1,xscale=1]

\draw  [fill={rgb, 255:red, 255; green, 255; blue, 255 }  ,fill opacity=1 ][line width=1.2] [general shadow={fill=black,shadow xshift=2.25pt,shadow yshift=-2.25pt}] (0,60) -- (60,60) -- (60,150) -- (0,150) -- cycle ;
\draw  [fill={rgb, 255:red, 255; green, 255; blue, 255 }  ,fill opacity=1 ][line width=1.2] [general shadow={fill=black,shadow xshift=2.25pt,shadow yshift=-2.25pt}] (250,60) -- (310,60) -- (310,150) -- (250,150) -- cycle ;

\draw  [->]  (140,20) -- (30,55) ;
\draw  [->]  (170,20) -- (280,55) ;

\draw  [<-]  (64,80) -- (250,80) ;
\draw  [->]  (60,105) -- (249,105) ;
\draw  [<-]  (64,130) -- (250,130) ;

\draw  [->]  (280,150) -- (280,190) ;

\draw (30,75) node   [align=left] {证明者};
\draw (280,75) node   [align=left] {验证者};
\draw (155,10) node    {$G$};
\draw (280,193) node [anchor=north] [inner sep=0.75pt]   [align=left] {$\mathsf{accept}$ 或 $\mathsf{reject}$};
\draw (83,34.1) node [anchor=south east] [inner sep=0.75pt]    {$sk$};
\draw (227,34.1) node [anchor=south west] [inner sep=0.75pt]    {$vk$};


\end{tikzpicture}
    \caption{身份认证协议中的证明者与验证者}
    \label{fig:18—1}
\end{figure}

\begin{definition}
一个\textbf{身份识别协议} $\mathcal{I}=(G,P,V)$ 由三部分组成:
\begin{itemize}
    \item $G$ 是一个概率性的\textbf{密钥生成}算法,不接受任何输入,并输出一对密钥 $(vk,sk)$,其中 $vk$ 被称为\textbf{验证密钥 (verification key)},$sk$ 被称为\textbf{私钥 (secret key)};
    \item $P$ 是一个交互式协议算法,被称为\textbf{证明者 (prover)},它接受 $sk$ 作为输入;
    \item $V$ 是一个交互式协议算法,被称为\textbf{验证者 (verifier)},它接受 $vk$ 作为输入,输出 $\mathsf{accept}$ 或 $\mathsf{reject}$。
\end{itemize}
我们要求,当 $P(sk)$ 和 $V(vk)$ 能够正确交互时,$V(vk)$ 总是输出 $\mathsf{accept}$。也就是说,对于 $G$ 的所有可能输出 $(vk,sk)$,如果 $P$ 由 $sk$ 初始化,$V$ 由 $vk$ 初始化,那么当 $P$ 和 $V$ 交互结束时,$V$ 输出 $\mathsf{accept}$ 的概率为 $1$。

\end{definition}