\section{从无前缀安全PRF到完全安全PRF(方法2):无前缀编码}\label{sec:6-6}

将无前缀安全的 PRF 转换为安全 PRF 的另一种方法是对 PRF 的输入进行编码,使得没有任何一条编码后的输入是另一条的前缀。我们定义以下术语:
\begin{itemize}
	\item 如果 $S\subseteq\mathcal{X}^{\leq\ell}$ 中的任何元素都不是其他元素的前缀,我们就称 $S$ 是一个\textbf{无前缀集 (prefix-free set)}。例如,如果 $(x_1,x_2,x_3)$ 属于一个无前缀集 $S$,那么 $x_1$ 和 $(x_1,x_2)$ 都不在 $S$ 中。
	\item 令 $\mathcal{X}^{\leq\ell}_{_{>0}}$ 表示$\mathcal{X}$ 上所有长度不超过 $\ell$ 的非空序列的集合。如果映射 $pf$ 是一个单射(即一对一函数),且 $pf$ 的像是一个无前缀集,我们就称 $pf:\mathcal{M}\to\mathcal{X}^{\leq\ell}_{_{>0}}$ 是一个\textbf{无前缀编码 (prefix-free encoding)}。
\end{itemize}

令 $PF$ 是一个定义在 $(\mathcal{K},\mathcal{X}^{\leq\ell},\mathcal{Y})$ 上的无前缀安全 PRF,且 $pf:\mathcal{M}\to\mathcal{X}^{\leq\ell}_{_{>0}}$ 是一个无前缀编码。定义派生的 PRF $F$ 为:
\[
F(k,m):=PF(k,pf(m))
\]
那么 $F$ 定义在 $(\mathcal{K},\mathcal{M},\mathcal{Y})$ 上。于是,我们就可以得到下面的定理。

\begin{theorem}\label{theo:6-8}
如果 $PF$ 是一个无前缀安全的 PRF,$pf$ 是一个无前缀编码,那么 $F$ 是一个安全的 PRF。
\end{theorem}

\subsection{无前缀编码}

为了使用定理 \ref{theo:6-8} 构造 PRF,我们介绍两种无前缀编码 $pf:\mathcal{M}\to\mathcal{X}^{\leq\ell}$。我们假设 $\mathcal{X}=\{0,1\}^n$,其中 $n$ 是某个整数。

\begin{snote}[方法 1:前置长度。]
设 $\mathcal{M}:=\mathcal{X}^{\leq\ell-1}$,并令 $m=(a_1,\dots,a_v)\in\mathcal{M}$。定义:
\[
pf(m):=(\langle v\rangle,a_1,\dots,a_v)\quad\in\mathcal{X}^{\leq\ell}_{_{>0}}
\]
其中 $\langle v\rangle\in\mathcal{X}$ 是 $v$ 的二进制表示,即 $m$ 的长度。我们假设 $\ell<2^n$,那么消息长度可以被编码为 $n$ 比特的二进制序列。

我们下面论证 $pf$ 是一个无前缀编码。首先,$pf$ 显然是一个单射。为了说明 $pf$ 的像是一个无前缀集,令 $pf(x)$ 和 $pf(y)$ 是 $pf$ 的像中的两个元素。如果 $pf(x)$ 和 $pf(y)$ 包含相同数量的分组,那么它们中的任何一个都不可能是另一个的真前缀。如果 $pf(x)$ 和 $pf(y)$ 包含不同数量的分组,它们就必须在第一个分组上就有所不同,此时它们也同样都不可能是另一个的真前缀。因此,$pf$ 是一个无前缀编码。

这种无前缀编码在实践中并不常用,因为这样所产生的 MAC 不是一个流式 MAC:使用这种 MAC 的应用程序必须提前把消息的长度承诺给 MAC。正如我们之前所说,这对于流媒体应用来说极不实用,因为在这种应用中,数据包的长度可能无法提前预知。
\end{snote}

\begin{snote}[方法 2:停止比特。]
设 $\mathcal{\bar{X}}:=\{0,1\}^{n-1}$,并令 $\mathcal{M}=\mathcal{\bar{X}}^{\leq\ell}_{_{>0}}$。对于 $m=(a_1,\dots,a_v)\in\mathcal{M}$,定义:
\[
pf(m):=\big((a_1\,\Vert\,0),\,(a_2\,\Vert\,0),\,\dots,\,(a_{v-1}\,\Vert\,0),\,(a_v\,\Vert\,1)\big)\quad\in\mathcal{X}^{\leq\ell}_{_{>0}}
\]
显然,$pf$ 是一个单射。为了说明 $pf$ 的像是一个无前缀集,令 $pf(x)$ 和 $pf(y)$ 是 $pf$ 的像中的两个元素。设 $v$ 是 $pf(x)$ 中分组的数量。如果 $pf(y)$ 包含 $v$ 个或更少的分组,那么 $pf(x)$ 必然不是 $pf(y)$ 的真前缀。如果 $pf(y)$ 包含多于 $v$ 个分组,那么 $pf(y)$ 中的 $v$ 号分组以 $0$ 结束,而 $pf(x)$ 中的 $v$ 号分组以 $1$ 结束。因此,$pf(x)$ 和 $pf(y)$ 在 $v$ 号分组上有所不同,所以 $pf(x)$ 不是 $pf(y)$ 的真前缀。

这种无前缀编码所产生的 MAC 是一个流式 MAC。然而,这种编码使 MAC 的长度增加了 $v$ 比特。当使用 CBC 或级联计算一个长消息的 MAC 时,这种编码将导致对底层 PRF(比如 AES)的额外计算。相对地,\ref{sec:6-5} 节中介绍的加密 PRF 方法只增加了一次对底层 PRF 的计算。例如,使用 ECBC-AES 和 $pf$ 对一兆字节 ($2^{20}$ 字节) 的消息计算 MAC,除了加密 PRF 方法所需要的,还需要对 AES 进行额外的 $511$ 次评估。在实践中,情况甚至可能更糟。由于计算机更喜欢字节对齐的数据,我们很可能需要在每个分组中附加一整个字节,而不是仅仅一个比特。那么,使用 ECBC-AES 和 $pf$ 对一兆字节的消息计算 MAC,将导致比加密 PRF 方法多出 $4096$ 次 AES 的计算,大约多 $6\%$ 的开销。
\end{snote}