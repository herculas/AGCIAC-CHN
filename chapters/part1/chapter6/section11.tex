\section{PMAC:一种并行的 MAC}\label{sec:6-11}

迄今为止,我们所介绍的所有 MAC 构造,包括 ECBC,CMAC 和 NMAC,本质上都是串行性的。也就是说,在第 $i-1$ 个分组的计算完成之前,无论如何都无法计算第 $i$ 个分组。这就使得我们很难利用硬件的并行性或流水线来加速 MAC 的生成和认证。在本节中,我们将介绍一种非常适合并行架构的安全 MAC,称作 PMAC。更具体地,我们将介绍一种名为 $\rm PMAC_0$ 的算法,因为它更容易描述。

令 $F_1$ 是一个定义在 $(\mathcal{K}_1,\mathbb{Z}_p,\mathcal{Y})$ 上的 PRF,其中 $p$ 是一个素数,$\mathcal{Y}:=\{0,1\}^n$。令 $F_2$ 是一个定义在 $(\mathcal{K}_2,\mathcal{Y},\mathcal{Z})$ 上的 PRF。

下面,我们构建一个新的 PRF,称作 $\rm PMAC_0$,它将一个密钥和一条 $\mathbb{Z}^{\leq\ell}_p$ 上的消息作为输入,输出 $\mathcal{Z}$ 上的一个元素。一个 $\rm PMAC_0$ 的密钥由 $k\in\mathbb{Z}_p$,$k_1\in\mathcal{K}_2$ 和 $k_2\in\mathcal{K}_2$ 组成。$\rm PMAC_0$ 构造的工作原理如下:

\vspace{5pt}

\hspace*{5pt} 输入:$m=(a_1,\dots,a_v)\in\mathbb{Z}^v_p$,其中 $0\leq v\leq\ell$,以及\\
\hspace*{75pt} 密钥 $\vec{k}=(k,k_1,k_2)$,其中 $k\in\mathbb{Z}_p$,$k_1\in\mathcal{K}_1$,$k_2\in\mathcal{K}_2$\\
\hspace*{26pt} 输出:$\mathcal{Z}$ 中的一个标签

\vspace{5pt}

\hspace*{5pt} ${\rm PMAC}_0(\vec{k},m)$:\\
\hspace*{50pt} 令 $t\leftarrow 0^n\in\mathcal{Y}$,$mask\leftarrow 0\in\mathbb{Z}_p$\\
\hspace*{50pt} 对于 $i=1,\dots,v$:\\
\hspace*{75pt} 令 $mask\leftarrow mask+k$ \quad\quad // $mask=i\cdot k\in\mathbb{Z}_p$\\
\hspace*{75pt} 令 $r\leftarrow a_i+mask$ \quad\quad\quad\; // $r=a_i+i\cdot k\in\mathbb{Z}_p$\\
\hspace*{75pt} 令 $t\leftarrow t\oplus F_1(k_1,r)$\\
\hspace*{50pt} 输出 $F_2(k_2,t)$

\vspace{5pt}

\noindent
在评估 $F_1$ 之前,主循环将掩码 $k,2k,3k,\dots$ 加到每个消息分组上。在一个串行设备上,这需要在每轮迭代中进行两次模 $p$ 加法。而在并行设备上,每个处理器都可以独立计算 $a_i+i\cdot k$,然后将其应用到 $F_1$ 上,如图 \ref{fig:6-9} 所示。

$\rm PMAC_0$ 是一个安全的 PRF,因此对于长的消息来说,$\rm PMAC_0$ 可以作为一个安全的 MAC。根据下一章的定理 \ref{theo:7-7},我们很容易给出它的安全性证明。现在,我们先说明该定理,并将对它的证明留到 \ref{subsec:7-3-3} 小节。

\begin{figure}
  \centering
  

\tikzset{every picture/.style={line width=0.75pt}} %set default line width to 0.75pt        

\begin{tikzpicture}[x=0.75pt,y=0.75pt,yscale=-1,xscale=1]
%uncomment if require: \path (0,291); %set diagram left start at 0, and has height of 291

%Shape: Rectangle [id:dp9774685149738991] 
\draw  [fill={rgb, 255:red, 255; green, 255; blue, 255 }  ,fill opacity=1 ][line width=1.2] [general shadow={fill=black,shadow xshift=2.25pt,shadow yshift=-2.25pt}] (10,100) -- (60,100) -- (60,150) -- (10,150) -- cycle ;
%Shape: Rectangle [id:dp510720189942172] 
\draw  [fill={rgb, 255:red, 255; green, 255; blue, 255 }  ,fill opacity=1 ][line width=1.2] [general shadow={fill=black,shadow xshift=2.25pt,shadow yshift=-2.25pt}] (0,0.5) -- (70,0.5) -- (70,20.5) -- (0,20.5) -- cycle ;
%Straight Lines [id:da3081370845461646] 
\draw    (35,20) -- (35,45.5) ;
\draw [shift={(35,48.5)}, rotate = 270] [fill={rgb, 255:red, 0; green, 0; blue, 0 }  ][line width=0.08]  [draw opacity=0] (7.14,-3.43) -- (0,0) -- (7.14,3.43) -- cycle    ;
%Shape: Rectangle [id:dp9734206144602773] 
\draw  [fill={rgb, 255:red, 255; green, 255; blue, 255 }  ,fill opacity=1 ][line width=1.2]  (10,50) -- (60,50) -- (60,70) -- (10,70) -- cycle ;
%Straight Lines [id:da323846154585246] 
\draw    (35,70) -- (35,95.5) ;
\draw [shift={(35,98.5)}, rotate = 270] [fill={rgb, 255:red, 0; green, 0; blue, 0 }  ][line width=0.08]  [draw opacity=0] (7.14,-3.43) -- (0,0) -- (7.14,3.43) -- cycle    ;
%Shape: Rectangle [id:dp05290723093570815] 
\draw  [fill={rgb, 255:red, 255; green, 255; blue, 255 }  ,fill opacity=1 ][line width=1.2] [general shadow={fill=black,shadow xshift=2.25pt,shadow yshift=-2.25pt}] (100,100) -- (150,100) -- (150,150) -- (100,150) -- cycle ;
%Shape: Rectangle [id:dp1093845690627071] 
\draw  [fill={rgb, 255:red, 255; green, 255; blue, 255 }  ,fill opacity=1 ][line width=1.2] [general shadow={fill=black,shadow xshift=2.25pt,shadow yshift=-2.25pt}] (90,0) -- (160,0) -- (160,20) -- (90,20) -- cycle ;
%Straight Lines [id:da7052343660066145] 
\draw    (125,20) -- (125,45.5) ;
\draw [shift={(125,48.5)}, rotate = 270] [fill={rgb, 255:red, 0; green, 0; blue, 0 }  ][line width=0.08]  [draw opacity=0] (7.14,-3.43) -- (0,0) -- (7.14,3.43) -- cycle    ;
%Shape: Rectangle [id:dp58977118062097] 
\draw  [fill={rgb, 255:red, 255; green, 255; blue, 255 }  ,fill opacity=1 ][line width=1.2]  (100,49.5) -- (150,49.5) -- (150,69.5) -- (100,69.5) -- cycle ;
%Straight Lines [id:da7686201216226598] 
\draw    (125,70) -- (125,95.5) ;
\draw [shift={(125,98.5)}, rotate = 270] [fill={rgb, 255:red, 0; green, 0; blue, 0 }  ][line width=0.08]  [draw opacity=0] (7.14,-3.43) -- (0,0) -- (7.14,3.43) -- cycle    ;
%Shape: Rectangle [id:dp27955689134006323] 
\draw  [fill={rgb, 255:red, 255; green, 255; blue, 255 }  ,fill opacity=1 ][line width=1.2] [general shadow={fill=black,shadow xshift=2.25pt,shadow yshift=-2.25pt}] (190,100) -- (240,100) -- (240,150) -- (190,150) -- cycle ;
%Shape: Rectangle [id:dp7233359680399574] 
\draw  [fill={rgb, 255:red, 255; green, 255; blue, 255 }  ,fill opacity=1 ][line width=1.2] [general shadow={fill=black,shadow xshift=2.25pt,shadow yshift=-2.25pt}] (180,0) -- (250,0) -- (250,20) -- (180,20) -- cycle ;
%Straight Lines [id:da5883707293006084] 
\draw    (215,20.5) -- (215,46) ;
\draw [shift={(215,49)}, rotate = 270] [fill={rgb, 255:red, 0; green, 0; blue, 0 }  ][line width=0.08]  [draw opacity=0] (7.14,-3.43) -- (0,0) -- (7.14,3.43) -- cycle    ;
%Shape: Rectangle [id:dp7689392470835601] 
\draw  [fill={rgb, 255:red, 255; green, 255; blue, 255 }  ,fill opacity=1 ][line width=1.2]  (190,49.5) -- (240,49.5) -- (240,69.5) -- (190,69.5) -- cycle ;
%Straight Lines [id:da4636362963186851] 
\draw    (215,70) -- (215,95.5) ;
\draw [shift={(215,98.5)}, rotate = 270] [fill={rgb, 255:red, 0; green, 0; blue, 0 }  ][line width=0.08]  [draw opacity=0] (7.14,-3.43) -- (0,0) -- (7.14,3.43) -- cycle    ;
%Shape: Rectangle [id:dp5466938678848405] 
\draw  [fill={rgb, 255:red, 255; green, 255; blue, 255 }  ,fill opacity=1 ][line width=1.2] [general shadow={fill=black,shadow xshift=2.25pt,shadow yshift=-2.25pt}] (320,100) -- (370,100) -- (370,150) -- (320,150) -- cycle ;
%Shape: Rectangle [id:dp25346000691606374] 
\draw  [fill={rgb, 255:red, 255; green, 255; blue, 255 }  ,fill opacity=1 ][line width=1.2] [general shadow={fill=black,shadow xshift=2.25pt,shadow yshift=-2.25pt}] (310,0) -- (380,0) -- (380,20) -- (310,20) -- cycle ;
%Straight Lines [id:da6559643060823181] 
\draw    (345,20) -- (345,45.5) ;
\draw [shift={(345,48.5)}, rotate = 270] [fill={rgb, 255:red, 0; green, 0; blue, 0 }  ][line width=0.08]  [draw opacity=0] (7.14,-3.43) -- (0,0) -- (7.14,3.43) -- cycle    ;
%Shape: Rectangle [id:dp51010339222892] 
\draw  [fill={rgb, 255:red, 255; green, 255; blue, 255 }  ,fill opacity=1 ][line width=1.2]  (320,50) -- (370,50) -- (370,70) -- (320,70) -- cycle ;
%Straight Lines [id:da28081159414468204] 
\draw    (345,70) -- (345,95.5) ;
\draw [shift={(345,98.5)}, rotate = 270] [fill={rgb, 255:red, 0; green, 0; blue, 0 }  ][line width=0.08]  [draw opacity=0] (7.14,-3.43) -- (0,0) -- (7.14,3.43) -- cycle    ;
%Straight Lines [id:da9602785191580627] 
\draw    (35,150) -- (35,165) -- (180.08,199.31) ;
\draw [shift={(183,200)}, rotate = 193.31] [fill={rgb, 255:red, 0; green, 0; blue, 0 }  ][line width=0.08]  [draw opacity=0] (7.14,-3.43) -- (0,0) -- (7.14,3.43) -- cycle    ;
%Straight Lines [id:da050594365958000154] 
\draw    (125,150) -- (125,165) -- (185.29,193.71) ;
\draw [shift={(188,195)}, rotate = 205.46] [fill={rgb, 255:red, 0; green, 0; blue, 0 }  ][line width=0.08]  [draw opacity=0] (7.14,-3.43) -- (0,0) -- (7.14,3.43) -- cycle    ;
%Straight Lines [id:da1043914383262281] 
\draw    (215,150) -- (215,165) -- (193.83,192.62) ;
\draw [shift={(192,195)}, rotate = 307.48] [fill={rgb, 255:red, 0; green, 0; blue, 0 }  ][line width=0.08]  [draw opacity=0] (7.14,-3.43) -- (0,0) -- (7.14,3.43) -- cycle    ;
%Straight Lines [id:da1372396274984624] 
\draw    (345,150) -- (345,165) -- (199.92,199.31) ;
\draw [shift={(197,200)}, rotate = 346.69] [fill={rgb, 255:red, 0; green, 0; blue, 0 }  ][line width=0.08]  [draw opacity=0] (7.14,-3.43) -- (0,0) -- (7.14,3.43) -- cycle    ;
%Straight Lines [id:da08210012733710736] 
\draw    (190,205) -- (190,230) -- (227,230) ;
\draw [shift={(230,230)}, rotate = 180] [fill={rgb, 255:red, 0; green, 0; blue, 0 }  ][line width=0.08]  [draw opacity=0] (7.14,-3.43) -- (0,0) -- (7.14,3.43) -- cycle    ;
%Shape: Rectangle [id:dp11932353610641866] 
\draw  [fill={rgb, 255:red, 255; green, 255; blue, 255 }  ,fill opacity=1 ][line width=1.2] [general shadow={fill=black,shadow xshift=2.25pt,shadow yshift=-2.25pt}] (230,205) -- (280,205) -- (280,255) -- (230,255) -- cycle ;
%Straight Lines [id:da6074386369886411] 
\draw    (280,230) -- (327,230) ;
\draw [shift={(330,230)}, rotate = 180] [fill={rgb, 255:red, 0; green, 0; blue, 0 }  ][line width=0.08]  [draw opacity=0] (7.14,-3.43) -- (0,0) -- (7.14,3.43) -- cycle    ;

% Text Node
\draw (35,125) node  [font=\small]  {$F_{1}( k_{1} ,\cdot )$};
% Text Node
\draw (35,10.5) node  [font=\small]  {$a_{1}$};
% Text Node
\draw (35,60) node  [font=\small]  {$a_{1} +k$};
% Text Node
\draw (125,125) node  [font=\small]  {$F_{1}( k_{1} ,\cdot )$};
% Text Node
\draw (125,10) node  [font=\small]  {$a_{2}$};
% Text Node
\draw (125,59.5) node  [font=\small]  {$a_{2} +2k$};
% Text Node
\draw (215,125) node  [font=\small]  {$F_{1}( k_{1} ,\cdot )$};
% Text Node
\draw (215,10) node  [font=\small]  {$a_{3}$};
% Text Node
\draw (215,59.5) node  [font=\small]  {$a_{3} +3k$};
% Text Node
\draw (345,125) node  [font=\small]  {$F_{1}( k_{1} ,\cdot )$};
% Text Node
\draw (345,10) node  [font=\small]  {$a_{v}$};
% Text Node
\draw (345,60) node  [font=\small]  {$a_{v} +vk$};
% Text Node
\draw (190,200) node  [font=\large]  {$\oplus $};
% Text Node
\draw (255,230) node  [font=\small]  {$F_{2}( k_{2} ,\cdot )$};
% Text Node
\draw (330,226.6) node [anchor=south] [inner sep=0.75pt]  [font=\small]  {$tag$};
% Text Node
\draw (280,10) node    {$\cdots $};
% Text Node
\draw (280,60) node    {$\cdots $};
% Text Node
\draw (280,125) node    {$\cdots $};


\end{tikzpicture}
  \caption{$\rm PMAC_0$ 构造}
  \label{fig:6-9}
\end{figure}

\begin{theorem}\label{theo:6-11}
如果 $F_1$ 和 $F_2$ 都是安全的 PRF,且 $|\mathcal{Y}|$ 和素数 $p$ 都是超多项式的,那么 $\rm PMAC_0$ 对于任何多项式约束的 $\ell$ 都是安全的 PRF。
\begin{quote}
特别地,对于每个按照攻击游戏 \ref{game:4-2} 攻击 $\rm PMAC_0$ 的 PRF 对手 $\mathcal{A}$,假设它最多能够向其挑战者发起 $Q$ 次查询,则必然存在两个 PRF 对手 $\mathcal{B}_1$ 和 $\mathcal{B}_2$,其中 $\mathcal{B}_1$ 和 $\mathcal{B}_2$ 都是围绕 $\mathcal{A}$ 的基本包装器,满足:
\end{quote}
\begin{equation}\label{eq:6-28}
{\rm PRF}\mathsf{adv}[\mathcal{A}, {\rm PMAC_0}]\leq{\rm PRF}\mathsf{adv}[\mathcal{B}_1,F_1]+{\rm PRF}\mathsf{adv}[\mathcal{B}_2,F_2]+\frac{Q^2}{2|\mathcal{Y}|}+\frac{Q^2\ell^2}{2p}
\end{equation}
\end{theorem}

当使用 $\rm PMAC_0$ 时,我们要把输入消息切分成多个分组,其中的每个分组都是 $\mathbb{Z}_p$ 中的元素。在实践中,这是很不方便的。相对地,将每条消息切分成 $\{0,1\}^n$ 中的 $n$ 比特序列就要容易得多。接下来介绍的一个更好的并行 MAC 构造,正是通过使用有限域 ${\rm GF}(2^n)$ 代替 $\mathbb{Z}_p$ 实现的。这是一个很好的例子,能够说明为什么有限域 ${\rm GF}(2^n)$ 在密码学中如此重要。出于安全考虑,我们经常需要将计算定义在某个域上,但是像 $\mathbb{Z}_p$ 这样的素阶有限域在实践中总是不太方便。${\rm GF}(2^n)$ 相对来说就要好很多,它上面的算术运算更快,还能很自然地让我们对 $n$ 比特序列进行操作。

\begin{snote}[比 $\rm PMAC_0$ 更好的 PMAC。]
尽管 $\rm PMAC_0$ 非常适合用在并行硬件上,但它仍然有改进空间。事实上,还有一些比 $\rm PMAC_0$  更优秀的实现,包括 PMAC 和 XECB,它们都是可并行的。具体地,PMAC 相对 $\rm PMAC_0$ 来说有如下的改进:
\begin{itemize}
	\item PMAC 使用有限域 ${\rm GF}(2^n)$ 上的算术运算,而不是 $\mathbb{Z}_p$。${\rm GF}(2^n)$ 中的元素可以表示为 $n$ 比特长的序列,域上的加法就是按位异或。正因如此,PMAC 实际上只是使用了 $F_1=F_2=F$,其中 $F$ 是定义在 $(\mathcal{K},\mathcal{Y},\mathcal{Y})$ 上的一个 PRF。PMAC 的输入空间由 $\mathcal{Y}=\{0,1\}^n$ 中的元素序列组成,而不是 $\mathbb{Z}_p$ 中的元素。
	\item PMAC 的第 $i$ 分组的掩码被定义为 $\gamma_i\cdot k$,其中 $\gamma_1,\gamma_2,\dots$ 是 ${\rm GF}(2^n)$ 中的固定常数。乘法也定义在 ${\rm GF}(2^n)$ 上。$\gamma_i$ 是经过特别选取的,为的是保证从 $\gamma_i\cdot k$ 计算 $\gamma_{i+1}\cdot k$ 成本很低。
	\item PMAC 的密钥 $k$ 是由 $k\leftarrow F(k_1,0^n)$ 派生而来的,并且令 $k_2\leftarrow k_1$。因此,PMAC 使用的密匙比 $\rm PMAC_0$ 的更短。
	\item PMAC 使用一个技巧来保存 $F$ 的一个应用。
	\item PMAC 使用 CMAC $rpf$ 的一个变体来提供按位 PRF。
\end{itemize}
结果就是,PMAC 在串行设备上与 ECBC 和 NMAC 的效率相近,但在并行或流水线设备上有更好的性能。PMAC 在本章所介绍的所有 PRF 中是最优秀的,它在各种计算机结构上都能很好地工作,对长消息和短消息都很有效。
\end{snote}

\begin{snote}[$\rm PMAC_0$ 是增量的。]
假设 Bob 为某一条长消息 $m$ 计算了标签 $t$,一段时间后他改变了 $m$ 中的一个分组,想要重新计算这条新消息 $m'$ 的标签。当使用 CBC-MAC 时,现在的标签 $t$ 对于新的计算是毫无用处的——Bob 必须从头开始计算 $m'$ 的新标签。但使用 $\rm PMAC_0$ 时,情况就会有很大的改观。假设用于构造 $\rm PMAC_0$ 的 PRF $F_2$ 是一个分组密码,比如 AES,令 $D$ 为该分组密码的解密算法。令 $m'$ 是将 $m$ 的第 $i$ 个分组从 $a_i$ 修改为 $a_i'$ 后得到的新消息。那么,我们很容易由 $m$ 的标签 $t:={\rm PMAC_0}(k,m)$ 推导出 $m'$ 的新标签 $t':={\rm PMAC_0}(k,m')$,方法如下:

\vspace{5pt}

\hspace*{5pt} 令 $t_1\leftarrow D(k_2,t)$\\
\hspace*{26pt} 令 $t_2\leftarrow t_1\oplus F_1(k_1,\;a_i+i\cdot k)\oplus F_1(k_1,\;a_i'+i\cdot k)$\\
\hspace*{26pt} 令 $t'\leftarrow F_2(k_2,t_2)$

\vspace{5pt}

\noindent
因此,给定某个长消息 $m$ 的标签(以及 MAC 的密钥),我们很容易推导出对 $m$ 进行局部修改后的新标签。具有这种性质的 MAC 被称作是\textbf{增量的(incremental)}。我们上面标明,使用分组密码实现的 $\rm PMAC_0$ 是增量的。
\end{snote}