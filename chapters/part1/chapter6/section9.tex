\section{案例研究:ANSI CBC-MAC}\label{sec:6-9}

当使用 PRF 构建 MAC 时,实现者经常会只输出 PRF 输出的 $w$ 个最高有效比特来缩短最终得到的标签。练习 4.4 表明,将一个安全的 PRF 的输出截短并不会影响其作为 PRF 的安全性。但是,这种截短会影响派生的 MAC。定理 \ref{theo:6-2} 表明,$w$ 越小,MAC 的安全性就越低。特别是,该定理在具体的安全边界中包含一个 $1/2^w$ 的误差项。

两个 ANSI 标准(ANSI X9.9 和 ANSI X9.19)和两个 ISO 标准(ISO 8731-1 和 ISO/IEC 9797)指定了 ECBC 的几个变体,它们使用 DES 作为底层的 PRF,提供消息认证的功能。这些标准都截短了 ECBC-DES 的最终 $64$ 比特输出,只使用输出的最左边 $w$ 个比特,其中 $w=32$,$48$ 或 $64$。这种设计以降低安全性为代价,减少了标签的长度。

ANSI 的两个 CBC-MAC 标准都指定了一个填充方案,用于填充长度不是 DES 或者 AES 分组长度整数倍的消息。该填充方案与 \ref{sec:6-8} 节中描述的函数 $inj$ 相同。在签署消息和验证消息-标签对时,使用的填充方案都应当是相同的。