\section{认证加密的引申义}\label{sec:9-2}

在构建 AE 安全的系统之前,让我们先再讨论一下定义 \ref{def:9-1},看看它到底意味着什么。考虑一个发送者 Alice 和一个接收者 Bob,他们共享一个密钥 $k$。Alice 通过公网向 Bob 发送了一连串的消息。每条消息都用一个 AE 安全的密码 $\mathcal{E}=(E,D)$ 加密,使用的密钥就是 $k$。

首先,考虑一个窃听对手 $\mathcal{A}$。由于 $\mathcal{E}$ 是 CPA 安全的,这并不能使得 $\mathcal{A}$ 获取任何新的关于 Alice 发送给 Bob 的消息的信息。

现在,考虑一个更具侵略性的对手 $\mathcal{A}$,它试图让 Bob 收到一条并不由 Alice 发出的消息。我们声称这不可能发生。为了了解原因,不妨考虑下面这个只有一条消息的例子:Alice 向 Bob 发送一条 $m$ 的加密消息,但密文 $c$ 在中途被 $\mathcal{A}$ 截获。对手的目标是创造某个 $\hat{c}$,使得 $\hat{m}:=D(k,\hat{c})\neq\mathsf{reject}$,同时 $\hat{m}\neq m$。这样的 $\hat{c}$ 就能够欺骗 Bob,使他认为 Alice 发送的是 $\hat{m}$,而不是 $m$。但这样一来,$\mathcal{A}$ 也能够就 $\mathcal{E}$ 赢得攻击游戏 \ref{game:9-1},而这与 $\mathcal{E}$ 的密文完整性相矛盾。因此,$\mathcal{A}$ 不可能修改 $c$ 而不被发现。更一般地说,将该论证应用于多条消息的场合,我们就能表明,$\mathcal{A}$ 不能使 Bob 接受任何不由 Alice 发出的消息。这里,更一般的结论是,\emph{密文完整性}就能够导出\emph{消息完整性}。

\subsection{选择密文攻击:一个例子}\label{subsec:9-2-1}

我们现在考虑一种更具侵略性的攻击类型,称为\textbf{选择密文攻击(chosen ciphertext attack)}。正如我们将要看到的,即使面对如此强大的攻击,AE 安全的密码也能为消息提供机密性和完整性。

为了说明选择密文攻击,假设 Alice 向 Bob 发送了一封电子邮件。简单起见,我们假设每封电子邮件都以字符 \texttt{To:} 为开头,后面紧跟着收件人的邮箱地址。因此,发送给 Bob 的邮件会以 \texttt{To:bob@mail.com} 作为开头,而发送给 Mel 的邮件则会从 \texttt{To:mel@mail.com} 开始。邮件服务器会解密收到的每一封邮件,并将其发送到收件人的收件箱:以 \texttt{To:bob@mail.com} 为开头的邮件会被送入 Bob 的收件箱,而以 \texttt{To:mel@mail.com} 为开头的邮件会被送入 Mel 的收件箱。

这个故事中的攻击者 Mel 想要阅读 Alice 发给 Bob 的邮件。对 Mel 来说不幸的是,Alice 很小心,她用一个只有 Alice 和邮件服务器知道的密钥加密了邮件。当邮件服务器收到密文 $c$ 时,它会解密密文,并将得到的明文发送到 Bob 的收件箱。因此,Mel 是无法阅读这封邮件的。

然而,我们表明,如果 Alice 用 CPA 安全的密码,比如随机化计数器模式或随机化 CBC 模式来加密邮件,Mel就能很容易地获得邮件内容。方法如下:Mel 在密文 $c$ 到达邮件服务器之前截获它,并对其进行修改,以获取一个新的密文 $\hat{c}$,并使 $\hat{c}$ 的密文以 \texttt{To:mel@mail.com} 为开头,但其余部分与原始消息相同。此后,Mel 再将 $\hat{c}$ 转发给邮件服务器。当邮件服务器收到 $\hat{c}$ 时,它会解密这条密文,并(错误地)将明文发送给 Mel 的收件箱,使得 Mel 可以轻松读取它。

为了成功发起这种攻击,Mel 必须首先解决以下问题:给定某条消息 $(u\,\Vert\,m)$ 的加密 $c$,其中 $u$ 是一个固定的已知前缀(在我们的例子中,$u:=\texttt{To:bob@mail.com}$),生成一条密文 $\hat{c}$,它的解密是 $(v\,\Vert\,m)$,其中,$v$ 是另一个前缀(在我们的例子中,$v:=\texttt{To:mel@mail.com}$)。

我们表明,如果加密方案是随机化计数器模式或随机化 CBC 模式,Mel 就可以轻松地解决这个问题。简单起见,我们假设 $u$ 和 $v$ 都是二进制序列,其长度与底层分组密码的分组大小相同。与之前一样,我们记 $c[0]$ 和 $c[1]$ 分别是 $c$ 的第一个和第二个分组,其中 $c[0]$ 就是随机 IV。Mel 构建 $\hat{c}$ 的方法如下:
\begin{itemize}
	\item 随机化计数器模式:$\hat{c}$ 与 $c$ 基本相同,只是 $\hat{c}[1]:=c[1]\oplus u\oplus v$。
	\item 随机化 CBC 模式:$\hat{c}$ 与 $c$ 基本相同,只是 $\hat{c}[0]:=c[0]\oplus u\oplus v$。
\end{itemize}
不难看出,在这两种情况下,对 $\hat{c}$ 的解密都是从前缀 $v$ 开始的(见 \ref{subsec:3-3-2} 小节)。现在,Mel 就能够获得 $\hat{c}$ 的解密,并且明文读取秘密消息 $m$。

刚才发生了什么?我们已经证明了,这两种加密模式都是 CPA 安全的,但是上面的介绍展示了破解它们的方法。这就是一个选择密文攻击的例子——通过查询对 $\hat{c}$ 的解密,Mel 就能够推断出对 $c$ 的解密。这种攻击再次展示了攻击者如何利用一个密码的\emph{易被控制性(malleability)}——我们曾经在 \ref{subsec:3-3-2} 小节中展示过一个基于易被控制性的攻击方式。

正如我们刚刚看到的,当攻击者可以解密某些密文时,即便他不能直接解密他感兴趣的密文,一个 CPA 安全的系统也会变得完全不安全。换句话说,密文完整性的缺失就会完全损害机密性——就算明文完整性不是明确的安全要求,情况也是如此。

我们非正式地论证,如果 Alice 使用的是一个 AE 安全的密码 $\mathcal{E}=(E,D)$,上述攻击就无法成功。假设 Mel 截获了一条密文 $c:=E(k,m)$。他试图创建另一条密文 $\hat{c}$,它满足 (1) $\hat{m}:=D(k,\hat{c})$ 从前缀 $v$ 开始,以及 (2) 对手可以从 $\hat{m}$ 恢复 $m$,特别地,$\hat{m}\neq\mathsf{reject}$。密文完整性——以及随之而来的 AE 安全性——意味着,攻击者无法创建这样的 $\hat{c}$。事实上,攻击者无法创建任何新的有效密文,因此,一个 AE 安全的密码就可以挫败这种攻击。

在下一小节中,我们将正式定义选择密文攻击。我们还将表明,如果一个密码是 AE 安全的,那么即使面对这种类型的攻击,它也是安全的。
 
\subsection{选择密文攻击:定义}\label{subsec:9-2-2}

在这一小节中,我们将正式定义选择密文攻击。在这样的攻击中,对手拥有选择明文攻击中攻击者所拥有的所有权力,除此之外,它还可以获取由它选取的密文的解密——但受到一个限制。回顾一下,在选择明文攻击中,对手向挑战者发起了一连串的加密查询,并获取了一些密文作为应答。我们施加的限制是,对手不得要求解密这些密文中的任意一条。虽然这样的限制对于使攻击游戏有意义来说是有必要的,但它看起来可能有点不直观:如果对手可以解密由它选择的密文,为什么它不能解密最重要的密文?我们将在后面(\ref{sec:9-3} 节)更深层次地解释这个定义背后的直觉。我们将在下面(\ref{subsec:9-2-3} 小节)表明,如果一个密码是 AE 安全的,它对选择密文攻击也是安全的。

下面是正式的攻击游戏:

\begin{game}[CCA 安全性]\label{game:9-2}
对于一个定义在 $(\mathcal{K},\mathcal{M},\mathcal{C})$ 上的给定密码 $\mathcal{E}=(E,D)$ 和一个给定对手 $\mathcal{A}$,我们定义两个实验。对于 $b=0,1$,我们定义:

\vspace*{5pt}

\noindent\textbf{实验 $b$:}
\begin{itemize}
	\item 挑战者随机选取 $k\overset{\rm R}\leftarrow\mathcal{K}$。
	\item $\mathcal{A}$ 向挑战者发起一连串查询。每个查询都属于下面两种类型中的一个:
	\begin{itemize}
		\item \emph{加密查询}:对于 $i=1,2,\dots$,第 $i$ 个加密查询由一对消息 $(m_{i0},m_{i1})\in\mathcal{M}^2$ 组成。挑战者计算 $c_i\overset{\rm R}\leftarrow E(k,m_{ib})$,并将 $c_i$ 发送给 $\mathcal{A}$。
		\item \emph{解密查询}:对于 $j=1,2,\dots$,第 $j$ 个解密查询由一条密文 $\hat{c}_j\in\mathcal{C}$ 组成,该密文不是对之前任意一次加密查询的应答,即:
			\[
			\hat{c}_j\notin\{c_1,c_2,\dots\}
			\]
		挑战者计算 $\hat{m}_j\leftarrow D(k,\hat{c}_j)$,并将 $\hat{m}_j$ 发送给 $\mathcal{A}$。
	\end{itemize}
	\item 在游戏结束时,对手输出一个比特 $\hat{b}\in\{0,1\}$。
\end{itemize}
令 $W_b$ 是 $\mathcal{A}$ 在实验 $b$ 中输出 $1$ 的事件。我们将 $\mathcal{A}$ 相对于 $\mathcal{E}$ 的\textbf{优势}定义为:
\[
\mathrm{CCA}\mathsf{adv}[\mathcal{A},\mathcal{E}]
:=
\big\lvert
\Pr[W_0]-\Pr[W_1]
\big\rvert
\]
\end{game}

我们强调,在上述攻击游戏中,加密查询和解密查询可以以任意次序交错进行。

\begin{definition}[CCA 安全性]\label{def:9-5}
如果对于所有的有效对手 $\mathcal{A}$,$\mathrm{CCA}\mathsf{adv}[\mathcal{A},\mathcal{E}]$ 的值都可忽略不计,我们就称密码 $\mathcal{E}$ \textbf{对选择密文攻击是语义安全的(semantically secure against chosen ciphertext attack)},简称为 \textbf{CCA 安全的 (CCA-secure)}。
\end{definition}

在某些情况下,每条消息都需要生成一个新的密钥,因此,一个特定的密钥 $k$ 只会被用于加密一条消息。当攻击者试图愚弄用户用一个 $k$ 来解密多条密文时,系统也想要对选择密文攻击保持安全性。对于这些情况,我们定义针对下述对手的安全性,它只能发起\emph{一次}加密查询,但是可以发起多次解密查询。

\begin{definition}[1CCA 安全性]\label{def:9-6}
在攻击游戏 \ref{game:9-2} 中,如果对手 $\mathcal{A}$ 被限制只能发起一次加密查询,我们就将其优势表示为 $\mathrm{1CCA}\mathsf{adv}[\mathcal{A},\mathcal{E}]$。如果对于所有的有效对手 $\mathcal{A}$,$\mathrm{1CCA}\mathsf{adv}[\mathcal{A},\mathcal{E}]$ 的值都可忽略不计,我们就称密码 $\mathcal{E}$ \textbf{对选择密文攻击是一次性语义安全的(one-time semantically secure against chosen ciphertext attack)},简称为 \textbf{1CCA 安全的 (CCA-secure)}。
\end{definition}

正如 \ref{subsec:2-2-5} 小节所讨论的,攻击游戏 \ref{game:9-2} 也可以被重构为一个``比特猜测"游戏,此时,挑战者不再拥有两个独立的实验,而是会随机选择一个 $b\in\{0,1\}$,然后针对对手 $\mathcal{A}$ 运行实验 $b$。在这个游戏中,我们定义 $\mathcal{A}$ 的\emph{比特猜测优势} $\mathrm{CCA}\mathsf{adv}^*[\mathcal{A},\mathcal{E}]$(以及 $\mathrm{1CCA}\mathsf{adv}^*[\mathcal{A},\mathcal{E}]$)为 $\lvert\Pr[\hat{b}=b]-1/2\lvert$。\ref{subsec:2-2-5} 小节中的一般结论(即式 \ref{eq:2-11})在此也同样适用:
\begin{equation}\label{eq:9-1}
\mathrm{CCA}\mathsf{adv}[\mathcal{A},\mathcal{E}]
=
2\cdot
\mathrm{CCA}\mathsf{adv}^*[\mathcal{A},\mathcal{E}]
\end{equation}
同样地,对于被限制只能进行单次加密查询的对手,我们有:
\begin{equation}\label{eq:9-2}
\mathrm{1CCA}\mathsf{adv}[\mathcal{A},\mathcal{E}]
=
2\cdot
\mathrm{1CCA}\mathsf{adv}^*[\mathcal{A},\mathcal{E}]
\end{equation}

\subsection{认证加密意味着选择密文安全性}\label{subsec:9-2-3}

下面我们表明,每个 AE 安全的系统都是 CCA 安全的。同样地,每个 1AE 安全的系统都是 1CCA 安全的。

\begin{theorem}\label{theo:9-1}
令 $\mathcal{E}=(E,D)$ 是一个密码。如果 $\mathcal{E}$ 是 AE 安全的,那么它也是 CCA 安全的。如果 $\mathcal{E}$ 是 1AE 安全的,那么它也是 1CCA 安全的。
\begin{quote}
特别地,假设 $\mathcal{A}$ 是一个针对 $\mathcal{E}$ 的 CCA 对手,它最多发起 $Q_\mathrm{e}$ 次加密查询和 $Q_\mathrm{d}$ 次解密查询。则存在一个 CPA 对手 $\mathcal{B}_\mathrm{cpa}$ 和一个 CI 对手 $\mathcal{B}_\mathrm{ci}$,其中,$\mathcal{B}_\mathrm{cpa}$ 和 $\mathcal{B}_\mathrm{ci}$ 都是围绕 $\mathcal{A}$ 的基本包装器,满足:
\end{quote}
\begin{equation}\label{eq:9-3}
\mathrm{CCA}\mathsf{adv}[\mathcal{A},\mathcal{E}]
\leq
\mathrm{CPA}\mathsf{adv}[\mathcal{B}_\mathrm{cpa},\mathcal{E}]+
2Q_\mathrm{d}\cdot
\mathrm{CI}\mathsf{adv}[\mathcal{B}_\mathrm{ci},\mathcal{E}]
\end{equation}
\begin{quote}
此外,$\mathcal{B}_\mathrm{cpa}$ 和 $\mathcal{B}_\mathrm{ci}$ 最多都只能发起 $Q_\mathrm{e}$ 次加密查询。
\end{quote}
\end{theorem}

在证明该定理之前,我们先指出它的一个逆命题:如果一个密码是 CCA 安全的,并且能够提供\emph{明文}完整性,它就一定是 AE 安全的。在练习 \ref{exer:9-15} 中,我们要求你证明这一命题。这两个结论共同为下面这一论断提供了一个非常强的支撑,即 AE 安全性对于在非安全信道上的通用通信需求来说是\emph{正确的}安全定义。我们还注意到,建立一个不提供密文(或明文)完整性的 CCA 安全密码也是有可能的,我们会在练习 \ref{exer:9-12} 中提供一个例子。

\begin{proof}[证明思路]
一个 CCA 对手 $\mathcal{A}$ 发起加密查询和被允许的解密查询。我们首先论证,所有对于这些解密查询的应答都必然是 $\mathsf{reject}$。想要知道为什么,不妨观察一下,如果对手曾经发起过一个有效的解密查询 $c_i$,并且对它的解密不是 $\mathsf{reject}$,那么这个 $c_i$ 就可以被用于赢得密文完整性游戏。因此,由于 $\mathcal{A}$ 的所有解密查询都会被拒绝,对手无法通过发起解密查询学到任何新的东西,因此还不如直接丢弃掉它们。在移除解密查询后,我们最终得到的就是一个标准的 CPA 游戏。由于 $\mathcal{E}$ 是 CPA 安全的,因此对手无法赢得该游戏。我们的结论就是,$\mathcal{A}$ 在赢得 CCA 游戏中的优势可忽略不计。
\end{proof}

\begin{proof}
假设 $\mathcal{A}$ 是一个有效 CCA 对手,它像在攻击游戏 \ref{game:9-2} 中那样攻击 $\mathcal{E}$,最多发起 $Q_\mathrm{e}$ 次加密查询和 $Q_\mathrm{d}$ 次解密查询。我们想证明,如果 $\mathcal{E}$ 是 AE 安全的,$\mathrm{CCA}\mathsf{adv}[\mathcal{A},\mathcal{E}]$ 就是可忽略不计的。下面,我们使用 CCA 和 CPA 攻击游戏的比特猜测版本,并表明:
\begin{equation}\label{eq:9-4}
\mathrm{CCA}\mathsf{adv}^*[\mathcal{A},\mathcal{E}]
\leq
\mathrm{CPA}\mathsf{adv}^*[\mathcal{B}_\mathrm{cpa},\mathcal{E}]+
Q_\mathrm{d}\cdot
\mathrm{CI}\mathsf{adv}[\mathcal{B}_\mathrm{ci},\mathcal{E}]
\end{equation}
其中,$\mathcal{B}_\mathrm{cpa}$ 和 $\mathcal{B}_\mathrm{ci}$ 是两个有效对手。这样,根据式 \ref{eq:9-4},\ref{eq:9-1} 和 \ref{eq:5-4},我们就能得到式 \ref{eq:9-3}。此外,正如我们将要看到的,对手 $\mathcal{B}_\mathrm{cpa}$ 最多发起 $Q_\mathrm{e}$ 次加密查询;因此,如果 $\mathcal{E}$ 是 1AE 安全的,它也一定是 1CCA 安全的。

\vspace*{5pt}

\noindent\textbf{游戏 $\mathbf{0}$。}
我们将游戏 $0$ 定义为攻击游戏 \ref{game:9-2} 的比特猜测版本。在该游戏中,挑战者工作如下:

\vspace*{10pt}

\hspace*{5pt} 随机选取 $b\overset{\rm R}\leftarrow\{0,1\}$
\hspace*{90pt} // \quad $\mathcal{A}$ \emph{会试着猜测} $b$\\
\hspace*{26pt} 随机选取 $k\overset{\rm R}\leftarrow\mathcal{K}$

\vspace*{5pt}

\hspace*{5pt} 当从 $\mathcal{A}$ 处收到第 $i$ 个加密查询 $(m_{i0},m_{i1})$ 时:\\
\hspace*{50pt} 将 $c_i\overset{\rm R}\leftarrow E(k,m_b)$ 发送给 $\mathcal{A}$

\vspace*{5pt}

\hspace*{5pt} 当从 $\mathcal{A}$ 处收到第 $j$ 个解密查询 $\hat{c}_j$ 时:\\
\hspace*{5pt} ($1$)
\hspace*{24.5pt} 将 $D(k,\hat{c}_j)$ 发送给 $\mathcal{A}$

\vspace*{10pt}

\noindent
最终,对手输出一个猜测比特 $\hat{b}\in\{0,1\}$。如果 $b=\hat{b}$,我们就称 $\mathcal{A}$ 赢得了该游戏,我们用 $W_0$ 表示这一事件。根据定义,比特猜测优势就是:
\begin{equation}\label{eq:9-5}
\mathrm{CCA}\mathsf{adv}^*[\mathcal{A},\mathcal{E}]
=
\big\lvert
\Pr[W_0]-1/2
\big\rvert
\end{equation}

\noindent\textbf{游戏 $\mathbf{1}$。} 现在,我们将挑战者逻辑的第 ($1$) 行修改如下:

\vspace*{10pt}

\noindent
\hspace*{5pt} ($1$)
\hspace*{24.5pt} 将 $\mathsf{reject}$ 发送给 $\mathcal{A}$

\vspace*{10pt}

\noindent
我们声称,$\mathcal{A}$ 无法将这个挑战者与原来的挑战者区分开来。令 $Z$ 为 $\mathcal{A}$ 在游戏 $1$ 中发出一个解密查询 $\hat{c}_j$,满足 $D(k,\hat{c}_j)\neq\mathsf{reject}$ 的事件。显然,只要 $Z$ 不发生,游戏 $0$ 和游戏 $1$ 的进程就是相同的。因此,根据差分引理(即定理 \ref{theo:4-7}),我们可以得到 $|\Pr[W_0]-\Pr[W_1]|\leq\Pr[Z]$。

使用与定理 \ref{theo:6-1} 的证明中类似的``猜测策略",我们可以利用 $\mathcal{A}$ 来建立一个 CI 对手 $\mathcal{B}_\mathrm{ci}$,它能以至少 $\Pr[Z]/Q_\mathrm{d}$ 的概率赢得 CI 攻击游戏。请注意,在游戏 $1$ 中,解密算法根本就没有被使用过。对手 $\mathcal{B}_\mathrm{ci}$ 的策略就是简单地猜测一个随机数 $\omega\in\{1,\dots,Q_\mathrm{d}\}$,然后扮演 $\mathcal{A}$ 的挑战者:
\begin{itemize}
	\item 当 $\mathcal{A}$ 发起加密查询时,$\mathcal{B}_\mathrm{ci}$ 将其转发给自己的挑战者,并将挑战者的应答返回给 $\mathcal{A}$。
	\item 当 $\mathcal{A}$ 发起解密查询 $\hat{c}_j$ 时,$\mathcal{B}_\mathrm{ci}$ 只是简单地将 $\mathsf{reject}$ 发送给 $\mathcal{A}$,除非 $j=\omega$,此时 $\mathcal{B}_\mathrm{ci}$ 会输出 $\hat{c}_j$ 并停机。
\end{itemize}
不难看出 $\mathrm{CI}\mathsf{adv}[\mathcal{B}_\mathrm{ci},\mathcal{E}]\geq\Pr[Z]/Q_\mathrm{d}$,因此:
\begin{equation}\label{eq:9-6}
\big\lvert
\Pr[W_0]-\Pr[W_1]
\big\rvert
\leq
\Pr[Z]
\leq
Q_\mathrm{d}\cdot
\mathrm{CI}\mathsf{adv}[\mathcal{B}_\mathrm{ci},\mathcal{E}]
\end{equation}

\noindent\textbf{最后的归约。}
由于在游戏 $1$ 中,所有的解密查询都被拒绝,所以它本质上就是一个 CPA 攻击游戏。更确切地说,我们可以构建一个 CPA 对手 $\mathcal{B}_\mathrm{cpa}$,它扮演 $\mathcal{A}$ 的挑战者:
\begin{itemize}
	\item 当 $\mathcal{A}$ 发起加密查询时,$\mathcal{B}_\mathrm{cpa}$ 将其转发给自己的挑战者,并将挑战者的应答返回给 $\mathcal{A}$。
	\item 当 $\mathcal{A}$ 发起解密查询时,$\mathcal{B}_\mathrm{cpa}$ 只是简单地将 $\mathsf{reject}$ 发送给 $\mathcal{A}$。
\end{itemize}
在游戏结束时,$\mathcal{B}_\mathrm{cpa}$ 只需输出 $\mathcal{A}$ 所输出的那个比特 $\hat{b}$。显然:
\begin{equation}\label{eq:9-7}
\big\lvert
\Pr[W_1]-1/2
\big\rvert
=\mathrm{CPA}\mathsf{adv}^*[\mathcal{B}_\mathrm{cpa},\mathcal{E}]
\end{equation}
综合式 \ref{eq:9-5},\ref{eq:9-6} 和 \ref{eq:9-7},我们就可以得到式 \ref{eq:9-4},故而定理得证。
\end{proof}