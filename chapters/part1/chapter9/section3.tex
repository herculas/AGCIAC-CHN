\section{作为抽象接口的加密}\label{sec:9-3}

为了深入理解认证加密的定义,我们表明,它准确地捕捉了一个直观的概念,即作为一个\emph{抽象接口(abstract interface)}的安全加密。AE 安全性意味着,对这种接口的实际实现可以被一个理想化实现取代。而在这个理想化实现中,消息实际上是从发送方直接跳变到接收方的,根本不需要经过任何网络通信(甚至是任何加密方式的通信)。下面,我们更系统地阐述这一想法。

假设发送方 $S$ 和接收方 $R$ 正在使用某个任意的、基于互联网的系统(比如游戏、拍卖、银行等等)。另外,我们假设 $S$ 和 $R$ 已经共享了一个随机的加密密钥 $k$。在协议过程中,$S$ 会向 $R$ 发送消息 $m_1,m_2,\dots$ 的加密。消息 $m_i$ 由 $S$ 所使用协议的逻辑决定,我们不管它到底是什么。我们可以想象,$S$ 将消息 $m_i$ 放在它的``发件箱"中,而发件箱的具体工作细节与 $S$ 无关。当然,我们还是能知道 $S$ 的发件箱中会发生什么:$m_i$ 会被 $k$ 加密成 $c_i$,而后者会通过一条电线被发送至 $R$。

在接收侧,当一条密文 $\hat{c}$ 抵达电线位于 $R$ 的那一端时,它就会被 $k$ 解密。如果解密的结果是一条消息 $\hat{m}\neq\mathsf{reject}$,这条消息就会被放到 $R$ 的``收件箱"中。每当有消息出现在 $R$ 的收件箱中,$R$ 都可以读取这条消息,并根据它的协议逻辑对消息进行处理,而不必担心消息是如何到达的。

攻击者可能试图以多种方式破坏 $S$ 和 $R$ 之间的通信:
\begin{itemize}
	\item 第一,攻击者可能会丢弃、重排或重放 $S$ 所发送的密文。
	\item 第二,攻击者可能会篡改由 $S$ 发送的密文,或注入一些凭空创建的新密文。
	\item 第三,攻击者可能掌握了由 $S$ 发送的某些消息的部分知识,或者甚至能够影响其中一些消息的选择。
	\item 第四,通过观察 $R$ 的行为,攻击者可能会收集到经过 $R$ 处理的某些消息的部分知识。即使是关于交付给 $R$ 的密文是否被拒绝的知识也可能是有用的。
\end{itemize}

在描述了一个抽象的加密接口及其实现之后,我们下面描述这个接口的一种\emph{理想实现(ideal implementation)},它以一种直观的方式捕捉到了认证加密所提供的安全性保证。当 $S$ 将 $m_i$ 放到它的``发件箱"中时,理想实现现在不再对 $m_i$ 进行加密,而是加密一条与 $m_i$ 无关的假消息 $\mathit{dummy}_i$(它们的长度应当相同),以此来创建一条密文 $c_i$。因此,$c_i$ 可以被视作是 $m_i$ 的一个``把手",但不包含任何(除了长度之外)的关于 $m_i$ 的信息。当 $c_i$ 到达 $R$ 时,对应的消息 $m_i$ 被神奇地从 $S$ 的发件箱复制到 $R$ 的收件箱中。如果一条密文到达 $R$,但不在先前生成的 $c_i$ 中,理想实现就会丢弃它。

这个理想实现其实只是一种思想实验。它显然不能以任何有效的方式在物理上实现(如果不先发明远程传输的话)。然而,正如我们将要论证的那样,如果底层密码 $\mathcal{E}$ 提供认证加密,理想实现——就所有实际目的而言——都等同于真正的实现。因此,协议设计者不需要担心真实实现的任何细节,或者加密定义的细微差别:他可以假装他正在使用抽象的加密接口及其理想实现,其中的密文都只是把手,而消息都会神奇地从 $S$ 跳变到 $R$。

请注意,就算是在理想实现中,攻击者仍然可能丢弃、重排或重放密文,而这些行为将导致对应的消息被丢弃、重排或重放。使用序列号和缓冲区处理这些问题并不困难,但这些工作要留给更高层的协议。

\vspace*{10pt}

下面,我们非正式地论证,当 $\mathcal{E}$ 能够提供认证加密时,现实世界中的实现与理想实现是没有任何区别的。该论证分三步进行。我们先从真实的实现开始,在每一步中,我们都会做一些轻微的修改。
\begin{itemize}
	\item 首先,我们对 $R$ 的收件箱的真实实现进行修改,如下所示。当一条密文 $\hat{c}$ 到达 $R$ 那一端时,由 $S$ 先前生成的密文列表 $c_1,c_2,\dots$ 会被扫描,如果 $\hat{c}=c_i$,相应的消息 $m_i$ 就会被神奇地从 $S$ 的发件箱复制到 $R$ 的收件箱中,而不需要实际运行解密算法。
	
	$\mathcal{E}$ 的正确性属性保证,修改后的行为与真正实现完全相同。
	\item 其次,我们再次修改 $R$ 的收件箱上的实现,此时,如果一条密文 $\hat{c}$ 到达 $R$ 那一端,但不在 $S$ 所生成的密文列表中,该实现就会丢弃 $\hat{c}$。
	
	对手能够区分这种修改和第一种修改的唯一方法是,它创造一条不会被拒绝,但不是由 $S$ 产生的密文。但这是不可能的,因为 $\mathcal{E}$ 有密文完整性。
	\item 第三,我们修改 $S$ 的发件箱的实现,用对 $\mathit{dummy}_i$ 的加密取代对 $m_i$ 的加密。$R$ 的收件箱的实现仍与第二次修改相同。请注意,解密算法在第二和第三种修改中都没有被使用。因此,一个能够区分这个修改和第二个修改的对手就可以被用来直接破解 $\mathcal{E}$ 的 CPA 安全性。因此,由于 $\mathcal{E}$ 是 CPA 安全的,这两种修改就是不可区分的。
\end{itemize}

由于第三个修改与理想实现相同,我们可以看到,从对手的角度来看,真实实现和理想实现是无法区分的。

我们没有考虑到的一个技术问题是,存在这样一种可能性,即由 $S$ 产生的 $c_i$ 不是唯一的。当然,如果我们要把 $c_i$ 看作是理想实现中的把手,唯一性似乎是一个基本属性。事实上,CPA 安全意味着,在理想实现中,生成的 $c_i$ 是唯一的概率是压倒性的,参见练习 \ref{exer:5-12}。