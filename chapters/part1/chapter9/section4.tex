\section{基于通用组合的认证加密密码}\label{sec:9-4}

在本节,我们试图通过组合一个 CPA 安全的密码和一个安全的 MAC 组合来构建认证加密。我们将会表明,先加密后 MAC 范式总是 AE 安全的,但是先 MAC 后加密范式却不是这样。

\subsection{先加密后MAC}\label{subsec:9-4-1}

令 $\mathcal{E}=(E,D)$ 是一个定义在 $(\mathcal{K}_\mathrm{e},\mathcal{M},\mathcal{C})$ 上的密码,$\mathcal{I}=(S,V)$ 是一个定义在 $(\mathcal{K}_\mathrm{m},\mathcal{C},\mathcal{T})$ 上的 MAC。先加密后 MAC 系统 $\mathcal{E}_\mathrm{EtM}=(E_\mathrm{EtM},D_\mathrm{EtM})$,简称为 $\mathrm{EtM}$,定义如下:

\vspace*{10pt}

\hspace*{20pt} $E_\mathrm{EtM}\big((k_\mathrm{e},k_\mathrm{m}),\;m\big)$
				\quad $:=$ \quad
				$c\overset{\rm R}\leftarrow E(k_\mathrm{e},m)$,
				\quad
				$t\overset{\rm R}\leftarrow S(k_\mathrm{m},c)$\\
\hspace*{165.5pt} 输出 $(c,t)$

\vspace*{5pt}

\hspace*{7pt} $D_\mathrm{EtM}\big((k_\mathrm{e},k_\mathrm{m}),\;(c,t)\big)$
				\quad $:=$ \quad
				如果 $V(k_\mathrm{m},c,t)=\mathsf{reject}$,则输出 $\mathsf{reject}$\\
\hspace*{165.5pt} 否则输出 $D(k_\mathrm{e},c)$

\vspace*{10pt}

\noindent
我们称 $\mathrm{EtM}$ 系统定义在 $(\mathcal{K}_\mathrm{e}\times\mathcal{K}_\mathrm{m},\;\mathcal{M},\;\mathcal{C}\times\mathcal{T})$ 上。下面的定理将表明,$\mathcal{E}_\mathrm{EtM}$ 能够提供认证加密。

\begin{theorem}\label{theo:9-2}
令 $\mathcal{E}=(E,D)$ 是一个密码,$\mathcal{I}=(S,V)$ 是一个 MAC 系统。如果 $\mathcal{E}$ 是 CPA 安全的,且 $\mathcal{I}$ 是一个安全的 MAC 系统,则 $\mathcal{E}_\mathrm{EtM}$ 就是 AE 安全的。此外,如果 $\mathcal{E}$ 是语义安全的,且 $\mathcal{I}$ 是一个一次性安全的 MAC 系统,则 $\mathcal{E}_\mathrm{EtM}$ 就是 1AE 安全的。
\begin{quote}
特别地,对于每个像攻击游戏 \ref{game:9-1} 中那样攻击 $\mathcal{E}_\mathrm{EtM}$ 的密文完整性对手 $\mathcal{A}_\mathrm{ci}$,都存在一个像攻击游戏 \ref{game:6-1} 中那样攻击 $\mathcal{I}$ 的 MAC 对手 $\mathcal{B}_\mathrm{mac}$,其中 $\mathcal{B}_\mathrm{mac}$ 是一个围绕 $\mathcal{A}_\mathrm{ci}$ 的基本包装器,由其发起的签名查询不多于由 $\mathcal{A}_\mathrm{ci}$ 发起的加密查询,满足:
\end{quote}
\[
\mathrm{CI}\mathsf{adv}[\mathcal{A}_\mathrm{ci},\mathcal{E}_\mathrm{EtM}]
=
\mathrm{MAC}\mathsf{adv}[\mathcal{B}_\mathrm{mac},\mathcal{I}]
\]
\begin{quote}
对于每个像攻击游戏 \ref{game:5-2} 中那样攻击 $\mathcal{E}_\mathrm{EtM}$ 的 CPA 对手 $\mathcal{A}_\mathrm{cpa}$,都存在一个像攻击游戏 \ref{game:5-2} 中那样攻击 $\mathcal{E}$ 的 CPA 对手 $\mathcal{B}_\mathrm{cpa}$,其中 $\mathcal{B}_\mathrm{cpa}$ 是一个围绕 $\mathcal{A}_\mathrm{cpa}$ 的基本包装器,由其发起的加密查询不多于由 $\mathcal{A}_\mathrm{cpa}$ 发起的加密查询,满足:
\end{quote}
\[
\mathrm{CPA}\mathsf{adv}[\mathcal{A}_\mathrm{cpa},\mathcal{E}_\mathrm{EtM}]
=
\mathrm{CPA}\mathsf{adv}[\mathcal{B}_\mathrm{cpa},\mathcal{E}]
\]
\end{theorem}

\begin{proof}
我们先证明 $\mathcal{E}_\mathrm{EtM}$ 能够提供密文完整性。该证明可以通过一个直接的归约实现。假设 $\mathcal{A}_\mathrm{ci}$ 是一个攻击 $\mathcal{E}_\mathrm{EtM}$ 的密文完整性对手。我们构建一个攻击 $\mathcal{I}$ 的 MAC 对手 $\mathcal{B}_\mathrm{mac}$。

对手 $\mathcal{B}_\mathrm{mac}$ 在 $\mathcal{I}$ 的 MAC 攻击游戏中扮演对手的角色,它与一个 MAC 挑战者 $\mathbf{C}_\mathrm{mac}$ 交互,后者在交互开始时随机选取一个 $k_\mathrm{m}\overset{\rm R}\leftarrow\mathcal{K}_\mathrm{m}$。然后,对手 $\mathcal{B}_\mathrm{mac}$ 模拟 $\mathcal{A}_\mathrm{ci}$ 的 $\mathcal{E}_\mathrm{EtM}$ 密文完整性挑战者,运行如下:

\vspace*{10pt}

\hspace*{5pt} 随机选取 $k_\mathrm{e}\overset{\rm R}\leftarrow\mathcal{K}_\mathrm{e}$\\
\hspace*{26pt} 当从 $\mathcal{A}_\mathrm{ci}$ 处收到一个查询 $m_i\in\mathcal{M}$ 时:\\
\hspace*{50pt} 令 $c_i\overset{\rm R}\leftarrow E(k_\mathrm{e},m_i)$\\
\hspace*{50pt} 在 $c_i$ 处查询 $\mathbf{C}_\mathrm{mac}$,并获得一个应答 $t_i\overset{\rm R}\leftarrow S(k_\mathrm{m},c_i)$\\
\hspace*{50pt} 将 $(c_i,t_i)$ 发送给 $\mathcal{A}_\mathrm{ci}$
\hspace*{10pt} // \quad \emph{于是有} $(c_i,t_i)=E_\mathrm{EtM}\big((k_\mathrm{e},k_\mathrm{m}),m_i\big)$\\
\hspace*{26pt} 最后,$\mathcal{A}_\mathrm{ci}$ 输出一条密文 $(c,t)\in\mathcal{C}\times\mathcal{T}$\\
\hspace*{26pt} 输出该消息-标签对 $(c,t)$

\vspace*{10pt}

\noindent
应该明确的是,$\mathcal{B}_\mathrm{mac}$ 就像在一个真正的密文完整性攻击游戏中一样对 $\mathcal{A}_\mathrm{ci}$ 的查询给出了应答。因此,对手 $\mathcal{A}_\mathrm{ci}$ 能以 $\mathrm{CI}\mathsf{adv}[\mathcal{A}_\mathrm{ci},\mathcal{E}_\mathrm{EtM}]$ 的概率输出一条能让它赢得攻击游戏 \ref{game:9-1} 的密文 $(c,t)$,满足 $(c,t)\notin\{(c_1,t_1),\dots\}$ 且 $V(k_\mathrm{m},c,t)=\mathsf{accept}$。由此可知,$(c,t)$ 是一个能让 $\mathcal{B}_\mathrm{mac}$ 赢得 MAC 攻击游戏的消息-标签对。因此,我们可得 $\mathrm{CI}\mathsf{adv}[\mathcal{A}_\mathrm{ci},\mathcal{E}_\mathrm{EtM}]=\mathrm{MAC}\mathsf{adv}[\mathcal{B}_\mathrm{mac},\mathcal{I}]$,正如定理所要求的。

剩下的工作就是证明,如果 $\mathcal{E}$ 是 CPA 安全的,则 $\mathcal{E}_\mathrm{EtM}$ 也是安全的。这就等于是说,密文中包含的那个使用密钥 $k_\mathrm{m}$ 计算出来的(因而根本不涉及加密密钥 $k_\mathrm{e}$)标签不会在攻击者破坏 $\mathcal{E}_\mathrm{EtM}$ 的 CPA 安全性时提供帮助。这部分的证明非常简单,我们将其留作练习(可参见练习 \ref{exer:5-20})。
\end{proof}

回顾一下我们在第\ref{chap:6}章中给出的安全的 MAC 的定义,我们要求,给定一个消息-标签对 $(c,t)$,攻击者无法制造一个新的标签 $t\neq t'$,同时使得 $(c,t')$ 也是一个合法的消息-标签对。在当时,这样的要求似乎有些奇怪:如果攻击者已经有了一个 $c$ 的有效标签,我们为什么还要关心它能否为 $c$ 找到另一个标签?现在,我们看到,如果攻击者能为 $c$ 找到另一个有效标签 $t'$,他就可以破坏 $\mathrm{EtM}$ 的密文完整性。攻击者可以使用 $\mathrm{EtM}$ 的密文 $(c,t)$ 来构建另一条有效密文 $(c,t')$,并赢得密文完整性游戏。我们对安全的 MAC 的定义能够确保,攻击者无法修改 $\mathrm{EtM}$ 的密文而不被发现。

\subsubsection{实现先加密后 MAC 时的常见错误}\label{subsubsec:9-4-1-1}

在实现先加密后 MAC 时,一个常见的错误是为密码和 MAC 选用相同的密钥,即设置 $k_\mathrm{e}=k_\mathrm{m}$。由此产生的系统无法提供认证加密,而且可能是不安全的,如练习 \ref{exer:9-8} 中所展示的那样。在定理 \ref{theo:9-2} 的证明中,我们利用了这样一个事实,即两个密钥 $k_\mathrm{e}$ 和 $k_\mathrm{m}$ 的选择是相互独立的。

另一种常见的错误是只对密文的一部分应用 MAC 签名算法。我们看一个例子。假设底层的 CPA 安全密码 $\mathcal{E}=(E,D)$ 是由随机化 CBC 模式(见 \ref{subsec:5-4-3} 小节)构建的,那么消息 $m$ 的加密就是 $(r,c)\overset{\rm R}\leftarrow E(k,m)$,这里的 $r$ 是一个随机 IV。当实现先加密后 MAC $\mathcal{E}_\mathrm{EtM}=(E_\mathrm{EtM},D_\mathrm{EtM})$ 时,加密算法被错误地定义为:
\[
E_\mathrm{EtM}\big((k_\mathrm{e},k_\mathrm{m}),\;m\big)
:=
\big\{
(r,c)\overset{\rm R}\leftarrow E(k_\mathrm{e},m),
\;
t\overset{\rm R}\leftarrow S(k_\mathrm{m},c),
\;
\text{输出}\,(r,c,t)
\big\}
\]
这里,$E(k_\mathrm{e},m)$ 输出了密文 $(r,c)$,但 MAC 签名算法只被应用到了 $c$ 上,而 IV 没有受到 MAC 的保护。这个错误会完全破坏密文完整性:给定一条密文 $(r,c,t)$,攻击者可以创建另一条有效密文 $(r',c,t)$,而 $r'\neq r$。解密算法无法检测到这种对 IV 的修改,更不会输出 $\mathsf{reject}$。相反,解密算法会输出 $D\big(k_\mathrm{e},(r',c)\big)$。由于 $(r',c,t)$ 是一条有效密文,对手就赢得了密文完整性游戏。更糟糕的是,假如 $(r,c,t)$ 是对一条消息 $m$ 的加密,那么对于任意的 $\Delta$,将 $(r,c,t)$ 改为 $(r\oplus\Delta,c,t)$ 都会使得 CBC 解密算法输出一条消息 $m'$,而 $m'[0]=m[0]\oplus\Delta$。这意味着,攻击者可以将 $m$ 的第一个分组中的头部信息改为它自己所选择的任何值。ISO 19772 认证加密标准的一个早期版本正是犯了这个错误 \cite{namprempre2014reconsidering}。类似地,在 2013 年,有人发现苹果公司的 iOS 系统中为数据加密而建立的 \texttt{RNCryptor} 设施使用了一个错误的先加密后 MAC,其中的 HMAC 没有被应用到加密 IV 上 \cite{napier2013rncryptor}。

在实现中,另一个需要注意的隐患是,在整条消息的完整性标签都被验证完成之前,系统不应当输出任何明文数据。\ref{sec:9-9} 节会介绍这方面的一个例子。

\subsection{先 MAC 后加密一般是不安全的:SSL 上的填充预言机攻击}\label{subsec:9-4-2}

接下来,我们考虑由一个 CPA 安全密码和一个安全的 MAC 的构成的先 MAC 后加密通用组合。我们将表明,这种构造不一定是 AE 安全的,而且可能会导致许多现实世界中的问题。

为了准确定义先 MAC 后加密范式,令 $\mathcal{I}=(S,V)$ 是一个定义在 $(\mathcal{K}_\mathrm{m},\mathcal{M},\mathcal{T})$ 上的 MAC,$E=(E,D)$ 是一个定义在 $(\mathcal{K}_\mathrm{e},\mathcal{M}\times\mathcal{T},\mathcal{C})$ 上的密码。先 MAC 后加密系统 $\mathcal{E}_\mathrm{MtE}=(E_\mathrm{MtE},D_\mathrm{MtE})$,或简称 $\mathrm{MtE}$,定义如下:

\vspace*{10pt}

\hspace*{20pt} $E_\mathrm{MtE}\big((k_\mathrm{e},k_\mathrm{m}),\;m\big)$
				\quad $:=$ \quad
				$t\overset{\rm R}\leftarrow E(k_\mathrm{m},m)$,
				\quad
				$c\overset{\rm R}\leftarrow S\big(k_\mathrm{e},\,(m,t)\big)$\\
\hspace*{165.5pt} 输出 $c$

\vspace*{5pt}

\hspace*{24pt} $D_\mathrm{MtE}\big((k_\mathrm{e},k_\mathrm{m}),\;c\big)$
				\quad $:=$ \quad
				$(m,t)\leftarrow D(k_\mathrm{e},c)$\\
\hspace*{165.5pt} 如果 $V(k_\mathrm{m},m,t)=\mathsf{reject}$,则输出 $\mathsf{reject}$\\
\hspace*{165.5pt} 否则输出 $D(k_\mathrm{e},c)$

\vspace*{10pt}

\noindent
我们称 $\mathrm{MtE}$ 系统定义在 $(\mathcal{K}_\mathrm{e}\times\mathcal{K}_\mathrm{m},\;\mathcal{M},\;\mathcal{C})$。

\begin{snote}[一种被彻底破解的 $\mathbf{MtE}$ 密码。]
我们表明,就算 $\mathcal{E}$ 是一个 CPA 安全的密码,$\mathcal{I}$ 是一个安全的 MAC,$\mathrm{MtE}$ 也不一定是 AE 安全的。事实上,对于广泛使用的密码和 MAC 来说,$\mathrm{MtE}$ 很可能是不安全的,而这在事实上已经导致了很多针对已部署系统的重大攻击。

考虑一下用于保护 WWW 流量的 SSL 3.0 协议,该协议已经被使用了二十多年(但在现代浏览器中被禁用)。SSL 3.0 使用 $\mathrm{MtE}$ 来组合随机化 CBC 模式加密和安全 MAC。我们曾在第\ref{chap:5}章中表明,随机化 CBC 模式加密是 CPA 安全的,尽管如此,这种组合仍然被彻底破解了:攻击者可以使用选择密文攻击有效地解密所有流量。这就导致了一种针对 SSL 3.0 的破坏性攻击,称为 \textbf{POODLE} \cite{moller2014poodle}。

我们假设 CBC 中所使用的底层分组密码运行在 $16$ 字节分组上,就像 AES 那样。回顾一下,CBC 模式加密将其输入填充到分组长度的整数倍,而在 SSL 3.0 中,具体的做法如下:如果需要一个长度为 $p>0$ 字节的填充序列,该方案就用一个长为 $p-1$ 字节的任意序列来填充消息,并且增加再额外增加一个字节,该字节的值就是 $(p-1)$。如果消息长度已经是分组长度($16$ 字节)的整数倍,SSL 3.0 就会添加一个长为 $16$ 字节的假分组,其最后一个字节被置为 $15$,而其前 $15$ 字节是任意内容。在解密过程中,算法会读取最后一个字节,并移除其值相应的那么多字节,以正确地移除填充。

具体来说,将 $\mathrm{MtE}$ 应用于随机化 CBC 模式加密和安全 MAC 得到的密码 $\mathcal{E}_\mathrm{MtE}=(E_\mathrm{MtE},D_\mathrm{MtE})$ 工作如下:
\begin{itemize}
	\item $E_\mathrm{MtE}\big((k_\mathrm{e},k_\mathrm{m}),\,m\big)$:首先,使用 MAC 签名算法为 $m$ 计算一个定长标签 $t\overset{\rm R}\leftarrow E(k_\mathrm{m},m)$。然后,用随机化 CBC 加密对 $m\,\Vert\,t$ 进行加密:对信息进行填充,然后使用密钥 $k_\mathrm{e}$ 和一个随机的 IV 在 CBC 模式下进行加密。因此,下面的数据会被加密以生成密文 $c$:
	\begin{equation}\label{eq:9-8}
		\boxed{\qquad\qquad\text{消息}\; m \qquad\qquad}\boxed{\quad\text{标签}\; t \quad}\boxed{\quad\text{填充}\; p \quad}
	\end{equation}
	
	请注意,标签 $t$ 并不保护填充的完整性。我们将利用这一点,用选择密文攻击来打破 CPA 安全性。
	\item $D_\mathrm{MtE}\big((k_\mathrm{e},k_\mathrm{m}),\,c\big)$:运行 CBC 解密以获得式 \ref{eq:9-8} 中的明文数据。然后,读取式 \ref{eq:9-8} 中最后一字节,并从数据中移除与其值相等长度的字节(即,如果最后一字节的值是 $3$,就移除该字节,再额外移除 $3$ 个字节),以完全移除填充 $p$。最后验证 MAC 标签,如果有效,就返回剩余的字节作为消息,否则就输出 $\mathsf{reject}$。
\end{itemize}
SSL 3.0 和 TLS 1.0 都使用了随机化 CBC 加密的一种有缺陷的变体,我们曾在练习 \ref{exer:5-13} 中讨论过这个问题,但它与我们这里的讨论无关。在这里,我们假设所使用的随机化 CBC 加密的实现是正确的。
\end{snote}

\begin{snote}[选择密文攻击。]
我们展示一种针对 $\mathcal{E}_\mathrm{MtE}$ 系统的选择密文攻击,它能让对手解密由其挑选的任何密文。考虑到这种攻击的存在,即使底层密码是 CPA 安全的,$\mathcal{E}_\mathrm{MtE}$ 也不一定是 AE 安全的。接下来,我们用 $(E,D)$ 表示用在 CBC 加密中的底层分组密码,它作用于 $16$ 字节的数据分组。

假设对手截获了一条对应于未知消息 $m$ 的有效密文 $c:=E_\mathrm{MtE}\big((k_\mathrm{e},k_\mathrm{m}),\,m)$。$m$ 的长度满足如下条件,即当 MAC 标签 $t$ 被添加到 $m$ 之后,$(m\,\Vert\,t)$ 的长度是 $16$ 字节的整数倍。这就意味着,在 CBC 加密的过程中,有一个完整的 $16$ 字节填充分组被添加到了消息后,且这个填充分组的最后一个字节的值是 $15$。因此,密文看起来就像下面这样:
\[
	c\quad = \quad
	\underbrace{\boxed{\quad c[0]\quad}}_{\text{IV}}\!
	\underbrace{\boxed{\quad c[1]\quad}\;\;\;\cdots\quad}_{m\,\text{的加密}}
	\underbrace{\quad\boxed{\; c[\ell-1]\;}}_{\text{加密标签}}\!
	\underbrace{\boxed{\quad c[\ell]\quad}}_{\text{加密填充}}
\]

我们首先证明,对手能够学到一些关于 $m[0]$($m$ 的第一个 $16$ 字节分组)的知识。这将打破 $\mathcal{E}_\mathrm{MtE}$ 的语义安全性。攻击者用 $c[1]$ 替换 $c$ 的最后一个分组,以准备一个选择密文查询 $\hat{c}$。也就是说:
\begin{equation}\label{eq:9-9}
\hat{c}\quad := \quad
\boxed{\quad c[0]\quad}
\boxed{\quad c[1]\quad}
\quad\cdots\quad
\boxed{\; c[\ell-1]\;}\!
\underbrace{\boxed{\quad c[1]\quad}}_{\text{加密填充?}}
\end{equation}
根据 CBC 解密的定义,解密 $\hat{c}$ 的最后一个分组可以得到 $16$ 字节的明文分组:
\[
v:=D\left(k_\mathrm{e},c[1]\right)\oplus c[\ell-1]=m[0]\oplus c[0]\oplus c[\ell-1]
\]
如果 $v$ 最后一字节的值是 $15$,那么在解密过程中,最后一个分组会被视为一个填充分组而被删除。剩下的序列是一个有效的消息-标签对,能够被正确地解密。如果 $v$ 最后一字节的值不是 $15$,那么对解密查询的应答很有可能就是 $\mathsf{reject}$。

换言之,如果对 $\hat{c}$ 的解密查询的应答不是 $\mathsf{reject}$,攻击者就能知道,$m[0]$ 的最后一个字节就等于 $u:=15\oplus c[0]\oplus c[\ell-1]$ 的最后一个字节。否则,攻击者也能知道 $m[0]$ 的最后一个字节不等于 $u$ 的最后一个字节。这就直接打破了 $\mathcal{E}_\mathrm{MtE}$ 的语义安全性:攻击者能够获得一些关于明文 $m$ 的知识。

读者可以用选择密文攻击中的对手(就像在攻击游戏 \ref{game:9-2} 中那样)重述上述攻击,我们将它留作一个启发性的练习。只要通过单次明文查询和单次密文查询,对手就能以 $1/256$ 的优势赢得游戏。这就证明 $\mathcal{E}_\mathrm{MtE}$ 是不安全的。

现在,假设攻击者使用另一个 IV 获得了对 $m$ 的另一个加密 $c'$。攻击者可以用密文 $c$ 和 $c'$ 来构造四个有用的选择密文查询:它可以用 $c[1]$ 或 $c'[1]$ 替换 $c$ 或 $c'$ 的最后一个分组。攻击者发出这四个密文查询,就可以了解到 $m[0]$ 的最后一字节是否等于以下四个值:
\[
15\oplus c[0]\oplus c[\ell-1],\qquad
15\oplus c[0]\oplus c'[\ell-1],\qquad
15\oplus c'[0]\oplus c[\ell-1],\qquad
15\oplus c'[0]\oplus c'[\ell-1]
\]
中某一个的最后一字节。如果这四个值各不相同,他们就能给攻击者提供四次学习 $m[0]$ 的最后一字节的机会。使用对消息 $m$ 的新加密多次重复这一过程,攻击者很快就能确定 $m[0]$ 的最后一字节。每次选择密文查询都能以 $1/256$ 的概率确定该字节。因此,平均来说,只要进行 $256$ 次选择密文查询,攻击者就能知道 $m[0]$ 的最后一字节的确切值。所以,攻击者可以打破语义安全性,更具体地说,它可以恢复明文的一个字节。接下来,假设攻击者能够请求对 $m$ 右移一比特后明文的加密,得到一条密文 $c_1$。将 $c1[1]$ 插入上一阶段密文(即对未移位的 $m$ 的加密)的最后一个分组,然后发出选择密文查询,攻击者就能揭示 $m[0]$ 的倒数第二个字节。对 $m$ 的每个字节重复该过程,就能够揭示 $m$ 的全部信息。下面我们表明,这能够导出一种针对 SSL 3.0 的真实攻击。
\end{snote}

\begin{snote}[彻底攻破 SSL 3.0。]
选择密文攻击似乎只在理论上可行,但事实上,它们经常被转化为极具破坏性的现实世界攻击。考虑一个网络浏览器和一个名为 \texttt{bank.com} 的受害网络服务器。在两者之间交换的信息使用 SSL 3.0 加密。浏览器和服务器之间共享一个被称为 cookie 的秘密,浏览器在它每一个发往 \texttt{bank.com} 的请求中都嵌入了这个 cookie。抽象地讲,浏览器发往 \texttt{bank.com} 的请求看起来就像:
\[
\boxed{\;\text{GET}\;\texttt{path}\quad\text{cookie:}\;\texttt{cookie}\;}
\]
其中,\texttt{path} 指浏览器向 \texttt{bank.com} 请求的资源的标识符。浏览器只会在它向 \texttt{bank.com} 发出的请求中插入该 cookie。

攻击者的目标是恢复秘密的 cookie。首先,它让浏览器访问 \texttt{attacker.com},在那里,它会向浏览器发送一个 JavaScript 程序。这个程序会迫使浏览器请求 \texttt{bank.com} 的资源 \texttt{/AA}。之所以请求这个路径,是为了确保消息和 MAC 的长度是分组长度($16$ 字节)的倍数,这是攻击所需要的。因此,浏览器向 \texttt{bank.com} 发送以下请求:
\begin{equation}\label{eq:9-10}
\boxed{\;\text{GET}\;\texttt{/AA}\quad\text{cookie:}\;\texttt{cookie}\;}
\end{equation}
该请求会被 SSL 3.0 加密。攻击者可以截获这个加密请求 $c$,并对 MtE 发起选择密文攻击,以了解 cookie 的一个字节。也就是说,攻击者会像式 \ref{eq:9-9} 那样准备好一个 $\hat{c}$,将 $\hat{c}$ 发送给 \texttt{bank.com},并查看\texttt{bank.com} 是否应答一个 SSL 报错信息。如果没有产生报错信息,攻击者就能知道 cookie 的一个字节。JavaScript 程序可以迫使浏览器重复发出式 \ref{eq:9-10} 中的请求,以给攻击者提供它所需的新鲜密文,直到最终揭示 cookie 的一个字节。

一旦对手知道了 cookie 的一个字节,它就可以让 JavaScript 程序向 \texttt{bank.com} 发出请求:
\[
\boxed{\;\text{GET}\;\texttt{/AAA}\quad\text{cookie:}\;\texttt{cookie}\;}
\]
以将 cookie 右移一个字节。这就为攻击者提供了一新的密文分组,不妨将其称作 $c_1[2]$,其中 cookie 被右移了一个字节。重新向服务器发送上一阶段的请求,但现在最后一个分组被替换为 $c_1[2]$,直到揭示 cookie 的第二个字节。对 cookie 的每一个字节重复这一过程,最终就能够揭示整个 cookie。

实际上,浏览器中的 JavaScript 程序能够为攻击者发动选择明文攻击提供充足的工具。而拦截网络中的数据包,对其进行修改并观察服务器的响应,就能为攻击者发动选择密文攻击提供充分的信息。这两者的结合就能完全打破 SSL 3.0 的 MtE 加密。

一个小细节是,每当 \texttt{bank.com} 应答一个 SSL 报错信息,SSL 会话都会关闭。但这并不构成问题:每当浏览器中的 JavaScript 程序向 \texttt{bank.com} 发起一个新的请求,都会自动启动一个新的 SSL 会话。因此,每个选择密文查询都是在不同的会话密钥下加密的,但这对攻击来说没有任何区别:每个查询都会检验 cookie 的一个字节是否等于一个已知的随机字节。只要有足够的查询,攻击者就能了解整个 cookie。
\end{snote}

\subsection{其他填充预言机攻击}\label{subsec:9-4-3}

TLS 1.0 是 SSL 3.0 的一个更新版本。它在填充中添加新的结构(见 \ref{subsec:5-4-4} 小节),以此来防御上一小节中的攻击:当填充 $p$ 个字节时,填充中所有字节的内容都会被置为 $p-1$。此外,在解密过程中,解密者需要检查所有填充字节的值是否都是正确的,如果不然就拒绝密文。这使得攻击者难以发动上一小节中介绍的攻击。当然,我们的目标只是想要表明 MtE 一般来说是不安全的,而 SSL 3.0 已经充分说明了这一点。

\begin{snote}[一种填充预言机计时攻击。]
尽管 TLS 1.0 加入了新的防御措施,但是对 MtE 解密的简陋实现仍然可能遭受攻击。假设这种实现是这样工作的:首先,它使用 CBC 解密所收到的密文;然后,它检查填充结构是否有效,如果不然,它就拒绝该密文;反之,如果填充是有效的,它就检查完整性标签,如果标签也是有效的,它就返回明文。在这个实现中,只有填充结构有效时,完整性标签才会被检查。这意味着,如果有一条密文包含无效的填充结构,而另一条密文包含有效的填充,但其标签是无效的,那么前者会比后者更快被拒绝。攻击者可以测量服务器应答一次选择密文查询所需的时间,如果很快就返回了一条 TLS 错误消息,它就能知道填充结构是无效的。否则,它至少能了解到这个填充是有效的。

这种计时信道被称为\textbf{填充预言机边信道(padding oracle side-channel)}。就像我们在 SSL 3.0 中所做的那样,我们也可以根据这种行为设计一个选择密文攻击,来完全解密一个秘密 cookie,读者可以将其当作一种很好的练习。为了了解如何实现这种攻击,不妨假设一个攻击者截获了一条加密 TLS 1.0 记录 $c$。令 $m$ 是 $c$ 的解密。假设攻击者想要检验 $m[2]$ 的最后一字节是否等于某个固定值 $b$。攻击者创建一个新的密文分组 $\hat{c}[1]:=c[1]\oplus B$,并将包含三个分组的记录 $\hat{c}=(c[0],\hat{c}[1],c[2])$ 发送给服务器。在对 $\hat{c}$ 进行 CBC 解密后,最后一个明文分组将是:
\[
\hat{m}[2]
:=\hat{c}[1]\oplus D\big(k,c[2]\big)
=m[2]\oplus B
\]
如果 $m[2]$ 的最后一字节等于 $b$,$\hat{m}[2]$ 的最后一比特就是 $0$,而这是一个有效的填充。服务器会尝试验证完整性标签,而这就会导致响应缓慢。如果 $m[2]$ 的最后一字节不等于 $b$,$\hat{m}[2]$ 就不以 $0$ 结尾,并且这很可能是一个无效的填充,而服务器很快就会给出应答。通过测量响应时间,攻击者就可以了解到 $m[2]$ 的最后一字节是否等于 $b$。就像我们对 SSL 3.0 所做的那样,用多个选择密文查询来重复这一过程,就可以揭示整个秘密 cookie。

有一种和用在 TLS 1.0 中的攻击类似,但更加复杂的 MtE 填充预言机计时攻击,被称作 Lucky13 \cite{al2013lucky}。想要在实现 TLS 1.0 解密时针对 Lucky13 攻击隐藏计时信息是相当有挑战性的。
\end{snote}

\begin{snote}[信息性报错消息。]
更糟糕的是,TLS 1.0 规范 \cite{dierks1999rfc2246} 中规定,当收到的密文因 MAC 验证错误而被拒绝时,服务器应发送一种特定类型(被称作 \texttt{bad\_record\_mac})的报错消息;而当密文因填充分组无效而被拒绝时,服务器应发送另一种类型(被称作 \texttt{decryption\_failed})的报错消息。理论上,这就能告诉攻击者,一条密文被拒绝到底是因为填充分组无效,还是因为完整性标签被损坏。这就可以使上述选择密文攻击成为可能,而且不需要借助于计时信息。唯一幸运的是,报错消息是加密的,攻击者无法看到错误代码。

尽管如此,这里仍然有一个重要的教训:当解密失败时,系统决不应该解释原因。应该发送一个通用的``\texttt{decryption\_failed}"代码,而不提供任何其他信息。这个问题在 TLS 1.1 中得到了承认和解决。此外,当解密失败时,无论失败的原因是什么,正确的实现总是应该花费相同的时间提供应答。
\end{snote}

\subsection{安全的先 MAC 后加密实例}\label{subsec:9-4-4}

尽管在应用于 CPA 安全的密码时,MtE 一般是不安全的,但是对于我们在第\ref{chap:5}章中讨论的一些特定的 CPA 密码来说,它仍然可以被证明是安全的。我们将在下面的定理 \ref{theo:9-3} 中表明,如果 $\mathcal{E}$ 恰好实现了随机化计数器模式,MtE 就是安全的。我们将在练习 \ref{exer:9-9} 中表明,假设没有消息填充,上述结论对于随机化 CBC 来说也是成立的。

定理 \ref{theo:9-3} 表明,即使 MAC 只是一次性安全的,具有随机化计数模式的先 MAC 后加密也是 AE 安全的。也就是说,使用一个较弱的,只对能发起\emph{单次}选择消息查询的对手安全的 MAC 也足够了。直观地说,我们能用这样的弱 MAC 证明安全性的原因是,MAC 值是被加密的,因而对手更难攻击 MAC。由于一次性 MAC 比多次性 MAC 更短更快,所以具有随机化计数器模式的先 MAC 后加密比先加密后 MAC 略有优势。尽管如此,上一小节中介绍的针对先 MAC 后加密的攻击表明,想要正确实现它仍然是很难的,因此我们不应该使用这种构造。

我们从一个如 \ref{subsec:5-4-2} 小节中讨论的随机化计数器模式密码 $\mathcal{E}=(E,D)$ 开始。我们假设 $\mathcal{E}$ 具有 \ref{subsec:5-4-2} 节末尾的 AES 计数器模式案例研究中所展示的一般结构。即我们使用一个计数器模式的变体,其中,密码 $\mathcal{E}$ 由一个定义在 $(\mathcal{K}_\mathrm{e},\,\mathcal{X}\times\mathbb{Z}_\ell,\,\mathcal{Y})$ 上的 PRF $F$ 构建,且 $\mathcal{Y}:=\{0,1\}^n$。更准确地说,对于消息 $m\in\mathcal{Y}^{\leq\ell}$,算法 $E$ 运行如下:
\[
E(k_\mathrm{e},\,m):=
\left\{
\begin{aligned}
& x\overset{\rm R}\leftarrow\mathcal{X}\\
& \text{对于}\;j=0,\dots,|m|-1:\\
& \qquad u[j]\leftarrow F\big(k_\mathrm{e},\,(x,j)\big)\oplus m[j]\\
& \text{输出}\;c:=(x,u)\in\mathcal{X}\times\mathcal{Y}^{|m|}
\end{aligned}
\right\}
\]
算法 $D(k_\mathrm{e},c)$ 的定义也与此类似。令 $\mathcal{I}=(S,V)$ 是一个定义在 $(\mathcal{K}_\mathrm{m},\,\mathcal{M},\,\mathcal{T})$ 上的一次性 MAC,其中 $\mathcal{M}:=\mathcal{Y}^{\leq\ell_\mathrm{m}}$,$\mathcal{T}:=\mathcal{Y}^{\ell_\mathrm{t}}$,并且 $\ell_\mathrm{m}+\ell_\mathrm{t}<\ell$。

基于 $F$ 和 $\mathcal{I}$ 的先 MAC 后加密密码 $\mathcal{E}_\mathrm{MtE}=(E_\mathrm{MtE},D_\mathrm{MtE})$ 接受 $\mathcal{M}$ 中的消息,定义如下:
\begin{equation}\label{eq:9-11}
\begin{aligned}
& E_\mathrm{MtE}\big((k_\mathrm{e},k_\mathrm{m}),\,m\big)
:=
\big\{\;
t\overset{\rm R}\leftarrow S(k_\mathrm{m},m),\quad
c\overset{\rm R}\leftarrow E\big(k_\mathrm{e},\,(m\,\Vert\,t)\big),\quad
\text{输出}\;c
\;\big\}\\
& D_\mathrm{MtE}\big((k_\mathrm{e},k_\mathrm{m}),\,c\big)
:=
\left\{
\begin{aligned}
& (m\,\Vert\,t)\leftarrow D(k_\mathrm{e},c)\\
& \text{如果}\;V(k_\mathrm{m},m,t)=\mathsf{reject}\text{,则输出}\;\mathsf{reject}\\
& \text{否则输出}\;m
\end{aligned}
\right\}
\end{aligned}
\end{equation}
正如我们在 \ref{subsec:9-4-1} 小节末尾和练习 \ref{exer:9-8} 中讨论的,两个密钥 $k_\mathrm{e}$ 和 $k_\mathrm{m}$ 的选择必须是相互独立的。如果设置 $k_\mathrm{e}=k_\mathrm{m}$,下面的安全性定理就会完全失效。

\begin{theorem}\label{theo:9-3}
假设 $\mathcal{I}$ 是一个安全的一次性 MAC,$F$ 是一个安全的 PRF,其中 $1/|\mathcal{X}|$ 可忽略不计,则式 \ref{eq:9-11} 中的由 PRF $F$ 和 MAC $\mathcal{I}$ 构建的密码 $\mathcal{E}_\mathrm{MtE}=(E_\mathrm{MtE},D_\mathrm{MtE})$ 能够提供认证加密。
\begin{quote}
特别地,对于每个在攻击游戏 \ref{game:9-1} 中攻击 $\mathcal{E}_\mathrm{MtE}$ 的 $Q$ 次查询密文完整性对手 $\mathcal{A}_\mathrm{ci}$,都存在两个如攻击游戏 \ref{game:6-1} 中那样攻击 $\mathcal{I}$ 的 MAC 对手 $\mathcal{B}_\mathrm{mac}$ 和 $\mathcal{B}_\mathrm{mac}'$,以及一个如攻击游戏 \ref{game:4-2} 中那样攻击 $F$ 的 PRF 对手 $\mathcal{B}_\mathrm{prf}$,它们都是围绕 $\mathcal{A}_\mathrm{ci}$ 的基本包装器,满足:
\end{quote}
\begin{equation}\label{eq:9-12}
\begin{aligned}
\mathrm{CI}\mathsf{adv}[\mathcal{A}_\mathrm{ci},\mathcal{E}_\mathrm{MtE}]
\leq
& \mathrm{PRF}\mathsf{adv}[\mathcal{B}_\mathrm{prf},F]+\\
& \;\; Q\cdot
\mathrm{MAC}_1\mathsf{adv}[\mathcal{B}_\mathrm{mac},\mathcal{I}]+
\mathrm{MAC}_1\mathsf{adv}[\mathcal{B}_\mathrm{mac}',\mathcal{I}]
+\frac{Q^2}{2|\mathcal{X}|}
\end{aligned}
\end{equation}
\begin{quote}
对于每个如攻击游戏 \ref{game:5-2} 中那样攻击 $\mathcal{E}_\mathrm{MtE}$ 的 CPA 对手 $\mathcal{A}_\mathrm{cpa}$,都存在一个如攻击游戏 \ref{game:5-2} 中那样攻击 $\mathcal{E}$ 的 CPA 对手 $\mathcal{B}_\mathrm{cpa}$,其中 $\mathcal{B}_\mathrm{cpa}$ 是一个围绕 $\mathcal{A}_\mathrm{cpa}$ 的基本包装器,满足:
\end{quote}
\[
\mathrm{CPA}\mathsf{adv}[\mathcal{A}_\mathrm{cpa},\mathcal{E}_\mathrm{MtE}]
=
\mathrm{CPA}\mathsf{adv}[\mathcal{B}_\mathrm{cpa},\mathcal{E}]
\]
\end{theorem}

\begin{proof}[证明思路]
该系统的 CPA 安全性可以由随机化计数器模式的 CPA 安全性立即得出。挑战在于,如何证明 $\mathcal{E}_\mathrm{MtE}$ 的密文完整性。令 $\mathcal{A}_\mathrm{ci}$ 是一个密文完整性对手。该对手发起一系列的查询 $m_1,\dots,m_Q$。对于每个 $m_i$,CI 挑战者都向 $\mathcal{A}_\mathrm{ci}$ 发送一条密文 $c_i=(x_i,u_i)$,其中 $x_i$ 是一个随机的 IV,$u_i$ 是使用 PRF $F$ 由 $x_i$ 派生的伪随机填充 $r_i$ 对数对 $m_i\,\Vert\,t_i$ 进行一次性填充加密的结果。这里,$t_i$ 是在 $m_i$ 上计算的 MAC 标签。在攻击游戏结束时,对手 $\mathcal{A}_\mathrm{ci}$ 输出一条密文 $c=(x,u)$,该密文不在 $c_i$ 之列,如果 $c$ 是一条有效密文,我们就称 $\mathcal{A}_\mathrm{ci}$ 获胜。这意味着,$u$ 使用从 $x$ 派生来的伪随机填充 $r$ 解密到 $m\,\Vert\,t$,而 $t$ 是一个 $m$ 上的有效标签。

现在,利用 PRF 安全性和 $x_i$ 不太可能重复的事实,我们可以有效地用真随机的填充取代伪随机的 $r_i$(和 $r$),同时不影响 $\mathcal{A}_\mathrm{ci}$ 的优势。这就是式 \ref{eq:9-12} 中 $\mathrm{PRF}\mathsf{adv}[\mathcal{B}_\mathrm{prf},F]$ 和 $Q^2/2|\mathcal{X}|$ 这两项的来源。请注意,在做了这一修改后,$t_i$ 就被完美地隐藏了起来,不会被对手发现。

下面,我们考虑 $\mathcal{A}_\mathrm{ci}$ 在这个修改后的攻击游戏中获胜的两种不同的方式:
\begin{itemize}
	\item 在第一种方式中,$\mathcal{A}_\mathrm{ci}$ 输出的 $x$ 值不在 $x_i$ 中。但是,在这种情况下,$\mathcal{A}_\mathrm{ci}$ 获胜的唯一方法就是希望随机消息上的随机标签是有效的。这就是式 \ref{eq:9-12} 中 $\mathrm{MAC}_1\mathsf{adv}[\mathcal{B}_\mathrm{mac}',\mathcal{I}]$ 这一项的由来。
	\item 在第二种情况下,对于某个 $j=1,\dots,Q$,$x$ 的值就等于 $x_j$。在这种情况下,为了获胜,$u$ 必须能在填充 $r_j$ 下被解密为 $m\,\Vert\,t$,其中 $t$ 是一个 $m$ 上的有效标签。此外,由于 $c\neq c_j$,我们有 $(m,t)\neq(m_j,t_j)$。为了把 $\mathcal{A}_\mathrm{ci}$ 变成一个一次性 MAC 对手,我们必须事先猜测索引 $j$:对于所有与猜测索引不同的索引 $i$,我们都可以用一个假标签来代替标签 $t_i$。这种猜测策略就是式 \ref{eq:9-12} 中 $Q\cdot\mathrm{MAC}_1\mathsf{adv}[\mathcal{B}_\mathrm{mac},\mathcal{I}]$ 这一项的来源。\qedhere
\end{itemize}
\end{proof}

\begin{proof}
为了证明密文完整性,我们让 $\mathcal{A}_\mathrm{ci}$ 与几个挑战者交互,这些挑战者之间密切相关。对于 $j=0,1,2,3,4$,我们定义 $W_j$ 为对手在游戏 $j$ 中获胜的事件。

\vspace*{5pt}
\noindent\textbf{游戏 $\mathbf{0}$。}和之前一样,我们首先让 $\mathcal{A}_\mathrm{ci}$ 与攻击游戏 \ref{game:9-1} 中的标准密文完整性挑战者交互,因为后者适用于 $\mathcal{E}_\mathrm{MtE}$,因此,我们有 $\Pr[W_0]= \mathrm{CI}\mathsf{adv}[\mathcal{A}_\mathrm{ci},\mathcal{E}_\mathrm{MtE}]$。

\vspace*{5pt}
\noindent\textbf{游戏 $\mathbf{1}$。}现在,我们用真正独立的一次性填充来替换计数器模式密码中的伪随机填充。由于 $F$ 是一个安全的 PRF,并且 $1/|\mathcal{X}|$ 可忽略不计,对手不会注意到这种差别。由此产生的 $\mathcal{E}_\mathrm{MtE}$ 的 CI 挑战者的工作方式如下:

\vspace*{10pt}

\hspace*{5pt} 随机选取 $k_\mathrm{m}\overset{\rm R}\leftarrow\mathcal{K}_\mathrm{m}$
\hspace*{141.5pt} // \quad \emph{选择一个随机 MAC 密钥}\\
\hspace*{26pt} 随机选取 $\omega\overset{\rm R}\leftarrow\{1,\dots,Q\}$
\hspace*{114.5pt} // \quad \emph{这个 $\omega$ 将在游戏 3 中使用}

\vspace*{5pt}

\hspace*{5pt} 当收到第 $i$ 个查询 $m_i\in\mathcal{Y}^{\leq\ell_\mathrm{m}}$ 时,对于 $i=1,2,\dots$:\\
\hspace*{7pt} ($1$)
\hspace*{22.5pt} 令 $t_i\leftarrow S(k_\mathrm{m},m_i)\in\mathcal{T}$
\hspace*{103pt} // \quad \emph{计算 $m_i$ 的标签}\\
\hspace*{7pt} ($2$) 
\hspace*{22.5pt} 随机选取 $x_i\overset{\rm R}\leftarrow\mathcal{X}$
\hspace*{126pt} // \quad \emph{选择一个随机的 IV}\\
\hspace*{47pt} \colorbox{gray!50}{随机选取 $r_i\overset{\rm R}\leftarrow\mathcal{Y}^{|m_i|+\ell_\mathrm{t}}$}
\hspace*{97pt} // \quad \emph{选择一个足够长的真随机一次性填充}\\
\hspace*{50pt} 令 $u_i\leftarrow (m_i\,\Vert\,t_i)\oplus r_i$,$c_i\leftarrow(x_i,u_i)$
\hspace*{42.5pt} // \quad \emph{构建密文}\\
\hspace*{50pt} 将 $c_i$ 发送给对手

\vspace*{5pt}

\hspace*{5pt} 当收到 $c=(x,u)\notin\{c_1,c_2,\dots\}$ 时:\\
\hspace*{50pt} // \quad \emph{解密密文 $c$}\\
\hspace*{7pt} ($3$)
\hspace*{22.5pt} 如果存在某个 $j$ 使得 $x=x_i$:\\
\hspace*{75pt} 则令 $(m\,\Vert\,t)\leftarrow u\oplus r_j$\\
\hspace*{7pt} ($4$)
\hspace*{47.2pt} 否则,随机选取 $r\overset{\rm R}\leftarrow\mathcal{Y}^{|u|}$,并令 $(m\,\Vert\,t)\leftarrow u\oplus r$

\vspace*{5pt}

\hspace*{29pt} // \quad \emph{检查产生的消息-标签对}\\
\hspace*{50pt} 如果 $V(k_\mathrm{m},m,t)=\mathsf{accept}$:\\
\hspace*{75pt} 则输出``获胜"\\
\hspace*{75pt} 否则输出``失败"

\vspace*{10pt}

请注意,在第 ($3$) 行中,可能会出现多个满足 $x=x_j$ 的 $j$,此时,我们可以取较小的那一个。

一个标准论证表明,存在一个有效的 PRF 对手 $\mathcal{B}_\mathrm{prf}$ 满足:
\begin{equation}\label{eq:9-13}
\big\lvert
\Pr[W_1]-\Pr[W_0]
\big\rvert
\leq
\mathrm{PRF}\mathsf{adv}[\mathcal{B}_\mathrm{prf},F]+\frac{Q^2}{2|\mathcal{X}|}
\end{equation}
请注意,如果想要再小心一点,我们也可以把这个论证分成两步。在第一步中,我们打出我们的``PRF牌",将 $F(k_\mathrm{e},\cdot)$ 替换成一个真随机函数 $f$,这就引入了式 \ref{eq:9-13} 中的 $\mathrm{PRF}\mathsf{adv}[\mathcal{B}_\mathrm{prf},F]$ 这一项。在第二步中,我们使用``健忘的侏儒"技术,让 $f$ 的所有输出都是相互独立的。对``所有 $x_i$ 都是各不相同的"这一事件使用差分引理,就会在式 \ref{eq:9-13} 中引入 ${Q^2}/{2|\mathcal{X}|}$ 这一项。

\vspace*{5pt}
\noindent\textbf{游戏 $\mathbf{2}$。}现在,我们限制对手的获胜条件,要求最终的密文 $c$ 中所使用的 IV 就是游戏期间发送给 $\mathcal{A}_\mathrm{ci}$ 的诸多 IV 中的一个。特别地,我们将第 ($4$) 行改为:

\vspace*{10pt}

\noindent
\hspace*{7pt} ($4$)
\hspace*{47.2pt} 否则,输出``失败"(并停机)

\vspace*{10pt}

\noindent
令 $Z_2$ 表示如下事件:在游戏 $2$ 中,尽管使用了先前未使用过的 $x\in\mathcal{X}$,但来自 $\mathcal{A}_\mathrm{ci}$ 的最终密文 $c=(x,u)$ 仍然是有效的。我们知道,除非事件 $Z_2$ 发生,否则这两个游戏的进程是完全相同的。当事件 $Z_2$ 在游戏 $2$ 中发生时,产生的数对 $(m,t)$ 随机均匀地分布在 $\mathcal{Y}^{|u|-\ell_\mathrm{t}}\times\mathcal{Y}^{\ell_\mathrm{t}}$ 上。这样的数对不太可能是一个有效的消息-标签对。不仅如此,游戏 $2$ 中的挑战者对第 ($1$) 行中生成的所有标签 $t_i$ 都使用一次性填充进行了有效的加密,所以即使用假标签替换这些标签,也不会影响 $Z_2$ 发生的概率。基于这些观察,我们很容易构建一个有效的 MAC 对手 $\mathcal{B}_\mathrm{mac}'$,使得 $\Pr[Z_2]\leq\mathrm{MAC}_1\mathsf{adv}[\mathcal{B}_\mathrm{mac}',\mathcal{I}]$。对手 $\mathcal{B}_\mathrm{mac}'$ 运行如下:它像在游戏 $2$ 中一样扮演 $\mathcal{A}_\mathrm{ci}$ 的挑战者,除了在上述第 ($1$) 行中,它会令 $t_i\leftarrow 0^{\ell_\mathrm{t}}$。当 $\mathcal{A}_\mathrm{ci}$ 输出 $c=(x,u)$ 时,对手 $\mathcal{B}_\mathrm{mac}'$ 输出一个 $\mathcal{Y}^{|u|-\ell_\mathrm{t}}\times\mathcal{Y}^{\ell_\mathrm{t}}$ 上的随机数对。因此,根据差分引理,我们有:
\begin{equation}\label{eq:9-14}
\big\lvert
\Pr[W_2]-\Pr[W_1]
\big\rvert
\leq
\mathrm{MAC}_1\mathsf{adv}[\mathcal{B}_\mathrm{mac}',\mathcal{I}]
\end{equation}

\noindent\textbf{游戏 $\mathbf{3}$。}我们进一步限制对手的获胜条件,要求密文伪造使用的是发送给 $\mathcal{A}_\mathrm{ci}$ 的第 $\omega$ 条密文中的 IV。这里的 $\omega$ 是 $\{1,\dots,Q\}$ 中的一个随机数,由挑战者选出。和游戏 $2$ 相比,获胜条件的唯一变化是将第 ($3$) 修改为:

\vspace*{10pt}

\noindent
\hspace*{7pt} ($3$)
\hspace*{22.5pt} 如果 $x=x_\omega$:\\

\vspace*{10pt}

\noindent
由于 $\omega$ 与 $\mathcal{A}_\mathrm{ci}$ 的观察完全无关,我们知道:
\begin{equation}\label{eq:9-15}
\Pr[W_3]\geq
(1/Q)\cdot\Pr[W_2]
\end{equation}


\noindent\textbf{游戏 $\mathbf{4}$。}最后,我们修改挑战者,使其只对 $\mathcal{A}_\mathrm{ci}$ 发出的第 $\omega$ 次查询计算有效标签。对于所有其他的查询,挑战者只会编造一个任意的(无效)标签。由于标签是用一次性填充加密的,对手无法分辨他得到的是不是无效标签的加密结果。特别地,和游戏 $3$ 相比,唯一的变化是用下面这两行替换掉第 ($1$) 行:

\vspace*{10pt}

\noindent
\hspace*{7pt} ($1$)
\hspace*{22.5pt} 令 $t_i\leftarrow(0^n)^{\ell_\mathrm{t}}\in\mathcal{T}$\\
\hspace*{50pt} 如果 $i=\omega$,则令 $t_i\leftarrow S(k_\mathrm{m},m_i)\in\mathcal{T}$
\qquad // \quad \emph{只会为 $m_\omega$ 计算正确的标签}

\vspace*{10pt}

\noindent
从对手的视角来看,这个游戏与游戏 $3$ 是完全相同的,因此:
\begin{equation}\label{eq:9-16}
\Pr[W_4]=\Pr[W_3]
\end{equation}

\noindent\textbf{最后的归约。}我们声称,存在一个有效的一次性 MAC 伪造者 $\mathcal{B}_\mathrm{mac}$,使得:
\begin{equation}\label{eq:9-17}
\Pr[W_4]=
\mathrm{MAC}_1\mathsf{adv}[\mathcal{B}_\mathrm{mac},\mathcal{I}]
\end{equation}
成立。对手 $\mathcal{B}_\mathrm{mac}$ 与 MAC 挑战者 $\mathbf{C}$ 交互,其工作方式如下:

\vspace*{10pt}

\hspace*{5pt} 随机选取 $\omega\overset{\rm R}\leftarrow\{1,\dots,Q\}$\\
\hspace*{26pt} 当收到第 $i$ 个查询 $m_i\in\{0,1\}^{\ell_\mathrm{m}}$ 时,对于 $i=1,2,\dots$:\\
\hspace*{50pt} 令 $t_i\leftarrow(0^n)^{\ell_\mathrm{t}}\in\mathcal{T}$\\
\hspace*{50pt} 如果 $i=\omega$,则向 $\mathbf{C}$ 查询 $m_i$ 上的标签,记应答为 $t_i\in\mathcal{T}$\\
\hspace*{50pt} 随机选取 $x\overset{\rm R}\leftarrow\mathcal{X}$
\hspace*{114.2pt} // \quad \emph{选择一个随机的 IV}\\
\hspace*{50pt} 随机选取 $r_i\overset{\rm R}\leftarrow\mathcal{Y}^{|m_i|+\ell_\mathrm{t}}$
\hspace*{85pt} // \quad \emph{选择一个足够长的随机一次性填充}\\
\hspace*{50pt} 令 $u_i\leftarrow(m_i\,\Vert\,t_i)\oplus r_i$,$c_i\leftarrow(x_i,u_i)$\\
\hspace*{50pt} 将 $c_i$ 发送给对手

\vspace*{5pt}

\hspace*{5pt} 当 $\mathcal{A}_\mathrm{ci}$ 输出 $c=(x,u)$ 时:\\
\hspace*{50pt} 如果 $x=x_\omega$:\\
\hspace*{75pt} 令 $(m\,\Vert\,t)\leftarrow u\oplus r_\omega$\\
\hspace*{75pt} 输出 $(m,t)$ 作为消息-标签伪造

\vspace*{10pt}

\noindent
由于 $c\neq c_\omega$,我们知道 $(m,t)\neq(m_\omega,t_\omega)$。因此,只要 $\mathcal{A}_\mathrm{ci}$ 赢得游戏 $4$,我们就知道 $\mathcal{B}_\mathrm{mac}$ 没有中止,而且会输出一个能够让它赢得一次性 MAC 攻击游戏的数对 $(m,t)$。由此可见 $\Pr[W_4]=\mathrm{MAC}_1\mathsf{adv}[\mathcal{B}_\mathrm{mac},\mathcal{I}]$,正如定理所要求的。
\end{proof}

\subsection{是先加密后 MAC,还是先 MAC 后加密?}\label{subsec:9-4-5}

到目前为止,我们证明了以下几个关于 MtE 和 EtM 范式的事实:
\begin{itemize}
	\item 只要密码是 CPA 安全的,MAC 是安全的,EtM 就能提供认证加密。密文上的 MAC 可以防止对密文的任何篡改。
	\item MtE 通常不是安全的——存在一些基于 CPA 安全密码的非 AE 安全 MtE。此外,由于潜在的计时边信道可能会导致严重的选择密文攻击,正确实现 MtE 也是很困难的。尽管如此,对于特定的密码,比如随机化计数器模式和随机化 CBC 模式的密码,即使 MAC 只是一次性安全的,MtE 也是 AE 安全的。
	\item 练习 \ref{exer:9-10} 中讨论了第三种范式,称为加密并 MAC (EaM)。该练习表明,如果使用的是随机化计数器模式的密码,只要 MAC 是安全的 PRF,EaM 就是安全的。EaM 在各方面都比 EtM 差,因此不应该被使用。
\end{itemize}

这些事实,以及对 MtE 的攻击实例都表明,EtM 是更好的应用范式。当然,至关重要的是,底层密码首先必须是 CPA 安全的,底层 MAC 也必须是安全的 MAC。否则,EtM 可能根本无法提供安全性。

鉴于历史上实现这些范式时曾出现过的种种错误,我们建议开发者不要自己去实现 EtM。开发者最好使用一个加密标准,比如说 GCM(见 \ref{sec:9-7} 节),它使用 EtM 来提供开箱即用的认证加密。
