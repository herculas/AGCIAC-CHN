\section{案例研究:IPsec}\label{sec:9-11}

IPsec 协议能够为互联网 IP 数据包提供机密性和完整性保证。该协议最初于 1998 年发布,随后在 2005 年进行了更新。IPsec 协议由许多子协议组成,但它们与我们这里的讨论无关。在本节中,我们将重点讨论最常用的 IPsec 协议,即隧道模式下的\textbf{封装安全载荷 (encapsulated security payload, ESP)}。

虚拟专用网 (virtual private network, VPN) 是 IPsec 的一个重要应用。VPN 让两个分支机构能够借由公网信道进行安全通信,如图 \ref{fig:9-5} 所示。在图示的例子中,来自机器 $1$,$2$,$3$ 的数据包会在西侧的网关被 IPsec 加密,然后在公共信道上传输。东侧网关会解密每个接收到的数据包,并将其转发到位于东侧分支内部的目的地址,即 $4$,$5$,$6$ 中的某一个。我们可以注意到,所有从西侧发送到东侧的数据包都使用相同的密钥 $k_{w\to e}$ 加密。从东侧发往西侧的数据包也与此类似,只是会使用另一个密钥 $k_{e\to w}$。我们使用这个 VPN 的案例作为一个理解 IPsec 的例子。

\begin{figure}
  \centering
  \tikzset{every picture/.style={line width=0.75pt}}

\begin{tikzpicture}[x=0.75pt,y=0.75pt,yscale=-1,xscale=1]

\draw   (40,8.5) -- (130,8.5) -- (130,58.5) -- (40,58.5) -- cycle ;
\draw   (0,118.5) -- (35,118.5) -- (35,153.5) -- (0,153.5) -- cycle ;
\draw   (105,118.5) -- (140,118.5) -- (140,153.5) -- (105,153.5) -- cycle ;
\draw   (52.5,118.5) -- (87.5,118.5) -- (87.5,153.5) -- (52.5,153.5) -- cycle ;
\draw   (260,118.5) -- (295,118.5) -- (295,153.5) -- (260,153.5) -- cycle ;
\draw   (365,118.5) -- (400,118.5) -- (400,153.5) -- (365,153.5) -- cycle ;
\draw   (312.5,118.5) -- (347.5,118.5) -- (347.5,153.5) -- (312.5,153.5) -- cycle ;

\draw  [fill={rgb, 255:red, 155; green, 155; blue, 155 }  ,fill opacity=0.5 ] (85,8.5) -- (130,8.5) -- (130,28.5) -- (85,28.5) -- cycle ;
\draw   (270,8.5) -- (360,8.5) -- (360,58.5) -- (270,58.5) -- cycle ;
\draw  [fill={rgb, 255:red, 155; green, 155; blue, 155 }  ,fill opacity=0.5 ] (270,8.5) -- (315,8.5) -- (315,28.5) -- (270,28.5) -- cycle ;

\draw    (85,58.5) -- (85,88.5) ;
\draw    (315,58.5) -- (315,88.5) ;
\draw    (17.5,88.5) -- (17.5,118.5) ;
\draw    (70,88.5) -- (70,118.5) ;
\draw    (122.5,88.5) -- (122.5,118.5) ;
\draw    (277.5,88.5) -- (277.5,118.5) ;
\draw    (330,88.5) -- (330,118.5) ;
\draw    (382.5,88.5) -- (382.5,118.5) ;
\draw [line width=2]    (0,88.5) -- (140,88.5) ;
\draw [line width=2]    (260,88.5) -- (400,88.5) ;
\draw [line width=4]    (130,18.5) -- (270,18.5) ;


\draw (107.5,18.5) node   [align=left] {IPsec};
\draw (292.5,18.5) node   [align=left] {IPsec};
\draw (85,43.5) node   [align=left] {网关};
\draw (315,43.5) node   [align=left] {网关};
\draw (17.5,136) node    {$1$};
\draw (70,136) node    {$2$};
\draw (122.5,136) node    {$3$};
\draw (277.5,136) node    {$4$};
\draw (330,136) node    {$5$};
\draw (382.5,136) node    {$6$};
\draw (70,178.5) node   [align=left] {西侧分支};
\draw (330,178.5) node   [align=left] {东侧分支};
\draw (200,15) node [anchor=south] [inner sep=0.75pt]   [align=left] {互联网};


\end{tikzpicture}
  \caption{东部和西部办公室分支机构之间的虚拟专用网}
  \label{fig:9-5}
\end{figure}

想要理解 IPsec,读者可能首先需要对 IP 协议有一个最基本的了解。这里,我们主要关注目前部署最广泛的 IP 版本 4 (IPv4)。图 \ref{fig:9-6} 的左侧展示了一个 IPv4(明文)数据包。这个数据包包含一个包头和一个载荷。包头中包含了一大堆字段,但只有几个与我们的讨论有关:
\begin{itemize}
	\item 前 $4$ 比特表示\textbf{版本号},对于 IPv4,版本号被置为 $4$。
	\item $2$ 字节的\textbf{数据包长度}字段揭示了整个数据包(包括包头)的字节长度。
	\item $1$ 字节的\textbf{协议}字段描述了数据包的载荷,例如,协议 $=6$ 表示一个 TCP 载荷。
	\item $2$ 字节的\textbf{报头校验和}字段包含对所有报头字节的校验和(不包括校验和字段自身)。该校验和用于检测包头中的随机传输错误。校验和无效的数据包会被接收方丢弃。任何人都可以计算校验和,因此无法针对攻击者提供完整性。事实上,当数据包在路由器之间传输时,路由器经常会改变包头中字段的值,并重新计算校验和。
	\item \textbf{源 IP} 和\textbf{目的 IP} 字段表示数据包的源地址和目的地址。
	\item \textbf{载荷}字段包含数据包内容,长度可变。
\end{itemize}

\begin{snote}[IPsec 封装安全载荷 (ESP)。]
图 \ref{fig:9-6} 的右侧展示了在隧道模式下用 ESP 加密一个数据包的结果。我们先描述加密数据包中的各个字段,再描述加密过程。
\end{snote}

\begin{figure}
  \centering
  \tikzset{every picture/.style={line width=0.75pt}}

\begin{tikzpicture}[x=0.75pt,y=0.75pt,yscale=-1,xscale=1]


\draw   (0,135) -- (240,135) -- (240,165) -- (0,165) -- cycle ;
\draw  [line width=2.25]  (0,135) -- (240,135) -- (240,285) -- (0,285) -- cycle ;
\draw   (0,165) -- (240,165) -- (240,195) -- (0,195) -- cycle ;
\draw  [line width=2.25]  (0,285) -- (240,285) -- (240,375) -- (0,375) -- cycle ;
\draw   (0,255) -- (240,255) -- (240,285) -- (0,285) -- cycle ;
\draw   (0,225) -- (240,225) -- (240,255) -- (0,255) -- cycle ;
\draw   (0,195) -- (240,195) -- (240,225) -- (0,225) -- cycle ;
\draw   (0,135) -- (30,135) -- (30,165) -- (0,165) -- cycle ;
\draw   (30,135) -- (60,135) -- (60,165) -- (30,165) -- cycle ;
\draw   (60,135) -- (120,135) -- (120,165) -- (60,165) -- cycle ;
\draw   (0,165) -- (120,165) -- (120,195) -- (0,195) -- cycle ;
\draw   (0,195) -- (60,195) -- (60,225) -- (0,225) -- cycle ;
\draw   (60,195) -- (120,195) -- (120,225) -- (60,225) -- cycle ;
\draw   (120,165) -- (240,165) -- (240,195) -- (120,195) -- cycle ;
\draw   (120,195) -- (240,195) -- (240,225) -- (120,225) -- cycle ;
\draw   (120,135) -- (240,135) -- (240,165) -- (120,165) -- cycle ;
\draw   (140,165) -- (240,165) -- (240,195) -- (140,195) -- cycle ;
\draw  [line width=1.5]  (360,0) -- (600,0) -- (600,390) -- (360,390) -- cycle ;
\draw   (360,0) -- (390,0) -- (390,30) -- (360,30) -- cycle ;
\draw   (390,0) -- (420,0) -- (420,30) -- (390,30) -- cycle ;
\draw   (420,0) -- (480,0) -- (480,30) -- (420,30) -- cycle ;
\draw   (480,0) -- (600,0) -- (600,30) -- (480,30) -- cycle ;
\draw   (360,30) -- (480,30) -- (480,60) -- (360,60) -- cycle ;
\draw   (480,30) -- (600,30) -- (600,60) -- (480,60) -- cycle ;
\draw   (360,60) -- (420,60) -- (420,90) -- (360,90) -- cycle ;
\draw   (420,60) -- (480,60) -- (480,90) -- (420,90) -- cycle ;
\draw   (360,90) -- (600,90) -- (600,120) -- (360,120) -- cycle ;
\draw   (360,120) -- (600,120) -- (600,150) -- (360,150) -- cycle ;
\draw   (360,150) -- (600,150) -- (600,180) -- (360,180) -- cycle ;
\draw   (360,180) -- (600,180) -- (600,210) -- (360,210) -- cycle ;
\draw  [fill={rgb, 255:red, 155; green, 155; blue, 155 }  ,fill opacity=0.5 ] (360,210) -- (600,210) -- (600,300) -- (360,300) -- cycle ;
\draw  [fill={rgb, 255:red, 155; green, 155; blue, 155 }  ,fill opacity=0.5 ] (360,300) -- (600,300) -- (600,345) -- (360,345) -- cycle ;
\draw   (360,345) -- (600,345) -- (600,390) -- (360,390) -- cycle ;
\draw  [line width=4.5]  (360,150) -- (600,150) -- (600,345) -- (360,345) -- cycle ;

\draw   (540,322.5) -- (600,322.5) -- (600,345) -- (540,345) -- cycle ;
\draw   (480,322.5) -- (540,322.5) -- (540,345) -- (480,345) -- cycle ;

\draw  [<->]  (1,400) -- (239,400) ;
\draw [shift={(240,400)}, rotate = 180] (0,5) -- (0,-5);
\draw [shift={(0,400)}, rotate = 0] (0,5) -- (0,-5) ;
\draw  [<->]  (361,415) -- (599,415) ;
\draw [shift={(600,415)}, rotate = 180] (0,5) -- (0,-5);
\draw [shift={(360,415)}, rotate = 0](0,5) -- (0,-5);

\draw  [draw opacity=0][fill={rgb, 255:red, 255; green, 255; blue, 255 }  ,fill opacity=1 ] (90,390) -- (150,390) -- (150,410) -- (90,410) -- cycle ;
\draw  [draw opacity=0][fill={rgb, 255:red, 255; green, 255; blue, 255 }  ,fill opacity=1 ] (450,405) -- (510,405) -- (510,425) -- (450,425) -- cycle ;

\draw  [dash pattern={on 4.5pt off 4.5pt}]  (240,135) -- (360,210) ;
\draw  [dash pattern={on 4.5pt off 4.5pt}]  (240,375) -- (360,300) ;


\draw (15,150) node   [align=left] [font=\small] {版本};
\draw (180,150) node   [align=left] [font=\small]{数据包长度};
\draw (90,210) node   [align=left] [font=\small]{协议};
\draw (180,210) node   [align=left][font=\small] {包头校验和};
\draw (120,240) node   [align=left][font=\small] {源 IP 地址};
\draw (120,270) node   [align=left][font=\small] {目的 IP 地址};
\draw (120,330) node   [align=left][font=\small] {载荷};
\draw (375,15) node   [align=left][font=\small] {版本};
\draw (540,15) node   [align=left][font=\small] {数据包长度};
\draw (450,75) node   [align=left][font=\small] {prot=\texttt{ESP}};
\draw (480,105) node   [align=left][font=\small] {源 IP 地址};
\draw (480,135) node   [align=left][font=\small] {目的 IP 地址};
\draw (480,367.5) node   [align=left][font=\small] {完整性标签};
\draw (480,255) node   [align=left][font=\small] {数据包};
\draw (480,195) node   [align=left][font=\small] {序列号};
\draw (480,165) node   [align=left][font=\small] {安全参数索引 (SPI)};
\draw (510,333) node   [align=left][font=\small] {填充长度};
\draw (570,333) node   [align=left][font=\small] {下一包头};
\draw (420,320) node   [align=left][font=\small] {填充};
\draw (120,400) node   [align=left][font=\small] {$32$ 比特};
\draw (480,415) node   [align=left][font=\small] {$32$ 比特};
\draw (52,40) node [anchor=west] [inner sep=0.75pt]   [align=left] {灰色区域是加密的};
\draw (52,60) node [anchor=west] [inner sep=0.75pt]   [align=left] {框内区域被完整性标签认证};

\end{tikzpicture}
  \caption{明文 IPv4 数据包和 IPsec ESP 数据包}
  \label{fig:9-6}
\end{figure}

\begin{snote}[IPsec 密钥管理——SPI 字段。]
每个 ESP 端点都会维护一个\textbf{安全关联数据库 (security association database, SAD)}。SAD 中的记录被称为\textbf{安全关联 (security association, SA)},由一个被称为\textbf{安全参数索引 (security parameters index, SPI)} 的 $32$ 比特标识符唯一识别。SAD 记录(即 SA)中包含许多与连接相关的参数,比如 ESP 加密算法(例如 3DES-CBC 或 AES-CBC),ESP 密钥(比如 $k_{w\to e}$ 或者 $k_{e\to w}$),源 IP 地址和目的 IP 地址,SPI,以及各种密钥交换参数。

当东侧分支网关发送数据包时,它使用数据包的目的 IP 地址和其他参数在其安全关联数据库 (SAD) 中选择一条安全关联 (SA)。网关将所选的 SA 的 $32$ 比特 SPI 嵌入到数据包头中,并使用 SA 中指定的密钥加密数据包。当数据包到达目的地址时,接收方使用以下算法在其 SAD 中找到一个合适的 SA:
\begin{enumerate}
	\item 首先,寻找一个与收到的 (SPI,目的地址,源地址) 相匹配的 SA;
	\item 如果没有找到匹配,接收方根据 (SPI,目的地址) 寻找匹配;
	\item 否则,只根据 SPI 寻找匹配。
\end{enumerate}
如果收到的数据包不存在 SA,则丢弃该数据包。否则,网关使用所选中的 SA 中指定的密密钥解密数据包。在大多数情况下,SA 只用于在一个方向上传输数据包,比如说从东到西。东西侧之间的双向 TCP 连接需要使用两个 SA——一个用于从东到西的数据包,另一个用于从西到东的数据包。一般来说,一个 ESP 端点会为每个对等实体维护两条 SAD 记录。

对特定主机来说,SAD 是半手动管理的。其中一些参数由人工管理,而另一些则由通信主机之间协商得到。具体地说,SA 密钥可以在两个端点上手动设置,也可以使用称为 IKE 的 IPsec 密钥交换协议进行协商 \cite{kaufman2014internet}。在本节中,我们不会讨论 SAD 的管理。
\end{snote}

\begin{snote}[ESP防重放——序列号字段。]
序列号让接收方能够检测并丢弃重复的数据包。重复可能是由网络错误造成的,也可能是由攻击者故意重放旧数据包造成的。每个 ESP 端点都会为每条安全关联维护一个\textbf{序列号}。在默认情况下,序列号的长度为 $64$ 比特(称为扩展序列号),而旧版的 ESP 则使用较短的 $32$ 比特序列号。序列号在创建安全关联时被初始化为 $0$,然后,每当使用安全协议发送一个新数据包时,就会被递增 $1$。整个 $64$ 比特都会被包含在 MAC 的计算中。然而,只有 $32$ 个最小有效比特会被包含在 ESP 数据包头中。换言之,ESP 端点维护一个 $64$ 比特的计数器,其中 $32$ 个 MSB 在数据包头中是隐式的,而 $32$ 个 LSB 在包头中是显式的。

在我们讨论序列号时,我们假定,对于每条安全关联,最多都只有一台主机在发送数据包。因此,对于一条特定的 SA,不存在两个主机发送相同序列号的数据包的危险。请注意,多个主机可以为某一个特定的 SA 接收数据包,就像在组播的情况下一样。我们只是不允许多个主机使用一个 SA 发送数据包。

对于一条特定的 SA,接收方必须丢弃任何包含已经出现在之前的数据包中的 $32$ 比特序列号的那些数据包。由于数据包可能不按顺序到达,接收方在验证序列号的单一性时,需要付出一些努力。RFC 4303 建议接收方维护一个大小为 $32$ 的窗口(比如比特向量)。窗口的 ``右"侧代表在该 SA 上收到的最高的、经过验证的序列号值。如果一个新数据包中的序列号低于窗口的``左"侧边界,就将其丢弃。当收到的数据包落在窗口内,就将其与窗口内的所有数据包进行比对,如果其数据包已经出现在了窗口中,就将其丢弃。每当收到一个序列号高于当前窗口``右"侧的有效数据包,就右移窗口。这样,即使丢失了一长串数据包,接收方也能从中恢复状态。

如果连续丢失了超过 $2^{32}$ 个数据包,发送方和接收方的 $64$ 比特序列号就会失去同步——两者的隐式 $32$ MSB 将会不同。因此,所有此后的数据包都将因为 MAC 验证失败而被拒绝。这就解释了为什么 ESP 的设计者选择只在数据包头中包含 $32$ 比特——连续丢失掉 $2^{32}$ 个数据包是基本不可能的。但如果继续减少比特数(比如 $16$ 比特),就会大大增加通信失败的可能性。
\end{snote}

\begin{snote}[填充及下一包头字段。]
ESP 会附加一个填充字段,以确保待加密数据的长度是所选加密算法的分组长度的整数倍(例如,对于 AES-CBC 来说,是 $16$ 字节的整数倍)。它还能确保产生的密文长度是 $4$ 字节的整数倍。填充的长度可以是 $0$ 到 $255$ 字节之间的任何值。此外,ESP 还会添加一个额外的填充长度字节,用于表明之前的填充字节数。最后,ESP 会添加一个下一包头 (\texttt{next-hdr}) 字段,用于指示载荷类型。在大多数情况下,载荷类型都是 IPv4 数据包,在这种情况下,\texttt{next-hdr}=$4$。

ESP 还支持一种可选服务,称为\textbf{流量机密性 (traffic flow confidentiality, TFC)},其中,发送方可以尝试隐藏明文数据包的长度。为了做到这一点,发送方可以在填充之前向载荷中添加假的(未指定的)字节。TFC 填充的长度是任意的。明文 IP 包头中的数据包长度字段能够揭示 TFC 填充的起点。TFC 填充会在解密后被移除。

ESP还支持使用``假"数据包来预防流量分析。其目的是防止观察者分辨出发送方传输数据的时机。比如说,我们可以指示发送方每隔一毫秒发送一个数据包,无论它是否真的有数据要发送。当没有数据时,发送方就会发送一个``假"数据包,它可以用 \texttt{next-hdr}=$59$ 来识别。由于 \texttt{next-hdr} 字段是加密的,观察者就无法分辨数据包是真的还是假的。尽管如此,当到达目的地址后,所有的假数据包都会在解密后被立即丢弃。
\end{snote}

\begin{snote}[加密过程。]
ESP 通过四个步骤实现先加密后 MAC 方法。我们依次讨论每个步骤:
\begin{enumerate}
	\item \textbf{填充。}填充,以及可选的 TFC 填充和下一包头字段,会被附加到明文 IP 数据包中。
	\item \textbf{加密。}图 \ref{fig:9-6} 中的灰色区域会被 SA 中指定的算法和密钥加密。ESP 支持各种加密算法,但至少需要支持 3DES-CBC、AES-CBC 和 AES 计数器模式。对于 CBC 模式,IV 会被附加在加密载荷之前,并以明文方式发送。加密算法可以设置为 NULL,数据包在这种情况下不会被加密。这在只需要 ESP 提供完整性,而不需要机密性时使用。
	\item \textbf{MAC。}使用 SA 中指定的算法和密钥计算完整性标签。该完整性标签覆盖以下数据:
		\[
			\mathrm{SPI}\;\Vert\;\text{64 比特序列号}\;\Vert\;\text{密文}
		\]
	其中的密文是第 $2$ 步的结果。请注意,尽管数据包中只嵌入了序列号的 $32$ 个比特,但标签仍然是基于 $64$ 比特序列号计算的。由此产生的标签被放置在密文之后的完整性标签字段中。ESP 支持各种 MAC 算法,但至少需要支持 HMAC-SHA1-96、HMAC-MD5-96 和 AES-XCBC-MAC-96(XCBC-MAC 是 CMAC 的一个变体)。完整性标签字段是可选的,如果加密算法已经能够提供认证加密(比如 GCM),也可以省略它。
	\item \textbf{封装。}最后,将 IPv4 包头添加到前面,得到一个 ESP 数据包,如图 \ref{fig:9-6} 的右侧所示。IPv4 包头中的协议字段会被置为 $50$,表示 ESP 载荷。
\end{enumerate}

解密也遵循类似的过程。接收方首先会检查 $32$ 比特序列号。如果数值重复或超出允许的窗口范围,就丢弃该数据包。接下来,接收方检查标签字段,如果 MAC 验证失败,则拒绝该数据包。然后解密数据包并移除填充。如果该数据包是一个``假"数据包(即 \texttt{next-hdr}=$59$),也丢弃该数据包。最后,重建明文数据包,并将其发送到目的地址。请注意,原则上,序列号字段也是可以被加密的。ESP 的设计者选择以明文方式发送该字段,以缩短拒绝重复数据包的时间。
\end{snote}

\begin{snote}[安全性。]
IP数据包可以以任何顺序到达,被重放,甚至被篡改。依靠先加密后 MAC 范式和序列号机制,ESP 能够确保接收方看到的数据流与发送方传输的数据流是完全相同的。困扰 ESP 的一个问题是,如何在不进行完整性检查的情况下提供 CPA 安全加密。RFC 4303 指出:
\begin{quote}
ESP 允许只加密的 SA,因为这可以提供相当优异的性能,并且仍能提供充分的安全性,比如,当上层能够独立提供认证/完整性保护时。
\end{quote}
然而,依靠更高的应用层来提供完整性保证是相当危险的。在发送端,应用层会在将数据传递给 IP 层之前对其进行处理。因此,这事实上等于是实现了一个先 MAC 后加密范式,而从理论上看,我们已经知道,它可能是不安全的。更重要的是,在实践中,假定上层会保护整个 IP 数据包是很危险的。比如说,像 SSL 这样的上层完全可以只提供完整性而不进行任何加密。将仅有加密的 ESP 和仅有完整性保护的 SSL 结合起来将是完全不安全的,因为 SSL 层不会为加密的包头提供完整性保护。因此,攻击者就能够篡改加密数据包中的目的 IP 字段。这样,接收方的 IPsec 网关就会解密数据包,并将结果转发到非预期的目的地址,从而导致严重的隐私泄露。\cite{bellovin1996problem,paterson2006cryptography} 讨论了ESP 纯加密模式的这种隐患,以及其他的一些风险。

然而,我们注意到,当所使用的密码能够提供认证加密(比如 GCM 模式)时,使用没有完整性检查的加密也是完全可以的,因为密码已经提供了认证加密。
\end{snote}