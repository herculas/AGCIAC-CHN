\section{认证加密:定义}\label{sec:9-1}

首先,对于一个密码 $\mathcal{E}$,我们定义什么叫做 $\mathcal{E}$ 提供认证加密。事实上,它必须满足两个属性。第一,$\mathcal{E}$ 必须是 CPA 安全的。第二,$\mathcal{E}$ 必须提供密文完整性,其定义如下。密文完整性是一个新的属性,它抓住这样一个事实,即 $\mathcal{E}$ 应当具备与 MAC 相似的属性。令 $\mathcal{E}=(E,D)$ 是一个定义在 $(\mathcal{K},\mathcal{M},\mathcal{C})$ 上的密码。我们使用下面的攻击游戏来定义密文完整性,如图 \ref{fig:9-2} 所示。该游戏与 MAC 攻击游戏 \ref{game:6-1} 相似。

\begin{game}[密文完整性]\label{game:9-1}
对于一个定义在 $(\mathcal{K},\mathcal{M},\mathcal{C})$ 上的给定密码 $\mathcal{E}=(E,D)$ 和一个给定对手 $\mathcal{A}$,攻击游戏运行如下:
\begin{itemize}
	\item 挑战者随机选择一个 $k\overset{\rm R}\leftarrow\mathcal{K}$。
	\item $\mathcal{A}$ 向挑战者发起多次查询。对于 $i=1,2,\dots$,第 $i$ 次查询包含一条消息 $m_i\in\mathcal{M}$。挑战者计算 $c_i\overset{\rm R}\leftarrow E(k, m_i)$,并将 $c_i$ 发送给 $\mathcal{A}$。
	\item 最后,$\mathcal{A}$ 输出一个候选密文 $c\in\mathcal{C}$,该文本不在 $\mathcal{A}$ 所得到的密文集合中,即:
	\[
	c\notin\{c_1,c_2,\dots\}
	\]
\end{itemize}
如果 $c$ 是用 $k$ 加密得到的一条有效密文,即 $D(k,c)\neq\mathsf{reject}$,我们就称 $\mathcal{A}$ 赢得游戏。我们将 $\mathcal{A}$ 相对于 $\mathcal{E}$ 的优势定义为 $\mathrm{CI}\mathsf{adv}[\mathcal{A},\mathcal{E}]$,即 $\mathcal{A}$ 赢得游戏的概率。最后,如果 $\mathcal{A}$ 最多发起 $Q$ 次加密查询,我们就称 $\mathcal{A}$ 是一个 \textbf{$Q$ 次查询对手 ($Q$-query adversary)}。
\end{game}

\begin{definition}\label{def:9-1}
如果对于每个有效对手 $\mathcal{A}$,$\mathrm{CI}\mathsf{adv}[\mathcal{A},\mathcal{E}]$ 的值都可忽略不计,我们就称密码 $\mathcal{E}=(E,D)$ 提供\textbf{密文完整性 (ciphertext integrity, CI)}。
\end{definition}

CPA 安全性和密文完整性是认证加密所必须的属性。这在下面的定义中得到了体现。

\begin{figure}
  \centering
  \tikzset{every picture/.style={line width=0.75pt}}

\begin{tikzpicture}[x=0.75pt,y=0.75pt,yscale=-1,xscale=1]


\draw  [fill={rgb, 255:red, 255; green, 255; blue, 255 }  ,fill opacity=1 ][line width=1.2] [general shadow={fill=black,shadow xshift=2.25pt,shadow yshift=-2.25pt}] (30,30) -- (90,30) -- (90,180) -- (30,180) -- cycle ;
\draw  [fill={rgb, 255:red, 255; green, 255; blue, 255 }  ,fill opacity=1 ][line width=1.2] [general shadow={fill=black,shadow xshift=2.25pt,shadow yshift=-2.25pt}] (240,30) -- (300,30) -- (300,180) -- (240,180) -- cycle ;
\draw  [->]  (231.07,104.86) .. controls (229.39,108.09) and (227.28,110) .. (225,110) .. controls (219.48,110) and (215,98.81) .. (215,85) .. controls (215,71.19) and (219.48,60) .. (225,60) .. controls (227.65,60) and (230.06,62.58) .. (231.85,66.78) ;

\draw  [<-]  (94,70) -- (240,70) ;
\draw  [->]  (90,105) -- (238,105) ;
\draw  [<-]  (94,140) -- (240,140) ;

\draw (60,45) node    {$k\overset{\mathrm{R}}{\leftarrow }\mathcal{K}$};
\draw (60,15) node   [align=left] {挑战者};
\draw (270,15) node   [align=left] {对手 $\mathcal{A}$};
\draw (166,66.6) node [anchor=south] [inner sep=0.75pt]    {$m_{i}$};
\draw (165,101.6) node [anchor=south] [inner sep=0.75pt]    {$c_{i}\leftarrow E(k,m_{i})$};
\draw (166,136.6) node [anchor=south] [inner sep=0.75pt]    {$c$};

\end{tikzpicture}
  \caption{密文完整性攻击游戏(攻击游戏 \ref{game:9-1})}
  \label{fig:9-2}
\end{figure}

\begin{definition}\label{def:9-2}
对于一个密码 $\mathcal{E}=(E,D)$,如果 $\mathcal{E}$ (1) 在选择明文攻击下是语义安全的,以及 (2) 能够提供密文完整性,我们就称 $\mathcal{E}$ 能够提供\textbf{认证加密},或者简单地称它是 \textbf{AE 安全}的。
\end{definition}

为什么定义 \ref{def:9-2} 是正确的?特别地,为什么我们要求\emph{密文}完整性,而不是其他的什么\emph{明文}完整性的概念(这可能看起来更自然)?在 \ref{sec:9-2} 节中,我们会描述一类非常阴险的攻击,称为\emph{选择密文攻击},我们将看到,我们对 AE 安全性的定义足以(事实上是必要的)防止这种攻击。在 \ref{sec:9-3} 节中,我们将对该定义给出一个更高层次的理由。

\subsection{一次性认证加密}\label{subsec:9-1-1}

在实践中,我们经常使用对称密钥对单一消息进行加密。该密钥不会被再次使用。例如,在发送加密电子邮件时,我们经常选择一个临时密钥,并用这个临时密钥来加密电子邮件的正文。随后,这个临时密钥也会被加密,然后附在电子邮件的首部中被一起发送出去。每发送一封电子邮件,都会生成一个新的临时密钥。

在这种设置下,我们可以使用一次性的加密方案,比如流密码。所选择的密码必须是语义安全的,但不一定需要是 CPA 安全的。同样地,该密码提供一次性的密文完整性就足够了,这是一个比密文完整性更弱的概念。特别是,我们可以修改攻击游戏 \ref{game:9-1},使得对手只能获得单独的某条消息 $m$ 的加密。

\begin{definition}\label{def:9-3}
如果对于每个有效的单次查询对手 $\mathcal{A}$,$\mathrm{CI}\mathsf{adv}[\mathcal{A},\mathcal{E}]$ 的值都可忽略不计,我们就称密码 $\mathcal{E}=(E,D)$ 提供\textbf{一次性密文完整性 (one-time ciphertext integrity)}。
\end{definition}

\begin{definition}\label{def:9-4}
对于一个密码 $\mathcal{E}=(E,D)$,如果 $\mathcal{E}$ 是语义安全的,并且能够提供一次性密文完整性,我们就称 $\mathcal{E}$ 能够提供\textbf{一次性认证加密(one-time authenticated encryption)},或者称它是 \textbf{1AE 安全}的。
\end{definition}

在那些对称密钥只会被使用一次的应用中,1AE 安全性就足够了。我们将会表明,对于图 \ref{fig:9-1} 中的先加密后 MAC 构造,如果所使用的是语义安全的密码和一次性 MAC,它就能够提供一次性认证加密。用一次性 MAC 替代 MAC 能够显著提高效率。