\section{案例研究:802.11b WEP,一个千疮百孔的系统}\label{sec:9-10}

于 1999 年获批的 IEEE 802.11b 标准为短距离无线通信 (WiFi) 提供了一个协议。其安全性由 802.11b 数据帧封装的有线等效加密 (Wired Equivalent Privacy, WEP) 协议提供。WEP 的设计目标是提供能够媲美有线网络的数据隐私性。然而,WEP 在这方面彻头彻尾地失败了,并且它为我们提供了一个极好的案例,展示了一个孱弱的设计是如何导致灾难性的后果的。

当启用 WEP 时,无线网络中的所有成员都会共享一个长期密钥 $k$。该标准支持 $40$ 或 $128$ 比特密钥。$40$ 比特的版本符合标准起草时美国的出口限制。我们将使用以下符号来描述 WEP:
\begin{itemize}
	\item WEP 加密使用 RC4 流密码。我们用 $\mathrm{RC4}(s)$ 表示当给定种子 $s$ 时,RC4 所产生的伪随机序列。
	\item 我们用 $CRC(m)$ 表示消息 $m\in\{0,1\}^*$ 的 $32$ 比特 CRC 校验和。CRC 的实现细节与我们这里的讨论无关,读者只需将 CRC 视为从任意比特序列映射到 $\{0,1\}^{32}$ 的某个特定函数。
\end{itemize}

令 $m$ 是一个 802.11b 明文帧。$m$ 的前几个比特编码了 $m$ 的长度。为了加密一个 802.11b 帧 $m$,发送者选择一个 $24$ 比特的 IV,并计算:
\[
\begin{aligned}
& c\leftarrow\big(m\,\Vert\,\mathrm{CRC}(m)\big)\oplus\mathrm{RC4}(\mathrm{IV}\,\Vert\,k)\\
& c_\mathrm{full}\leftarrow(\mathrm{IV},\,c)
\end{aligned}
\]
WEP 的加密过程如图 \ref{fig:9-4} 所示。接收者首先计算 $c\oplus\mathrm{RC4}(\mathrm{IV}\,\Vert\,k)$ 来解密得到数对 $(m,s)$。如果 $s=\mathrm{CRC}(m)$,接收方就接受该帧,否则就拒绝它。

\begin{snote}[攻击 1:IV 碰撞。]
WEP 的设计者明白,流密码的密钥不应该被重复使用。因此,他们使用一个 $24$ 比特的 IV 来派生出每一帧的密钥 $k_\mathrm{f}:=\mathrm{IV}\,\Vert\,k$。然而,该标准并没有规定如何选择 IV,而许多实现都做得不够好。我们说,每当一个无线基站碰巧发送了两个帧,例如第 $i$ 帧和第 $j$ 帧,而它们都是由相同的 IV 加密得到的,IV 碰撞就会发生。由于 IV 是以透明方式发送的,所以窃听者很容易发现 IV 碰撞。此外,一旦 IV 碰撞发生,攻击者就可以使用 \ref{subsec:3-3-1} 小节中讨论的两次性密码本攻击来解密 $i$ 和 $j$ 这两个帧。

那么,IV 碰撞发生的概率到底有多大?根据生日悖论,如果为每一帧选择一个随机 IV,那么预期每隔 $\sqrt{2^{24}}=2^{12}=4096$ 帧,就会发生一次 IV 碰撞。由于每帧最长是 $1156$ 字节,那么平均每传输大约 4 MB 的数据,就会发生一次碰撞。

除此之外,我们也可以使用一个计数器来生成 IV。该实现会在发送 $2^{24}$ 帧后耗尽整个 IV 空间,对于一个满负荷工作的无线接入点来说,这只需要大约一天时间。更糟糕的是,一些使用计数器方法的无线网卡在开机时会将计数器重置为 $0$。因此,这些网卡会经常重复使用较小的 IV,而这会让流量极易遭受两次性密码本攻击。
\end{snote}

\begin{figure}
  \centering
  \tikzset{every picture/.style={line width=0.75pt}}

\begin{tikzpicture}[x=0.75pt,y=0.75pt,yscale=-1,xscale=1]


\draw   (70,0) -- (390,0) -- (390,40) -- (70,40) -- cycle ;
\draw   (390,0) -- (470,0) -- (470,40) -- (390,40) -- cycle ;
\draw   (70,55) -- (470,55) -- (470,95) -- (70,95) -- cycle ;
\draw   (20,125) -- (65,125) -- (65,165) -- (20,165) -- cycle ;
\draw  [fill={rgb, 255:red, 155; green, 155; blue, 155 }  ,fill opacity=0.5 ] (70,125) -- (470,125) -- (470,165) -- (70,165) -- cycle ;

\draw [line width=2.5]    (60,110) -- (480,110) ;

\draw   (40,48) .. controls (40,44.69) and (42.69,42) .. (46,42) .. controls (49.31,42) and (52,44.69) .. (52,48) .. controls (52,51.31) and (49.31,54) .. (46,54) .. controls (42.69,54) and (40,51.31) .. (40,48) -- cycle ;
\draw   (40,48) -- (52,48) ;
\draw   (46,42) -- (46,54) ;

\draw (220,20) node   [align=left] {明文载荷 $m$};
\draw (270,145) node   [align=left] {加密帧};
\draw (42.5,145) node   [align=left] {IV};
\draw (430,20) node    {$\mathrm{CRC}(m)$};
\draw (270,75) node    {$\mathrm{RC4}(\mathrm{IV}\,\Vert\,k)$};


\end{tikzpicture}
  \caption{WEP 加密}
  \label{fig:9-4}
\end{figure}

\begin{snote}[攻击 2:相关密钥。]
对 WEP 加密的一种更具破坏性的攻击来自对相关 RC4 密钥的使用。在第\ref{chap:3}章中,我们曾经解释,必须为每条加密消息选择一个新的、\emph{随机的}流密码密钥。然而,WEP 所使用的密钥 $1\,\Vert\,k$,$2\,\Vert\,k$,$\dots$ 都是密切相关的——它们都有相同的后缀 $k$。RC4 从来就不是为这种目的而设计的,事实上,它在这些情况下是完全不安全的。Fluhrer、Mantin 和 Shamir 表明,在发送了大约一百万个 WEP 帧后,窃听者就可以恢复整个长期密钥 $k$ \cite{fluhrer2001weaknesses}。该攻击由 Stubblefield、Ioannidis 和 Rubin 实现 \cite{stubblefield2004key},现在可被用于各种黑客工具,如 \textsc{WepCrack} 和 \textsc{AirSnort}。

应该使用一个 PRF 来生成每个帧的帧密钥,比如说将第 $i$ 帧的密钥设置为 $k_i:=F(k,\mathrm{IV})$——所产生的密钥和随机独立的密钥是无法区分的。当然,虽然这种方法确实可以防止相关密钥问题,但无法解决上面讨论的 IV 碰撞问题,以及接下来讨论的可塑性问题。
\end{snote}

\begin{snote}[攻击 3:易被控制性。]
回顾一下,WEP 试图使用 CRC 校验来为认证加密提供完整性。在某种意义上,WEP 使用的是先 MAC 后加密的方法,但它使用的是 CRC,而不是 MAC。我们表明,尽管包含加密步骤,但这种构造其实完全无法提供密文完整性。

针对它的攻击利用了 CRC 的线性特性。也就是说,对于某条消息 $m$,给定 $\mathrm{CRC}(m)$,我们很容易为任何的 $\Delta$ 计算出 $\mathrm{CRC}(m\oplus\delta)$。更确切地说,存在一个公共函数 $L$,对于任何的 $m$ 和 $\Delta\in\{0,1\}^\ell$,我们都有:
\[
\mathrm{CRC}(m\oplus\Delta)=\mathrm{CRC}(m)\oplus L(\Delta)
\]
这一属性使攻击者可以对 WEP 密文进行任意的修改,而不会被接收方发现。令 $c$ 是一条 WEP 密文,即:
\[
c=\big(m,\mathrm{CRC}(m)\big)\oplus\mathrm{RC4}(\mathrm{IV}\,\Vert\,k)
\]
因此,$c'$ 可以完全无误地被解密为 $m\oplus\Delta$。我们可以发现,给定对 $m$ 的加密,攻击者就可以为它所选择的任何一个 $\Delta$ 创建一个对 $m\oplus\Delta$ 的有效加密。我们已经在 \ref{subsec:3-3-2} 小节解释过,这很有可能会导致严重的攻击。
\end{snote}

\begin{snote}[攻击4:选择的密码文本攻击。]
该协议还容易受到一种被称为\textbf{断续攻击 (chop-chop)}的选择密文攻击,这种攻击可以让攻击者解密由其选择的加密帧。我们会在练习 \ref{exer:9-5} 中描述这种攻击的一个简单版本。
\end{snote}

\vspace*{-10pt}

\begin{snote}[攻击5: 拒绝服务。]
我们简单提一下,802.11b 曾遭受过许多严重的拒绝服务 (Denial of Service, DoS) 攻击。比如说,在 802.11b 中,一旦客户端结束了网络调用,无线客户端就会向无线基站发送一条``断联"的消息。这是为了让无线基站能够释放分配给该客户端的内存资源。不幸的是,``断联"信息不会经过任何认证,也就是说,任何人都可以代表其他人发送断联消息。一旦断联,受害者将需要几秒钟才能和基站重新建立连接。因此,只要每隔几秒钟就向基站发送一次``断联"消息,攻击者就可以阻止任何由其选定的计算机连接到无线网络中。一些 802.11b 工具,比如 \texttt{Void11},就实现了这种攻击。
\end{snote}

\begin{snote}[802.11i。]
在 802.11b WEP 协议失败后,一个名为 802.11i 的新标准在 2004 年得到批准。802.11i 使用一种称作 CCM 的先加密后 MAC 模式来提供认证加密。具体地说,CCM 使用(原生)CBC-MAC 进行 MAC,并使用计数器模式进行加密。802.11i 使用 AES 作为底层的 PRF 来实现这两者。后来,CCM 被 NIST 采纳为联邦标准 \cite{dworkin2004sp}。
\end{snote}