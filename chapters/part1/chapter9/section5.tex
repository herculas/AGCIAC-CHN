\section{带有关联数据的基于nonce的认证加密}\label{sec:9-5}

在这一节中,我们将扩展认证加密的语法来符合其常用的形式。首先,正如我们对加密和 MAC 所做的那样,我们会定义基于 nonce 的认证加密,它能使得加解密算法具有确定性,但是会将唯一的 nonce 作为输入。这种方法不但可以缩短密文,也能提高安全性。

其次,为了扩展加密算法,我们会为加密算法额外提供一条输入消息,称为\textbf{关联数据(associated data)},其完整性受到密文的保护,但机密性则不然。很多设置都会需要关联数据。例如,当在网络协议中加密数据包时,认证加密能够保护数据包主体,但数据包的首部必须以透明方式传输,以便网络能够将数据包传送到其预定的目的地。尽管如此,我们还是要确保数据包首部的完整性。此时,首部就可以作为关联数据被输入到加密算法中。

支持关联数据的密码被称为 \textbf{AD 密码}。基于 nonce 的 AD 密码的语法如下:
\[
c=E(k,m,d,{\scriptstyle\mathpzc{N}})
\]
其中,$c\in\mathcal{C}$ 是密文,$k\in\mathcal{K}$ 是密钥,$m\in\mathcal{M}$ 是消息,$d\in\mathcal{D}$ 是关联数据,${\scriptstyle\mathpzc{N}}\in\mathpzc{N}$ 是 nonce。此外,我们要求加密算法 $E$ 是确定性的。类似地,解密的语法变成了:
\[
D(k,c,d,{\scriptstyle\mathpzc{N}})
\]
它输出一条消息 $m$ 或者 $\mathsf{reject}$。我们称基于 nonce 的 AD 密码定义在 $(\mathcal{K},\mathcal{M},\mathcal{D},\mathcal{C},\mathpzc{N})$ 上。和之前一样,我们要求,\emph{只要被赋予相同的 nonce 和关联数据},由 $E$ 产生的密文就一定能被 $D$ 正确地解密。也就是说,对于所有的密钥 $k$,所有的消息 $m$,所有的关联数据 $d$,以及所有的 nonce 值 ${\scriptstyle\mathpzc{N}}\in\mathpzc{N}$,我们都有:
\[
D\big(k,\,E(k,m,d,{\scriptstyle\mathpzc{N}}),\,d,\,{\scriptstyle\mathpzc{N}}\big)=m
\]

如果输入到加密算法中的消息 $m$ 是一条空消息,那么密码 $(E,D)$ 本质上就变成了用于关联数据 $d$ 的 MAC 系统。

\begin{snote}[CPA 安全性。]
对于一个基于 nonce 的 AD 密码,假设在加密过程中\emph{没有 nonce 会被多次使用},如果该密码不会向窃听者泄露任何有用的信息,我们就称它是 CPA 安全的。基于 nonce 的 AD 密码的 CPA 安全性定义与基于 nonce 的密码的 CPA 安全性定义(见 \ref{sec:5-5} 节)基本一致。唯一的区别在加密查询方面。对于 $b=0,1$,在实验 $b$ 中,加密查询会按如下方式处理:
\begin{quote}
第 $i$ 次加密查询的内容是一对长度相同的消息 $m_{i0},m_{i1}\in\mathcal{M}$,关联数据 $d_i\in\mathcal{D}$,以及一个唯一的 nonce ${\scriptstyle\mathpzc{N}}_i\in\mathpzc{N}\setminus\{{\scriptstyle\mathpzc{N}}_1,\dots,{\scriptstyle\mathpzc{N}}_{i-1}\}$。

挑战者计算 $c_i\leftarrow E(k,m_{ib},d_i,{\scriptstyle\mathpzc{N}}_i)$,并将 $ci$ 发送给对手。
\end{quote}
除此之外,其他部分都与 \ref{sec:5-5} 节中的定义相同。请注意,关联数据 $d_i$ 是在对手的控制之下的,nonce 值 ${\scriptstyle\mathpzc{N}}_i$ 也是如此,只需要保证其唯一性即可。对于 $b=0,1$,令 $W_b$ 是对手 $\mathcal{A}$ 在实验 $b$ 中输出 $1$ 的事件。我们将 $\mathcal{A}$ 相对于 $\mathcal{E}$ 的优势定义为:
\[
\mathrm{nCPA}_\mathrm{ad}\mathsf{adv}[\mathcal{A},\mathcal{E}]
:=
\big\lvert
\Pr[W_0]-\Pr[W_1]
\big\rvert
\]
\end{snote}

\begin{definition}[CPA 安全性]\label{def:9-7}
如果对于所有的有效对手 $\mathcal{A}$,$\mathrm{nCPA}_\mathrm{ad}\mathsf{adv}[\mathcal{A},\mathcal{E}]$ 的值都可忽略不计,我们就称基于 nonce 的 AD 密码\textbf{对选择明文攻击是语义安全的(semantically secure against chosen plaintext attack)},简称为是 \textbf{CPA 安全的 (CPA-secure)}。
\end{definition}

\begin{snote}[密文完整性。]
如果一个攻击者能够在密钥 $k$ 下请求由它所选择的消息、关联数据和 nonce 的加密结果,但仍然无法输出一个能够被解密算法接受的新三元组 $(c,d,{\scriptstyle\mathpzc{N}})$,我们就称基于 nonce 的 AD 密码能够提供密文完整性。尽管如此,对手决不能使用曾经使用过的 nonce 发起加密查询。

更确切地说,我们按如下方式修改密文完整性游戏(攻击游戏 \ref{game:9-1}):
\end{snote}

\begin{game}[密文完整性]\label{game:9-3}
对于一个定义在 $(\mathcal{K},\mathcal{M},\mathcal{D},\mathcal{C},\mathpzc{N})$ 上的给定 AD 密码 $\mathcal{E}=(E,D)$ 和一个给定对手 $\mathcal{A}$,攻击游戏运行如下:
\begin{itemize}
	\item 挑战者随机选取一个 $k\overset{\rm R}\leftarrow\mathcal{K}$。
	\item $\mathcal{A}$ 对挑战者发起一系列查询。对于 $i=1,2,\dots$,第 $i$ 次查询包含一条消息 $m_i\in\mathcal{M}$,关联数据 $d_i\in\mathcal{D}$,以及一个先前未被使用过的 nonce ${\scriptstyle\mathpzc{N}}_i\in\mathpzc{N}\setminus\{{\scriptstyle\mathpzc{N}}_1,\dots,{\scriptstyle\mathpzc{N}}_{i-1}\}$。挑战者计算 $c_i\leftarrow E(k,m_i,d_i,{\scriptstyle\mathpzc{N}}_i)$,并将 $ci$ 交给 $\mathcal{A}$。
	\item 最终,$\mathcal{A}$ 输出一个候选三元组 $(c,d,{\scriptstyle\mathpzc{N}})$,其中 $c\in\mathcal{C}$,$d\in\mathcal{D}$,${\scriptstyle\mathpzc{N}}\in\mathpzc{N}$,且该三元组与它先前收到的所有三元组都不同,即:
		\[
		(c,d,{\scriptstyle\mathpzc{N}})\notin
		\big\{
			(c_1,d_1,{\scriptstyle\mathpzc{N}}_1),\;
			(c_2,d_2,{\scriptstyle\mathpzc{N}}_2),\;
			\cdots
		\big\}
		\]
\end{itemize}
如果 $D(k,c,d,{\scriptstyle\mathpzc{N}})\neq\mathsf{reject}$,我们就称 $\mathcal{A}$ 赢得游戏。我们将 $\mathcal{A}$ 相对于 $\mathcal{E}$ 的优势定义为 $\mathrm{nCI}_\mathrm{ad}\mathsf{adv}[\mathcal{A},\mathcal{E}]$,即 $\mathcal{A}$ 赢得游戏的概率。
\end{game}

\begin{definition}\label{def:9-8}
如果对于所有的有效对手 $\mathcal{A}$,$\mathrm{nCI}_\mathrm{ad}\mathsf{adv}[\mathcal{A},\mathcal{E}]$ 的值都可忽略不计,我们就称基于 nonce 的 AD 密码 $\mathcal{E}=(E,D)$ 具有\textbf{密文完整性}。
\end{definition}

\begin{snote}[认证加密。]
现在,我们可以为 AD 密码定义基于 nonce 的认证加密。我们将其称作\textbf{基于 nonce 的 AEAD 密码(nonce-based AEAD cipher)},其中,AEAD 指的是\textbf{带有关联数据的认证加密 (authenticated encryption with associated data)}。
\end{snote}

\begin{definition}\label{def:9-9}
如果基于 nonce 的 AD 密码 $\mathcal{E}=(E,D)$ 是 CPA 安全的,并且具有密文完整性,我们就称 $\mathcal{E}$ 提供认证加密,或将其简称为\textbf{基于 nonce 的 AEAD 密码(nonce-based AEAD cipher)}。
\end{definition}

\begin{snote}[通用先加密后 MAC 组合。]
我们可以将一个基于 nonce 的 CPA 安全密码 $(E,D)$(如 \ref{sec:5-5} 节)和一个基于 nonce 的安全 MAC $(S,V)$(如 \ref{sec:7-5} 节)组合起来,以构建一个基于 nonce 的 AEAD 密码 $\mathcal{E}=(E_\mathrm{EtM},D_\mathrm{EtM})$,方法如下:

\vspace*{10pt}

\hspace*{20pt} $E_\mathrm{EtM}\big((k_\mathrm{e},k_\mathrm{m}),\;m,d,{\scriptstyle\mathpzc{N}}\big)$
				\quad $:=$ \quad
				$c\leftarrow E(k_\mathrm{e},m,{\scriptstyle\mathpzc{N}})$,
				\quad
				$t\leftarrow S\big(k_\mathrm{m},\;(c,d),\;{\scriptstyle\mathpzc{N}}\big)$\\
\hspace*{187pt} 输出 $(c,t)$

\vspace*{5pt}

\hspace*{7pt} $D_\mathrm{EtM}\big((k_\mathrm{e},k_\mathrm{m}),\;(c,t),d,{\scriptstyle\mathpzc{N}}\big)$
				\quad $:=$ \quad
				如果 $V\big(k_\mathrm{m},\;(c,d),\;t,\;{\scriptstyle\mathpzc{N}}\big)=\mathsf{reject}$,则输出 $\mathsf{reject}$\\
\hspace*{187pt} 否则输出 $D(k_\mathrm{e},c,d,{\scriptstyle\mathpzc{N}})$

\vspace*{10pt}

\noindent
EtM 系统定义在 $(\mathcal{K}_\mathrm{e}\times\mathcal{K}_\mathrm{m},\mathcal{M},\mathcal{D},\mathcal{C}\times\mathcal{T},\mathpzc{N})$上。下面的定理将表明,$\mathcal{E}_\mathrm{EtM}$ 是一个安全的 AEAD 密码。
\end{snote}


\begin{theorem}\label{theo:9-4}
令 $\mathcal{E}=(E,D)$ 是一个基于 nonce 的密码,$\mathcal{I}=(S,V)$ 是一个基于 nonce 的 MAC 系统。假设 $\mathcal{E}$ 是 CPA 安全的,$\mathcal{I}$ 是一个安全的 MAC,则 $\mathcal{E}_\mathrm{EtM}$ 是一个基于 nonce 的 AEAD 密码。
\end{theorem}

定理 \ref{theo:9-4} 的证明与定理 \ref{theo:9-2} 基本相同。