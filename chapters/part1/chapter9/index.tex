\chapter{认证加密}\label{chap:9}

本章,我们终于来到了对称加密的最高点。在这里,我们将构建能够同时确保数据机密性和完整性的系统,即使面对的是那些能够同时与发送方和接收方进行恶意交互的,极具攻击性的攻击者。我们将这样的系统称为\textbf{认证加密(authenticated encryption)}。或者简单地称它们是 AE 安全的。本章将会收束我们对对称加密的讨论,还会展示如何在现实世界中正确地进行安全的加密。

回顾一下,当我们在第\ref{chap:5}章中讨论 CPA 安全性时,我们强调,CPA 安全性并不提供任何完整性保证。攻击者可以篡改 CPA 安全密码的输出,而不被解密者发现。在本章中,我们将会介绍许多现实世界中的设置,其中,未被发现的密文篡改会损害消息的机密性和完整性。因此,单靠 CPA 安全性几乎无法满足任何应用的实际需求。相对地,应用程序总是应该使用认证加密来确保消息的机密性和完整性。我们强调,即使一个系统只要求机密性,CPA 安全性也是远远不够的。

在本章中,我们将讨论认证加密的概念,并构建几个 AE 系统。有两种构建 AE 系统的一般范式。第一种被称为\textbf{通用组合(generic composition)},它可以将 CPA 安全的密码与安全的 MAC 相结合。有许多方法可以结合这两个密码学原语,但并非所有的组合都是安全的。下面,我们简单考虑两个例子。

令 $(E,D)$ 是一个密码,$(S,V)$ 是一个 MAC。令 $k_\mathrm{enc}$ 是一个密码密钥,$k_\mathrm{mac}$ 是一个 MAC 密钥。我们立即就可以想到两种将加密与完整性相结合的方案,如图 \ref{fig:9-1} 所示,它们运行如下:
\begin{description}
	\item [先加密后 MAC] 使用 $c\overset{\rm R}\leftarrow E(k_\mathrm{enc},m)$ 加密消息,然后用 $\mathrm{tag}\overset{\rm R}\leftarrow S(k_\mathrm{mac},c)$ 计算密文的 MAC;结果是密文-标签对 $(c,\mathrm{tag})$。这种方法在 TLS 1.2 协议和后续版本中,以及在 IPsec 协议和广泛使用的 NIST GCM 标准中都得到了支持(见 \ref{sec:9-7} 节)。
	\item [先 MAC 后加密] 使用 $\mathrm{tag}\overset{\rm R}\leftarrow S(k_\mathrm{mac},m)$ 计算消息的 MAC,然后使用 $c\overset{\rm R}\leftarrow E\big(k_\mathrm{enc},(m,\mathrm{tag})\big)$ 加密消息-标签对;结果是密文 $c$。这种方法被用在了 TLS 的旧版本(例如 SSL 3.0 和后续的 TLS 1.0 版本)和 802.11i WiFi 加密协议中。
\end{description}
事实证明,对于任意 CPA 安全的密码和安全的 MAC 的组合,都只有第一种方法是安全的。其直觉是,密文上的 MAC 可以防止对密文的任何篡改。我们将表明,第二种方法可能是不安全的——MAC 和密码之间的交互可能会出现问题,导致所产生的系统不是 AE 安全的。这使得一些已经被广泛部署的系统反复遭到攻击。

\begin{figure}
	\centering
	\tikzset{every picture/.style={line width=0.75pt}}

\begin{tikzpicture}[x=0.75pt,y=0.75pt,yscale=-1,xscale=1]

\draw [line width=1.2]  (0,0) -- (140,0) -- (140,20) -- (0,20) -- cycle ;
\draw  [fill={rgb, 255:red, 155; green, 155; blue, 155 }  ,fill opacity=0.5 ][line width=1.2] (0,50) -- (140,50) -- (140,80) -- (0,80) -- cycle ;
\draw  [fill={rgb, 255:red, 155; green, 155; blue, 155 }  ,fill opacity=0.5 ][line width=1.2] (0,120) -- (140,120) -- (140,140) -- (0,140) -- cycle ;
\draw [line width=1.2] (140,120) -- (190,120) -- (190,140) -- (140,140) -- cycle ;

\draw    (0,20) -- (0,120) ;
\draw    (140,0) -- (140,80) ;
\draw    (140,80) -- (190,120) ;

\draw  [line width=1.2] (280,0.67) -- (420,0.67) -- (420,20.67) -- (280,20.67) -- cycle ;
\draw  [fill={rgb, 255:red, 155; green, 155; blue, 155 }  ,fill opacity=0.5 ][line width=1.2] (280,110) -- (470,110) -- (470,140) -- (280,140) -- cycle ;
\draw  [line width=1.2] (280,50) -- (420,50) -- (420,70) -- (280,70) -- cycle ;
\draw  [line width=1.2] (420,50) -- (470,50) -- (470,70) -- (420,70) -- cycle ;

\draw    (280,0.67) -- (280,140) ;
\draw    (420,20.67) -- (470,50) ;
\draw    (470,70) -- (470,110) ;

\draw (70,10) node    {$m$};
\draw (70,130) node    {$c$};
\draw (165,130) node    {$\mathrm{tag}$};
\draw (70,65) node    {$c\leftarrow E(k_{\mathrm{enc}},m)$};
\draw (80,100) node    {$\mathrm{tag}\leftarrow S( k_{\mathrm{mac}},c)$};
\draw (350,10.67) node   {$m$};
\draw (350,60) node    {$m$};
\draw (445,60) node    {$\mathrm{tag}$};
\draw (375,125) node    {$c\leftarrow E\big(k_{\mathrm{enc}} ,(m,\mathrm{tag})\big)$};
\draw (360,35) node    {$\mathrm{tag}\leftarrow S(k_{\mathrm{mac}} ,m)$};
\draw (95,170) node   [align=left] {先加密后 MAC};
\draw (375,170) node   [align=left] {先 MAC 后加密};

\end{tikzpicture}
	\caption{两种组合加密和 MAC 的方法}
	\label{fig:9-1}
\end{figure}

构建认证加密的第二种范式是直接利用分组密码或 PRF,而不需要先构建一个独立的密码或 MAC。这种方法有时被称为\textbf{集成方案(integrated schemes)}。OCB 加密模式是这一范式的一个典型案例(见练习 \ref{exer:9-17})。其他的例子包括 IAPM、XCBC、CCFB 等方案。

\begin{snote}[认证加密标准。]
像 OpenSSL 这样的密码学库通常会提供一个 CPA 安全加密的接口(比如带有随机 IV 的计数器模式),以及一个单独用来计算消息的 MAC 的接口。在过去,开发者需要自己正确地组合这两种密码学原语来实现认证加密。每个系统的做法都不尽相同,而且,并非所有在实践中使用的实例都是安全的。

最近,一些安全认证加密标准被陆续提出。一种被称为 Galois 计数模式 (Galois Counter Mode, GCM) 的流行方法使用先加密后 MAC 的方法来组合随机计数器模式加密和 Carter-Wegman MAC(见 \ref{sec:9-7} 节)。我们将在本章的稍后部分研究这种结构的细节及其安全性。我们鼓励开发者使用由底层密码学库提供的认证加密模式,而不要自己去实现它们。
\end{snote}


\section{认证加密的定义}\label{sec:9-1}

\begin{game}[密文完整性]\label{game:9-1}
	
\end{game}

\begin{definition}\label{def:9-1}
	
\end{definition}

\begin{definition}\label{def:9-2}
	
\end{definition}

\subsection{一次性认证加密}\label{subsec:9-1-1}

\begin{definition}\label{def:9-3}
	
\end{definition}

\begin{definition}\label{def:9-4}

\end{definition}
\section{认证加密的含义}
\section{作为抽象接口的加密}\label{sec:9-3}

为了深入理解认证加密的定义,我们表明,它准确地捕捉了一个直观的概念,即作为一个\emph{抽象接口(abstract interface)}的安全加密。AE 安全性意味着,对这种接口的实际实现可以被一个理想化实现取代。而在这个理想化实现中,消息实际上是从发送方直接跳变到接收方的,根本不需要经过任何网络通信(甚至是任何加密方式的通信)。下面,我们更系统地阐述这一想法。

假设发送方 $S$ 和接收方 $R$ 正在使用某个任意的、基于互联网的系统(比如游戏、拍卖、银行等等)。另外,我们假设 $S$ 和 $R$ 已经共享了一个随机的加密密钥 $k$。在协议过程中,$S$ 会向 $R$ 发送消息 $m_1,m_2,\dots$ 的加密。消息 $m_i$ 由 $S$ 所使用协议的逻辑决定,我们不管它到底是什么。我们可以想象,$S$ 将消息 $m_i$ 放在它的``发件箱"中,而发件箱的具体工作细节与 $S$ 无关。当然,我们还是能知道 $S$ 的发件箱中会发生什么:$m_i$ 会被 $k$ 加密成 $c_i$,而后者会通过一条电线被发送至 $R$。

在接收侧,当一条密文 $\hat{c}$ 抵达电线位于 $R$ 的那一端时,它就会被 $k$ 解密。如果解密的结果是一条消息 $\hat{m}\neq\mathsf{reject}$,这条消息就会被放到 $R$ 的``收件箱"中。每当有消息出现在 $R$ 的收件箱中,$R$ 都可以读取这条消息,并根据它的协议逻辑对消息进行处理,而不必担心消息是如何到达的。

攻击者可能试图以多种方式破坏 $S$ 和 $R$ 之间的通信:
\begin{itemize}
	\item 第一,攻击者可能会丢弃、重排或重放 $S$ 所发送的密文。
	\item 第二,攻击者可能会篡改由 $S$ 发送的密文,或注入一些凭空创建的新密文。
	\item 第三,攻击者可能掌握了由 $S$ 发送的某些消息的部分知识,或者甚至能够影响其中一些消息的选择。
	\item 第四,通过观察 $R$ 的行为,攻击者可能会收集到经过 $R$ 处理的某些消息的部分知识。即使是关于交付给 $R$ 的密文是否被拒绝的知识也可能是有用的。
\end{itemize}

在描述了一个抽象的加密接口及其实现之后,我们下面描述这个接口的一种\emph{理想实现(ideal implementation)},它以一种直观的方式捕捉到了认证加密所提供的安全性保证。当 $S$ 将 $m_i$ 放到它的``发件箱"中时,理想实现现在不再对 $m_i$ 进行加密,而是加密一条与 $m_i$ 无关的假消息 $\mathit{dummy}_i$(它们的长度应当相同),以此来创建一条密文 $c_i$。因此,$c_i$ 可以被视作是 $m_i$ 的一个``把手",但不包含任何(除了长度之外)的关于 $m_i$ 的信息。当 $c_i$ 到达 $R$ 时,对应的消息 $m_i$ 被神奇地从 $S$ 的发件箱复制到 $R$ 的收件箱中。如果一条密文到达 $R$,但不在先前生成的 $c_i$ 中,理想实现就会丢弃它。

这个理想实现其实只是一种思想实验。它显然不能以任何有效的方式在物理上实现(如果不先发明远程传输的话)。然而,正如我们将要论证的那样,如果底层密码 $\mathcal{E}$ 提供认证加密,理想实现——就所有实际目的而言——都等同于真正的实现。因此,协议设计者不需要担心真实实现的任何细节,或者加密定义的细微差别:他可以假装他正在使用抽象的加密接口及其理想实现,其中的密文都只是把手,而消息都会神奇地从 $S$ 跳变到 $R$。

请注意,就算是在理想实现中,攻击者仍然可能丢弃、重排或重放密文,而这些行为将导致对应的消息被丢弃、重排或重放。使用序列号和缓冲区处理这些问题并不困难,但这些工作要留给更高层的协议。

\vspace*{10pt}

下面,我们非正式地论证,当 $\mathcal{E}$ 能够提供认证加密时,现实世界中的实现与理想实现是没有任何区别的。该论证分三步进行。我们先从真实的实现开始,在每一步中,我们都会做一些轻微的修改。
\begin{itemize}
	\item 首先,我们对 $R$ 的收件箱的真实实现进行修改,如下所示。当一条密文 $\hat{c}$ 到达 $R$ 那一端时,由 $S$ 先前生成的密文列表 $c_1,c_2,\dots$ 会被扫描,如果 $\hat{c}=c_i$,相应的消息 $m_i$ 就会被神奇地从 $S$ 的发件箱复制到 $R$ 的收件箱中,而不需要实际运行解密算法。
	
	$\mathcal{E}$ 的正确性属性保证,修改后的行为与真正实现完全相同。
	\item 其次,我们再次修改 $R$ 的收件箱上的实现,此时,如果一条密文 $\hat{c}$ 到达 $R$ 那一端,但不在 $S$ 所生成的密文列表中,该实现就会丢弃 $\hat{c}$。
	
	对手能够区分这种修改和第一种修改的唯一方法是,它创造一条不会被拒绝,但不是由 $S$ 产生的密文。但这是不可能的,因为 $\mathcal{E}$ 有密文完整性。
	\item 第三,我们修改 $S$ 的发件箱的实现,用对 $\mathit{dummy}_i$ 的加密取代对 $m_i$ 的加密。$R$ 的收件箱的实现仍与第二次修改相同。请注意,解密算法在第二和第三种修改中都没有被使用。因此,一个能够区分这个修改和第二个修改的对手就可以被用来直接破解 $\mathcal{E}$ 的 CPA 安全性。因此,由于 $\mathcal{E}$ 是 CPA 安全的,这两种修改就是不可区分的。
\end{itemize}

由于第三个修改与理想实现相同,我们可以看到,从对手的角度来看,真实实现和理想实现是无法区分的。

我们没有考虑到的一个技术问题是,存在这样一种可能性,即由 $S$ 产生的 $c_i$ 不是唯一的。当然,如果我们要把 $c_i$ 看作是理想实现中的把手,唯一性似乎是一个基本属性。事实上,CPA 安全意味着,在理想实现中,生成的 $c_i$ 是唯一的概率是压倒性的,参见练习 \ref{exer:5-12}。
\section{基于通用组合的认证加密密码}\label{sec:9-4}

在本节,我们试图通过组合一个 CPA 安全的密码和一个安全的 MAC 组合来构建认证加密。我们将会表明,先加密后 MAC 范式总是 AE 安全的,但是先 MAC 后加密范式却不是这样。

\subsection{先加密后MAC}\label{subsec:9-4-1}

令 $\mathcal{E}=(E,D)$ 是一个定义在 $(\mathcal{K}_\mathrm{e},\mathcal{M},\mathcal{C})$ 上的密码,$\mathcal{I}=(S,V)$ 是一个定义在 $(\mathcal{K}_\mathrm{m},\mathcal{C},\mathcal{T})$ 上的 MAC。先加密后 MAC 系统 $\mathcal{E}_\mathrm{EtM}=(E_\mathrm{EtM},D_\mathrm{EtM})$,简称为 $\mathrm{EtM}$,定义如下:

\vspace*{10pt}

\hspace*{20pt} $E_\mathrm{EtM}\big((k_\mathrm{e},k_\mathrm{m}),\;m\big)$
				\quad $:=$ \quad
				$c\overset{\rm R}\leftarrow E(k_\mathrm{e},m)$,
				\quad
				$t\overset{\rm R}\leftarrow S(k_\mathrm{m},c)$\\
\hspace*{165.5pt} 输出 $(c,t)$

\vspace*{5pt}

\hspace*{7pt} $D_\mathrm{EtM}\big((k_\mathrm{e},k_\mathrm{m}),\;(c,t)\big)$
				\quad $:=$ \quad
				如果 $V(k_\mathrm{m},c,t)=\mathsf{reject}$,则输出 $\mathsf{reject}$\\
\hspace*{165.5pt} 否则输出 $D(k_\mathrm{e},c)$

\vspace*{10pt}

\noindent
我们称 $\mathrm{EtM}$ 系统定义在 $(\mathcal{K}_\mathrm{e}\times\mathcal{K}_\mathrm{m},\;\mathcal{M},\;\mathcal{C}\times\mathcal{T})$ 上。下面的定理将表明,$\mathcal{E}_\mathrm{EtM}$ 能够提供认证加密。

\begin{theorem}\label{theo:9-2}
令 $\mathcal{E}=(E,D)$ 是一个密码,$\mathcal{I}=(S,V)$ 是一个 MAC 系统。如果 $\mathcal{E}$ 是 CPA 安全的,且 $\mathcal{I}$ 是一个安全的 MAC 系统,则 $\mathcal{E}_\mathrm{EtM}$ 就是 AE 安全的。此外,如果 $\mathcal{E}$ 是语义安全的,且 $\mathcal{I}$ 是一个一次性安全的 MAC 系统,则 $\mathcal{E}_\mathrm{EtM}$ 就是 1AE 安全的。
\begin{quote}
特别地,对于每个像攻击游戏 \ref{game:9-1} 中那样攻击 $\mathcal{E}_\mathrm{EtM}$ 的密文完整性对手 $\mathcal{A}_\mathrm{ci}$,都存在一个像攻击游戏 \ref{game:6-1} 中那样攻击 $\mathcal{I}$ 的 MAC 对手 $\mathcal{B}_\mathrm{mac}$,其中 $\mathcal{B}_\mathrm{mac}$ 是一个围绕 $\mathcal{A}_\mathrm{ci}$ 的基本包装器,由其发起的签名查询不多于由 $\mathcal{A}_\mathrm{ci}$ 发起的加密查询,满足:
\end{quote}
\[
\mathrm{CI}\mathsf{adv}[\mathcal{A}_\mathrm{ci},\mathcal{E}_\mathrm{EtM}]
=
\mathrm{MAC}\mathsf{adv}[\mathcal{B}_\mathrm{mac},\mathcal{I}]
\]
\begin{quote}
对于每个像攻击游戏 \ref{game:5-2} 中那样攻击 $\mathcal{E}_\mathrm{EtM}$ 的 CPA 对手 $\mathcal{A}_\mathrm{cpa}$,都存在一个像攻击游戏 \ref{game:5-2} 中那样攻击 $\mathcal{E}$ 的 CPA 对手 $\mathcal{B}_\mathrm{cpa}$,其中 $\mathcal{B}_\mathrm{cpa}$ 是一个围绕 $\mathcal{A}_\mathrm{cpa}$ 的基本包装器,由其发起的加密查询不多于由 $\mathcal{A}_\mathrm{cpa}$ 发起的加密查询,满足:
\end{quote}
\[
\mathrm{CPA}\mathsf{adv}[\mathcal{A}_\mathrm{cpa},\mathcal{E}_\mathrm{EtM}]
=
\mathrm{CPA}\mathsf{adv}[\mathcal{B}_\mathrm{cpa},\mathcal{E}]
\]
\end{theorem}

\begin{proof}
我们先证明 $\mathcal{E}_\mathrm{EtM}$ 能够提供密文完整性。该证明可以通过一个直接的归约实现。假设 $\mathcal{A}_\mathrm{ci}$ 是一个攻击 $\mathcal{E}_\mathrm{EtM}$ 的密文完整性对手。我们构建一个攻击 $\mathcal{I}$ 的 MAC 对手 $\mathcal{B}_\mathrm{mac}$。

对手 $\mathcal{B}_\mathrm{mac}$ 在 $\mathcal{I}$ 的 MAC 攻击游戏中扮演对手的角色,它与一个 MAC 挑战者 $\mathbf{C}_\mathrm{mac}$ 交互,后者在交互开始时随机选取一个 $k_\mathrm{m}\overset{\rm R}\leftarrow\mathcal{K}_\mathrm{m}$。然后,对手 $\mathcal{B}_\mathrm{mac}$ 模拟 $\mathcal{A}_\mathrm{ci}$ 的 $\mathcal{E}_\mathrm{EtM}$ 密文完整性挑战者,运行如下:

\vspace*{10pt}

\hspace*{5pt} 随机选取 $k_\mathrm{e}\overset{\rm R}\leftarrow\mathcal{K}_\mathrm{e}$\\
\hspace*{26pt} 当从 $\mathcal{A}_\mathrm{ci}$ 处收到一个查询 $m_i\in\mathcal{M}$ 时:\\
\hspace*{50pt} 令 $c_i\overset{\rm R}\leftarrow E(k_\mathrm{e},m_i)$\\
\hspace*{50pt} 在 $c_i$ 处查询 $\mathbf{C}_\mathrm{mac}$,并获得一个应答 $t_i\overset{\rm R}\leftarrow S(k_\mathrm{m},c_i)$\\
\hspace*{50pt} 将 $(c_i,t_i)$ 发送给 $\mathcal{A}_\mathrm{ci}$
\hspace*{10pt} // \quad \emph{于是有} $(c_i,t_i)=E_\mathrm{EtM}\big((k_\mathrm{e},k_\mathrm{m}),m_i\big)$\\
\hspace*{26pt} 最后,$\mathcal{A}_\mathrm{ci}$ 输出一条密文 $(c,t)\in\mathcal{C}\times\mathcal{T}$\\
\hspace*{26pt} 输出该消息-标签对 $(c,t)$

\vspace*{10pt}

\noindent
应该明确的是,$\mathcal{B}_\mathrm{mac}$ 就像在一个真正的密文完整性攻击游戏中一样对 $\mathcal{A}_\mathrm{ci}$ 的查询给出了应答。因此,对手 $\mathcal{A}_\mathrm{ci}$ 能以 $\mathrm{CI}\mathsf{adv}[\mathcal{A}_\mathrm{ci},\mathcal{E}_\mathrm{EtM}]$ 的概率输出一条能让它赢得攻击游戏 \ref{game:9-1} 的密文 $(c,t)$,满足 $(c,t)\notin\{(c_1,t_1),\dots\}$ 且 $V(k_\mathrm{m},c,t)=\mathsf{accept}$。由此可知,$(c,t)$ 是一个能让 $\mathcal{B}_\mathrm{mac}$ 赢得 MAC 攻击游戏的消息-标签对。因此,我们可得 $\mathrm{CI}\mathsf{adv}[\mathcal{A}_\mathrm{ci},\mathcal{E}_\mathrm{EtM}]=\mathrm{MAC}\mathsf{adv}[\mathcal{B}_\mathrm{mac},\mathcal{I}]$,正如定理所要求的。

剩下的工作就是证明,如果 $\mathcal{E}$ 是 CPA 安全的,则 $\mathcal{E}_\mathrm{EtM}$ 也是安全的。这就等于是说,密文中包含的那个使用密钥 $k_\mathrm{m}$ 计算出来的(因而根本不涉及加密密钥 $k_\mathrm{e}$)标签不会在攻击者破坏 $\mathcal{E}_\mathrm{EtM}$ 的 CPA 安全性时提供帮助。这部分的证明非常简单,我们将其留作练习(可参见练习 \ref{exer:5-20})。
\end{proof}

回顾一下我们在第\ref{chap:6}章中给出的安全的 MAC 的定义,我们要求,给定一个消息-标签对 $(c,t)$,攻击者无法制造一个新的标签 $t\neq t'$,同时使得 $(c,t')$ 也是一个合法的消息-标签对。在当时,这样的要求似乎有些奇怪:如果攻击者已经有了一个 $c$ 的有效标签,我们为什么还要关心它能否为 $c$ 找到另一个标签?现在,我们看到,如果攻击者能为 $c$ 找到另一个有效标签 $t'$,他就可以破坏 $\mathrm{EtM}$ 的密文完整性。攻击者可以使用 $\mathrm{EtM}$ 的密文 $(c,t)$ 来构建另一条有效密文 $(c,t')$,并赢得密文完整性游戏。我们对安全的 MAC 的定义能够确保,攻击者无法修改 $\mathrm{EtM}$ 的密文而不被发现。

\subsubsection{实现先加密后 MAC 时的常见错误}\label{subsubsec:9-4-1-1}

在实现先加密后 MAC 时,一个常见的错误是为密码和 MAC 选用相同的密钥,即设置 $k_\mathrm{e}=k_\mathrm{m}$。由此产生的系统无法提供认证加密,而且可能是不安全的,如练习 \ref{exer:9-8} 中所展示的那样。在定理 \ref{theo:9-2} 的证明中,我们利用了这样一个事实,即两个密钥 $k_\mathrm{e}$ 和 $k_\mathrm{m}$ 的选择是相互独立的。

另一种常见的错误是只对密文的一部分应用 MAC 签名算法。我们看一个例子。假设底层的 CPA 安全密码 $\mathcal{E}=(E,D)$ 是由随机化 CBC 模式(见 \ref{subsec:5-4-3} 小节)构建的,那么消息 $m$ 的加密就是 $(r,c)\overset{\rm R}\leftarrow E(k,m)$,这里的 $r$ 是一个随机 IV。当实现先加密后 MAC $\mathcal{E}_\mathrm{EtM}=(E_\mathrm{EtM},D_\mathrm{EtM})$ 时,加密算法被错误地定义为:
\[
E_\mathrm{EtM}\big((k_\mathrm{e},k_\mathrm{m}),\;m\big)
:=
\big\{
(r,c)\overset{\rm R}\leftarrow E(k_\mathrm{e},m),
\;
t\overset{\rm R}\leftarrow S(k_\mathrm{m},c),
\;
\text{输出}\,(r,c,t)
\big\}
\]
这里,$E(k_\mathrm{e},m)$ 输出了密文 $(r,c)$,但 MAC 签名算法只被应用到了 $c$ 上,而 IV 没有受到 MAC 的保护。这个错误会完全破坏密文完整性:给定一条密文 $(r,c,t)$,攻击者可以创建另一条有效密文 $(r',c,t)$,而 $r'\neq r$。解密算法无法检测到这种对 IV 的修改,更不会输出 $\mathsf{reject}$。相反,解密算法会输出 $D\big(k_\mathrm{e},(r',c)\big)$。由于 $(r',c,t)$ 是一条有效密文,对手就赢得了密文完整性游戏。更糟糕的是,假如 $(r,c,t)$ 是对一条消息 $m$ 的加密,那么对于任意的 $\Delta$,将 $(r,c,t)$ 改为 $(r\oplus\Delta,c,t)$ 都会使得 CBC 解密算法输出一条消息 $m'$,而 $m'[0]=m[0]\oplus\Delta$。这意味着,攻击者可以将 $m$ 的第一个分组中的头部信息改为它自己所选择的任何值。ISO 19772 认证加密标准的一个早期版本正是犯了这个错误 \cite{namprempre2014reconsidering}。类似地,在 2013 年,有人发现苹果公司的 iOS 系统中为数据加密而建立的 \texttt{RNCryptor} 设施使用了一个错误的先加密后 MAC,其中的 HMAC 没有被应用到加密 IV 上 \cite{napier2013rncryptor}。

在实现中,另一个需要注意的隐患是,在整条消息的完整性标签都被验证完成之前,系统不应当输出任何明文数据。\ref{sec:9-9} 节会介绍这方面的一个例子。

\subsection{先 MAC 后加密一般是不安全的:SSL 上的填充预言机攻击}\label{subsec:9-4-2}

接下来,我们考虑由一个 CPA 安全密码和一个安全的 MAC 的构成的先 MAC 后加密通用组合。我们将表明,这种构造不一定是 AE 安全的,而且可能会导致许多现实世界中的问题。

为了准确定义先 MAC 后加密范式,令 $\mathcal{I}=(S,V)$ 是一个定义在 $(\mathcal{K}_\mathrm{m},\mathcal{M},\mathcal{T})$ 上的 MAC,$E=(E,D)$ 是一个定义在 $(\mathcal{K}_\mathrm{e},\mathcal{M}\times\mathcal{T},\mathcal{C})$ 上的密码。先 MAC 后加密系统 $\mathcal{E}_\mathrm{MtE}=(E_\mathrm{MtE},D_\mathrm{MtE})$,或简称 $\mathrm{MtE}$,定义如下:

\vspace*{10pt}

\hspace*{20pt} $E_\mathrm{MtE}\big((k_\mathrm{e},k_\mathrm{m}),\;m\big)$
				\quad $:=$ \quad
				$t\overset{\rm R}\leftarrow E(k_\mathrm{m},m)$,
				\quad
				$c\overset{\rm R}\leftarrow S\big(k_\mathrm{e},\,(m,t)\big)$\\
\hspace*{165.5pt} 输出 $c$

\vspace*{5pt}

\hspace*{24pt} $D_\mathrm{MtE}\big((k_\mathrm{e},k_\mathrm{m}),\;c\big)$
				\quad $:=$ \quad
				$(m,t)\leftarrow D(k_\mathrm{e},c)$\\
\hspace*{165.5pt} 如果 $V(k_\mathrm{m},m,t)=\mathsf{reject}$,则输出 $\mathsf{reject}$\\
\hspace*{165.5pt} 否则输出 $D(k_\mathrm{e},c)$

\vspace*{10pt}

\noindent
我们称 $\mathrm{MtE}$ 系统定义在 $(\mathcal{K}_\mathrm{e}\times\mathcal{K}_\mathrm{m},\;\mathcal{M},\;\mathcal{C})$。

\begin{snote}[一种被彻底破解的 $\mathbf{MtE}$ 密码。]
我们表明,就算 $\mathcal{E}$ 是一个 CPA 安全的密码,$\mathcal{I}$ 是一个安全的 MAC,$\mathrm{MtE}$ 也不一定是 AE 安全的。事实上,对于广泛使用的密码和 MAC 来说,$\mathrm{MtE}$ 很可能是不安全的,而这在事实上已经导致了很多针对已部署系统的重大攻击。

考虑一下用于保护 WWW 流量的 SSL 3.0 协议,该协议已经被使用了二十多年(但在现代浏览器中被禁用)。SSL 3.0 使用 $\mathrm{MtE}$ 来组合随机化 CBC 模式加密和安全 MAC。我们曾在第\ref{chap:5}章中表明,随机化 CBC 模式加密是 CPA 安全的,尽管如此,这种组合仍然被彻底破解了:攻击者可以使用选择密文攻击有效地解密所有流量。这就导致了一种针对 SSL 3.0 的破坏性攻击,称为 \textbf{POODLE} \cite{moller2014poodle}。

我们假设 CBC 中所使用的底层分组密码运行在 $16$ 字节分组上,就像 AES 那样。回顾一下,CBC 模式加密将其输入填充到分组长度的整数倍,而在 SSL 3.0 中,具体的做法如下:如果需要一个长度为 $p>0$ 字节的填充序列,该方案就用一个长为 $p-1$ 字节的任意序列来填充消息,并且增加再额外增加一个字节,该字节的值就是 $(p-1)$。如果消息长度已经是分组长度($16$ 字节)的整数倍,SSL 3.0 就会添加一个长为 $16$ 字节的假分组,其最后一个字节被置为 $15$,而其前 $15$ 字节是任意内容。在解密过程中,算法会读取最后一个字节,并移除其值相应的那么多字节,以正确地移除填充。

具体来说,将 $\mathrm{MtE}$ 应用于随机化 CBC 模式加密和安全 MAC 得到的密码 $\mathcal{E}_\mathrm{MtE}=(E_\mathrm{MtE},D_\mathrm{MtE})$ 工作如下:
\begin{itemize}
	\item $E_\mathrm{MtE}\big((k_\mathrm{e},k_\mathrm{m}),\,m\big)$:首先,使用 MAC 签名算法为 $m$ 计算一个定长标签 $t\overset{\rm R}\leftarrow E(k_\mathrm{m},m)$。然后,用随机化 CBC 加密对 $m\,\Vert\,t$ 进行加密:对信息进行填充,然后使用密钥 $k_\mathrm{e}$ 和一个随机的 IV 在 CBC 模式下进行加密。因此,下面的数据会被加密以生成密文 $c$:
	\begin{equation}\label{eq:9-8}
		\boxed{\qquad\qquad\text{消息}\; m \qquad\qquad}\boxed{\quad\text{标签}\; t \quad}\boxed{\quad\text{填充}\; p \quad}
	\end{equation}
	
	请注意,标签 $t$ 并不保护填充的完整性。我们将利用这一点,用选择密文攻击来打破 CPA 安全性。
	\item $D_\mathrm{MtE}\big((k_\mathrm{e},k_\mathrm{m}),\,c\big)$:运行 CBC 解密以获得式 \ref{eq:9-8} 中的明文数据。然后,读取式 \ref{eq:9-8} 中最后一字节,并从数据中移除与其值相等长度的字节(即,如果最后一字节的值是 $3$,就移除该字节,再额外移除 $3$ 个字节),以完全移除填充 $p$。最后验证 MAC 标签,如果有效,就返回剩余的字节作为消息,否则就输出 $\mathsf{reject}$。
\end{itemize}
SSL 3.0 和 TLS 1.0 都使用了随机化 CBC 加密的一种有缺陷的变体,我们曾在练习 \ref{exer:5-13} 中讨论过这个问题,但它与我们这里的讨论无关。在这里,我们假设所使用的随机化 CBC 加密的实现是正确的。
\end{snote}

\begin{snote}[选择密文攻击。]
我们展示一种针对 $\mathcal{E}_\mathrm{MtE}$ 系统的选择密文攻击,它能让对手解密由其挑选的任何密文。考虑到这种攻击的存在,即使底层密码是 CPA 安全的,$\mathcal{E}_\mathrm{MtE}$ 也不一定是 AE 安全的。接下来,我们用 $(E,D)$ 表示用在 CBC 加密中的底层分组密码,它作用于 $16$ 字节的数据分组。

假设对手截获了一条对应于未知消息 $m$ 的有效密文 $c:=E_\mathrm{MtE}\big((k_\mathrm{e},k_\mathrm{m}),\,m)$。$m$ 的长度满足如下条件,即当 MAC 标签 $t$ 被添加到 $m$ 之后,$(m\,\Vert\,t)$ 的长度是 $16$ 字节的整数倍。这就意味着,在 CBC 加密的过程中,有一个完整的 $16$ 字节填充分组被添加到了消息后,且这个填充分组的最后一个字节的值是 $15$。因此,密文看起来就像下面这样:
\[
	c\quad = \quad
	\underbrace{\boxed{\quad c[0]\quad}}_{\text{IV}}\!
	\underbrace{\boxed{\quad c[1]\quad}\;\;\;\cdots\quad}_{m\,\text{的加密}}
	\underbrace{\quad\boxed{\; c[\ell-1]\;}}_{\text{加密标签}}\!
	\underbrace{\boxed{\quad c[\ell]\quad}}_{\text{加密填充}}
\]

我们首先证明,对手能够学到一些关于 $m[0]$($m$ 的第一个 $16$ 字节分组)的知识。这将打破 $\mathcal{E}_\mathrm{MtE}$ 的语义安全性。攻击者用 $c[1]$ 替换 $c$ 的最后一个分组,以准备一个选择密文查询 $\hat{c}$。也就是说:
\begin{equation}\label{eq:9-9}
\hat{c}\quad := \quad
\boxed{\quad c[0]\quad}
\boxed{\quad c[1]\quad}
\quad\cdots\quad
\boxed{\; c[\ell-1]\;}\!
\underbrace{\boxed{\quad c[1]\quad}}_{\text{加密填充?}}
\end{equation}
根据 CBC 解密的定义,解密 $\hat{c}$ 的最后一个分组可以得到 $16$ 字节的明文分组:
\[
v:=D\left(k_\mathrm{e},c[1]\right)\oplus c[\ell-1]=m[0]\oplus c[0]\oplus c[\ell-1]
\]
如果 $v$ 最后一字节的值是 $15$,那么在解密过程中,最后一个分组会被视为一个填充分组而被删除。剩下的序列是一个有效的消息-标签对,能够被正确地解密。如果 $v$ 最后一字节的值不是 $15$,那么对解密查询的应答很有可能就是 $\mathsf{reject}$。

换言之,如果对 $\hat{c}$ 的解密查询的应答不是 $\mathsf{reject}$,攻击者就能知道,$m[0]$ 的最后一个字节就等于 $u:=15\oplus c[0]\oplus c[\ell-1]$ 的最后一个字节。否则,攻击者也能知道 $m[0]$ 的最后一个字节不等于 $u$ 的最后一个字节。这就直接打破了 $\mathcal{E}_\mathrm{MtE}$ 的语义安全性:攻击者能够获得一些关于明文 $m$ 的知识。

读者可以用选择密文攻击中的对手(就像在攻击游戏 \ref{game:9-2} 中那样)重述上述攻击,我们将它留作一个启发性的练习。只要通过单次明文查询和单次密文查询,对手就能以 $1/256$ 的优势赢得游戏。这就证明 $\mathcal{E}_\mathrm{MtE}$ 是不安全的。

现在,假设攻击者使用另一个 IV 获得了对 $m$ 的另一个加密 $c'$。攻击者可以用密文 $c$ 和 $c'$ 来构造四个有用的选择密文查询:它可以用 $c[1]$ 或 $c'[1]$ 替换 $c$ 或 $c'$ 的最后一个分组。攻击者发出这四个密文查询,就可以了解到 $m[0]$ 的最后一字节是否等于以下四个值:
\[
15\oplus c[0]\oplus c[\ell-1],\qquad
15\oplus c[0]\oplus c'[\ell-1],\qquad
15\oplus c'[0]\oplus c[\ell-1],\qquad
15\oplus c'[0]\oplus c'[\ell-1]
\]
中某一个的最后一字节。如果这四个值各不相同,他们就能给攻击者提供四次学习 $m[0]$ 的最后一字节的机会。使用对消息 $m$ 的新加密多次重复这一过程,攻击者很快就能确定 $m[0]$ 的最后一字节。每次选择密文查询都能以 $1/256$ 的概率确定该字节。因此,平均来说,只要进行 $256$ 次选择密文查询,攻击者就能知道 $m[0]$ 的最后一字节的确切值。所以,攻击者可以打破语义安全性,更具体地说,它可以恢复明文的一个字节。接下来,假设攻击者能够请求对 $m$ 右移一比特后明文的加密,得到一条密文 $c_1$。将 $c1[1]$ 插入上一阶段密文(即对未移位的 $m$ 的加密)的最后一个分组,然后发出选择密文查询,攻击者就能揭示 $m[0]$ 的倒数第二个字节。对 $m$ 的每个字节重复该过程,就能够揭示 $m$ 的全部信息。下面我们表明,这能够导出一种针对 SSL 3.0 的真实攻击。
\end{snote}

\begin{snote}[彻底攻破 SSL 3.0。]
选择密文攻击似乎只在理论上可行,但事实上,它们经常被转化为极具破坏性的现实世界攻击。考虑一个网络浏览器和一个名为 \texttt{bank.com} 的受害网络服务器。在两者之间交换的信息使用 SSL 3.0 加密。浏览器和服务器之间共享一个被称为 cookie 的秘密,浏览器在它每一个发往 \texttt{bank.com} 的请求中都嵌入了这个 cookie。抽象地讲,浏览器发往 \texttt{bank.com} 的请求看起来就像:
\[
\boxed{\;\text{GET}\;\texttt{path}\quad\text{cookie:}\;\texttt{cookie}\;}
\]
其中,\texttt{path} 指浏览器向 \texttt{bank.com} 请求的资源的标识符。浏览器只会在它向 \texttt{bank.com} 发出的请求中插入该 cookie。

攻击者的目标是恢复秘密的 cookie。首先,它让浏览器访问 \texttt{attacker.com},在那里,它会向浏览器发送一个 JavaScript 程序。这个程序会迫使浏览器请求 \texttt{bank.com} 的资源 \texttt{/AA}。之所以请求这个路径,是为了确保消息和 MAC 的长度是分组长度($16$ 字节)的倍数,这是攻击所需要的。因此,浏览器向 \texttt{bank.com} 发送以下请求:
\begin{equation}\label{eq:9-10}
\boxed{\;\text{GET}\;\texttt{/AA}\quad\text{cookie:}\;\texttt{cookie}\;}
\end{equation}
该请求会被 SSL 3.0 加密。攻击者可以截获这个加密请求 $c$,并对 MtE 发起选择密文攻击,以了解 cookie 的一个字节。也就是说,攻击者会像式 \ref{eq:9-9} 那样准备好一个 $\hat{c}$,将 $\hat{c}$ 发送给 \texttt{bank.com},并查看\texttt{bank.com} 是否应答一个 SSL 报错信息。如果没有产生报错信息,攻击者就能知道 cookie 的一个字节。JavaScript 程序可以迫使浏览器重复发出式 \ref{eq:9-10} 中的请求,以给攻击者提供它所需的新鲜密文,直到最终揭示 cookie 的一个字节。

一旦对手知道了 cookie 的一个字节,它就可以让 JavaScript 程序向 \texttt{bank.com} 发出请求:
\[
\boxed{\;\text{GET}\;\texttt{/AAA}\quad\text{cookie:}\;\texttt{cookie}\;}
\]
以将 cookie 右移一个字节。这就为攻击者提供了一新的密文分组,不妨将其称作 $c_1[2]$,其中 cookie 被右移了一个字节。重新向服务器发送上一阶段的请求,但现在最后一个分组被替换为 $c_1[2]$,直到揭示 cookie 的第二个字节。对 cookie 的每一个字节重复这一过程,最终就能够揭示整个 cookie。

实际上,浏览器中的 JavaScript 程序能够为攻击者发动选择明文攻击提供充足的工具。而拦截网络中的数据包,对其进行修改并观察服务器的响应,就能为攻击者发动选择密文攻击提供充分的信息。这两者的结合就能完全打破 SSL 3.0 的 MtE 加密。

一个小细节是,每当 \texttt{bank.com} 应答一个 SSL 报错信息,SSL 会话都会关闭。但这并不构成问题:每当浏览器中的 JavaScript 程序向 \texttt{bank.com} 发起一个新的请求,都会自动启动一个新的 SSL 会话。因此,每个选择密文查询都是在不同的会话密钥下加密的,但这对攻击来说没有任何区别:每个查询都会检验 cookie 的一个字节是否等于一个已知的随机字节。只要有足够的查询,攻击者就能了解整个 cookie。
\end{snote}

\subsection{其他填充预言机攻击}\label{subsec:9-4-3}

TLS 1.0 是 SSL 3.0 的一个更新版本。它在填充中添加新的结构(见 \ref{subsec:5-4-4} 小节),以此来防御上一小节中的攻击:当填充 $p$ 个字节时,填充中所有字节的内容都会被置为 $p-1$。此外,在解密过程中,解密者需要检查所有填充字节的值是否都是正确的,如果不然就拒绝密文。这使得攻击者难以发动上一小节中介绍的攻击。当然,我们的目标只是想要表明 MtE 一般来说是不安全的,而 SSL 3.0 已经充分说明了这一点。

\begin{snote}[一种填充预言机计时攻击。]
尽管 TLS 1.0 加入了新的防御措施,但是对 MtE 解密的简陋实现仍然可能遭受攻击。假设这种实现是这样工作的:首先,它使用 CBC 解密所收到的密文;然后,它检查填充结构是否有效,如果不然,它就拒绝该密文;反之,如果填充是有效的,它就检查完整性标签,如果标签也是有效的,它就返回明文。在这个实现中,只有填充结构有效时,完整性标签才会被检查。这意味着,如果有一条密文包含无效的填充结构,而另一条密文包含有效的填充,但其标签是无效的,那么前者会比后者更快被拒绝。攻击者可以测量服务器应答一次选择密文查询所需的时间,如果很快就返回了一条 TLS 错误消息,它就能知道填充结构是无效的。否则,它至少能了解到这个填充是有效的。

这种计时信道被称为\textbf{填充预言机边信道(padding oracle side-channel)}。就像我们在 SSL 3.0 中所做的那样,我们也可以根据这种行为设计一个选择密文攻击,来完全解密一个秘密 cookie,读者可以将其当作一种很好的练习。为了了解如何实现这种攻击,不妨假设一个攻击者截获了一条加密 TLS 1.0 记录 $c$。令 $m$ 是 $c$ 的解密。假设攻击者想要检验 $m[2]$ 的最后一字节是否等于某个固定值 $b$。攻击者创建一个新的密文分组 $\hat{c}[1]:=c[1]\oplus B$,并将包含三个分组的记录 $\hat{c}=(c[0],\hat{c}[1],c[2])$ 发送给服务器。在对 $\hat{c}$ 进行 CBC 解密后,最后一个明文分组将是:
\[
\hat{m}[2]
:=\hat{c}[1]\oplus D\big(k,c[2]\big)
=m[2]\oplus B
\]
如果 $m[2]$ 的最后一字节等于 $b$,$\hat{m}[2]$ 的最后一比特就是 $0$,而这是一个有效的填充。服务器会尝试验证完整性标签,而这就会导致响应缓慢。如果 $m[2]$ 的最后一字节不等于 $b$,$\hat{m}[2]$ 就不以 $0$ 结尾,并且这很可能是一个无效的填充,而服务器很快就会给出应答。通过测量响应时间,攻击者就可以了解到 $m[2]$ 的最后一字节是否等于 $b$。就像我们对 SSL 3.0 所做的那样,用多个选择密文查询来重复这一过程,就可以揭示整个秘密 cookie。

有一种和用在 TLS 1.0 中的攻击类似,但更加复杂的 MtE 填充预言机计时攻击,被称作 Lucky13 \cite{al2013lucky}。想要在实现 TLS 1.0 解密时针对 Lucky13 攻击隐藏计时信息是相当有挑战性的。
\end{snote}

\begin{snote}[信息性报错消息。]
更糟糕的是,TLS 1.0 规范 \cite{dierks1999rfc2246} 中规定,当收到的密文因 MAC 验证错误而被拒绝时,服务器应发送一种特定类型(被称作 \texttt{bad\_record\_mac})的报错消息;而当密文因填充分组无效而被拒绝时,服务器应发送另一种类型(被称作 \texttt{decryption\_failed})的报错消息。理论上,这就能告诉攻击者,一条密文被拒绝到底是因为填充分组无效,还是因为完整性标签被损坏。这就可以使上述选择密文攻击成为可能,而且不需要借助于计时信息。唯一幸运的是,报错消息是加密的,攻击者无法看到错误代码。

尽管如此,这里仍然有一个重要的教训:当解密失败时,系统决不应该解释原因。应该发送一个通用的``\texttt{decryption\_failed}"代码,而不提供任何其他信息。这个问题在 TLS 1.1 中得到了承认和解决。此外,当解密失败时,无论失败的原因是什么,正确的实现总是应该花费相同的时间提供应答。
\end{snote}

\subsection{安全的先 MAC 后加密实例}\label{subsec:9-4-4}

\begin{theorem}\label{theo:9-3}
	
\end{theorem}

\subsection{是先加密后MAC还是先MAC后加密?}\label{subsec:9-4-5}
\section{包含相关数据的基于nonce的认证加密}\label{sec:9-5}
\section{另一个变体:包含相关数据的CCA安全密码}\label{sec:9-6}
\section{案例研究:Galois 计数器模式 (GCM)}\label{sec:9-7}


Galois 计数器模式 (GCM) 是一种流行的基于 nonce 的 AEAD 密码,它于 2007 年被 NIST 标准化。GCM 是一个先加密后 MAC 密码,组合了一个 CPA 安全密码和一个安全的 MAC。这个 CPA 安全密码是基于 nonce 的计数器模式密码,通常使用 AES 实现。而安全 MAC 通常用一个 Carter-Wegman MAC 实现,它的底层是一个名为 GHASH 的带密钥哈希函数,后者是 \ref{sec:7-4} 节中 $H_\mathrm{xpoly}$ 函数的一个变体。当对空消息进行加密时,该密码就变成了一个称为 \textbf{GMAC} 的 MAC 系统,它能为关联数据提供完整性。

GCM 在底层使用了一个分组密码 $\mathcal{E}=(E,D)$,比如定义在 $(\mathcal{K},\mathcal{X})$ 上的 AES,其中 $\mathcal{X}:=\{0,1\}^{128}$。该分组密码会被用于计数器模式加密和 Carter-Wegman MAC。GHASH 函数定义在 $(\mathcal{X},\mathcal{X}^{\leq\ell},\mathcal{X})$ 上,其中 $\ell:=2^{32}-1$。

GCM 可以接受不同长度的 nonce,但我们先描述一个使用 $96$ 比特 nonce ${\scriptstyle\mathpzc{N}}$ 的 GCM,这是一个最简单的例子。GCM 的加密算法运行如下:

\vspace*{10pt}

\hspace*{5pt} 输入:密钥 $k\in\mathcal{K}$,消息 $m$,关联数据 $d$,以及 nonce ${\scriptstyle\mathpzc{N}}\in\{0,1\}^{96}$

\vspace*{5pt}

\hspace*{5pt} 令 $k_\mathrm{m}\leftarrow E(k,\,0^{128})$
\hspace*{150pt} // \quad \emph{首先生成 GHASH 密钥}

\vspace*{5pt}

\hspace*{5pt} 计算计数器模式加密中计数器的初始值:\\
\hspace*{50pt} 令 $x\leftarrow({\scriptstyle\mathpzc{N}}\,\Vert\,0^{31}1)\in\{0,1\}^{128}$\\
\hspace*{50pt} 令 $x'\leftarrow x+1$
\hspace*{152pt} // \quad \emph{计数器的初始值}

\vspace*{5pt}

\hspace*{5pt} 令 $c\leftarrow\{$ 令计数器从 $x'$ 开始,用计数器模式对 $m$ 进行加密的结果 $\}$\\
\hspace*{26pt} 令 $d'\leftarrow\{$ 用 $0$ 将 $d$ 填充到 $128$ 比特的最小整数倍 $\}$\\
\hspace*{26pt} 令 $c'\leftarrow\{$ 用 $0$ 将 $c$ 填充到 $128$ 比特的最小整数倍 $\}$

\vspace*{5pt}

\hspace*{5pt} 计算 Carter-Wegman MAC:\\
\hspace*{5pt} ($*$)
\hspace*{24.5pt} 令 $h\leftarrow\mathrm{GHASH}\Big(k_\mathrm{m},\;\big(d'\,\Vert\,c'\,\Vert\,\mathrm{length}(d)\,\Vert\,\mathrm{length}(c)\big)\Big)\quad\in\{0,1\}^{128}$\\
\hspace*{50pt} 令 $t\leftarrow h\oplus E(k,x)\quad\in\{0,1\}^{128}$

\vspace*{5pt}

\hspace*{5pt} 输出 $(c,t)$
\hspace*{193pt} // \quad \emph{先加密后 MAC 的密文}

\vspace*{10pt}

第 ($*$) 行中的每个 $\mathrm{length}$ 字段都是一个 $64$ 比特的值,表示各自字段的长度(以字节计)。如果输入 nonce ${\scriptstyle\mathpzc{N}}$ 的长度不是 $96$ 比特,它就会被填充到 $128$ 的最小整数倍,产生一个填充后的序列 ${\scriptstyle\mathpzc{N}}'$。此外,计数器的初始值 $x$ 由 $x\leftarrow\mathrm{GHASH}\big(k_\mathrm{m},({\scriptstyle\mathpzc{N}}'\,\Vert\,\mathrm{length}({\scriptstyle\mathpzc{N}}))\big)$ 产生,这是一个 $\{0,1\}^{128}$ 上的值。

和之前一样,完整性标签 $t$ 可以被截断成任何所需的长度。标签 $t$ 越短,系统就越容易受到密文完整性攻击。

被加密的消息必须短于 $2^{32}$ 个分组(即消息必须在 $\mathcal{X}_v$ 中,其中 $v<2^{32}$)。标准中建议,不要使用相同的密钥 $k$ 加密超过 $2^{32}$ 条消息。

GCM 解密算法的输入包括一个密钥 $k\in\mathcal{K}$,一条密文 $(c,t)$,相关数据 $d$,以及一个 nonce ${\scriptstyle\mathpzc{N}}$。它的操作与先加密后 MAC 是一样的:它先计算 $k_\mathrm{m}\leftarrow E(k,0^{128})$,然后检查 Carter-Wegman 完整性标签 $t$。如果标签是有效的,它就输出对 $c$ 的计数器模式解密。我们强调,解密必须是原子性的:在整条消息全部通过完整性标签的验证之前,不应当输出任何明文数据。

\begin{snote}[GHASH。]
剩下的工作就是描述定义在 $(\mathcal{X},\mathcal{X}^{\leq\ell},\mathcal{X})$ 上的带密钥哈希函数 GHASH。这个哈希函数被用于 Carter-Wegman MAC 中,因此安全起见,它必须是一个 DUF。我们曾在 \ref{sec:7-4} 节中表明,函数 $H_\mathrm{xpoly}$ 是一个 DUF,而 GHASH 本质上和它是一样的。回顾一下,$H_\mathrm{xpoly}(k,z)$ 的工作原理是在 $k$ 点处评估一个由 $z$ 派生而来的多项式。我们在描述 $H_\mathrm{xpoly}$ 时使用了模素数 $p$ 的算术运算,所以 $z$ 的两个分组和输出都应当是 $\mathbb{Z}_p$ 上的元素。

哈希函数 GHASH 与 $H_\mathrm{xpoly}$ 几乎完全相同,只是输入消息分组和输出都是 $\{0,1\}^{128}$ 上的元素。另外,在 GHASH 中,DUF 属性对异或运算符 $\oplus$(而不是模算术减法)成立。正如我们在备注 \ref{remark:7-4} 中所讨论的,为了建立一个 XOR-DUF,我们可以使用定义在有限域 $\mathrm{GF}(2^{128})$ 上的多项式。这是一个由 $2^{128}$ 个元素组成的域,称为 \textbf{Galois 域},这也是 GCM 名字的来源。这个域由既约多项式 $g(X):=X^{128}+X^7+X^2+X+1$ 定义。$\mathrm{GF}(2^{128})$ 的元素是 $\mathrm{GF}(2)$ 上的小于 $128$ 阶的多项式,其算术运算都以 $g(X)$ 为模。虽然这听起来好像花哨,但 $\mathrm{GF}(2^{128})$ 中的所有元素都可以很方便地用一个 $128$ 比特序列表示(每一比特编码多项式的一个系数)。域中的加法就是异或计算,而乘法可能稍微复杂一点,但仍然不是太难(见下文——许多现代计算机都提供了直接的硬件支持)。

有了上述符号,对于 $k\in\mathrm{GF}(2^{128})$ 和 $z\in\big(\mathrm{GF}(2^{128})\big)^v$,函数 $\mathrm{GHASH}(k,z)$ 就是 $\mathrm{GF}(2^{128})$ 上的多项式评估:
\begin{equation}\label{eq:9-18}
\mathrm{GHASH}(k,z):=
z[0]k^v+z[1]k^{v-1}+\cdots+z[v-1]k
\in\mathrm{GF}(2^{128})
\end{equation}
就是这样。在第($*$)行中,在 GHASH 的输入中加入两个 $\mathrm{length}$ 字段可以确保,即使对于不同长度的消息,XOR-DUF 属性也能得到保证。
\end{snote}

\begin{snote}[安全性。]
GCM 的 AEAD 安全性与我们对先加密后 MAC 通用组合的分析(定理 \ref{theo:9-4})类似,可以从作为 PRF 的底层分组密码的安全性得到。GCM 与我们的通用组合的主要区别在于,GCM 在涉及到密钥时``走了一些弯路":它只使用了一个密钥 $k$,并使用 $E(k,0^n)$ 作为 GHASH 的密钥,并使用 $E(k,x)$ 作为掩盖 GHASH 输出的填充,这与 Carter-Wegman 的做法类似,但不完全相同。重要的是,计数器模式加密是从计数器值 $x':=x+1$ 开始的,因此我们可以确保,用于加密消息的 PRF 输入与用于派生 GHASH 密钥和填充的输入是不同的。上面的讨论聚焦于 nonce 是 $96$ 比特的情况。另一种情况,即 GHASH 被应用于 nonce 中以计算 $x$,需要更多的分析——见练习 \ref{exer:9-14}。

GCM 没有抵抗 nonce 重复使用的能力。如果一个 nonce 意外地被用在了两条不同的消息上,这些消息的所有机密性都会丧失掉。更糟的是,GHASH 的密钥 $k_\mathrm{m}$ 也会被暴露(见练习 \ref{exer:7-13}),而这就可以被用来破坏密文完整性。因此在 GCM 中,至关重要的一件事就是不重复使用 nonce 值。
\end{snote}

\begin{snote}[优化和性能。]
有很多方法可以对 GCM 和 GHASH 的实现进行优化。在实践中,我们通常使用 Horner 方法评估式 \ref{eq:9-18} 中的多项式,因此,处理每个明文分组都只需要在 $\mathrm{GF}(2^{128})$ 上做一次加法和一次乘法。

最近,英特尔在其指令集中增加了一条特殊指令(称为 \texttt{PCLMULQDQ})以快速进行二元多项式乘法。这条指令不能直接被用来实现 GHASH,因为它与 $\mathrm{GF}(2^{128})$ 中元素的标准表示方式不兼容。幸运的是,Gueron 的工作展示了如何克服这些困难并使用 \texttt{PCLMULQDQ} 指令来加速英特尔平台上的 GHASH 的方法。

由于 GHASH 对每个分组只需要在 $\mathrm{GF}(2^{128})$ 上做一次加法和一次乘法,所以有人可能会认为,GCM 加解密过程中的大部分时间都花在了计数器模式的 AES 上。然而,由于 AES 硬件实现的改进,特别是 AES-NI 指令的流水线化,情况并非总是如此。在英特尔(于 2013 年推出)的 Haswell 处理器上,由于 GHASH 的额外开销,GCM 会比纯计数器模式慢三倍左右。然而,即将到来的 \texttt{PCLMULQDQ} 实现优化可能会使 GCM 只比纯计数器模式稍微昂贵一点点,这是人们所能期望的最好结果。

我们应该指出,目前已经有可能实现成本并不显著高于 AES 计数器模式的安全认证加密——这可以通过 OCB 这样的集成方案实现(见练习 \ref{exer:9-17})。
\end{snote}
\section{案例研究:TLS 1.3 记录协议}\label{sec:9-8}

传输层安全 (Transport Layer Security, TLS) 协议是迄今为止部署最广泛的安全协议。几乎所有的网上购物都受到 TLS 的保护。尽管 TLS 主要用于保护网络流量,但它其实是一个通用的协议,可以用来保护许多其他类型的流量,比如电子邮件、即时通信等等。

TLS 的原始版本由网景 (Netscape) 设计,它在当时被称作安全套接字层协议 (Secure Socket Layer protocol, SSL)。SSL 2.0 于 1994 年问世,用于保护网络电商流量。1995 年提出的 SSL 3.0 纠正了 SSLv2 的几个重大安全问题。例如,SSL 2.0 对密码和 MAC 使用相同的密钥。这是一种非常不好的做法,因为它会使对 MtE 和 EtM 的安全证明失效;与此同时,它还意味着,如果我们使用一个弱的密码密钥(比方说由于出口限制),那么 MAC 密钥也必须是弱的。SSL 2.0 只支持少量的算法,具体地说,它只支持基于 MD5 的 MAC。

此后,互联网工程任务组 (Internet Engineering Task Force, IETF) 创建了一个传输层安全工作组,目标是标准化一个类 SSL 协议。该工作组于 1999 年制定了 TLS 1.0 协议规范 \cite{dierks1999rfc2246}。TLS 1.0 其实只是 SSL 3.0 的一个小升级,因此也常被称作 SSL 3.1。在此之后,TLS 又先后在 2006 年和 2008 年引入了若干小的更新,这将其升级到了 TLS 1.2 版本。TLS 1.2 由于其中存在的几个安全漏洞,于 2017 年被推翻,被一个更强的 TLS 1.3 取代。现在,TLS 已经变得无处不在,它被用于全世界的许多软件工程中。在本节中,我们主要关注 TLS 1.3 版本。

\begin{snote}[TLS 1.3 记录协议。]
抽象地讲,TLS 由两个部分组成。第一部分被称为 \textbf{TLS 会话设置 (TLS session setup)},它负责协商即将用于加密会话的密码套件,然后在浏览器和服务器之间建立一个共享秘密。第二部分被称为 \textbf{TLS 记录协议 (TLS record protocol)},它使用上述共享秘密在双方之间安全地传输数据。TLS 会话设置会使用公钥密码学技术,我们会在第\ref{chap:21}章讨论这个部分。在本节中,我们主要关注 TLS 记录协议。

在 TLS 术语中,会话建立过程中所产生的共享秘密被称为\textbf{主秘密 (master-secret)}。这个高熵的主秘密被用于推导出两个密钥 $k_{b\to s}$ 和 $k_{s\to b}$。密钥 $k_{b\to s}$ 负责加密从浏览器到服务器的消息,而 $k_{s\to b}$ 则负责反方向的通信。TLS 使用主秘密和其他随机元作为种子来派生这两个密钥,这个密钥派生函数被称为 HKDF(见 \ref{subsec:8-10-5} 小节),它能够派生出足够多的伪随机比特,用来构造这两个密钥。这一步骤由浏览器和服务器共同完成,因此,当协商结束时,双方都将持持有密钥 $k_{b\to s}$ 和 $k_{s\to b}$。

TLS 记录协议以记录形式发送数据,每条记录最长为 $2^{14}$ 字节。如果一方需要传输超过 $2^{14}$ 字节的数据,记录协议就会将数据分割成多个记录,每个的长度都不长于 $2^{14}$ 字节。每一通信方都需要维护一个 $64$ 比特的\textbf{写序列号 (write sequence number)},它的初始值是 $0$,每发送一条记录,它就会递增 $1$。

TLS 1.3 使用一个基于 nonce 的 AEAD 密码 $(E,D)$ 来加密所有记录。具体选择哪一个基于 nonce 的 AEAD 密码,需要由通信双方在 TLS 会话设置期间协商决定。AEAD 加密算法会被赋予以下参数:
\begin{itemize}
	\item 密钥:$k_{b\to s}$ 或 $k_{s\to b}$,取决于正在运行加密的是浏览器还是服务器。
	\item 明文数据:最长 $2^{14}$ 字节。
	\item 关联数据:空(长度为 $0$)。
	\item nonce($8$ 字节或更长):nonce 的计算方法是:(1) 在加密方的 $64$ 比特写序列号左侧填充 $0$,以达到预期的 nonce长度;(2) 将这个填充后的序列号与一个随机序列(称为 \texttt{client\_write\_iv} 或 \texttt{server\_write\_iv},取决于谁正在加密)进行异或,该随机序列是在会话设置期间从主秘密中得到的,并且在会话期间是固定的。TLS 1.3 本可以使用一个效果相同但更容易理解的方法:随机选择一个初始 nonce 值,然后为每条记录依次递增。但 TLS 1.3 所使用的方法相比来说更容易实现。
\end{itemize}

AEAD 密码会输出一条密文 $c$,它会被格式化成一条加密 TLS 记录,其格式如下:
\begin{center}
\tikzset{every picture/.style={line width=0.5pt}}

\begin{tikzpicture}[x=0.75pt,y=0.75pt,yscale=-1,xscale=1]

\draw   (0,0) -- (40,0) -- (40,20) -- (0,20) -- cycle ;
\draw   (40,0) -- (110,0) -- (110,20) -- (40,20) -- cycle ;
\draw   (110,0) -- (180,0) -- (180,20) -- (110,20) -- cycle ;
\draw   (180,0) -- (360,0) -- (360,20) -- (180,20) -- cycle ;

\draw (20,11) node   [align=left] {\texttt{type}};
\draw (75,9.5) node   [align=left] {\texttt{version}};
\draw (145,11) node   [align=left] {\texttt{length}};
\draw (270,10) node   [align=left] {密文 $c$};

\end{tikzpicture}
\end{center}
其中,\texttt{type} 是一个 $1$ 字节的记录类型(握手记录或应用数据记录),\texttt{version} 是一个历史遗留的 $2$ 字节字段,总是会被置为 \texttt{0301},\texttt{length} 是一个 $2$ 字节字段,表示 $c$ 的长度,而 $c$ 是密文本身。类型、版本和长度字段都以透明方式发送。请注意,nonce 不是加密 TLS 记录的一部分。接收方会自行计算 nonce。

为什么我们需要将 nonce 的初始值置为一个随机数,而不是简单地将其置为 $0$?原因在于,在网络协议中,通过 TLS 发送的第一个消息分组通常是一个固定的公共值。如果 nonce 被置为 $0$,那么第一条密文就是用 $c_0\leftarrow E(k,m_0,d,0)$ 计算得到的,而对手已经知道了 $m_0$ 和关联数据 $d$。这就意味着,对手可以针对密钥 $k$ 进行\emph{时空权衡}的穷举搜索的攻击,后者将会在 \ref{sec:18-7} 节中讨论。这种攻击表明,依靠大量的预计算和足够的存储,攻击者能以不可忽略不计的优势迅速地从 $c_0$ 中恢复 $k$——对于 $128$ 比特的密钥,这种攻击可能在不远的将来变得可行。将初始 nonce 随机化,就可以使 TLS ``在未来也能"免受此类攻击。

当收到一条记录时,接收方运行 AEAD 解密算法来解密 $c$。如果解密的结果是 $\mathsf{reject}$,接收方就会向对方发送一个 \texttt{bad\_record\_mac} 致命警报,然后关闭 TLS 会话。
\end{snote}

\begin{snote}[长度字段。]
和早期版本的 TLS 一样,在 TLS 1.3 中,记录的长度是以透明方式发送的。一些基于流量分析的攻击会利用记录长度来推断关于记录内容的信息。比如说,如果一条加密 TLS 记录中包含了两张不同大小的图片其中的某一张,长度就会向窃听者泄露加密的是哪张图片。Chen 等人表明,加密记录的长度可以揭示用户提供给云应用程序的私人数据的大量信息 \cite{chen2010side}。他们以一个在线报税系统作为例子。还有一些工作对许多其他系统发起了类似的攻击。由于尚无针对该问题的完整解决方案,它因此经常被大众忽视。

当加密一条 TLS 记录时,长度字段并不是关联数据的一部分,因此其完整性无法得到保护。原因在于,由于变长填充的存在,在加密算法终止之前,我们可能无法得知 $c$ 的长度。因此,我们无法将长度作为加密算法的输入。但这并不影响安全性:一个安全的 AEAD 密码会拒绝一条长度字段被篡改的密文。
\end{snote}

\begin{snote}[防止重放。]
攻击者可能会试图重放以前的记录,以促使接收方出现错误的举动。例如,攻击者可以简单地重放包含采购订单的记录,以试图让接收方重复处理同一张采购订单。TLS 使用 $64$ 比特的写序列号来拒绝这种重复的数据包。TLS 假定记录是按顺序传递的,这样,不需要记录中的任何额外信息,接收方也能对即将收到的序列号有一个正确的预期。重复或者失序的记录都会被丢弃,因为此时,AEAD 解密算法会以错误的 nonce 为输入,而这将导致它最终拒绝密文。
\end{snote}

\begin{snote}[cookie 切割机攻击。]
TLS 提供了一个流式接口,在这种接口中,记录一准备好就会被发送出去。虽然对记录的重放、重排和流中的删除都会被 $64$ 比特的序列号组织,但是对流中\emph{最后一条}记录的删除却没有任何防御措施。特别是,一个主动攻击者可以在一个会话的中途关闭网络连接,对另一个参与方来说,这看起来就像是对话正常结束了一样。这可能会导致一种被称为 \textbf{cookie 切割机}的现实世界攻击。为了说明这种攻击是如何工作的,考虑一个受害网站和一个受害网络浏览器。受害浏览器访问了一个恶意网站,该网站会指示浏览器连接到 \texttt{victim.com}。假设受害网站的加密响应看起来是这样的:

\vspace*{10pt}

\hspace*{29pt} \texttt{HTTP/1.1 302 Redirect}\\
\hspace*{50pt} \texttt{Location: http://victim.com/path}\\
\hspace*{50pt} \texttt{Set-Cookie: SID=[AuthenticationToken]; secure}\\
\hspace*{50pt} \texttt{Content-Length: 0\;\;\;\textbackslash r\textbackslash n\textbackslash r\textbackslash n}
         
\vspace*{10pt}
        
\noindent
前两行表明了响应的类型。注意,第二行包含一个从浏览器请求中复制的``\texttt{path}"值。第三行设置了一个将被储存在浏览器上的 cookie。在这里,``\texttt{secure}"属性表明,这个 cookie 只能通过一个加密 TLS 会话被发送到 \texttt{victim.com}。第四行表示响应结束。

假设在最初的浏览器请求中,``\texttt{path}"值相当长,那么服务器的响应就会被切分到两个 TLS 帧中:

\vspace*{10pt}

\hspace*{1pt} \texttt{frame 1:}
\hspace*{10.5pt} \texttt{HTTP/1.1 302 Redirect}\\
\hspace*{85pt} \texttt{Location: http://victim.com/path}\\
\hspace*{85pt} \texttt{Set-Cookie: SID=[AuthenticationToken]}
         
\vspace*{10pt}

\hspace*{1pt} \texttt{frame 2:}
\hspace*{10.5pt} \texttt{; secure}\\
\hspace*{85pt} \texttt{Content-Length: 0\;\;\;\textbackslash r\textbackslash n\textbackslash r\textbackslash n}

\vspace*{10pt}

\noindent
网络攻击者在发送第一帧后关闭了连接,因此第二帧永远不会到达浏览器。这导致浏览器将 cookie 标记为非安全的。现在,攻击者将浏览器引向 \texttt{victim.com} 的明文 (http) 版本,浏览器就会以明文方式发送 SID cookie,攻击者就可以轻易地获取到它。

实际上,对手能够让浏览器收到并不是由服务器发出的消息:服务器发送了两个帧,但浏览器只收到了其中的一个,并将其作为有效消息接受。尽管每一帧都用上了适当的认证加密,但这种攻击仍然可能发生。

TLS 假设应用层会防御这种攻击。特别是,服务器响应当以一个形如 \texttt{\textbackslash r\textbackslash n\textbackslash r\textbackslash n} 的消息结束标志 (end-of-message, EOM) 来标记结束。浏览器在看到 EOM 之前不应当处理传入的消息。但是在实践中,浏览器总是会尝试一收到报头就开始处理请求,这就会导致上述漏洞。每个使用 TLS 的应用程序都必须意识到这个问题,并使用 EOM 或类似的机制来防御它。
\end{snote}
\section{针对 SSH 中非原子性解密的一种攻击}
\section{案例研究:802.11b WEP,一个千疮百孔的系统}\label{sec:9-10}

于 1999 年获批的 IEEE 802.11b 标准为短距离无线通信 (WiFi) 提供了一个协议。其安全性由 802.11b 数据帧封装的有线等效加密 (Wired Equivalent Privacy, WEP) 协议提供。WEP 的设计目标是提供能够媲美有线网络的数据隐私性。然而,WEP 在这方面彻头彻尾地失败了,并且它为我们提供了一个极好的案例,展示了一个孱弱的设计是如何导致灾难性的后果的。

当启用 WEP 时,无线网络中的所有成员都会共享一个长期密钥 $k$。该标准支持 $40$ 或 $128$ 比特密钥。$40$ 比特的版本符合标准起草时美国的出口限制。我们将使用以下符号来描述 WEP:
\begin{itemize}
	\item WEP 加密使用 RC4 流密码。我们用 $\mathrm{RC4}(s)$ 表示当给定种子 $s$ 时,RC4 所产生的伪随机序列。
	\item 我们用 $CRC(m)$ 表示消息 $m\in\{0,1\}^*$ 的 $32$ 比特 CRC 校验和。CRC 的实现细节与我们这里的讨论无关,读者只需将 CRC 视为从任意比特序列映射到 $\{0,1\}^{32}$ 的某个特定函数。
\end{itemize}

令 $m$ 是一个 802.11b 明文帧。$m$ 的前几个比特编码了 $m$ 的长度。为了加密一个 802.11b 帧 $m$,发送者选择一个 $24$ 比特的 IV,并计算:
\[
\begin{aligned}
& c\leftarrow\big(m\,\Vert\,\mathrm{CRC}(m)\big)\oplus\mathrm{RC4}(\mathrm{IV}\,\Vert\,k)\\
& c_\mathrm{full}\leftarrow(\mathrm{IV},\,c)
\end{aligned}
\]
WEP 的加密过程如图 \ref{fig:9-4} 所示。接收者首先计算 $c\oplus\mathrm{RC4}(\mathrm{IV}\,\Vert\,k)$ 来解密得到数对 $(m,s)$。如果 $s=\mathrm{CRC}(m)$,接收方就接受该帧,否则就拒绝它。

\begin{snote}[攻击 1:IV 碰撞。]
WEP 的设计者明白,流密码的密钥不应该被重复使用。因此,他们使用一个 $24$ 比特的 IV 来派生出每一帧的密钥 $k_\mathrm{f}:=\mathrm{IV}\,\Vert\,k$。然而,该标准并没有规定如何选择 IV,而许多实现都做得不够好。我们说,每当一个无线基站碰巧发送了两个帧,例如第 $i$ 帧和第 $j$ 帧,而它们都是由相同的 IV 加密得到的,IV 碰撞就会发生。由于 IV 是以透明方式发送的,所以窃听者很容易发现 IV 碰撞。此外,一旦 IV 碰撞发生,攻击者就可以使用 \ref{subsec:3-3-1} 小节中讨论的两次性密码本攻击来解密 $i$ 和 $j$ 这两个帧。

那么,IV 碰撞发生的概率到底有多大?根据生日悖论,如果为每一帧选择一个随机 IV,那么预期每隔 $\sqrt{2^{24}}=2^{12}=4096$ 帧,就会发生一次 IV 碰撞。由于每帧最长是 $1156$ 字节,那么平均每传输大约 4 MB 的数据,就会发生一次碰撞。

除此之外,我们也可以使用一个计数器来生成 IV。该实现会在发送 $2^{24}$ 帧后耗尽整个 IV 空间,对于一个满负荷工作的无线接入点来说,这只需要大约一天时间。更糟糕的是,一些使用计数器方法的无线网卡在开机时会将计数器重置为 $0$。因此,这些网卡会经常重复使用较小的 IV,而这会让流量极易遭受两次性密码本攻击。
\end{snote}

\begin{figure}
  \centering
  \tikzset{every picture/.style={line width=0.75pt}}

\begin{tikzpicture}[x=0.75pt,y=0.75pt,yscale=-1,xscale=1]


\draw   (70,0) -- (390,0) -- (390,40) -- (70,40) -- cycle ;
\draw   (390,0) -- (470,0) -- (470,40) -- (390,40) -- cycle ;
\draw   (70,55) -- (470,55) -- (470,95) -- (70,95) -- cycle ;
\draw   (20,125) -- (65,125) -- (65,165) -- (20,165) -- cycle ;
\draw  [fill={rgb, 255:red, 155; green, 155; blue, 155 }  ,fill opacity=0.5 ] (70,125) -- (470,125) -- (470,165) -- (70,165) -- cycle ;

\draw [line width=2.5]    (60,110) -- (480,110) ;

\draw   (40,48) .. controls (40,44.69) and (42.69,42) .. (46,42) .. controls (49.31,42) and (52,44.69) .. (52,48) .. controls (52,51.31) and (49.31,54) .. (46,54) .. controls (42.69,54) and (40,51.31) .. (40,48) -- cycle ;
\draw   (40,48) -- (52,48) ;
\draw   (46,42) -- (46,54) ;

\draw (220,20) node   [align=left] {明文载荷 $m$};
\draw (270,145) node   [align=left] {加密帧};
\draw (42.5,145) node   [align=left] {IV};
\draw (430,20) node    {$\mathrm{CRC}(m)$};
\draw (270,75) node    {$\mathrm{RC4}(\mathrm{IV}\,\Vert\,k)$};


\end{tikzpicture}
  \caption{WEP 加密}
  \label{fig:9-4}
\end{figure}

\begin{snote}[攻击 2:相关密钥。]
对 WEP 加密的一种更具破坏性的攻击来自对相关 RC4 密钥的使用。在第\ref{chap:3}章中,我们曾经解释,必须为每条加密消息选择一个新的、\emph{随机的}流密码密钥。然而,WEP 所使用的密钥 $1\,\Vert\,k$,$2\,\Vert\,k$,$\dots$ 都是密切相关的——它们都有相同的后缀 $k$。RC4 从来就不是为这种目的而设计的,事实上,它在这些情况下是完全不安全的。Fluhrer、Mantin 和 Shamir 表明,在发送了大约一百万个 WEP 帧后,窃听者就可以恢复整个长期密钥 $k$ \cite{fluhrer2001weaknesses}。该攻击由 Stubblefield、Ioannidis 和 Rubin 实现 \cite{stubblefield2004key},现在可被用于各种黑客工具,如 \textsc{WepCrack} 和 \textsc{AirSnort}。

应该使用一个 PRF 来生成每个帧的帧密钥,比如说将第 $i$ 帧的密钥设置为 $k_i:=F(k,\mathrm{IV})$——所产生的密钥和随机独立的密钥是无法区分的。当然,虽然这种方法确实可以防止相关密钥问题,但无法解决上面讨论的 IV 碰撞问题,以及接下来讨论的可塑性问题。
\end{snote}

\begin{snote}[攻击 3:易被控制性。]
回顾一下,WEP 试图使用 CRC 校验来为认证加密提供完整性。在某种意义上,WEP 使用的是先 MAC 后加密的方法,但它使用的是 CRC,而不是 MAC。我们表明,尽管包含加密步骤,但这种构造其实完全无法提供密文完整性。

针对它的攻击利用了 CRC 的线性特性。也就是说,对于某条消息 $m$,给定 $\mathrm{CRC}(m)$,我们很容易为任何的 $\Delta$ 计算出 $\mathrm{CRC}(m\oplus\delta)$。更确切地说,存在一个公共函数 $L$,对于任何的 $m$ 和 $\Delta\in\{0,1\}^\ell$,我们都有:
\[
\mathrm{CRC}(m\oplus\Delta)=\mathrm{CRC}(m)\oplus L(\Delta)
\]
这一属性使攻击者可以对 WEP 密文进行任意的修改,而不会被接收方发现。令 $c$ 是一条 WEP 密文,即:
\[
c=\big(m,\mathrm{CRC}(m)\big)\oplus\mathrm{RC4}(\mathrm{IV}\,\Vert\,k)
\]
因此,$c'$ 可以完全无误地被解密为 $m\oplus\Delta$。我们可以发现,给定对 $m$ 的加密,攻击者就可以为它所选择的任何一个 $\Delta$ 创建一个对 $m\oplus\Delta$ 的有效加密。我们已经在 \ref{subsec:3-3-2} 小节解释过,这很有可能会导致严重的攻击。
\end{snote}

\begin{snote}[攻击4:选择的密码文本攻击。]
该协议还容易受到一种被称为\textbf{断续攻击 (chop-chop)}的选择密文攻击,这种攻击可以让攻击者解密由其选择的加密帧。我们会在练习 \ref{exer:9-5} 中描述这种攻击的一个简单版本。
\end{snote}

\vspace*{-10pt}

\begin{snote}[攻击5: 拒绝服务。]
我们简单提一下,802.11b 曾遭受过许多严重的拒绝服务 (Denial of Service, DoS) 攻击。比如说,在 802.11b 中,一旦客户端结束了网络调用,无线客户端就会向无线基站发送一条``断联"的消息。这是为了让无线基站能够释放分配给该客户端的内存资源。不幸的是,``断联"信息不会经过任何认证,也就是说,任何人都可以代表其他人发送断联消息。一旦断联,受害者将需要几秒钟才能和基站重新建立连接。因此,只要每隔几秒钟就向基站发送一次``断联"消息,攻击者就可以阻止任何由其选定的计算机连接到无线网络中。一些 802.11b 工具,比如 \texttt{Void11},就实现了这种攻击。
\end{snote}

\begin{snote}[802.11i。]
在 802.11b WEP 协议失败后,一个名为 802.11i 的新标准在 2004 年得到批准。802.11i 使用一种称作 CCM 的先加密后 MAC 模式来提供认证加密。具体地说,CCM 使用(原生)CBC-MAC 进行 MAC,并使用计数器模式进行加密。802.11i 使用 AES 作为底层的 PRF 来实现这两者。后来,CCM 被 NIST 采纳为联邦标准 \cite{dworkin2004sp}。
\end{snote}
\section{案例研究:IPsec}\label{sec:9-11}
\section{一个有趣的应用:隐私信息检索}
\section{笔记}
\section{练习}