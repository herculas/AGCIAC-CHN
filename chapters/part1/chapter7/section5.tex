\section{基于nonce的MAC}\label{sec:7-5}

在Carter-Wegman构造(见 \ref{sec:7-4} 节)中,我们只要求随机元满足一个基本属性,即它们是各不相同的。这就促使我们研究基于 nonce 的MAC,它其实就是对基于 nonce 的加密(见 \ref{sec:5-5} 节)的一种类比。这种方法不仅可以减少标签的大小,还可以提高安全性。

\textbf{基于 nonce 的 MAC} 与普通的 MAC 类似,由一对\emph{确定性}算法 $S$ 和 $V$ 组成,分别用于签署和验证标签。然而,这些算法还需要一个额外的输入 ${\scriptstyle\mathpzc{N}}$,称为 nonce,它位于一个 \textbf{nonce 空间} $\mathpzc{N}$ 中。算法 $S$ 和 $V$ 的工作方式如下:
\begin{itemize}
	\item $S$ 接受一个密钥 $k\in\mathcal{K}$,一条消息 $m\in\mathcal{M}$ 和一个 nonce ${\scriptstyle\mathpzc{N}}\in\mathpzc{N}$ 作为输入,输出一个标签 $t\in\mathcal{T}$。
	\item $V$ 接受四个值 $k,m,t,{\scriptstyle\mathpzc{N}}$ 作为输入,其中 $k$ 是一个密钥,$m$ 是一条消息,$t$ 是一个标签,而 ${\scriptstyle\mathpzc{N}}$ 是一个 nonce。它输出 $\mathsf{accept}$ 或 $\mathsf{reject}$。
\end{itemize}
我们称基于 nonce 的 MAC 定义在 $(\mathcal{K},\mathcal{M},\mathcal{T},\mathpzc{N})$ 上。和之前一样,我们要求,\emph{只要 $S$ 和 $V$ 被赋予相同的 nonce},那么由 $S$ 生成的标签总是会被 $V$ 接受。MAC 必须满足下面所述的\textbf{正确性属性}:对于所有的密钥 $k$,所有的消息 $m$ 以及所有的 nonce ${\scriptstyle\mathpzc{N}}\in\mathpzc{N}$,都有:
\[
\Pr\big[
V(k,\,m,\,S(k,m,{\scriptstyle\mathpzc{N}}),\,{\scriptstyle\mathpzc{N}})=\mathsf{accept}
\big]
=1
\]

正如 \ref{sec:5-5} 节所述,为了确保安全性,发送方应避免(在同一密钥上)使用两次相同的 nonce。如果发送方能够维护状态,那么它就可以用一个简单的计数器来实现 nonce。此外,它也可以随机选择 nonce,只要 nonce 空间足够大,它就可以确保两次产生相同 nonce 的概率可忽略不计。

\subsection{安全的基于 nonce 的 MAC}\label{subsec:7-5-1}

当 nonce 是由对手选取的时候,基于 nonce 的 MAC 在选择消息攻击下必然是存在性不可伪造的。然而,对手决不可以使用以前曾经被使用过的 nonce 来请求标签。这抓住了一个想法,即只要 nonce 不被重复使用,就可以由任意的选取得到。基于 nonce 的 MAC 的安全性由下面的攻击游戏定义。

\begin{game}[基于 nonce 的 MAC 的安全性]\label{game:7-4}
对于一个定义在 $(\mathcal{K},\mathcal{M},\mathcal{T},\mathpzc{N})$ 上的给定的基于 nonce 的 MAC 系统 $\mathcal{I}=(S,V)$ 以及一个给定对手 $\mathcal{A}$,攻击游戏运行如下:
\begin{itemize}
	\item 挑战者随机选取一个 $k\overset{\rm R}\leftarrow\mathcal{K}$。
	\item $\mathcal{A}$ 对挑战者发起多次查询。对于 $i=1,2,\dots$,第 $i$ 次签名查询是一个数对 $(m_i,{\scriptstyle\mathpzc{N}}_i)$,其中 $m_i\in\mathcal{M}$,${\scriptstyle\mathpzc{N}}_i\in\mathpzc{N}$。我们要求,对于所有的 $j<i$,都有 ${\scriptstyle\mathpzc{N}}_i\neq{\scriptstyle\mathpzc{N}}_j$。挑战者计算 $t_i\overset{\rm R}\leftarrow S(k,m_i,{\scriptstyle\mathpzc{N}}_i)$,并将 $t_i$ 发送给 $\mathcal{A}$。 
	\item 最终,$\mathcal{A}$ 输出一个候选的伪造三元组$(m,t,{\scriptstyle\mathpzc{N}})\in\mathcal{M}\times\mathcal{T}\times\mathpzc{N}$,其中:
	\[
	(m,t,{\scriptstyle\mathpzc{N}})\notin\{(m_1,t_1,{\scriptstyle\mathpzc{N}}_1),\,(m_2,t_1,{\scriptstyle\mathpzc{N}}_2),\,\dots\}
	\]
\end{itemize}
如果$V(k,m,t,{\scriptstyle\mathpzc{N}})=\mathsf{accept}$,我们就称 $\mathcal{A}$ 赢得了该游戏。我们将 $\mathcal{A}$ 就 $\mathcal{I}$ 的优势定义为 $\mathrm{nMAC}\mathsf{adv}[\mathcal{A},\mathcal{I}]$,即 $\mathcal{A}$ 赢得该游戏的概率。
\end{game}

\begin{definition}\label{def:7-6}
如果对于所有的有效对手 $\mathcal{A}$,$\mathrm{nMAC}\mathsf{adv}[\mathcal{A},\mathcal{I}]$ 的值都可忽略不计,我们就称基于 nonce 的 MAC 系统 $\mathcal{I}$ 是安全的。
\end{definition}

\begin{snote}[基于 nonce 的 Carter-Wegman MAC。]
Carter-Wegman MAC(见 \ref{sec:7-4} 节)可以被重构为一个基于 nonce 的 MAC:我们简单地将随机元 $r\in\mathcal{R}$ 视作一个 nonce,将它作为输入,而不是标签的一个随机生成的部分,提供给签名算法。使用 \ref{sec:7-4} 节中的符号,这样的 MAC 系统就形如:
\[
\begin{aligned}
	S\big((k_1,k_2),\,m,\,{\scriptstyle\mathpzc{N}}\big) & := H(k_1,m)+F(k_2,{\scriptstyle\mathpzc{N}})\\
	V\big((k_1,k_2),\,m,\,t,\,{\scriptstyle\mathpzc{N}}\big) & :=
	\left\{
	\begin{array}{ll}
		\mathsf{accept}, & t=S\big((k_1,k_2),\,m,\,{\scriptstyle\mathpzc{N}}\big)\\
		\mathsf{reject}, & \text{otherwise}
	\end{array}
	\right.
\end{aligned}
\]
我们可以得到下面的安全定理,它是定理 \ref{theo:7-9} 的一个基于 nonce 的版本。该定理的证明与定理 \ref{theo:7-9} 基本相同。
\end{snote}

\begin{theorem}\label{theo:7-10}
使用定理 \ref{theo:7-9} 中的符号,我们有以下约束:
\[
\mathrm{nMAC}\mathsf{adv}[\mathcal{A},\mathcal{I}_\mathrm{CW}]\leq\mathrm{PRF}\mathsf{adv}[\mathcal{B}_F,F]+\mathrm{DUF}\mathsf{adv}[\mathcal{B}_H,H]+\frac{1}{|\mathcal{T}|}
\]
\end{theorem}

这个约束比式 \ref{eq:7-24} 要严格得多:$Q^2$ 这一项已经没有了。当然,它之所以消失了,是因为我们坚持同一个 nonce 不会被使用两次。如果 nonce 实际上是由签名者随机生成的,那么 $Q^2$ 项就仍然需要加回来;然而,如果签名者将 nonce 作为一个计数器来实现,那么我们就可以避免 $Q^2$ 这项——唯一的要求是签名者不会签署超过 $|\mathcal{R}|$ 个值。关于 $F$ 的实现,请参见练习 7.12,该练习探讨了一个比较具体的问题。

与备注 \ref{remark:7-6} 中的讨论类似,当使用基于 nonce 的 Carter-Wegman MAC 时,至关重要的是 nonce 不能被重复用于不同的消息上。如果发生这种情况,Carter-Wegman MAC 就可能会被完全破坏——见练习 7.13 和 7.14。