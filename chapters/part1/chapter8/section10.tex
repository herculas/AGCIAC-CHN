\section{密钥派生与随机预言机模型}\label{sec:8-10}

尽管像 SHA256 这样的哈希函数在起初只是为了提供抗碰撞性而被设计的,但是正如我们在 \ref{sec:8-7} 节中看到的,业界经常用它们来解决其他的一些问题。直观地说,像 SHA256 这样的哈希函数能够``彻底扰乱"其输入,因此这种方法似乎有一定的意义。事实上,在 \ref{sec:8-7} 节中,我们研究了把一个无密钥哈希函数变成一个带密钥函数的问题,后者同时也是一个安全的 PRF,我们还发现,在合理的假设下,我们确实可以给出一个安全分析。

在这一节中,我们研究另一个问题,称为\textbf{密钥派生 (key derivation)}问题。粗略的说,这个问题是这样的:我们现在有一些机密数据,并且想要把它转换成一个 $n$ 比特的序列,而后者可以作为密钥被用到其他密码学原语——比如 AES——中。现在,这种机密数据在某种意义上说可能是随机的(至少在某种程度上是难以猜测的),但它可能根本不像是一个均匀分布的、随机的、$n$ 比特的序列。那么,我们如何从这样一个秘密的 $s$ 得到一个密码学密钥 $t$ 呢? 当然是哈希。在实践中,我们需要一个哈希函数 $H$,例如 SHA256(或者,正如我们后面将要推荐的一些由 SHA256 构建的函数),并计算 $t\leftarrow H(s)$。

在介绍上述问题的同时,我们还将介绍\emph{随机预言机模型(random oracle model)},它是一个启发式工具,不仅对分析密钥派生问题有帮助,对分析其他一系列问题也是有用的。

\subsection{密钥派生问题}\label{subsec:8-10-1}

让我们更详细地考察密钥派生问题。高层次地说,该问题就是将某个难以猜测的离散数据转换为一个 $n$ 比特的序列,后者可以作为一些标准的密码学原语(比如 AES)的密钥。在所有情况下,解决方案都是对秘密进行哈希以获得密钥。我们从一些启发性的例子开始。
\begin{itemize}
	\item 秘密可能是一个口令。虽然这样的口令可能有点难猜,但直接使用它作为 AES 密钥可能是很危险的。即使口令均匀分布在一个很大的字典中(这本身已经是一个很可疑的假设了),但将它编码为一个比特序列后,编码也肯定不是均匀分布的。很可能有相当一部分的口令对应于 AES 的``弱密钥",从而使其容易受到攻击。回顾一下,AES 的设计以随机比特序列作为密钥,所以它在这样的口令上的表现完全是另一回事。
	\item 秘密可能是一台运行中的计算机上的各种类型的系统事件记录(例如各种中断的时间,如按下的按键或鼠标移动引起的中断)。同样地,身处计算机系统之外的攻击者可能很难准确预测这种日志的内容。然而,直接使用日志作为 AES 密钥是有问题的:它可能太长了,而且远不是均匀分布的。
	\item 秘密可能是一个已被部分破坏的密码学密钥。想象一下,一个用户有一个 $128$ 比特的密钥,但其中的 $64$ 比特已经被泄露给了对手。该密钥仍然相当难以猜测,但从对手的角度来看,它仍然不是均匀分布的,因此不应当再被直接用作 AES 密钥。
	\item 稍后,我们将看到在公钥密码学中被广泛使用的数论变换的例子。稍微跳跃一点,我们在后面将会看到,对于一个大的合模数 $N$,如果 $x$ 是一个随机选择的模 $N$ 数,而一个对手得到了 $y:=x^3 \bmod N$,它很难计算出 $x$。我们可以把 $x$ 看作是秘密,与前面的例子类似,我们可以把 $y$ 看作是泄露给对手的信息。即使在信息论意义上,$y$ 的值完全决定了 $x$,它仍然被广泛认为是难以计算的。因此,我们可能想把 $x$ 当作秘密数据,与前面例子中的方式完全相同。许多和之前一样的问题在这里也会出现,其中最重要的是,$x$ 通常比 AES 密钥长得多(通常有几千比特长)。
\end{itemize}

如前所述,实践中采用的解决方案是简单地使用哈希函数 $H$ 对秘密 $s$ 进行哈希以获得密钥 $t\leftarrow H(s)$。

现在,我们给出一个我们所追求的安全属性的正式定义。

我们假设,秘密 $s$ 是根据一些固定(且公开的)概率分布 $P$ 采样得到的。我们假设,任何这样的秘密数据都可以被编码为某个有限集 $\mathcal{S}$ 上的元素。此外,我们引入一个函数 $I$ 来对泄露 $s$ 的部分信息的事实进行建模,因此,一个试图猜测 $s$ 的对手能够获取侧信息 $I(s)$。

\begin{game}[猜测优势]\label{game:8-3}
令 $P$ 是一个定义在有限集 $\mathcal{S}$ 上的概率分布,$I$ 是一个定义在 $\mathcal{S}$ 上的函数。对于一个给定对手 $\mathcal{A}$,攻击游戏运行如下:
\begin{itemize}
	\item 挑战者根据 $P$ 随机选择 $s$,并将 $I(s)$ 发送给 $\mathcal{A}$。
	\item 对手输出对 $s$ 的猜测 $\hat{s}$,如果 $\hat{s}=s$,它就赢得该游戏。
\end{itemize}
我们称 $\mathcal{A}$ 赢得该游戏的概率为其\textbf{猜测优势(guessing advantage)},记为 $\mathrm{Guess}\mathsf{adv}[\mathcal{A},P,I]$。
\end{game}

在上面的第一个例子中,我们可以简单地将 $s$ 建模为一个口令,这个口令均匀分布在某个(已被编码的)字典 $D$ 上。在这种情况下,对手没有得到任何侧信息,不管它的算力有多强,猜测优势都是 $1/|D|$。

在上面的第二个例子中,我们似乎很难对猜测优势给出一个有意义的、可靠的估计。

在上面的第三个例子中,$s$ 均匀分布在 $\{0,1\}^{128}$ 上,而 $I(s)$ 是 $s$ 的前 $64$ 比特。显然,任何对手,无论其算力多么强大,猜测优势都不会超过 $2^{-64}$。

在上面的第四个例子中,$s$ 是值 $x$,$I(s)$ 是值 $y$。由于 $y$ 完全决定了 $x$,所以对手有可能通过暴力搜索由 $I(s)$ 恢复 $s$。当然也有更聪明和更快的算法,但目前还没有已知的有效算法可以做到这一点。所以对于所有有效对手,猜测优势似乎都是可忽略不计的。

现在,假设我们使用一个哈希函数 $H:\mathcal{S}\to\mathcal{T}$ 从 $s$ 派生出密钥 $t$。直观地说,我们想让 $t$ ``看起来是随机的"。为了正式地定义这个直观的概念,我们使用 \ref{sec:3-11} 节中定义的计算上不可区分性的概念。因此,正式地说,我们想要的属性是:如果 $s$ 是根据 $P$ 采样得到的,而 $t$ 是从 $\mathcal{T}$ 中随机选出的,那么两个分布 $(I(s),H(s))$ 和 $(I(s),t)$ 在计算上是不可区分的。对于对手 $\mathcal{A}$,令 $\mathrm{Dist}\mathsf{adv}[\mathcal{A},P,I,H]$ 是对手在攻击游戏 \ref{game:3-3} 中区分这两种分布的优势。

对于我们希望能够证明的定理,它的类型是,大体上说,如果 $H$ 满足某个特定属性,或许 $P$ 和 $I$ 也被施加了某种约束,那么对于每个对手 $\mathcal{A}$,都存在一个对手 $\mathcal{B}$(它是一个 $\mathcal{A}$ 的基本包装器),满足 $\mathrm{Dist}\mathsf{adv}[\mathcal{A},P,I,H]$ 不比 $\mathrm{Guess}\mathsf{adv}[\mathcal{B},P,I]$ 大太多。事实上,在某些情况下,证明这样一个定理\emph{是}可能的。我们将在后面的 \ref{subsec:8-10-4} 小节中讨论这个结论——现在,我们简单地说,由于一些原因,这种严格的方法在实践中并没有被广泛地使用。相反,我们将更详细地研究使用 SHA256 这样``现成"的哈希函数来派生密钥的启发式方法。

\begin{snote}[子密钥的派生。]
在继续我们的讨论之前,我们考虑以下相关的问题:如何处理从 $s$ 派生来的密钥 $t$。在某些应用中,我们可能会直接使用 $t$ 作为——比方说——AES 的密钥。然而,在其他应用中,我们可能需要好几个密钥:例如一个加密密钥和一个 MAC 密钥,或者用于安全双工通信的两个不同的加密密钥(因此,Alice 有一个密钥用于向 Bob 发送加密消息,而 Bob 使用另一个密钥向 Alice 发送加密消息)。因此,一旦我们派生出一个``就所有意图和目的而言"都表现得像一个随机比特序列的单一密钥 $t$,我们就想再派生出几个子密钥。我们称之为\textbf{子密钥派生问题(sub-key derivation problem)},以区别于密钥派生问题。对于子密钥派生问题,我们假设从一个真随机的密钥 $t$ 开始,但是 $t$ 很可能并不是真随机的,但是当 $t$ 与真随机的密钥在计算上不可区分时,这个假设就是合理的。

幸运的是,对于子密钥派生问题,我们已经掌握了我们所需的所有工具。事实上,我们可以使用 PRG 或 PRF 从 $t$ 派生出子密钥。例如,在上面的例子中,如果 Alice 和 Bob 已经有了一个共享密钥 $t$,而它是由一个秘密 $s$ 派生而来的,那么他们就可以按如下方式使用 PRF $F$:
\begin{itemize}
	\item 派生出一个 MAC 密钥 $k_\mathrm{mac}\overset{\rm R}\leftarrow F(t,\,\texttt{"MAC}\text{-}\texttt{KEY"})$。
	\item 派生出一个 Alice 到 Bob 的加密密钥 $k_\mathrm{AB}\overset{\rm R}\leftarrow F(t,\texttt{"AB}\text{-}\texttt{KEY"})$。
	\item 派生出一个 Bob 到 Alice 的加密密钥 $k_\mathrm{BA}\overset{\rm R}\leftarrow F(t,\texttt{"BA}\text{-}\texttt{KEY"})$。
\end{itemize}
假设 $F$ 是一个安全的 PRF,那么密钥 $k_\mathrm{mac}$、$k_\mathrm{AB}$ 和 $k_\mathrm{BA}$ 的行为,就所有意图和目的而言,都像是独立的随机密钥。为了实现 $F$,我们甚至可以使用基于哈希的 PRF(如 HMAC),所以我们可以使用单一的``现成"哈希函数,如 SHA256,来完成我们所需要的一切任务,即密钥派生和子密钥派生。

因此,一旦我们解决了密钥派生的问题,我们就可以使用所构建的工具来解决子密钥派生的问题。不幸的是,使用``现成"的哈希函数进行密钥派生的做法并没有得到很好的理解和分析。尽管如此,还是有一些有用的启发式模型可供探索。
\end{snote}

\subsection{随机预言机:一种有用的启发法}\label{subsec:8-10-2}

我们现在介绍一种启发式算法,它可以在各种应用中被用来模拟哈希函数,包括密钥派生。正如我们将要在后面的章节中看到的,它已经成为一种流行的启发式算法,被用来证明许多密码学构造的合理性。

我们的想法是,简单地对哈希函数 $H$ 进行建模,\emph{就好像}它是一个真随机函数 $\mathcal{O}$ 一样。如果 $H$ 将 $\mathcal{M}$ 映射成 $\mathcal{T}$,那么 $\mathcal{O}$ 就是从集合 $\mathrm{Funs}[\mathcal{M},\mathcal{T}]$ 中均匀随机选出的。我们可以将任何攻击游戏转化为它的随机预言机版本:挑战者用 $\mathcal{O}$ 代替 $H$ 进行所有的计算,此外,对手被允许在它所选定的任意输入点上查询并获取 $\mathcal{O}$ 的值。这个函数 $\mathcal{O}$ 被称为\textbf{随机预言机 (random oracle)}。在这种设定下,我们称安全性在\textbf{随机预言机模型 (random oracle model)}下成立。由于函数 $\mathcal{O}$ 实在太大,我们无法将它写下来,更不可能将它用在真正的构造中。相对地,只有在对实际使用 $H$ 的系统进行启发式的安全分析时,我们才会用到 $\mathcal{O}$。

这种对使用哈希函数的构造进行分析的方法类似于 \ref{sec:4-7} 节所介绍的\emph{理想密码模型}。回顾一下,在理想密码模型中,我们用一个随机置换族 $\{\Pi_\mathpzc{k}\}_{\mathpzc{k}\in\mathcal{K}}$ 来替代一个定义在 $(\mathcal{K},\mathcal{X})$ 上的分组密码 $\mathcal{E}=(E,D)$。

正如我们之前所说的,随机预言机模型在现代密码学中用得很多。如果我们能使用一个``现成"的哈希函数 $H$,并将其建模为一个随机预言机,那就更好了。然而,如果我们想要一个真正通用的工具,我们就必须得小心一点,特别是如果我们想把 $H$ 建模为一个接受\emph{变长输入}的随机预言机的话。基本的经验法则是,不要\emph{直接}将 Merkle-Damg{\aa}rd 哈希用作通用的随机预言机。我们将在 \ref{subsec:8-10-3} 小节中讨论如何安全地(但同样也是启发式地)将 Merkle-Damg{\aa}rd 哈希用作通用的随机语言机。此外,我们还将看到,海绵构造(见 \ref{sec:8-8} 节)可以直接被``原样"使用。

我们强调,即使随机预言机上的安全结论都是严格的数学定理,它们仍然只是启发式的结果,而不能为使用特定哈希函数构建的系统提供任何安全性保证。尽管如此,这些安全结论确实能够排除针对系统的``通用攻击",如果哈希函数\emph{是}一个随机预言机的话,后者确实\emph{是}可行的。因此,这样的结论虽然不能排除所有攻击,但确实能够排除一般的攻击,这显然比对系统的安全性完全没有任何说法要好得多。事实上,在现实世界中,如果需要在两个系统 $\mathcal{S}_1$ 和 $\mathcal{S}_2$ 之间做一个选择,其中 $\mathcal{S}_1$ 有一个在随机预言机模型下的安全证明,而 $\mathcal{S}_2$ 有一个真正的安全证明,但是它的运行时间却是 $\mathcal{S}_1$ 的两倍,那么大多数业者都会(非常合理地)选择 $\mathcal{S}_1$,而不是 $\mathcal{S}_2$。

\begin{snote}[在随机预言机模型下定义安全性。]
假设我们有某种类型的密码学方案 $\mathcal{S}$,其实现使用了一个子程序,用于计算一个定义在 $(\mathcal{M},\mathcal{T})$ 上的哈希函数 $H$。方案 $\mathcal{S}$ 在其选择的任意点上评估 $H$,但不会考察 $H$ 的内部实现。我们称 \textbf{$\mathcal{S}$ 将 $H$ 用作一个预言机 ($\mathcal{S}$ uses $H$ as an oracle)}。比如说我们曾在 \ref{sec:8-7} 节中简要考察的方案 $F_\mathrm{pre}(k,x):=H(k\,\Vert\,x)$,它就是一个将哈希函数 $H$ 用作预言机的 PRF。

我们想要分析 $\mathcal{S}$ 的安全性。我们假设我们所感兴趣的任何安全属性,比如说一个``属性 $X$",(像往常一样)被建模为挑战者(特定于属性 $X$)和一个任意的对手 $\mathcal{A}$ 之间的游戏。在应答特定查询时,挑战者想必会计算与方案 $\mathcal{S}$ 相关的各种函数,而这些函数又可能需要在某些点上对 $H$ 进行评估。这个游戏定义了一个优势 $\mathrm{X}\mathsf{adv}[\mathcal{A},\mathcal{S}]$,而关于属性 $X$ 的安全性意味着,对于所有有效对手 $\mathcal{A}$,该优势都应当是可忽略不计的。

如果我们想要在随机预言机模型下分析 $\mathcal{S}$,定义安全性的攻击游戏就需要被修改,使用一个\emph{真随机函数} $\mathcal{O}\in\mathrm{Funs}[\mathcal{M},\mathcal{T}]$ 有效地替代 $H$,而对手和挑战者都可以对后者发起预言机访问。更确切地说,该游戏应当被修改如下:
\begin{itemize}
	\item 在游戏开始时,挑战者随机选择一个 $\mathcal{O}\in\mathrm{Funs}[\mathcal{M},\mathcal{T}]$。
	\item 除了标准查询外,对手 $\mathcal{A}$ 还可以提交\emph{随机预言机查询 (random oracle queries)}:它向挑战者发送 $m\in\mathcal{M}$,而挑战者用 $t=\mathcal{O}(m)$ 应答。对手可以发起任意数量的随机预言机查询,并且可以与标准查询以任意次序交错进行。
	\item 在处理标准查询时,挑战者用 $\mathcal{O}$ 来代替 $H$ 来进行计算。
\end{itemize}
定义对手优势的规则与之前相同,但使用 $\mathrm{X}^\mathrm{io}\mathsf{adv}[\mathcal{A},\mathcal{S}]$ 来表示,以强调这是随机预言机模型下的优势。\emph{随机预言机模型下}的安全性意味着,对于所有有效对手 $\mathcal{A}$,$\mathrm{X}^\mathrm{io}\mathsf{adv}[\mathcal{A},\mathcal{S}]$ 都是可忽略不计的。
\end{snote}

\begin{snote}[一个简单的例子:随机预言机模型下的 PRF。]
下面,我们说明如何应用随机预言机框架来构建安全的 PRF。特别地,我们将表明,在随机预言机模型下,$F_\mathrm{pre}$ 是一个安全的 PRF。我们首先修改标准的 PRF 安全游戏,以获得随机预言机模型下的 PRF 安全游戏。为了更清楚地表述问题,如果我们有一个将哈希函数 $H$ 用作预言机的 PRF,我们就用 $F^\mathcal{O}$ 表示使用随机预言机 $\mathcal{O}$ 代替 $H$ 后得到的函数。
\end{snote}

\begin{game}[随机预言机模型下的 PRF]\label{game:8-4}
令 $F$ 是一个定义在 $(\mathcal{K},\mathcal{X},\mathcal{Y})$ 上的 PRF,它将一个定义在 $(\mathcal{M},\mathcal{T})$ 上的哈希函数 $H$ 用作预言机。对于一个给定对手 $\mathcal{A}$,我们定义两个实验:实验 $0$ 和实验 $1$。对于 $b=0,1$,我们定义:

\noindent\textbf{实验 $b$:}
\begin{itemize}
	\item 挑战者按如下方式选择 $f\in{\rm Funs}[\mathcal{X},\mathcal{Y}]$:
	
	\hspace*{26pt} 如果 $b=0$:随机选取 $k\overset{\rm R}\leftarrow\mathcal{K}$,令 $f\leftarrow F^\mathcal{O}(k,\cdot)$;\\
	\hspace*{26pt} 如果 $b=1$:随机选取 $f\overset{\rm R}\leftarrow{\rm Funs}[\mathcal{X},\mathcal{Y}]$。
	
	\item 对手向挑战者发起一系列查询。
	\begin{itemize}
		\item $F$-查询:对于查询 $x\in\mathcal{X}$,应答 $y=f(x)\in\mathcal{Y}$。
		\item $\mathcal{O}$-查询:对于查询 $m\in\mathcal{M}$,应答 $t=\mathcal{O}(m)\in\mathcal{T}$。
	\end{itemize}
	\item 对手计算并输出一个比特 $\hat b\in\{0,1\}$。
\end{itemize}
对于 $b=0,1$,令 $W_b$ 为 $\mathcal{A}$ 在实验 $b$ 中输出 $1$ 的事件。我们将 $\mathcal{A}$ 就 $F$ 的优势定义为:
\[
\mathrm{PRF}^\mathrm{ro}\mathsf{adv}[\mathcal{A},F]:=
\big\lvert
\Pr[W_0]-\Pr[W_1]
\big\rvert
\]
\end{game}

\begin{definition}\label{def:8-5}
如果对于所有有效对手 $\mathcal{A}$,$\mathrm{PRF}^\mathrm{ro}\mathsf{adv}[\mathcal{A},F]$ 的值都是可忽略不计的,我们就称 PRF $F$ 在随机预言机模型下是安全的。
\end{definition}

重新考虑 PRF $F_\mathrm{pre}(k,x):=H(k\,\Vert\,x)$。我们假设 $F_\mathrm{pre}$ 定义在 $(\mathcal{K},\mathcal{X},\mathcal{T})$ 上,其中 $\mathcal{K}=\{0,1\}^\kappa$,$\mathcal{X}=\{0,1\}^{\leq L}$,并且 $H$ 定义在 $(\mathcal{M},\mathcal{T})$ 上,其中 $\mathcal{M}$ 包括所有最长为 $\kappa+L$ 的比特序列。

我们将证明,在随机预言机模型下,该 PRF 是一个安全的 PRF。但是等等!我们已经在 \ref{sec:8-7} 节中论证过,当 $H$ 是一个 Merkle-Damg{\aa}rd 哈希时,$F_\mathrm{pre}$ 完全是不安全的。这似乎出现了矛盾。问题在于,正如我们之前提到的,直接使用 Merkle-Damg{\aa}rd 哈希作为随机预言机是不安全的。我们将在 \ref{subsec:8-10-3} 节介绍解决这个问题的方法。

\begin{theorem}\label{theo:8-9}
如果 $\mathcal{K}$ 很大,那么当 $H$ 被建模为一个随机预言机时,$F_\mathrm{pre}$ 就是一个安全的 PRF。
\begin{quote}
特别地,如果 $\mathcal{A}$ 是一个如攻击游戏 \ref{game:8-4} 中那样的随机预言机 PRF 对手,并且最多可以发起 $Q_\mathrm{ro}$ 次预言机查询,则有:
\end{quote}
\[
\mathrm{PRF}^\mathrm{ro}\mathsf{adv}[\mathcal{A},F_\mathrm{pre}]
\leq
Q_\mathrm{ro}/|\mathcal{K}|
\]
\end{theorem}

请注意,定理 \ref{theo:8-9} 是无条件成立的,也就是说,对 $\mathcal{A}$ 的唯一约束就是随即预言机查询次数的限制:它不依赖于任何复杂性假设。

\begin{proof}[证明思路]
一旦将 $H$ 替换成 $\mathcal{O}$,对手就必须在不持有密钥 $k$ 的情况下将 $\mathcal{O}(k\,\Vert\,\cdot)$ 和 $\mathrm{Funs}[\mathcal{X},\mathcal{T}]$ 中的一个随机函数区分开来。由于 $\mathcal{O}(k\,\Vert\,\cdot)$ 也是 $\mathrm{Funs}[\mathcal{X},\mathcal{T}]$ 上的一个随机函数,因此对手唯一的希望,就是发起对 $\mathcal{O}$ 的查询,并且以某种方式利用应答所透露的信息。如果 $k'=k$,我们就称 $\mathcal{O}$-查询 $k'\,\Vert\,x'$ 是相关的。很显然,如果一个对 $\mathcal{O}$ 的查询不是相关的,它对于区分 $\mathcal{O}(k\,\Vert\,\cdot)$ 和随机函数就不会提供帮助,因为这时查询所返回的值与函数 $\mathcal{O}(k\,\Vert\,\cdot)$ 是无关的。此外,经过 $Q_\mathrm{ro}$ 次查询后,对手成功发起一次相关查询的概率最大是 $Q_\mathrm{ro}/|\mathcal{K}|$。
\end{proof}

\begin{proof}
为了严格表述上述证明思路,我们令 $\mathcal{A}$ 与两个 PRF 挑战者交互。对于 $j=0,1$,令 $W_j$ 表示 $\mathcal{A}$ 在游戏 $j$ 中输出 $1$ 的事件。

\vspace{5pt}

\noindent\textbf{游戏 $\mathbf{0}$。}
我们将游戏 $0$ 中的挑战者按如下方式表述,使得它等价于攻击游戏 \ref{game:8-4} 中的实验 $0$,但是更加便于分析。我们假设对手不会发起两次相同的 $F_\mathrm{pre}$-查询。此外,利用我们以经使用过很多次的``忠实的侏儒"的思路,我们使用一个关联数组 $Map:\mathcal{M}\to\mathcal{T}$ 来建立临时的随机预言机。下面是挑战者的运行逻辑:

\vspace{5pt}

\hspace*{5pt} 初始化:\\
\hspace*{50pt} 初始化一个空的关联数组 $Map:\mathcal{M}\to\mathcal{T}$\\
\hspace*{50pt} 随机选取 $k\overset{\rm R}\leftarrow\mathcal{K}$\\
\hspace*{26pt} 当收到一个对 $x\in\{0,1\}^{\leq L}$ 的 $F_\mathrm{pre}$-查询时:\\
\hspace*{50pt} 随机选取 $t\overset{\rm R}\leftarrow\mathcal{T}$\\
\hspace*{5pt} ($1$)
\hspace*{24.5pt} 如果 $(k\,\Vert\,x)\in\mathrm{Domain}(Map)$,就令 $t\leftarrow Map[k\,\Vert\,x]$\\
\hspace*{5pt} ($2$)
\hspace*{24.5pt} 令 $Map[k\,\Vert\,x]\leftarrow t$\\
\hspace*{50pt} 将 $t$ 发送给 $\mathcal{A}$\\
\hspace*{26pt} 当收到一个 $\mathcal{O}$-查询 $m\in\mathcal{M}$ 时:\\
\hspace*{50pt} 随机选取 $t\overset{\rm R}\leftarrow\mathcal{T}$\\
\hspace*{50pt} 如果 $m\in\mathrm{Domain}(Map)$,就令 $t\leftarrow Map[m]$\\
\hspace*{50pt} 令 $Map[m]\leftarrow t$\\
\hspace*{50pt} 将 $t$ 发送给 $\mathcal{A}$

\vspace{5pt}

\noindent
很显然,该挑战者就等价于攻击游戏 \ref{game:8-4} 的实验 $0$ 中的挑战者。在游戏 $0$ 中,每当挑战者(在处理 $F_\mathrm{pre}$-查询或 $\mathcal{O}$-查询时)需要在某个输入位置对随机预言机进行采样,它都会生成一个随机的``默认输出"。如果它发现预言机曾经在该输入位置被采样过,就会覆盖这个默认输出;在任何情况下,关联数组都会记录输入/输出对。

\vspace{5pt}

\noindent\textbf{游戏 $\mathbf{1}$。}
我们让我们的侏儒变得``健忘",方法是修改游戏 $0$,删除标有 ($1$) 和 ($2$) 的两行。现在观察一下,在游戏 $1$ 中,挑战者在应答 $F_\mathrm{pre}$-查询时,不会再使用 $Map$ 或 $k$:它只会返回一个随机值。所以很明显(根据假设,$\mathcal{A}$ 不会发起两次相同的 $F_\mathrm{pre}$-查询),游戏 $1$ 就等同于攻击游戏 \ref{game:8-4} 的实验 $1$,因此有:
\[
\mathrm{PRF}^\mathrm{ro}\mathsf{adv}[\mathcal{A},F_\mathrm{pre}]=
\big\lvert
\Pr[W_1]-\Pr[W_0]
\big\rvert
\]
令 $Z$ 为对手\emph{在游戏 $1$ 中}对一个点 $m=(k\,\Vert\,\hat{x})$ 发起 $\mathcal{O}$-查询的事件。很明显,除非 $Z$ 发生,否则上述两个游戏的结果是一样的。所以,根据差分引理,我们有:
\[
\big\lvert
\Pr[W_1]-\Pr[W_0]
\big\rvert
\leq\Pr[Z]
\]
由于密钥 $k$ 完全独立于 $\mathcal{A}$ 在游戏 $1$ 中的观察,因此每个 $\mathcal{O}$-查询都会以 $1/|\mathcal{K}|$ 的概率命中该密钥,此外,简单应用联合约束,我们就可以得到:
\[
\Pr[Z]\leq
Q_\mathrm{ro}/|\mathcal{K}|
\]
这就完成了证明。
\end{proof}

\begin{snote}[随机预言机模型下的密钥派生。]
现在,我们回到 \ref{subsec:8-10-1} 小节中讨论的密钥派生问题。同样地,我们有一个采样自某个分布 $P$ 的秘密 $s$,而信息 $I(s)$ 被泄露给了对手。我们要论证的是,如果 $H$ 被建模为一个随机预言机,那么对手在区分 $(I(s),H(s))$ 和 $(I(s),t)$(其中 $t$ 是真随机的)上的优势,不会比对手只用 $I(s)$(而不是 $H(s)$)猜出秘密 $s$ 的优势多太多。

为了将 $H$ 建模为一个随机预言机 $\mathcal{O}$,我们将计算性不可区分性攻击游戏 \ref{game:3-3} 转换为随即预言机模型,所以现在,攻击者被给予对随即预言机 $\mathcal{O}$ 的访问,并试图区分 $(I(s),\mathcal{O}(s))$ 和 $(I(s),t)$。与该攻击相关的优势被记为 $\mathrm{Dist}^\mathrm{ro}\mathsf{adv}[\mathcal{A},P,I,H]$。

在陈述我们的安全定理之前,我们不妨先对攻击游戏 \ref{game:8-3} 进行推广,允许对手输出一个猜测列表 $\hat{s}_1,\dots,\hat{s}_Q$。如果对于某个 $i=1,\dots,Q$,有 $\hat{s}_i=s$ 成立,我们就称对手赢得游戏。对手 $\mathcal{A}$ 在这个游戏中获胜的概率被称为他的\textbf{列表猜测优势(list guessing advantage)},记为 $\mathrm{ListGuess}\mathsf{adv}[\mathcal{A},P,I]$。

显然,如果一个对手 $\mathcal{A}$ 能以概率 $\epsilon$ 赢得上述列表猜测游戏,我们就可以将它转换成一个以概率 $\epsilon/Q$ 赢得单次猜测游戏的对手:我们只需运行 $\mathcal{A}$ 并得到一个列表 $\hat{s}_1,\dots,\hat{s}_Q$,然后随机选择一个 $i=1,\dots,Q$,最后输出 $\hat{s}_i$。然而有时,我们说不定能做得更好:使用部分信息 $I(s)$,我们也许就能排除一些 $\hat{s}_i$;而且,在某些情况下,我们可能能够唯一确定正确的 $\hat{s}_i$。这取决于具体的应用。
\end{snote}

\begin{theorem}\label{theo:8-10}
如果 $H$ 被建模为一个随机预言机,那么对于每个最多发起 $Q_\mathrm{ro}$ 次随机预言机查询的区分对手 $\mathcal{A}$,都存在一个列表猜测对手 $\mathcal{B}$,它是一个围绕 $\mathcal{A}$ 的基本包装器,满足:
\[
\mathrm{Dist}^\mathrm{ro}\mathsf{adv}[\mathcal{A},P,I,H]
\leq
\mathrm{ListGuess}\mathsf{adv}[\mathcal{B},P,I]
\]
并且,$\mathcal{B}$ 所输出的列表最多有 $Q_\mathrm{ro}$ 项。特别地,存在一个猜测对手 $\mathcal{B}'$,它是一个围绕 $\mathcal{A}$ 的基本包装器,满足:
\[
\mathrm{Dist}^\mathrm{ro}\mathsf{adv}[\mathcal{A},P,I,H]
\leq
Q_\mathrm{ro}\cdot\mathrm{Guess}\mathsf{adv}[\mathcal{B}',P,I]
\]
\end{theorem}

\begin{proof}
该证明与定理 \ref{theo:8-9} 的证明几乎完全相同。我们定义两个游戏,对于 $j=0,1$,令 $W_j$ 表示 $\mathcal{A}$ 在游戏 $j$ 中输出 $1$ 的事件。

\vspace{5pt}

\noindent\textbf{游戏 $\mathbf{0}$。}
我们将游戏 $0$ 中的挑战者按如下方式表述,使得它等价于 $(I(s),H(s))$ 和 $(I(s),t)$ 区分游戏的实验 $0$。我们使用一个关联数组 $Map:\mathcal{M}\to\mathcal{T}$ 来建立临时的随机预言机。下面是挑战者的运行逻辑:

\vspace{5pt}

\hspace*{5pt} 初始化:\\
\hspace*{50pt} 初始化一个空的关联数组 $Map:\mathcal{M}\to\mathcal{T}$\\
\hspace*{50pt} 根据分布 $P$ 生成 $s$\\
\hspace*{50pt} 随机选取 $t\overset{\rm R}\leftarrow\mathcal{T}$\\
\hspace*{5pt} ($*$)
\hspace*{24.5pt} 令 $Map[s]\leftarrow t$\\
\hspace*{50pt} 将 $(I(s),t)$ 发送给 $\mathcal{A}$\\
\hspace*{26pt} 当收到一个 $\mathcal{O}$-查询 $\hat{s}\in\mathcal{S}$ 时:\\
\hspace*{50pt} 随机选取 $\hat{t}\overset{\rm R}\leftarrow\mathcal{T}$\\
\hspace*{50pt} 如果 $\hat{s}\in\mathrm{Domain}(Map)$,就令 $\hat{t}\leftarrow Map[\hat{s}]$\\
\hspace*{50pt} 令 $Map[\hat{s}]\leftarrow\hat{t}$\\
\hspace*{50pt} 将 $\hat{t}$ 发送给 $\mathcal{A}$

\vspace{5pt}

\noindent\textbf{游戏 $\mathbf{1}$。}
我们删除标有 ($*$) 的那一行。该游戏就等同于区分游戏的实验 $1$,因为现在的 $t$ 值是真正独立于随机预言机的。此外,除非游戏 $1$ 中的对手 $\mathcal{A}$ 在 $s$ 处发起了一个 $\mathcal{O}$-查询,否则这两个游戏的结果就是一样的。所以,我们的列表猜测对手 $\mathcal{B}$ 只是从自己的挑战者那里得到值 $I(s)$,并像游戏 $1$ 中那样扮演 $\mathcal{A}$ 的挑战者的角色。在游戏结束时,$\mathcal{B}$ 简单地输出 $\mathrm{Domain}(Map)$,即 $\mathcal{A}$ 发起的所有 $\mathcal{O}$-查询的位置。重要的一点是:我们的 $\mathcal{B}$ 在扮演这个角色时,除了 $I(s)$ 之外,没有获得任何其他关于 $s$ 的信息,此外,它记录了 $\mathcal{A}$ 所发起的所有 $\mathcal{O}$-查询。所以,根据查分引理,我们有:
\[
\mathrm{Dist}^\mathrm{ro}\mathsf{adv}[\mathcal{A}]=
\big\lvert
\Pr[W_0]-\Pr[W_1]
\big\rvert
\leq
\mathrm{ListGuess}\mathsf{adv}[\mathcal{B}]\qedhere
\]
\end{proof}

\subsection{随机预言机:安全的操作模式}\label{subsec:8-10-3}

我们已经看到,在随机预言机模型下,$F_\mathrm{pre}(k,x):=H(k\,\Vert\,x)$ 是安全的。然而,我们还知道,如果 $H$ 是一个 Merkle-Damg{\aa}rd 哈希,它就完全是不安全的。问题在于,Merkle-Damg{\aa}rd 构造包含一个非常简单的迭代结构,这将它暴露在``扩展攻击"之下。虽然从抗碰撞的角度来看,这种构造并不是问题,但它表明,拿一个现成的哈希函数并把它当作一个随机预言机来使用是一个非常危险的举措。

在本节中,我们将讨论如何安全地将一个 Merkle-Damg{\aa}rd 哈希函数用作一个随机预言机。我们还将看到,海绵构造(见 \ref{sec:8-8} 节)已经可以安全地``按原样"使用;事实上,海绵构造正是为该目的设计的:提供一个可以直接被用作随机预言机的,针对变长输入和变长输出的哈希函数。

假设 $H$ 是一个由压缩函数 $h:\{0,1\}^n\times\{0,1\}^\ell\to\{0,1\}^n$ 构建的 Merkle-Damg{\aa}rd 哈希。一个推荐的操作模式是使用 HMAC 和零密钥:
\[
\mathrm{HMAC}_0(m)
:=\mathrm{HMAC}(0^\ell,m)
=H\big(\mathsf{opad}\,\Vert\,H(\mathsf{ipad}\,\Vert\,m)\big)
\]
虽然这个构造能够挫败明显的扩展攻击,但我们凭什么相信 $\mathrm{HMAC}_0$ 作为一个通用的随机预言机是安全的?事实上,我们只能给出启发式的证据。从本质上讲,我们想要论证的是,$\mathrm{HMAC}_0$ 不存在内生的结构性弱点,它能招致将底层压缩函数本身作为随机预言机的通用攻击——或者,更现实地说——将底层压缩函数作为一个基于理想密码的 Davies-Meyer 构造的通用攻击。

因此我们想表明,利用某些特定的操作模式,我们可以由一个``小"的随机预言机,一个理想密码,甚至一个理想置换,建立一个``大"的随机预言机。

用来完成这种任务的数学工具被称为\textbf{无差别性(indifferentiability)}。我们在这里介绍此概念的一个简化版本。假设我们试图由一个较小的原语 $\rho$ 建立一个``大"的随机预言机 $\mathcal{O}$,其中 $\rho$ 可以是一个小领域上的随机预言机,一个理想密码,或者一个理想置换。我们用 $F[\rho]$ 表示基于理想原语 $\rho$ 的随机预言机的一个特定构造。

现在,考虑由某个挑战者 $\mathbf{C}$ 和对手 $\mathcal{A}$ 定义的一个通用攻击游戏。我们将 $\mathbf{C}$ 和 $\mathcal{A}$ 之间的交互记为 $\langle\mathbf{C},\mathcal{A}\rangle$。我们假设交互的结果是一个输出比特。我们接下来的所有安全定义都以这种形式的游戏为模型。

在有大随机预言机 $\mathcal{O}$ 的攻击游戏的随机预言机版本中,我们会允许挑战者和对手对随机函数 $\mathcal{O}$ 发起预言机访问,我们将交互记为 $\langle\mathbf{C}^\mathcal{O},\mathcal{A}^\mathcal{O}\rangle$。然而,如果我们使用构造 $F[\rho]$ 来实现上述的大随机预言机,那么,尽管挑战者只会通过构造 $F$ 来访问 $\rho$,对手却能够直接查询 $\rho$。我们将这种交互记为 $\langle\mathbf{C}^{F[\rho]},\mathcal{A}^\rho\rangle$。

例如,在 $\mathrm{HMAC}_0$ 构造中,压缩函数 $h$ 被建模为一个随机预言机 $\rho$;或者,如果 $h$ 本身是通过 Davies-Meyer 构建的,那么底层的分组密码就被建模为一个理想密码 $\rho$。 在任何一种情况下,$F[\rho]$ 都对应于 $\mathrm{HMAC}_0$ 构造本身。请注意,这里存在一种不对称性:在任何攻击游戏中,挑战者都只能通过 $F[\rho]$(本例中为 $\mathrm{HMAC}_0$)间接地访问 $\rho$,而对手却可以直接访问 $\rho$ 本身(即压缩函数 $h$ 或底层的分组密码)。

如果以下结论成立,我们就称 $F[\rho]$ 与 $\mathcal{O}$ 是\textbf{无差别(indifferentiable)}的:
\begin{quote}
\emph{对于每个有效挑战者 $\mathbf{C}$ 和有效对手 $\mathcal{A}$,都存在一个有效对手 $\mathcal{B}$,它是一个围绕 $\mathcal{A}$ 的基本包装器,满足:}
\end{quote}
\[
\big\lvert
\Pr[\langle\mathbf{C}^{F[\rho]},\mathcal{A}^\rho\rangle \text{ output } 1]
-
\Pr[\langle\mathbf{C}^\mathcal{O},\mathcal{B}^\mathcal{O}\rangle  \text{ output } 1]
\big\rvert
\]
\begin{quote}
\emph{是可忽略不计的。}
\end{quote}

从定义中应该可以看出,如果我们能够在随机预言机模型下为大的随机预言机 $\mathcal{O}$ 证明某个密码学方案的安全性,那么如果我们用 $F[\rho]$ 来实现 $\mathcal{O}$,该方案就仍然是安全的:如果一个对手 $\mathcal{A}$ 能够攻破使用 $F[\rho]$ 的方案,那么上面的对手 $\mathcal{B}$ 就能够攻破使用 $\mathcal{O}$ 的方案。

\begin{snote}[一些安全的模式。]
如果我们将压缩函数 $h$ 本身建模为一个随机预言机,或者 $h$ 是通过 Davies-Meyer 构建的,并且我们将底层分组密码建模为一个理想密码,我们就可以证明 $\mathrm{HMAC}_0$ 构造与变长输入上的随机预言机是无差别的。

将 $\mathrm{HMAC}_0$ 直接用作随机预言机的一个问题是,它的\emph{输出}相当短。幸运的是,使用 $\mathrm{HMAC}_0$ 来获得一个具有较长输出的随机预言机其实并不困难。方法如下。假设 $\mathrm{HMAC}_0$ 的输出有 $n$ 比特,而我们需要一个长输出(比如说 $N>n$ 比特)的随机预言机。设置 $q:=\lceil N/n\rceil$。令 $e_0,e_1,\dots,e_q$ 是整数 $0,1,\dots,q$ 的定长编码。我们的新哈希函数 $H'$ 按如下方式工作。对于输入 $m$,我们计算 $t\leftarrow\mathrm{HMAC}_0(e_0\,\Vert\,m)$。然后,对于 $i=1,\dots,q$,我们计算 $t\leftarrow\mathrm{HMAC}_0(e_i\,\Vert\,t)$。最后,我们输出 $t_1\,\Vert\,t_2\,\Vert\,\cdots\,\Vert\,t_q$ 的前 $N$ 比特。我们可以证明,$H'$ 与一个有 $N$ 比特输出的随机预言机是无差别的。如果我们用任何一个本身与 $N$ 比特输出的随机预言机是无差别的哈希函数代替 $\mathrm{HMAC}_0$,这个结果仍然是成立的。还要注意的是,当应用于长输入时,$H'$ 是相当有效的:它只需要对长输入计算一次 $\mathrm{HMAC}_0$ 即可。

已经有证明表示,如果我们把底层置换建模为一个理想置换(假设 $2^c$ 是超多项式的,其中 $c$ 为置换的大小),那么海绵构造与变长输入上的随机预言机是无差别的。这包括 \ref{subsec:8-8-2} 小节介绍的(用于定长输出的)标准化 SHA3 实现和(用于变长输出的)SHAKE 变体。SHA3 和 SHAKE 规范中使用的特殊填充规则能够确保它们所有的变体都能被用作独立的随机预言机。

我们有时需要输出元素均匀分布在某个特定集合上的随机预言机。例如,我们可能希望输出均匀分布在某个正整数集 $S=\{0,\dots,d-1\}$ 上,其中 $d$ 为某个正整数。为了实现这一点,我们可以使用一个有 $n$ 比特输出的哈希函数 $H$,其中,输出可被视作一个数的 $n$ 比特二进制编码,并定义 $H'(m):=H(m) \bmod d$。如果 $H$ 与一个具有 $n$ 比特输出的随机预言机是无差别的,并且 $2^n/d$ 是超多项式的,那么哈希函数 $H'$ 与一个输出是 $S$ 中的元素的随机预言机也是无差别的。
\end{snote}

\subsection{剩余哈希引理}\label{subsec:8-10-4}

现在,我们重新回到密钥派生问题上。在适当的情况下,我们可以不依靠启发法和任何计算假设就解决密钥派生问题。而且这个解决方案是对通用哈希函数的一个令人惊讶且优雅的应用(见 \ref{sec:7-1} 节)。这个结论被称为\textbf{剩余哈希引理 (leftover hash lemma)},它指出,如果我们使用一个 $\epsilon$-UHF 来哈希一个被猜中的概率最多为 $\gamma$ 的秘密,那么只要 $\epsilon$ 和 $\gamma$ 足够小,哈希的输出与一个真随机值就是统计上不可区分的。回忆一下,UHF 有一个密钥,我们通常认为它是一个私钥;然而,在这个结论中,密钥是可以被公开的——事实上,它可以被看作是一个公共的系统参数,当它被一次性地生成之后,就可以在之后被反复地使用。

在这里,我们的目标是简单地陈述该结论,并指出什么时候,什么地方才可以(或者不可以)使用它。为了阐明该结论,我们需要使用我们在 \ref{sec:3-11} 节中介绍的两个随机变量间的统计距离的概念。此外,如果 $\mathsf{s}$ 是一个在集合 $\mathcal{S}$ 中取值的随机变量,我们就将 \textbf{$\mathsf{s}$ 的猜测概率(guessing probability of $\mathsf{s}$)} 定义为 $\mathrm{max}_{x\in\mathcal{S}}\Pr[\mathsf{s}=x]$。

\begin{theorem}[剩余哈希引理]\label{theo:8-11}
令 $H$ 是一个定义在 $(\mathcal{K},\mathcal{S},\mathcal{T})$ 上的带密钥哈希函数。假设 $H$ 是一个 $(1+\alpha)/N$-UHF,其中 $N:=|\mathcal{T}|$。令 $\mathsf{k},\mathsf{s}_1,\dots,\mathsf{s}_m$ 是相互独立的随机变量,其中 $\mathsf{k}$ 均匀分布在 $\mathcal{K}$ 上,并且每个 $\mathsf{s}_i$ 的猜测概率至多是 $\gamma$。令 $\delta$ 是:
\[
\big(\mathsf{k},H(\mathsf{k},\mathsf{s}_1),\dots,H(\mathsf{k},\mathsf{s}_m)\big)
\]
与 $\mathcal{K}\times\mathcal{T}^m$ 上的均匀分布之间的统计距离。那么,我们有:
\[
\delta\leq
\frac{1}{2}m\sqrt{N\gamma+\alpha}
\]
\end{theorem}

让我们看看,当 $m=1$ 时,该引理意味着什么。我们有一个秘密 $s$,当给定任何关于 $s$ 的侧信息 $I(s)$ 时,它被猜中的概率最多是 $\gamma$。为了应用该引理,对猜测概率的约束 $\gamma$ 必须对所有对手,甚至是计算上无界的对手成立。然后,我们使用一个随机的哈希密钥 $k$ 对 $s$ 进行哈希。至关重要的是,$s$(给定 $I(s)$)和 $k$ 是相互独立的——尽管我们在这里并没有讨论这种可能性,但在一些潜在的用例中,$s$ 或者函数 $I$ 的分布可能以某种取决于 $k$ 的方式被对手偏离,其中 $k$ 被假定是公开的,且已经为对手所知。因此,为了应用该引理,我们必须确保 $s$(给定 $I(s)$)和 $k$ 真的是相互独立的。如果所有这些条件都能得到满足,那么该引理就意味着,对于任何对手 $\mathcal{A}$,甚至是计算上无界的对手,它区分 $(k,I(s),H(k,s))$ 和 $(k,I(s),t)$(其中 $t$ 是 $\mathcal{T}$ 上的一个真随机的元素)的优势被 $\delta$ 约束,正如引理所强调的。

现在,让我们引入一些现实世界中的数字。如果我们想把输出用作 AES 的密钥,我们需要 $N=2^{128}$。我们知道如何建立 $(1/N)$-UHF,所以我们可以取 $\alpha=0$(见练习 \ref{exer:7-18}——如果 $\alpha$ 是一个非零但仍然相当小的值,我们就可以用一个短得多的哈希密钥来解决)。如果我们希望 $\delta\leq2^{-64}$,我们就需要让猜测概率 $\gamma$ 大约是 $2^{-256}$。

因此,除了上面列出的所有条件外,我们确实还需要一个极小的猜测概率,以使该引理是适用的。\ref{subsec:8-10-1} 小节中讨论的例子都不符合这些要求:要么猜测概率不够小,要么对无界对手不是无条件成立的,要么只能被启发式地估计。因此,剩余哈希引理的实际适用性是有限的——但当它确实适用时,它将成为一个非常强大的工具。另外,我们注意到,通过在 $m>1$ 时使用该引理,在合适的条件下,我们可以模拟这样的情况:使用同一个哈希密钥从许多互相独立的,猜测概率都很小的秘密中派生出许多密钥。并不令人惊讶的是,区分概率会随着派生次数的增加而线性增长。

由于这些实际的限制,使用被建模为随机预言机的密码学哈希函数——而不是 UHF——来进行密钥派生是更加典型的。事实上,如果我们使用一个 UHF,而上面所讨论的某一个假设被证明是错误的,就很可能导致一个灾难性的安全漏洞。使用密码学哈希函数,就算只能启发式地保证密钥派生的安全,也更容易得到谅解。
 
\subsection{案例研究:HKDF}\label{subsec:8-10-5}

HKDF是RFC 5869中规定的一个密钥派生函数,并被部署在许多标准中。HKDF是以HMAC结构来指定的(见8.7节)。因此,它使用函数HMAC(k, m),其中k和m是可变长度的字节字符串,它本身是以的Merkle-Damg̊ard哈希H,如SHA256。

HKDF的输入包括一个秘密S,一个可选的盐值盐(在下面讨论),一个可选的信息字段(也在下面讨论)。
可选的信息字段(也在下面讨论),以及一个输出长度参数L。
HKDF的执行包括两个阶段,称为提取(相当于我们所说的密钥派生)和扩展(相当于我们所说的子密钥派生)。
在提取阶段,HKDF使用盐和s来计算t←HMAC(盐 , s)。
使用中间密钥t和信息,扩展(或子密钥派生)阶段计算L字节的输出数据,如下所示。
q ← ⌈L/HashLen⌉ // HashLen是H的输出长度(字节),将z0初始化为空字符串
for i ← 1 to q do:
zi ← HMAC(t,zi-1 ∥ info ∥ Octet(i)) // Octet(i)是一个单字节,其值为i 输出z1 ∥ ... ∥zq的前L个八位数

当盐为空时,HKDF的提取阶段与我们在8.10.3节中所说的HMAC0相同。正如那里所讨论的,HMAC0可以启发式地被看作是一个随机预言机,因此我们可以使用第8.10.2节的分析来证明这是一个随机预言机模型中的安全密钥派生程序。这一点,如果s是很难猜到的,那么t就与随机的无法区分。

HKDF的用户可以选择提供非零的盐。盐的作用类似于左旋哈希法则中使用的随机哈希密钥(见第8.10.4节);特别是,它不需要是秘密的,而且可以重复使用。然而,重要的是,盐值是独立于秘密S的,不能被对手操纵。我们的想法是,在这些情况下,HKDF的提取阶段的输出似乎更有可能与随机无异,而不需要依赖随机预言机模型的全部力量。不幸的是,已知的安全证明适用于有限的环境,所以在一般情况下,这仍然是有点启发式的。

正如我们在第8.10.1节末尾讨论的那样,扩展阶段只是简单地应用HMAC作为PRF来派生子密钥。信息参数可以用来 "命名 "衍生的子密钥,以确保用于不同目的的密钥的独立性。由于底层哈希的输出长度是固定的,所以使用一个简单的迭代方案来产生更长的输出。这个阶段可以在中间密钥t与随机不可区分,以及HMAC是一个安全的PRF的假设下进行严格的分析--我们已经知道,在对H的压缩函数的合理假设下,HMAC是一个安全的PRF。