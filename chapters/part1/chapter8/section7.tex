\section{案例研究:HMAC}\label{sec:8-7}

这一节,我们回过头来探讨之前的问题,即为长消息建立一个安全的 MAC。以 SHA256 为代表的 Merkle-Damg{\aa}rd 哈希函数已经被部署到了各种系统中。大多数密码学库都包含多种 Merkle-Damg{\aa}rd 函数的实现,而且这些实现往往都很快:通常情况下,用 SHA256 对一个很长的消息进行哈希要比用 AES 对相同的消息进行 CBC-MAC 要快得多。

当然,我们可以使用 \ref{sec:8-2} 节中所介绍的先哈希后MAC构造。回顾一下,在这个构造中,我们将一个安全的 MAC 系统 $\mathcal{I}=(S,V)$ 和一个抗碰撞哈希函数 $H$ 结合。由此产生的签名算法先使用 $H$ 对消息进行哈希,得到一个短的摘要 $H(m)$,然后使用 $S$ 对 $H(m)$ 进行签名,得到 MAC 标签 $t=S(k,H(m))$。正如我们在定理 \ref{theo:8-1} 中说明的,这样产生的构造是安全的。然而,这种构造并没有得到非常广泛的应用,这是为什么呢?

首先,正如我们在定理 \ref{theo:8-1} 之后所讨论的,如果我们可以找到 $H$ 的碰撞,先哈希后 MAC 构造就完全被破坏了。用于寻找碰撞的攻击,比如生日攻击(见 \ref{sec:8-3} 节),或者其他更复杂的攻击,都可以完全\emph{离线}进行,即不需要与任何系统用户交互。相对地,\emph{在线}攻击则需要对手和诚实的系统用户之间进行许多次交互。一般来说,我们认为离线攻击是特别危险的,因为对手可以在很长一段时间内投入巨大的计算资源:在对先哈希后 MAC 的攻击中,攻击者可以花上几个月的时间在许多机器上悄悄计算以找到 $H$ 上的碰撞,而不会引起任何怀疑。

另一个不直接使用先哈希后 MAC 构造的原因是,我们既需要一个哈希函数 $H$,也需要一个 MAC 系统 $\mathcal{I}$。因此,一个实现可能需要软件和/或硬件来执行哈希算法(比如 SHA256)和 MAC 算法(比如基于 AES 的 CBC-MAC)。在其他条件相同的情况下,如果能简单地只使用一种算法作为 MAC 的基础就更好了。

这就很自然地将我们导向这样一个问题:怎样基于一个\emph{无密钥}的 Merkle-Damg{\aa}rd 哈希函数(如 SHA256),以某种方式实现一个\emph{带密钥}的函数,它可以用作一个安全的 MAC,甚至是一个安全的 PRF?此外,我们希望能够在一个比抗碰撞更弱(至少是定性地)的假设下证明这个构造的安全性;特别地,这个构造应当不易受到针对底层压缩函数的离线碰撞查找攻击。

假设 $H$ 是一个由压缩函数 $h:\{0,1\}^n\times\{0,1\}^\ell\to\{0,1\}^n$ 构建的 Merkle-Damg{\aa}rd 哈希。我们可以想到下面几种简单的方法:
\begin{description}
	\item [前缀密钥:] $F_\mathrm{pre}(k,M):=H(k\,\Vert\,M)$。这种思路完全是不安全的,因为存在这样的一种\emph{扩展攻击}:给定 $F\mathrm{pre}(k,M)$,我们很容易为任意的 $M'$ 计算出 $F_\mathrm{pre}(k,M\,\Vert\,\mathrm{PB}\,\Vert\,M')$。这里的 $\mathrm{PB}$ 是消息 $k\,\Vert\,M$ 的 Merkle-Damg{\aa}rd 填充分组。如果排除这种扩展攻击,那么在合理的假设条件下,这个构造是安全的(见练习 \ref{exer:8-18})。
	\item [后缀密钥:] $F_\mathrm{post}(k,M):=H(M\,\Vert\,k)$。这种构造有点类似于先哈希后 MAC 构造,并且依赖于 $h$ 的抗碰撞性。事实上,该构造很容易受到离线碰撞查找攻击:假设我们能找到两个不同的 $\ell$ 比特序列 $M_0$ 和 $M_1$ 满足 $h(\mathrm{IV},M_0)=h(\mathrm{IV},M_1)$,我们就有 $F_\mathrm{post}(k,M_0)=F_\mathrm{post}(k,M_1)$。因此,这个构造也无法解决我们的问题。然而,在恰当的假设条件下(当然,也包括对 $h$ 抗碰撞性的假设),我们仍然能够得到该构造的安全证明(见练习 \ref{exer:8-19})。
	\item [密钥包裹:] $F_\mathrm{env}(k,M):=H(k\,\Vert\,M\,\Vert\,k)$。在对 $h$ 合理的伪随机性假设以及某种格式假设($k$是一个 $\ell$ 比特序列,$M$ 被填充为一个长度为 $\ell$ 的整数倍的序列)下,可以证明该构造是一个安全的PRF。见练习 \ref{exer:8-17}。
	\item [双层密钥嵌套:] $F_\mathrm{nest}((k_1,k_2),M):=H(k_2\,\Vert\,H(k_1\,\Vert\,M))$。在对 $h$ 合理的伪随机性假设和种格式假设($k_1$ 和 $k2$ 都是 $\ell$ 比特序列)下,该构造也可以被证明是一个安全的 PRF。
\end{description}

双层密钥嵌套与一个被称为 HMAC 的经典 MAC 构造密切相关。HMAC 是互联网上应用最广泛的 MAC。它被用于 SSL、TLS、IPsec、SSH 和许多其他的安全协议中。TLS 和 IPsec 在会话设置过程中使用 HMAC 来生成会话密钥。下面,我们分析双层密钥嵌套的安全性,然后讨论它与 HMAC 的关系。

\begin{figure}
	\centering
	\tikzset{every picture/.style={line width=0.75pt}}

\begin{tikzpicture}[x=0.75pt,y=0.75pt,yscale=-1,xscale=1]

\draw  [fill={rgb, 255:red, 255; green, 255; blue, 255 }  ,fill opacity=1 ][line width=1.2] [general shadow={fill=black,shadow xshift=2.25pt,shadow yshift=-2.25pt}] (110,85) -- (110,110) -- (60,110) -- (60,70) -- cycle ;
\draw  [fill={rgb, 255:red, 255; green, 255; blue, 255 }  ,fill opacity=1 ][line width=1.2] [general shadow={fill=black,shadow xshift=2.25pt,shadow yshift=-2.25pt}] (210,85) -- (210,110) -- (160,110) -- (160,70) -- cycle ;
\draw  [fill={rgb, 255:red, 255; green, 255; blue, 255 }  ,fill opacity=1 ][line width=1.2] [general shadow={fill=black,shadow xshift=2.25pt,shadow yshift=-2.25pt}] (380,85) -- (380,110) -- (330,110) -- (330,70) -- cycle ;
\draw  [fill={rgb, 255:red, 255; green, 255; blue, 255 }  ,fill opacity=1 ][line width=1.2] [general shadow={fill=black,shadow xshift=2.25pt,shadow yshift=-2.25pt}] (380,225) -- (380,250) -- (330,250) -- (330,210) -- cycle ;
\draw  [fill={rgb, 255:red, 255; green, 255; blue, 255 }  ,fill opacity=1 ][line width=1.2] [general shadow={fill=black,shadow xshift=2.25pt,shadow yshift=-2.25pt}] (480,225) -- (480,250) -- (430,250) -- (430,210) -- cycle ;

\draw  [line width=1.2]  (10,10) -- (80,10) -- (80,30) -- (10,30) -- cycle ;
\draw  [line width=1.2]  (110,10) -- (180,10) -- (180,30) -- (110,30) -- cycle ;
\draw  [line width=1.2]  (280,10) -- (350,10) -- (350,30) -- (280,30) -- cycle ;
\draw  [line width=1.2]  (280,150) -- (350,150) -- (350,170) -- (280,170) -- cycle ;
\draw  [line width=1.2]  (380,150) -- (450,150) -- (450,170) -- (380,170) -- cycle ;

\draw    (210,100) -- (250,100) ;
\draw  [->]  (10,100) -- (59,100) ;
\draw  [->]  (110,100) -- (159,100) ;
\draw  [->]  (290,100) -- (329,100) ;
\draw  [->]  (480,240) -- (520,240) ;
\draw  [->]  (280,240) -- (329,240) ;
\draw  [->]  (380,240) -- (429,240) ;

\draw  [->]  (45,30) -- (45,80) -- (59,80) ;
\draw  [->]  (145,30) -- (145,80) -- (159,80) ;
\draw  [->]  (315,30) -- (315,80) -- (329,80) ;
\draw  [->]  (315,170) -- (315,220) -- (329,220) ;
\draw  [->]  (415,170) -- (415,220) -- (429,220) ;
\draw  [->]  (380,100) -- (415,100) -- (415,149) ;

\draw   (64.95,80) -- (59.95,83.5) -- (59.95,76.5) -- cycle ;
\draw   (165,100) -- (160,103.5) -- (160,96.5) -- cycle ;
\draw   (335,100) -- (330,103.5) -- (330,96.5) -- cycle ;
\draw   (435,240) -- (430,243.5) -- (430,236.5) -- cycle ;
\draw   (334.95,220) -- (329.95,223.5) -- (329.95,216.5) -- cycle ;

\draw  [dash pattern={on 4.5pt off 4.5pt}][line width=0.75]  (95,0) -- (95,60) -- (120,60) -- (120,120) -- (0,120) -- (0,0) -- cycle ;
\draw  [dash pattern={on 4.5pt off 4.5pt}][line width=0.75]  (365,140) -- (365,200) -- (390,200) -- (390,260) -- (270,260) -- (270,140) -- cycle ;
\draw  [dash pattern={on 0.84pt off 2.51pt}]  (250,100) -- (290,100) ;


\draw (85,95) node  [font=\small]  {$h$};
\draw (185,95) node  [font=\small]  {$h$};
\draw (355,95) node  [font=\small]  {$h$};
\draw (45,20) node  [font=\small]  {$k_{1}$};
\draw (145,20) node  [font=\small]  {$m_{1}$};
\draw (315,20) node  [font=\small]  {$m_{s} \ \| \ \mathrm{PB}_{\mathrm{i}}$};
\draw (12,96.6) node [anchor=south west] [inner sep=0.75pt]  [font=\small]  {$\mathrm{IV}$};
\draw (135,96.6) node [anchor=south] [inner sep=0.75pt]  [font=\small]  {$k'_{1}$};
\draw (230,20) node    {$\cdots $};
\draw (355,235) node  [font=\small]  {$h$};
\draw (455,235) node  [font=\small]  {$h$};
\draw (315,160) node  [font=\small]  {$k_{2}$};
\draw (405,236.6) node [anchor=south] [inner sep=0.75pt]  [font=\small]  {$k'_{2}$};
\draw (282,236.6) node [anchor=south west] [inner sep=0.75pt]  [font=\small]  {$\mathrm{IV}$};
\draw (415,160) node  [font=\small]  {$t\ \| \ \mathrm{PB}_{\mathrm{o}}$};
\draw (397.5,96.62) node [anchor=south] [inner sep=0.75pt]  [font=\small]  {$t$};

\end{tikzpicture}
	\caption{双层密钥嵌套}
	\label{fig:8-9}
\end{figure}

\subsection{双层密钥嵌套的安全性}\label{subsec:8-7-1}

现在我们证明,在对 $h$ 进行适当的伪随机性假设的情况下,双层密钥嵌套确实是一个安全的 PRF。首先,考虑到 $H$ 是由 $h$ 建立的 Merkle-Damg{\aa}rd 哈希,我们试着``拆开" $F_\mathrm{nest}((k_1,k_2),M)$ 的定义,见图 \ref{fig:8-9}。读者应该仔细研究这张图。我们假设密钥 $k_1$ 和 $k_2$ 都是 $\ell$ 比特的序列,所以它们各自占据一个完整的消息分组。$H$ 内圈计算的输入是填充后的序列 $k_1\,\Vert\,M\,\Vert\,\mathrm{PB}_\mathrm{i}$,它会被拆分为一系列长为 $\ell$ 比特的分组。$H$ 内圈计算的输出是一个 $n$ 比特序列 $t$。$H$ 外圈计算的输入是填充后的序列 $k_2\,\Vert\,t\,\Vert\,\mathrm{PB}_\mathrm{o}$。我们假设 $n$ 远小于 $\ell$,所以 $t\,\Vert\,\mathrm{PB}_\mathrm{o}$ 就是一个单独的 $\ell$ 比特分组,如图所示。

下面介绍我们所需的伪随机性假设。我们定义下面两个由 $h$ 派生的 PRF $h_\mathrm{bot}$ 和 $h_\mathrm{top}$:
\begin{equation}\label{eq:8-6}
h_\mathrm{bot}(k,m):=h(k,m),
\qquad
h_\mathrm{top}(k,m):=h(m,k)
\end{equation}
对于 PRF $h_\mathrm{bot}$,PRF 密钥 $k$ 被视作 $h$ 的第一个输入,即 $n$ 比特的链式变量输入,它是图 \ref{fig:8-9} 中 $h$ 盒的\emph{底部}输入。对于 PRF $h_\mathrm{top}$,PRF密钥 $k$ 被视为 $h$ 的第二个输入,即 $\ell$ 比特的消息分组输入,它是图中 $h$ 盒的\emph{顶部}输入。为了使该图更容易理解,我们用一个 $\triangleright$ 符号来标记 $h$ 盒的输入,用于指示被视作 PRF 密钥的那一个输入。事实上,读者很容易注意到,我们将虚线框内的两个 $h$ 计算视作 PRF $h_\mathrm{top}$ 的计算,因此,图中标有 $k_1'$ 和 $k_2'$ 的值由 $k_1'\leftarrow h_\mathrm{top}(k_1,\mathrm{IV})$ 和 $k_1'\leftarrow h_\mathrm{top}(k_2,\mathrm{IV})$ 计算而来。图中所有其他的 $h$ 计算都被视作对 $h_\mathrm{bot}$ 的计算。

我们的假设是,$h_\mathrm{bot}$ 和 $h_\mathrm{top}$ 都是安全的 PRF。稍后,我们将使用理想密码模型对 Davies-Meyer 压缩函数证明这一假设(见 \ref{subsec:8-7-3} 小节)。

现在,我们先对以下结论给出一个简单的证明:
\begin{quote}
\emph{如果 $h_\mathrm{bot}$ 和 $h_\mathrm{top}$ 都是安全的 PRF,那么双层密钥嵌套也是安全的。}
\end{quote}

第一个观察是,密钥 $k_1$ 和 $k_2$ 只被用于派生 $k_1'$ 和 $k_2'$,即 $k_1'=h_\mathrm{top}(k_1,\mathrm{IV})$,$k_2'=h_\mathrm{top}(k_2,\mathrm{IV})$。假设 $h_\mathrm{top}$ 是一个安全的 PRF,就意味着在 PRF 攻击游戏中,我们可以有效地用真随机的 $n$ 比特序列替换 $k_1'$ 和 $k_2'$。图 \ref{fig:8-10} 展示了由此产生的构造。这里,我们所做的就是去掉图 \ref{fig:8-9} 中所有虚线框内的元素。这个新构造中的函数将两个密钥 $k_1'$ 和 $k_2'$ 以及一条消息 $M$ 作为输入。根据以上观察,我们只需证明图 \ref{fig:8-10} 中的构造是一个安全的 PRF 即可。

\begin{figure}
	\centering
	\tikzset{every picture/.style={line width=0.75pt}}

\begin{tikzpicture}[x=0.75pt,y=0.75pt,yscale=-1,xscale=1]


\draw  [fill={rgb, 255:red, 255; green, 255; blue, 255 }  ,fill opacity=1 ][line width=1.2] [general shadow={fill=black,shadow xshift=2.25pt,shadow yshift=-2.25pt}] (100,85) -- (100,110) -- (50,110) -- (50,70) -- cycle ;
\draw  [fill={rgb, 255:red, 255; green, 255; blue, 255 }  ,fill opacity=1 ][line width=1.2] [general shadow={fill=black,shadow xshift=2.25pt,shadow yshift=-2.25pt}] (270,85) -- (270,110) -- (220,110) -- (220,70) -- cycle ;
\draw  [fill={rgb, 255:red, 255; green, 255; blue, 255 }  ,fill opacity=1 ][line width=1.2] [general shadow={fill=black,shadow xshift=2.25pt,shadow yshift=-2.25pt}] (370,225) -- (370,250) -- (320,250) -- (320,210) -- cycle ;

\draw  [line width=1.2]  (0,10) -- (70,10) -- (70,30) -- (0,30) -- cycle ;
\draw  [line width=1.2]  (170,10) -- (240,10) -- (240,30) -- (170,30) -- cycle ;
\draw  [line width=1.2]  (270,150) -- (340,150) -- (340,170) -- (270,170) -- cycle ;

\draw    (100,100) -- (140,100) ;
\draw  [->]  (0,100) -- (49,100) ;
\draw  [->]  (180,100) -- (219,100) ;
\draw  [->]  (270,240) -- (319,240) ;
\draw  [->]  (370,240) -- (410,240) ;
\draw  [->]  (35,30) -- (35,80) -- (49,80) ;
\draw  [->]  (205,30) -- (205,80) -- (219,80) ;
\draw  [->]  (270,100.04) -- (305,100) -- (305,149) ;
\draw  [->]  (305,170) -- (305,220) -- (319,220) ;

\draw   (55,100) -- (50,103.5) -- (50,96.5) -- cycle ;
\draw   (225,100) -- (220,103.5) -- (220,96.5) -- cycle ;
\draw   (325,240) -- (320,243.5) -- (320,236.5) -- cycle ;

\draw  [dash pattern={on 0.84pt off 2.51pt}]  (140,100) -- (180,100) ;


\draw (75,95) node  [font=\small]  {$h$};
\draw (245,95) node  [font=\small]  {$h$};
\draw (35,20) node  [font=\small]  {$m_{1}$};
\draw (205,20) node  [font=\small]  {$m_{s} \ \| \ \mathrm{PB}_{\mathrm{i}}$};
\draw (2,96.6) node [anchor=south west] [inner sep=0.75pt]  [font=\small]  {$k'_{1}$};
\draw (120,20) node    {$\cdots $};
\draw (345,235) node  [font=\small]  {$h$};
\draw (272,236.6) node [anchor=south west] [inner sep=0.75pt]  [font=\small]  {$k'_{2}$};
\draw (305,160) node  [font=\small]  {$t\ \| \ \mathrm{PB}_{\mathrm{o}}$};
\draw (287.5,96.62) node [anchor=south] [inner sep=0.75pt]  [font=\small]  {$t$};


\end{tikzpicture}
	\caption{NMAC 的一个按比特版本}
	\label{fig:8-10}
\end{figure}

希望即使不看标题,读者也能认出图 \ref{fig:8-10} 中的构造正是将我们在 \ref{subsec:6-5-1} 小节中介绍的 NMAC 应用在 $h_\mathrm{bot}$ 上的结果(特别是,请注意图 \ref{fig:6-5-b})。实际上,图 \ref{fig:8-10} 中的构造是 NMAC 的一个按比特版本,我们可以借由填充从按分组版本中得到它(如 \ref{sec:6-8} 节所述)。因此,在假设 $h_\mathrm{bot}$ 是一个安全 PRF 的情况下,双层密钥嵌套的安全性可以直接由 NMAC 安全定理(定理 \ref{theo:6-7})得到。

\subsection{HMAC 标准}\label{subsec:8-7-2}

HMAC 标准与双层密钥嵌套(图 \ref{fig:8-9})基本相同,但有一个重要的区别:在 HMAC 中,密钥 $k_1$ 和 $k_2$ 并不相互独立,而是以一种有点特别的方式由单一密钥 $k$ 派生而来。

为了更详细地描述这一点,我们首先观察到,HMAC 本身在某种程度上是面向字节的,所以所有序列都是字节序列。底层 Merkle-Damg{\aa}rd 哈希的消息分组被假定为 $B$ 字节(而不是 $\ell$ 比特)。HMAC 的密钥 $k$ 是一个任意长度的序列。为了得到密钥 $k_1$ 和 $k_2$(都是长为 $B$字节的序列),我们首先迫使 $k$ 的长度正好是 $B$ 字节:如果 $k$ 的长度小于或等于 $B$ 字节,我们就用零字节来填充它;否则,我们就用零字节填充的 $H(k)$ 来代替它。然后,我们计算:
\[
k_1\leftarrow k\oplus\mathsf{ipad},
\qquad
k_2\leftarrow k\oplus\mathsf{opad}
\]
其中,$\mathsf{ipad}$ 和 $\mathsf{opad}$(`i' 和 `o' 分别代表``内部" 和 ``外部")都是 $B$ 字节的常量序列,定义如下:
\[
\begin{aligned}
& \mathsf{ipad}:=\texttt{0x36}\;\text{\emph{重复}}\;B\;\text{\emph{次}}\\
& \mathsf{opad}:=\texttt{0x5C}\;\text{\emph{重复}}\;B\;\text{\emph{次}}
\end{aligned}
\]

使用哈希函数 $H$ 实现的 HMAC 被称作 HMAC-$H$。在实践中最常用的 HMAC 是 HMAC-SHA1 和 HMAC-SHA256。HMAC 标准还允许对 HMAC 的输出进行截短。例如,当把 SHA1 的输出截短到 $80$ 比特时,相应的 HMAC 函数就被称作 HMAC-SHA1-80。举例来说,TLS 1.0 的实现就被要求能够支持 HMAC-SHA1-96。

\begin{snote}[HMAC 的安全性。]
由于密钥 $k_1'$ 和 $k_2'$ 是相关的(它们的异或就等于 $\mathsf{opad}\oplus\mathsf{ipad}$),我们为双层密钥嵌套所提出的安全证明就不再能够适用:在所述假设下,我们不能证明派生出的密钥 $k_1'$ 与 $k_2'$ 和随机元是不可区分的。一个解决方案是对压缩函数 $h$ 做一个更强的假设——我们需要假设 $h_\mathrm{top}$ 在关联密钥攻击下仍然是一个 PRF(如 Bellare 和 Kohno 的定义)。如果 $h$ 本身是一个 Davies-Meyer 压缩函数,那么在理想密码模型下,这个更强的假设仍能得到证明。
\end{snote}

\subsection{Davies-Meyer 在理想密码模型下是安全的 PRF}\label{subsec:8-7-3}

剩下的工作就是证明我们的假设,即式 \ref{eq:8-6} 中由 $h$ 派生的 $h_\mathrm{bot}$ 和 $h_\mathrm{top}$ 都是安全的 PRF。假设压缩函数 $h$ 是一个 Davies-Meyer 函数,即 $h(x,y):=E(y,x)\oplus x$,其中,$\mathcal{E}=(E,D)$ 是某个分组密码。那么:
\begin{itemize}
	\item $h_\mathrm{bot}(k,m):=h(k,m)=E(m,k)\oplus k$ 是一个定义在 $(\mathcal{X},\mathcal{K},\mathcal{X})$ 上的 PRF,并且
	\item $h_\mathrm{top}(k,m):=h(m,k)=E(k,m)\oplus m$ 是一个定义在 $(\mathcal{K},\mathcal{X},\mathcal{X})$ 上的 PRF。
\end{itemize}
当 $\mathcal{E}$ 是一个安全的分组密码时,$h_\mathrm{top}$ 显然是一个安全的 PRF(见练习 \ref{exer:4-1} 的 (c) 部分)。但 $h_\mathrm{bot}$ 是一个安全 PRF 的事实有点令人惊讶——作为 $h_\mathrm{bot}$ 输入的消息 $m$ 被用作 $E$ 的密钥。因此,即使 $\mathcal{E}$ 是一个安全的分组密码,$h_\mathrm{bot}$ 也无法提供安全性保证。尽管如此,我们仍然可以证明 $h_\mathrm{bot}$ 是一个安全的 PRF,但这需要在理想密码模型下。仅仅假设 $\mathcal{E}$ 是一个安全分组密码是不充分的。

如有必要,读者可以回顾 \ref{sec:4-7} 节中介绍的理想密码模型的基本概念。我们在本章的稍早部分也使用了理想密码模型(见 \ref{subsec:8-5-3} 小节)。

在理想密码模型中,我们启发式地将一个定义在 $(\mathcal{K},\mathcal{X})$ 上的分组密码 $\mathcal{E}=(E,D)$ 建模为一个随机置换族 $\{\Pi_\mathpzc{k}\}_{\mathpzc{k}\in\mathcal{K}}$。我们改编 PRF 攻击游戏 \ref{game:4-2},让它适用于理想密码模型。挑战者除了应答标准查询外,还要应答 $\Pi$-查询和 $\Pi^{-1}$-查询:一个$\Pi$-查询是一个数对 $(\mathpzc{k},\mathpzc{a})$,挑战者以 $\mathpzc{b}:=\Pi_\mathpzc{k}(\mathpzc{a})$ 作为应答。$\Pi^{-1}$-查询是一个数对 $(\mathpzc{k},\mathpzc{b})$,挑战者以 $\mathpzc{a}:=\Pi_\mathpzc{k}^{-1}(\mathpzc{b})$ 作为应答。对于一个标准查询 $m$,挑战者以 $v:=f(m)$ 作为应答:在攻击游戏的实验 $0$ 中,$f$ 是 $F(k,\cdot)$,其中的 $F$ 是一个 PRF,$k$ 是一个随机选出的密钥;在实验 $1$ 中,$f$ 是一个真随机函数。此外,在实验 $0$ 中,用于评估 $F$ 的随机置换就是在构建 $F$ 时用作 $E$ 和 $D$ 的置换。对于我们的 PRF,我们有 $h_\mathrm{bot}(k,m)=E(m,k)\oplus k=\Pi_m(k)\oplus k$。

对于对手 $\mathcal{A}$,我们将它在修改后的 PRF 攻击游戏中的优势定义为 $\mathrm{PRF}^\mathrm{ic}\mathsf{adv}[\mathcal{A},F]$,理想密码模型上的安全性意味着,该优势对于所有有效对手来说都可忽略不计。

\begin{theorem}[$h_\mathrm{bot}$ 的安全性]\label{theo:8-5}
令 $\mathcal{E}=(E,D)$ 是一个定义在 $(\mathcal{K},\mathcal{X})$ 上的分组密码,其中 $|\mathcal{X}|$ 是大的。那么在理想密码模型下,$h_\mathrm{bot}(k,m):=E(m,k)\oplus k$ 是一个安全的 PRF。
\begin{quote}
特别地,对于每个攻击 $h_\mathrm{bot}$ 的 PRF 对手 $\mathcal{A}$,如果它最多只能发起 $Q_\mathrm{ic}$ 次理想密码查询,我们就有:
\end{quote}
\[
\mathrm{PRF}^\mathrm{ic}\mathsf{adv}[\mathcal{A},h_\mathrm{bot}]\leq\frac{2Q_\mathrm{ic}}{|\mathcal{X}|}
\]
\end{theorem}

该定理中的上界是相当严格的,因为对密钥进行暴力搜索能非常接近该约束。

\begin{proof}
该证明与对 Evan-Mansour/$\mathcal{E}$X 构造的分析如出一辙(见 \ref{subsec:4-7-4} 小节的定理 \ref{theo:4-14}),特别是,该证明会用到领域分离定理(见第 \ref{subsec:4-7-4} 小节的定理 \ref{theo:4-15})。

令 $\mathcal{A}$ 是一个定理声明中的对手。记 $p_b$ 为 $\mathcal{A}$ 在攻击游戏 \ref{game:4-2} 的实验 $b$ 中输出 $1$ 的概率,其中 $b=0,1$。所以,根据定义,我们有:
\begin{equation}\label{eq:8-7}
\mathrm{PRF}^\mathrm{ic}\mathsf{adv}[\mathcal{A},h_\mathrm{bot}]=|p_0-p_1|
\end{equation}

下面,我们构造一个有两个游戏组成的游戏序列,并应用领域分离引理来证明该定理。

\vspace{5pt}

\noindent\textbf{游戏 $\mathbf{0}$。}
该游戏对应于理想密码模型下 PRF 攻击游戏的实验 $0$。我们可以将挑战者的逻辑表述如下:

\vspace{5pt}

\hspace*{5pt} 初始化:\\
\hspace*{50pt} 对于每个 $\mathpzc{k}\in\mathcal{K}$,随机选取 $\Pi_\mathpzc{k}\overset{\rm R}\leftarrow\mathrm{Perms}[\mathcal{X}]$\\
\hspace*{50pt} 随机选取 $k\overset{\rm R}\leftarrow\mathcal{X}$

\vspace{5pt}

\hspace*{5pt} 标准 $h_\mathrm{bot}$-查询 $m$:\\
\hspace*{26pt} 1.\qquad 令 $c\leftarrow\Pi_m(k)$\\
\hspace*{26pt} 2.\qquad 令 $v\leftarrow c\oplus k$\\
\hspace*{26pt} 3.\qquad 返回 $v$

\vspace{5pt}

游戏 $0$ 中的挑战者\emph{完全按照定理 \ref{theo:4-14} 的证明中的游戏 $0$} 来处理理想密码查询:

\vspace{5pt}

\hspace*{5pt} 理想密码 $\Pi$-查询 $\mathpzc{k},\mathpzc{a}$:\\
\hspace*{26pt} 1.\qquad 令 $\mathpzc{b}\leftarrow\Pi_\mathpzc{k}(\mathpzc{a})$\\
\hspace*{26pt} 2.\qquad 返回 $\mathpzc{b}$

\vspace{5pt}

\hspace*{5pt} 理想密码 $\Pi^{-1}$-查询 $\mathpzc{k},\mathpzc{b}$:\\
\hspace*{26pt} 1.\qquad 令 $\mathpzc{a}\leftarrow\Pi_\mathpzc{k}^{-1}(\mathpzc{b})$\\
\hspace*{26pt} 2.\qquad 返回 $\mathpzc{a}$

\vspace{5pt}

令 $W_0$ 为 $\mathcal{A}$ 在游戏 $0$ 结束时输出 $1$ 的事件。从构造上看,显然有:
\begin{equation}\label{eq:8-8}
\Pr[W_0]=p_0
\end{equation}

\noindent\textbf{游戏 $\mathbf{1}$。}
同定理 \ref{theo:4-14} 的证明一样,我们``凭空"宣布,标准查询和理想密码查询都是使用相互独立的随机置换处理的。下面是具体的逻辑,与游戏 $0$ 不同的部分被高亮强调:

\vspace{5pt}

\hspace*{5pt} 初始化:\\
\hspace*{50pt} 对于每个 $\mathpzc{k}\in\mathcal{K}$,随机选取 \colorbox{gray!50}{$\Pi_{\mathrm{std},\mathpzc{k}}\overset{\rm R}\leftarrow\mathrm{Perms}[\mathcal{X}]$ 和 $\Pi_{\mathrm{ic},\mathpzc{k}}\overset{\rm R}\leftarrow\mathrm{Perms}[\mathcal{X}]$}\\
\hspace*{50pt} 随机选取 $k\overset{\rm R}\leftarrow\mathcal{X}$

\vspace{5pt}

\hspace*{5pt} 标准 $h_\mathrm{bot}$-查询 $m$:\\
\hspace*{26pt} 1.\qquad 令 $c\leftarrow$\colorbox{gray!50}{$\Pi_{\mathrm{std},m}(k)$}
\quad//\quad\emph{将 $k$ 添加到 $\Pi_{\mathrm{std},m}$ 的采样领域,将 $c$ 添加到 $\Pi_{\mathrm{std},m}$ 的采样范围}\\
\hspace*{26pt} 2.\qquad 令 $v\leftarrow c\oplus k$\\
\hspace*{26pt} 3.\qquad 返回 $v$

\vspace{5pt}

游戏 $1$ 中的挑战者\emph{完全按照定理 \ref{theo:4-14} 的证明中的游戏 $1$} 来处理理想密码查询:

\vspace{5pt}

\hspace*{5pt} 理想密码 $\Pi$-查询 $\mathpzc{k},\mathpzc{a}$:\\
\hspace*{26pt} 1.\qquad 令 $\mathpzc{b}\leftarrow$\colorbox{gray!50}{$\Pi_{\mathrm{ic},\mathpzc{k}}(\mathpzc{a})$}
\;\;\quad//\quad\emph{将 $\mathpzc{a}$ 添加到 $\Pi_{\mathrm{ic},\mathpzc{k}}$ 的采样领域,将 $\mathpzc{b}$ 添加到 $\Pi_{\mathrm{ic},\mathpzc{k}}$ 的采样范围}\\
\hspace*{26pt} 2.\qquad 返回 $\mathpzc{b}$

\vspace{5pt}

\hspace*{5pt} 理想密码 $\Pi^{-1}$-查询 $\mathpzc{k},\mathpzc{b}$:\\
\hspace*{26pt} 1.\qquad 令 $\mathpzc{a}\leftarrow$\colorbox{gray!50}{$\Pi_{\mathrm{ic},\mathpzc{k}}^{-1}(\mathpzc{b})$}
\;\;\quad//\quad\emph{将 $\mathpzc{a}$ 添加到 $\Pi_{\mathrm{ic},\mathpzc{k}}$ 的采样领域,将 $\mathpzc{b}$ 添加到 $\Pi_{\mathrm{ic},\mathpzc{k}}$ 的采样范围}\\
\hspace*{26pt} 2.\qquad 返回 $\mathpzc{a}$

\vspace{5pt}

令 $W_1$ 为 $\mathcal{A}$ 在游戏 $1$ 结束时输出 $1$ 的事件。考虑游戏 $1$ 中一次标准查询的一个输入/输出对 $(m,v)$。注意到,$k$ 是唯一一个被添加到 $\Pi_{\mathrm{std},m}(k)$ 的采样领域中的项,而 $c=v\oplus k$ 是唯一一个被添加到 $\Pi_{\mathrm{std},m}(k)$ 的采样范围中的项。特别是,$c$ 是随机生成的,而 $k$ 始终保持完美隐藏(即与对手的观察无关)。

因此,从对手的角度来看,标准查询的行为与随机函数相同,而理想密码查询的行为与一个\emph{独立的}理想密码的理想密码查询相同。特别地,我们有:
\begin{equation}\label{eq:8-9}
\Pr[W_1]=p_1
\end{equation}

最后,我们使用领域分离引理来分析 $|\Pr[W_0]-\Pr[W_1]|$。领域分离失败事件 $Z$ 是指在游戏 $1$ 中,$\Pi_{\mathrm{std},m}$ 的一个采样领域与 $\Pi_{\mathrm{ic},\mathpzc{k}}$ 的一个采样领域重叠,或者 $\Pi_{\mathrm{std},m}$ 的一个采样范围与 $\Pi_{\mathrm{ic},\mathpzc{k}}$ 的一个采样范围重叠的事件。领域分离引理告诉我们,有:
\begin{equation}\label{eq:8-10}
|\Pr[W_0]-\Pr[W_1]|\leq\Pr[Z]
\end{equation}

如果事件 $Z$ 发生,那么对于某个对应于一个理想密码查询的输入/输出三元组 $(\mathpzc{k},\mathpzc{a},\mathpzc{b})$,$\mathpzc{k}=m$ 是一个输出为 $v$ 的标准查询的输入,并且:
\begin{enumerate}[(i)]
	\item $\mathpzc{a}=k$,或者
	\item $\mathpzc{b}=v\oplus k$
\end{enumerate}
成立。对于任何固定三元组 $(\mathpzc{k},\mathpzc{a},\mathpzc{b})$,根据 $k$ 的独立性,条件 (i) 和 (ii) 成立的概率都是 $1/|\mathcal{X}|$,因此,根据联合约束,我们有:
\begin{equation}\label{eq:8-11}
\Pr[Z]\leq\frac{2Q_\mathrm{ic}}{|\mathcal{X}|}
\end{equation}

于是,根据式 \ref{eq:8-7},\ref{eq:8-8},\ref{eq:8-9},\ref{eq:8-10} 和 \ref{eq:8-11},该定理成立。
\end{proof} 