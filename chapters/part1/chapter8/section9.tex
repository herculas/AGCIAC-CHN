\section{Merkle树:证明哈希列表的属性}\label{sec:8-9}

既然我们已经了解了如何构建抗碰撞函数,那么,让我们看看它们在数据完整性方面都有哪些应用。考虑一个大的可执行文件,它以 $\ell$ 比特的短分组 $x_1,\dots,x_n$ 的形式被存储在磁盘上。操作系统每次加载和运行这个可执行文件之前,都需要验证其内容没有被篡改。在本章的前言部分,我们讨论了如何在只读存储器中保存整个文件的短哈希值\footnotemark[3]。每当要运行文件时,系统首先重新计算文件的哈希值,并验证它是否与存储中的哈希值一致。我们说,抗碰撞哈希能确保对手无法篡改文件而不被发现。问题是,对于一个大文件来说,计算整个文件的哈希可能需要相当长的时间,而这将大大增加启动可执行文件的耗时。

\footnotetext[3]{回顾一下,只读存储器可以被对手读取,但是不能被修改。它可以被实现为一个独立的系统,向任何要求获得数据的人提供数据。或者,更简单地说,它可以通过使用数字签名方案(见第\ref{chap:13}章)签署数据来实现,并离线存储签名密钥。}

可以改进这一点吗?为了开始运行可执行文件,系统只需验证第一个分组 $x_1$。当执行流转移到其他分组时,系统再去验证相应的分组,以此类推。换句话说,与其一次性验证整个文件,不如让系统在加载相关内容时再去验证它。对于这种设计思路,一个可行的方案是预先计算所有分组 $x_1,\dots,x_n$ 的哈希值,并将产生的 $n$ 个哈希值存储在只读存储器中。这样,验证每个分组都很容易,但问题是,存储这 $n$ 个哈希值需要占用大量的只读空间。幸运的是,我们有一个更好的解决方案。

\begin{snote}[Merkle 树。]
为了重述这个的问题,我们有一个包含 $n$ 项的向量 $T:=(x_1,\dots,x_n)\in\mathcal{X}^n$,我们希望计算这个向量的一个短哈希值 $y$。之后,我们可能会被给予哈希值 $y$ 和数对 $(i,x)$,其中 $1\leq i\leq n$。我们需要验证 $x$ 是否是 $T$ 中的第 $i$ 项。

这个问题的解决方案是一个聪明的数据结构,称为 \textbf{Merkle 树},如图 \ref{fig:8-12} 所示。由此产生的哈希函数 $H$ 被称为 \textbf{Merkle 树哈希 (Merkle tree hash)}。

\begin{figure}
	\centering
	

\tikzset{every picture/.style={line width=0.75pt}} %set default line width to 0.75pt        

\begin{tikzpicture}[x=0.75pt,y=0.75pt,yscale=-1,xscale=1]
%uncomment if require: \path (0,249); %set diagram left start at 0, and has height of 249

%Straight Lines [id:da702920327532164] 
\draw [fill={rgb, 255:red, 255; green, 255; blue, 255 }  ,fill opacity=1 ][line width=0.75]    (222.4,54.69) -- (115,90) ;
\draw [shift={(225.25,53.76)}, rotate = 161.8] [fill={rgb, 255:red, 0; green, 0; blue, 0 }  ][line width=0.08]  [draw opacity=0] (7.14,-3.43) -- (0,0) -- (7.14,3.43) -- cycle    ;
%Straight Lines [id:da3073933646499085] 
\draw [fill={rgb, 255:red, 255; green, 255; blue, 255 }  ,fill opacity=1 ][line width=0.75]    (247.84,53.96) -- (355,90) ;
\draw [shift={(245,53.01)}, rotate = 18.59] [fill={rgb, 255:red, 0; green, 0; blue, 0 }  ][line width=0.08]  [draw opacity=0] (7.14,-3.43) -- (0,0) -- (7.14,3.43) -- cycle    ;
%Straight Lines [id:da6120087870514785] 
\draw [fill={rgb, 255:red, 255; green, 255; blue, 255 }  ,fill opacity=1 ][line width=0.75]    (25,170) -- (47.19,140.65) ;
\draw [shift={(49,138.26)}, rotate = 127.09] [fill={rgb, 255:red, 0; green, 0; blue, 0 }  ][line width=0.08]  [draw opacity=0] (7.14,-3.43) -- (0,0) -- (7.14,3.43) -- cycle    ;
%Straight Lines [id:da5183313825045153] 
\draw [fill={rgb, 255:red, 255; green, 255; blue, 255 }  ,fill opacity=1 ][line width=0.75]    (63.29,140.91) -- (85,170) ;
\draw [shift={(61.5,138.51)}, rotate = 53.27] [fill={rgb, 255:red, 0; green, 0; blue, 0 }  ][line width=0.08]  [draw opacity=0] (7.14,-3.43) -- (0,0) -- (7.14,3.43) -- cycle    ;
%Straight Lines [id:da3638120020023241] 
\draw [fill={rgb, 255:red, 255; green, 255; blue, 255 }  ,fill opacity=1 ][line width=0.75]    (166.71,140.91) -- (145,170) ;
\draw [shift={(168.5,138.51)}, rotate = 126.73] [fill={rgb, 255:red, 0; green, 0; blue, 0 }  ][line width=0.08]  [draw opacity=0] (7.14,-3.43) -- (0,0) -- (7.14,3.43) -- cycle    ;
%Straight Lines [id:da32018500957267415] 
\draw [fill={rgb, 255:red, 255; green, 255; blue, 255 }  ,fill opacity=1 ][line width=0.75]    (182.83,141.14) -- (205,170) ;
\draw [shift={(181,138.76)}, rotate = 52.47] [fill={rgb, 255:red, 0; green, 0; blue, 0 }  ][line width=0.08]  [draw opacity=0] (7.14,-3.43) -- (0,0) -- (7.14,3.43) -- cycle    ;
%Straight Lines [id:da11829710873612775] 
\draw [fill={rgb, 255:red, 255; green, 255; blue, 255 }  ,fill opacity=1 ][line width=0.75]    (103.26,97.68) -- (55,130) ;
\draw [shift={(105.75,96.01)}, rotate = 146.19] [fill={rgb, 255:red, 0; green, 0; blue, 0 }  ][line width=0.08]  [draw opacity=0] (7.14,-3.43) -- (0,0) -- (7.14,3.43) -- cycle    ;
%Straight Lines [id:da5550208993570565] 
\draw [fill={rgb, 255:red, 255; green, 255; blue, 255 }  ,fill opacity=1 ][line width=0.75]    (126.27,98.63) -- (175,130) ;
\draw [shift={(123.75,97.01)}, rotate = 32.77] [fill={rgb, 255:red, 0; green, 0; blue, 0 }  ][line width=0.08]  [draw opacity=0] (7.14,-3.43) -- (0,0) -- (7.14,3.43) -- cycle    ;
%Shape: Circle [id:dp6920813762808213] 
\draw  [fill={rgb, 255:red, 255; green, 255; blue, 255 }  ,fill opacity=1 ][line width=0.75] [general shadow={fill=black,shadow xshift=1.5pt,shadow yshift=-1.5pt}] (15,170) .. controls (15,164.48) and (19.48,160) .. (25,160) .. controls (30.52,160) and (35,164.48) .. (35,170) .. controls (35,175.52) and (30.52,180) .. (25,180) .. controls (19.48,180) and (15,175.52) .. (15,170) -- cycle ;
%Shape: Circle [id:dp9663292169740727] 
\draw  [fill={rgb, 255:red, 255; green, 255; blue, 255 }  ,fill opacity=1 ][line width=0.75] [general shadow={fill=black,shadow xshift=1.5pt,shadow yshift=-1.5pt}] (75,170) .. controls (75,164.48) and (79.48,160) .. (85,160) .. controls (90.52,160) and (95,164.48) .. (95,170) .. controls (95,175.52) and (90.52,180) .. (85,180) .. controls (79.48,180) and (75,175.52) .. (75,170) -- cycle ;
%Shape: Circle [id:dp6698750169919838] 
\draw  [fill={rgb, 255:red, 255; green, 255; blue, 255 }  ,fill opacity=1 ][line width=0.75] [general shadow={fill=black,shadow xshift=1.5pt,shadow yshift=-1.5pt}] (135,170) .. controls (135,164.48) and (139.48,160) .. (145,160) .. controls (150.52,160) and (155,164.48) .. (155,170) .. controls (155,175.52) and (150.52,180) .. (145,180) .. controls (139.48,180) and (135,175.52) .. (135,170) -- cycle ;
%Shape: Circle [id:dp408527637149501] 
\draw  [fill={rgb, 255:red, 255; green, 255; blue, 255 }  ,fill opacity=1 ][line width=0.75] [general shadow={fill=black,shadow xshift=1.5pt,shadow yshift=-1.5pt}] (195,170) .. controls (195,164.48) and (199.48,160) .. (205,160) .. controls (210.52,160) and (215,164.48) .. (215,170) .. controls (215,175.52) and (210.52,180) .. (205,180) .. controls (199.48,180) and (195,175.52) .. (195,170) -- cycle ;
%Shape: Circle [id:dp9073791707119769] 
\draw  [fill={rgb, 255:red, 255; green, 255; blue, 255 }  ,fill opacity=1 ][line width=0.75] [general shadow={fill=black,shadow xshift=1.5pt,shadow yshift=-1.5pt}] (45,130) .. controls (45,124.48) and (49.48,120) .. (55,120) .. controls (60.52,120) and (65,124.48) .. (65,130) .. controls (65,135.52) and (60.52,140) .. (55,140) .. controls (49.48,140) and (45,135.52) .. (45,130) -- cycle ;
%Shape: Circle [id:dp02478093212823307] 
\draw  [fill={rgb, 255:red, 255; green, 255; blue, 255 }  ,fill opacity=1 ][line width=0.75] [general shadow={fill=black,shadow xshift=1.5pt,shadow yshift=-1.5pt}] (165,130) .. controls (165,124.48) and (169.48,120) .. (175,120) .. controls (180.52,120) and (185,124.48) .. (185,130) .. controls (185,135.52) and (180.52,140) .. (175,140) .. controls (169.48,140) and (165,135.52) .. (165,130) -- cycle ;
%Shape: Circle [id:dp6156110109492565] 
\draw  [fill={rgb, 255:red, 255; green, 255; blue, 255 }  ,fill opacity=1 ][line width=0.75] [general shadow={fill=black,shadow xshift=1.5pt,shadow yshift=-1.5pt}] (105,90) .. controls (105,84.48) and (109.48,80) .. (115,80) .. controls (120.52,80) and (125,84.48) .. (125,90) .. controls (125,95.52) and (120.52,100) .. (115,100) .. controls (109.48,100) and (105,95.52) .. (105,90) -- cycle ;
%Straight Lines [id:da24603915152063838] 
\draw [fill={rgb, 255:red, 255; green, 255; blue, 255 }  ,fill opacity=1 ][line width=0.75]    (286.94,140.9) -- (265,170) ;
\draw [shift={(288.75,138.51)}, rotate = 127.02] [fill={rgb, 255:red, 0; green, 0; blue, 0 }  ][line width=0.08]  [draw opacity=0] (7.14,-3.43) -- (0,0) -- (7.14,3.43) -- cycle    ;
%Straight Lines [id:da9115181138506314] 
\draw [fill={rgb, 255:red, 255; green, 255; blue, 255 }  ,fill opacity=1 ][line width=0.75]    (302.59,141.13) -- (325,170) ;
\draw [shift={(300.75,138.76)}, rotate = 52.18] [fill={rgb, 255:red, 0; green, 0; blue, 0 }  ][line width=0.08]  [draw opacity=0] (7.14,-3.43) -- (0,0) -- (7.14,3.43) -- cycle    ;
%Straight Lines [id:da9185055776150515] 
\draw [fill={rgb, 255:red, 255; green, 255; blue, 255 }  ,fill opacity=1 ][line width=0.75]    (406.71,140.91) -- (385,170) ;
\draw [shift={(408.5,138.51)}, rotate = 126.73] [fill={rgb, 255:red, 0; green, 0; blue, 0 }  ][line width=0.08]  [draw opacity=0] (7.14,-3.43) -- (0,0) -- (7.14,3.43) -- cycle    ;
%Straight Lines [id:da263720662321004] 
\draw [fill={rgb, 255:red, 255; green, 255; blue, 255 }  ,fill opacity=1 ][line width=0.75]    (423.08,141.63) -- (445,170) ;
\draw [shift={(421.25,139.26)}, rotate = 52.31] [fill={rgb, 255:red, 0; green, 0; blue, 0 }  ][line width=0.08]  [draw opacity=0] (7.14,-3.43) -- (0,0) -- (7.14,3.43) -- cycle    ;
%Straight Lines [id:da320460555452849] 
\draw [fill={rgb, 255:red, 255; green, 255; blue, 255 }  ,fill opacity=1 ][line width=0.75]    (343.75,97.67) -- (295,130) ;
\draw [shift={(346.25,96.01)}, rotate = 146.45] [fill={rgb, 255:red, 0; green, 0; blue, 0 }  ][line width=0.08]  [draw opacity=0] (7.14,-3.43) -- (0,0) -- (7.14,3.43) -- cycle    ;
%Straight Lines [id:da4257765400493081] 
\draw [fill={rgb, 255:red, 255; green, 255; blue, 255 }  ,fill opacity=1 ][line width=0.75]    (366.76,98.4) -- (415,130) ;
\draw [shift={(364.25,96.76)}, rotate = 33.23] [fill={rgb, 255:red, 0; green, 0; blue, 0 }  ][line width=0.08]  [draw opacity=0] (7.14,-3.43) -- (0,0) -- (7.14,3.43) -- cycle    ;
%Shape: Circle [id:dp1109096379469161] 
\draw  [fill={rgb, 255:red, 255; green, 255; blue, 255 }  ,fill opacity=1 ][line width=0.75] [general shadow={fill=black,shadow xshift=1.5pt,shadow yshift=-1.5pt}] (255,170) .. controls (255,164.48) and (259.48,160) .. (265,160) .. controls (270.52,160) and (275,164.48) .. (275,170) .. controls (275,175.52) and (270.52,180) .. (265,180) .. controls (259.48,180) and (255,175.52) .. (255,170) -- cycle ;
%Shape: Circle [id:dp6670429507855935] 
\draw  [fill={rgb, 255:red, 255; green, 255; blue, 255 }  ,fill opacity=1 ][line width=0.75] [general shadow={fill=black,shadow xshift=1.5pt,shadow yshift=-1.5pt}] (315,170) .. controls (315,164.48) and (319.48,160) .. (325,160) .. controls (330.52,160) and (335,164.48) .. (335,170) .. controls (335,175.52) and (330.52,180) .. (325,180) .. controls (319.48,180) and (315,175.52) .. (315,170) -- cycle ;
%Shape: Circle [id:dp9489763101292301] 
\draw  [fill={rgb, 255:red, 255; green, 255; blue, 255 }  ,fill opacity=1 ][line width=0.75] [general shadow={fill=black,shadow xshift=1.5pt,shadow yshift=-1.5pt}] (375,170) .. controls (375,164.48) and (379.48,160) .. (385,160) .. controls (390.52,160) and (395,164.48) .. (395,170) .. controls (395,175.52) and (390.52,180) .. (385,180) .. controls (379.48,180) and (375,175.52) .. (375,170) -- cycle ;
%Shape: Circle [id:dp17009001398474677] 
\draw  [fill={rgb, 255:red, 255; green, 255; blue, 255 }  ,fill opacity=1 ][line width=0.75] [general shadow={fill=black,shadow xshift=1.5pt,shadow yshift=-1.5pt}] (435,170) .. controls (435,164.48) and (439.48,160) .. (445,160) .. controls (450.52,160) and (455,164.48) .. (455,170) .. controls (455,175.52) and (450.52,180) .. (445,180) .. controls (439.48,180) and (435,175.52) .. (435,170) -- cycle ;
%Shape: Circle [id:dp311641882516994] 
\draw  [fill={rgb, 255:red, 255; green, 255; blue, 255 }  ,fill opacity=1 ][line width=0.75] [general shadow={fill=black,shadow xshift=1.5pt,shadow yshift=-1.5pt}] (285,130) .. controls (285,124.48) and (289.48,120) .. (295,120) .. controls (300.52,120) and (305,124.48) .. (305,130) .. controls (305,135.52) and (300.52,140) .. (295,140) .. controls (289.48,140) and (285,135.52) .. (285,130) -- cycle ;
%Shape: Circle [id:dp32731070719881616] 
\draw  [fill={rgb, 255:red, 255; green, 255; blue, 255 }  ,fill opacity=1 ][line width=0.75] [general shadow={fill=black,shadow xshift=1.5pt,shadow yshift=-1.5pt}] (405,130) .. controls (405,124.48) and (409.48,120) .. (415,120) .. controls (420.52,120) and (425,124.48) .. (425,130) .. controls (425,135.52) and (420.52,140) .. (415,140) .. controls (409.48,140) and (405,135.52) .. (405,130) -- cycle ;
%Shape: Circle [id:dp6495027217472407] 
\draw  [fill={rgb, 255:red, 255; green, 255; blue, 255 }  ,fill opacity=1 ][line width=0.75] [general shadow={fill=black,shadow xshift=1.5pt,shadow yshift=-1.5pt}] (345,90) .. controls (345,84.48) and (349.48,80) .. (355,80) .. controls (360.52,80) and (365,84.48) .. (365,90) .. controls (365,95.52) and (360.52,100) .. (355,100) .. controls (349.48,100) and (345,95.52) .. (345,90) -- cycle ;
%Shape: Circle [id:dp7240544767999253] 
\draw  [fill={rgb, 255:red, 255; green, 255; blue, 255 }  ,fill opacity=1 ][line width=0.75] [general shadow={fill=black,shadow xshift=1.5pt,shadow yshift=-1.5pt}] (225,50) .. controls (225,44.48) and (229.48,40) .. (235,40) .. controls (240.52,40) and (245,44.48) .. (245,50) .. controls (245,55.52) and (240.52,60) .. (235,60) .. controls (229.48,60) and (225,55.52) .. (225,50) -- cycle ;
%Shape: Rectangle [id:dp8463728262042536] 
\draw   (5,200) -- (45,200) -- (45,215) -- (5,215) -- cycle ;
%Shape: Rectangle [id:dp37573295720289646] 
\draw   (65,200) -- (105,200) -- (105,215) -- (65,215) -- cycle ;
%Shape: Rectangle [id:dp39722658574356395] 
\draw  [line width=1.5]  (125,200) -- (165,200) -- (165,215) -- (125,215) -- cycle ;
%Shape: Rectangle [id:dp7495732290873869] 
\draw   (185,200) -- (225,200) -- (225,215) -- (185,215) -- cycle ;
%Shape: Rectangle [id:dp5653373151711263] 
\draw   (245,200) -- (285,200) -- (285,215) -- (245,215) -- cycle ;
%Shape: Rectangle [id:dp25965689342656884] 
\draw   (305,200) -- (345,200) -- (345,215) -- (305,215) -- cycle ;
%Shape: Rectangle [id:dp14199578106093447] 
\draw   (365,200) -- (405,200) -- (405,215) -- (365,215) -- cycle ;
%Shape: Rectangle [id:dp8809655361412605] 
\draw   (425,200) -- (465,200) -- (465,215) -- (425,215) -- cycle ;
%Straight Lines [id:da10877585246462518] 
\draw [fill={rgb, 255:red, 255; green, 255; blue, 255 }  ,fill opacity=1 ][line width=0.75]    (25,200) -- (25,183) ;
\draw [shift={(25,180)}, rotate = 90] [fill={rgb, 255:red, 0; green, 0; blue, 0 }  ][line width=0.08]  [draw opacity=0] (7.14,-3.43) -- (0,0) -- (7.14,3.43) -- cycle    ;
%Straight Lines [id:da44045096101998715] 
\draw [fill={rgb, 255:red, 255; green, 255; blue, 255 }  ,fill opacity=1 ][line width=0.75]    (85,200) -- (85,183) ;
\draw [shift={(85,180)}, rotate = 90] [fill={rgb, 255:red, 0; green, 0; blue, 0 }  ][line width=0.08]  [draw opacity=0] (7.14,-3.43) -- (0,0) -- (7.14,3.43) -- cycle    ;
%Straight Lines [id:da342077588771736] 
\draw [fill={rgb, 255:red, 255; green, 255; blue, 255 }  ,fill opacity=1 ][line width=0.75]    (145,200) -- (145,183) ;
\draw [shift={(145,180)}, rotate = 90] [fill={rgb, 255:red, 0; green, 0; blue, 0 }  ][line width=0.08]  [draw opacity=0] (7.14,-3.43) -- (0,0) -- (7.14,3.43) -- cycle    ;
%Straight Lines [id:da5894662403845001] 
\draw [fill={rgb, 255:red, 255; green, 255; blue, 255 }  ,fill opacity=1 ][line width=0.75]    (205,200) -- (205,183) ;
\draw [shift={(205,180)}, rotate = 90] [fill={rgb, 255:red, 0; green, 0; blue, 0 }  ][line width=0.08]  [draw opacity=0] (7.14,-3.43) -- (0,0) -- (7.14,3.43) -- cycle    ;
%Straight Lines [id:da6075272503855431] 
\draw [fill={rgb, 255:red, 255; green, 255; blue, 255 }  ,fill opacity=1 ][line width=0.75]    (265,200) -- (265,183) ;
\draw [shift={(265,180)}, rotate = 90] [fill={rgb, 255:red, 0; green, 0; blue, 0 }  ][line width=0.08]  [draw opacity=0] (7.14,-3.43) -- (0,0) -- (7.14,3.43) -- cycle    ;
%Straight Lines [id:da7118849250215102] 
\draw [fill={rgb, 255:red, 255; green, 255; blue, 255 }  ,fill opacity=1 ][line width=0.75]    (325,200) -- (325,183) ;
\draw [shift={(325,180)}, rotate = 90] [fill={rgb, 255:red, 0; green, 0; blue, 0 }  ][line width=0.08]  [draw opacity=0] (7.14,-3.43) -- (0,0) -- (7.14,3.43) -- cycle    ;
%Straight Lines [id:da7937224127475853] 
\draw [fill={rgb, 255:red, 255; green, 255; blue, 255 }  ,fill opacity=1 ][line width=0.75]    (385,200) -- (385,183) ;
\draw [shift={(385,180)}, rotate = 90] [fill={rgb, 255:red, 0; green, 0; blue, 0 }  ][line width=0.08]  [draw opacity=0] (7.14,-3.43) -- (0,0) -- (7.14,3.43) -- cycle    ;
%Straight Lines [id:da9640722903578183] 
\draw [fill={rgb, 255:red, 255; green, 255; blue, 255 }  ,fill opacity=1 ][line width=0.75]    (445,200) -- (445,183) ;
\draw [shift={(445,180)}, rotate = 90] [fill={rgb, 255:red, 0; green, 0; blue, 0 }  ][line width=0.08]  [draw opacity=0] (7.14,-3.43) -- (0,0) -- (7.14,3.43) -- cycle    ;
%Straight Lines [id:da7095996565493987] 
\draw [fill={rgb, 255:red, 255; green, 255; blue, 255 }  ,fill opacity=1 ][line width=0.75]    (235,40) -- (235,13) ;
\draw [shift={(235,10)}, rotate = 90] [fill={rgb, 255:red, 0; green, 0; blue, 0 }  ][line width=0.08]  [draw opacity=0] (7.14,-3.43) -- (0,0) -- (7.14,3.43) -- cycle    ;

% Text Node
\draw (115,90) node    {$h$};
% Text Node
\draw (235,50) node    {$h$};
% Text Node
\draw (55,130) node    {$h$};
% Text Node
\draw (25,170) node    {$h$};
% Text Node
\draw (85,170) node    {$h$};
% Text Node
\draw (145,170) node    {$h$};
% Text Node
\draw (205,170) node    {$h$};
% Text Node
\draw (265,170) node    {$h$};
% Text Node
\draw (325,170) node    {$h$};
% Text Node
\draw (385,170) node    {$h$};
% Text Node
\draw (445,170) node    {$h$};
% Text Node
\draw (175,130) node    {$h$};
% Text Node
\draw (295,130) node    {$h$};
% Text Node
\draw (415,130) node    {$h$};
% Text Node
\draw (355,90) node    {$h$};
% Text Node
\draw (25,207.5) node    {$x_{1}$};
% Text Node
\draw (85,207.5) node    {$x_{2}$};
% Text Node
\draw (145,207.5) node    {$x_{3}$};
% Text Node
\draw (205,207.5) node    {$x_{4}$};
% Text Node
\draw (265,207.5) node    {$x_{5}$};
% Text Node
\draw (325,207.5) node    {$x_{6}$};
% Text Node
\draw (385,207.5) node    {$x_{7}$};
% Text Node
\draw (445,207.5) node    {$x_{8}$};
% Text Node
\draw (37,152.73) node [anchor=south east] [inner sep=0.75pt]    {$y_{1}$};
% Text Node
\draw (80.38,114.6) node [anchor=south east] [inner sep=0.75pt]    {\colorbox{gray!50}{$y_{9}$}};
% Text Node
\draw (170.13,70.48) node [anchor=south east] [inner sep=0.75pt]    {$y_{13}$};
% Text Node
\draw (156.75,152.85) node [anchor=south east] [inner sep=0.75pt]    {$y_{3}$};
% Text Node
\draw (276.88,152.85) node [anchor=south east] [inner sep=0.75pt]    {$y_{5}$};
% Text Node
\draw (396.75,152.85) node [anchor=south east] [inner sep=0.75pt]    {$y_{7}$};
% Text Node
\draw (320.63,111.6) node [anchor=south east] [inner sep=0.75pt]    {$y_{11}$};
% Text Node
\draw (73.25,152.85) node [anchor=south west] [inner sep=0.75pt]    {$y_{2}$};
% Text Node
\draw (193,155.98) node [anchor=south west] [inner sep=0.75pt]    {\colorbox{gray!50}{$y_{4}$}};
% Text Node
\draw (312.88,152.98) node [anchor=south west] [inner sep=0.75pt]    {$y_{6}$};
% Text Node
\draw (433.13,153.23) node [anchor=south west] [inner sep=0.75pt]    {$y_{8}$};
% Text Node
\draw (149.38,112.1) node [anchor=south west] [inner sep=0.75pt]    {$y_{10}$};
% Text Node
\draw (389.63,111.98) node [anchor=south west] [inner sep=0.75pt]    {$y_{12}$};
% Text Node
\draw (300,73.1) node [anchor=south west] [inner sep=0.75pt]    {\colorbox{gray!50}{$y_{14}$}};
% Text Node
\draw (237,27) node [anchor=west] [inner sep=0.75pt]    {$y_{15}$};


\end{tikzpicture}
	\caption{一颗含有 $8$ 个叶子结点的 Merkle 树。$y_4,y_9,y_{14}$ 证明了 $x_3$ 的真实性。}
	\label{fig:8-12}
\end{figure}

Merkle 树哈希使用一个抗碰撞哈希函数 $h$,例如 SHA256,它输出一个 $\mathcal{Y}$ 上的元素。$h$ 的输入是一个 $\mathcal{X}$ 上的元素,或者一对 $\mathcal{Y}$ 上的元素。由 $h$ 派生的 Merkle 树哈希 $H$ 定义在 $(\mathcal{X}^n,\mathcal{Y})$ 上。简单起见,我们假设 $n$ 是 $2$ 的整数次幂。Merkle 树哈希的工作原理如图 \ref{fig:8-12} 所示:想要计算 $(x_1,\dots,x_n)\in\mathcal{X}^n$ 的哈希,首先对 $n$ 个输入元素计算 $h$,得到 $(y_1,\dots,y_n)\in\mathcal{Y}^n$,然后根据这些元素建立一个哈希树,如图所示。更确切地说,哈希函数 $H$ 的定义如下:

\vspace{5pt}

\hspace*{5pt} 输入:$x_1,\dots,x_n\in\mathcal{X}$,其中 $n$ 是 $2$ 的整数幂\\
\hspace*{26pt} 输出:$y\in\mathcal{Y}$

\vspace{5pt}

\hspace*{5pt} 对于 $i=1,\dots,n$:
\hspace*{24pt} 令 $y_i\leftarrow h(x_i)$
\hspace*{65.5pt} // \emph{初始化} $y_1,\cdots,y_n$\\
\hspace*{26pt} 对于 $j=1,\dots,n-1$:
\hspace*{5pt} 令 $y_{j+n}=h(y_{2j-1},\;y_{2j})$
\hspace*{20pt} // \emph{计算树结点} $y_{n+1},\dots,y_{2n-1}$

\vspace{5pt}

\hspace*{5pt} 输出 $y_{2n-1}\in\mathcal{Y}$

\vspace{5pt}

在练习 \ref{exer:8-9} 中,我们会介绍一个与之密切相关的哈希函数,它是为变长输入而设计的,并且,当假设 $h$ 是抗碰撞的时,该函数也是抗碰撞的。
\end{snote}

\begin{snote}[成员证明。]
Merkle 树哈希的显著特点是,给定一个哈希值 $y=H(x_1,\dots,x_n)$,我们很容易验证某个 $x\in\mathcal{X}$ 是否是 $T:=(x_1,\dots,x_n)$ 中某个特定位置的元素。比如说,为了证明图 \ref{fig:8-12} 中的 $x=x_3$,我们提供中间哈希值 $\pi:=(\boldsymbol{y_4},\,\boldsymbol{y_9},\,\boldsymbol{y_{14}})$。然后,验证者就可以计算:
\begin{equation}\label{eq:8-16}
\hat{y}_3\leftarrow h(x),
\quad
\hat{y}_{10}\leftarrow h(\hat{y}_3,\boldsymbol{y_4}),
\quad
\hat{y}_{13}\leftarrow h(\boldsymbol{y_9},\hat{y}_{10}),
\quad
\hat{y}_{15}\leftarrow h(\hat{y}_{13},\boldsymbol{y_{14}})
\end{equation}
如果 $y=\hat{y}_{15}$,验证者就接受 $x=x_3$。这个 $\pi$ 被称为 $x$ 在 $T$ 的索引 $3$ 处的 \textbf{Merkle 证明 (Merkle proof)}。

更一般地说,为了证明元素 $x$ 是 $T:=(x_1,\dots,x_n)$ 中位置为 $i$ 的元素,我们要将从 $i$ 号叶子结点到 Merkle 树根的路径上的所有结点的兄弟结点作为中间哈希值组合成证明 $\pi$。证明 $\pi$ 恰好包含 $\log_2n$ 个 $\mathcal{Y}$ 中的元素。验证者可以使用 $\pi$ 中提供的值来重新推导 $T$ 的 Merkle 哈希值。为此,它从 $i$ 号叶子结点开始计算路径上的哈希值,一直到根结点,如式 \ref{eq:8-16} 所示。如果最终计算出的 Merkle 哈希值与保存在只读存储器中的哈希值 $y$ 一致,验证者就接受 $x$ 是真实值(即 $x=x_i$)。

我们将在下面的定理 \ref{theo:8-8} 中表明,如果 $h$ 是抗碰撞的,那么对手无法给出一个 $x$ 和一个 $i$,以及一个证明 $\pi'$,从而欺骗验证者相信 $x$ 是 $T$ 中索引为 $i$ 的元素。

重新考察一下存储在磁盘上的可执行文件,我们将其视为一个分组序列 $x_1,\dots,x_n$,并假设系统在只读存储器中保存了 $y:=H(x_1,\dots,x_n)$。我们可以将 Merkle 树中的 $2n-1$ 个中间哈希值 $y_1,\dots,y_{2n-1}$ 与可执行文件一起保存在存储器中。之后,如果想要验证一个数据分组,系统就能够迅速地找到对应于该分组的 $\log_2n$ 个哈希值,并组成一个 Merkle 证明。通过计算这 $\log_2n$ 个哈希值,系统就可以计算出一个 Merkle 哈希,并将结果与存储的 $y$ 进行比较。在实践中,假设每个分组大小是 4 KB。那么即使可执行文件有 $2^{16}$ 个分组,我们也只需要最多为每个分组多保存两个哈希值(总共 $2n-1$ 个哈希值),即每个分组 $64$ 字节。当需要验证某个分组时,我们只需要计算 $16$ 个哈希值即可。

对于这个问题,还有其他的解决方案。例如,系统可以在每个分组旁存储一个 MAC 标签,并在执行分组之前验证该标签。然而,这要求系统要妥善管理 MAC 密钥,并确保它永远不会被对手获取到。虽然这在某些情况下可能是合理的,但 Merkle 树提供了一个更有效的解决方案,它不需要依赖任何密钥。
\end{snote}

\begin{snote}[为多个元素提供成员证明。]
和之前一样,假设 $y:=H(x_1,\dots,x_n)$ 被存储在只读存储器中,并令 $T:=(x_1,\dots,x_n)$。令 $L\subseteq\mathcal{X}$ 是一个元素集合。我们现在想要向验证者证明 $L$ 中的所有元素都在 $T$ 中。为此,我们可以为 $L$ 中的所有元素提供 Merkle 证明,总的证明大小为 $|L|\log_2n$ 个 $\mathcal{Y}$ 上的元素。然而,这些 Merkle 证明中可能会包含许多重复项,我们可以通过删除重复元素来缩小整个证明。下面的定理对最坏情况下的证明大小进行了约束。我们用 $L\subseteq T$ 来表示 $L$ 中的所有元素都包含在 $T$ 中这一事实。
\end{snote}

\begin{theorem}\label{theo:8-7}
令 $T\subseteq\mathcal{X}$ 是一个大小为 $n$ 的集合,其中 $n$ 是 $2$ 的整数次幂。对于每个 $1\leq r\leq n$ 和一个大小为 $r$ 的集合 $L\subseteq T$,证明 $L$ 中所有元素都在 $T$ 上的 Merkle 证明最多包含 $r\cdot\log_2({n}/{r})$ 个 $\mathcal{Y}$ 上的元素。
\end{theorem}

\begin{proof}
该定理就是定理 \ref{theo:5-8} 的一个直接的推论。令 $S:=T\setminus L$,则 $|S|=n-r$。不难看出。$L$ 的 Merkle 证明中的哈希值集合恰好对应于 ${\rm cover}(S)$ 中的结点。定理 \ref{theo:5-8} 中提供的对 $|{\rm cover}(S)|$ 的约束可以证明此定理。
\end{proof}

\begin{snote}[非成员证明。]
让我们来看看 Merkle 树的另一个应用。考虑一个信用卡处理中心,它维护着一个被撤销的信用卡号列表 $T:=(x_1,\dots,x_n)\in\mathcal{X}^n$。列表 $T$ 会被发送到世界各地不受信任的缓存服务器上,而且每个商家都会收到 Merkle 树哈希 $y:=H(x_1,\dots,x_n)$。这个哈希值 $y$ 被认为是由中心正确计算得到的。当商家需要处理客户的信用卡 $x$ 时,它将 $x$ 发送到最近的缓存服务器,询问 $x$ 是否已被撤销(即验证 $x$ 是否在 $T$ 中)。若果真如此,缓存服务器就会应答一个 Merkle 证明,说明 $x$ 已在 $T$ 中,这样商家就会拒绝该交易。Merkle 树方案的安全性意味着,恶意的缓存服务器不能欺骗商家,使其相信一张有效的信用卡被撤销了。更一般地,Merkle 树让我们可以在不受信任的缓存服务器之间复制数据集 $T$,而没有任何缓存服务器可以谎报该数据集的成员。

对于信用卡的应用场景来说,仅仅验证 $x$ 是否在 $T$ 中是远远不够的。缓存服务器还必须能够让商家相信,一张信用卡 $x$ 不在 $T$ 中(即没有被撤销)。令人惊讶的是,Merkle 树也可以用来提供集合的非成员证明,但要做到这一点,我们必须首先稍微修改一下 Merkle 树的结构。

假设 $T$ 中的元素都是整数,则 $\mathcal{X}\subseteq\mathbb{Z}$。在修改后的 Merkle 树哈希中,我们首先对树的叶子结点进行排序,使得 $x_1<x_2<\cdots<x_n$,如图 \ref{fig:8-13} 所示。然后,我们像之前一样计算树的哈希值 $y:=H(x_1,\dots,x_n)$。我们称其为\textbf{排序 Merkle 树哈希 (sorted Merkle tree hash)}。

\begin{figure}
	\centering
	

\tikzset{every picture/.style={line width=0.75pt}} %set default line width to 0.75pt        

\begin{tikzpicture}[x=0.75pt,y=0.75pt,yscale=-1,xscale=1]
%uncomment if require: \path (0,249); %set diagram left start at 0, and has height of 249

%Straight Lines [id:da702920327532164] 
\draw [fill={rgb, 255:red, 255; green, 255; blue, 255 }  ,fill opacity=1 ][line width=0.75]    (237.32,54.25) -- (115,90) ;
\draw [shift={(240.2,53.41)}, rotate = 163.71] [fill={rgb, 255:red, 0; green, 0; blue, 0 }  ][line width=0.08]  [draw opacity=0] (7.14,-3.43) -- (0,0) -- (7.14,3.43) -- cycle    ;
%Straight Lines [id:da3073933646499085] 
\draw [fill={rgb, 255:red, 255; green, 255; blue, 255 }  ,fill opacity=1 ][line width=0.75]    (262.68,55.04) -- (385,90) ;
\draw [shift={(259.8,54.21)}, rotate = 15.95] [fill={rgb, 255:red, 0; green, 0; blue, 0 }  ][line width=0.08]  [draw opacity=0] (7.14,-3.43) -- (0,0) -- (7.14,3.43) -- cycle    ;
%Straight Lines [id:da6120087870514785] 
\draw [fill={rgb, 255:red, 255; green, 255; blue, 255 }  ,fill opacity=1 ][line width=0.75]    (25,170) -- (47.19,140.65) ;
\draw [shift={(49,138.26)}, rotate = 127.09] [fill={rgb, 255:red, 0; green, 0; blue, 0 }  ][line width=0.08]  [draw opacity=0] (7.14,-3.43) -- (0,0) -- (7.14,3.43) -- cycle    ;
%Straight Lines [id:da5183313825045153] 
\draw [fill={rgb, 255:red, 255; green, 255; blue, 255 }  ,fill opacity=1 ][line width=0.75]    (63.29,140.91) -- (85,170) ;
\draw [shift={(61.5,138.51)}, rotate = 53.27] [fill={rgb, 255:red, 0; green, 0; blue, 0 }  ][line width=0.08]  [draw opacity=0] (7.14,-3.43) -- (0,0) -- (7.14,3.43) -- cycle    ;
%Straight Lines [id:da3638120020023241] 
\draw [fill={rgb, 255:red, 255; green, 255; blue, 255 }  ,fill opacity=1 ][line width=0.75]    (166.71,140.91) -- (145,170) ;
\draw [shift={(168.5,138.51)}, rotate = 126.73] [fill={rgb, 255:red, 0; green, 0; blue, 0 }  ][line width=0.08]  [draw opacity=0] (7.14,-3.43) -- (0,0) -- (7.14,3.43) -- cycle    ;
%Straight Lines [id:da32018500957267415] 
\draw [fill={rgb, 255:red, 255; green, 255; blue, 255 }  ,fill opacity=1 ][line width=0.75]    (182.83,141.14) -- (205,170) ;
\draw [shift={(181,138.76)}, rotate = 52.47] [fill={rgb, 255:red, 0; green, 0; blue, 0 }  ][line width=0.08]  [draw opacity=0] (7.14,-3.43) -- (0,0) -- (7.14,3.43) -- cycle    ;
%Straight Lines [id:da11829710873612775] 
\draw [fill={rgb, 255:red, 255; green, 255; blue, 255 }  ,fill opacity=1 ][line width=0.75]    (103.26,97.68) -- (55,130) ;
\draw [shift={(105.75,96.01)}, rotate = 146.19] [fill={rgb, 255:red, 0; green, 0; blue, 0 }  ][line width=0.08]  [draw opacity=0] (7.14,-3.43) -- (0,0) -- (7.14,3.43) -- cycle    ;
%Straight Lines [id:da5550208993570565] 
\draw [fill={rgb, 255:red, 255; green, 255; blue, 255 }  ,fill opacity=1 ][line width=0.75]    (126.27,98.63) -- (175,130) ;
\draw [shift={(123.75,97.01)}, rotate = 32.77] [fill={rgb, 255:red, 0; green, 0; blue, 0 }  ][line width=0.08]  [draw opacity=0] (7.14,-3.43) -- (0,0) -- (7.14,3.43) -- cycle    ;
%Shape: Circle [id:dp6920813762808213] 
\draw  [fill={rgb, 255:red, 255; green, 255; blue, 255 }  ,fill opacity=1 ][line width=0.75] [general shadow={fill=black,shadow xshift=1.5pt,shadow yshift=-1.5pt}] (15,170) .. controls (15,164.48) and (19.48,160) .. (25,160) .. controls (30.52,160) and (35,164.48) .. (35,170) .. controls (35,175.52) and (30.52,180) .. (25,180) .. controls (19.48,180) and (15,175.52) .. (15,170) -- cycle ;
%Shape: Circle [id:dp9663292169740727] 
\draw  [fill={rgb, 255:red, 255; green, 255; blue, 255 }  ,fill opacity=1 ][line width=0.75] [general shadow={fill=black,shadow xshift=1.5pt,shadow yshift=-1.5pt}] (75,170) .. controls (75,164.48) and (79.48,160) .. (85,160) .. controls (90.52,160) and (95,164.48) .. (95,170) .. controls (95,175.52) and (90.52,180) .. (85,180) .. controls (79.48,180) and (75,175.52) .. (75,170) -- cycle ;
%Shape: Circle [id:dp6698750169919838] 
\draw  [fill={rgb, 255:red, 255; green, 255; blue, 255 }  ,fill opacity=1 ][line width=0.75] [general shadow={fill=black,shadow xshift=1.5pt,shadow yshift=-1.5pt}] (135,170) .. controls (135,164.48) and (139.48,160) .. (145,160) .. controls (150.52,160) and (155,164.48) .. (155,170) .. controls (155,175.52) and (150.52,180) .. (145,180) .. controls (139.48,180) and (135,175.52) .. (135,170) -- cycle ;
%Shape: Circle [id:dp408527637149501] 
\draw  [fill={rgb, 255:red, 255; green, 255; blue, 255 }  ,fill opacity=1 ][line width=0.75] [general shadow={fill=black,shadow xshift=1.5pt,shadow yshift=-1.5pt}] (195,170) .. controls (195,164.48) and (199.48,160) .. (205,160) .. controls (210.52,160) and (215,164.48) .. (215,170) .. controls (215,175.52) and (210.52,180) .. (205,180) .. controls (199.48,180) and (195,175.52) .. (195,170) -- cycle ;
%Shape: Circle [id:dp9073791707119769] 
\draw  [fill={rgb, 255:red, 255; green, 255; blue, 255 }  ,fill opacity=1 ][line width=0.75] [general shadow={fill=black,shadow xshift=1.5pt,shadow yshift=-1.5pt}] (45,130) .. controls (45,124.48) and (49.48,120) .. (55,120) .. controls (60.52,120) and (65,124.48) .. (65,130) .. controls (65,135.52) and (60.52,140) .. (55,140) .. controls (49.48,140) and (45,135.52) .. (45,130) -- cycle ;
%Shape: Circle [id:dp02478093212823307] 
\draw  [fill={rgb, 255:red, 255; green, 255; blue, 255 }  ,fill opacity=1 ][line width=0.75] [general shadow={fill=black,shadow xshift=1.5pt,shadow yshift=-1.5pt}] (165,130) .. controls (165,124.48) and (169.48,120) .. (175,120) .. controls (180.52,120) and (185,124.48) .. (185,130) .. controls (185,135.52) and (180.52,140) .. (175,140) .. controls (169.48,140) and (165,135.52) .. (165,130) -- cycle ;
%Shape: Circle [id:dp6156110109492565] 
\draw  [fill={rgb, 255:red, 255; green, 255; blue, 255 }  ,fill opacity=1 ][line width=0.75] [general shadow={fill=black,shadow xshift=1.5pt,shadow yshift=-1.5pt}] (105,90) .. controls (105,84.48) and (109.48,80) .. (115,80) .. controls (120.52,80) and (125,84.48) .. (125,90) .. controls (125,95.52) and (120.52,100) .. (115,100) .. controls (109.48,100) and (105,95.52) .. (105,90) -- cycle ;
%Straight Lines [id:da24603915152063838] 
\draw [fill={rgb, 255:red, 255; green, 255; blue, 255 }  ,fill opacity=1 ][line width=0.75]    (316.94,140.9) -- (295,170) ;
\draw [shift={(318.75,138.51)}, rotate = 127.02] [fill={rgb, 255:red, 0; green, 0; blue, 0 }  ][line width=0.08]  [draw opacity=0] (7.14,-3.43) -- (0,0) -- (7.14,3.43) -- cycle    ;
%Straight Lines [id:da9115181138506314] 
\draw [fill={rgb, 255:red, 255; green, 255; blue, 255 }  ,fill opacity=1 ][line width=0.75]    (332.59,141.13) -- (355,170) ;
\draw [shift={(330.75,138.76)}, rotate = 52.18] [fill={rgb, 255:red, 0; green, 0; blue, 0 }  ][line width=0.08]  [draw opacity=0] (7.14,-3.43) -- (0,0) -- (7.14,3.43) -- cycle    ;
%Straight Lines [id:da9185055776150515] 
\draw [fill={rgb, 255:red, 255; green, 255; blue, 255 }  ,fill opacity=1 ][line width=0.75]    (436.71,140.91) -- (415,170) ;
\draw [shift={(438.5,138.51)}, rotate = 126.73] [fill={rgb, 255:red, 0; green, 0; blue, 0 }  ][line width=0.08]  [draw opacity=0] (7.14,-3.43) -- (0,0) -- (7.14,3.43) -- cycle    ;
%Straight Lines [id:da263720662321004] 
\draw [fill={rgb, 255:red, 255; green, 255; blue, 255 }  ,fill opacity=1 ][line width=0.75]    (453.08,141.63) -- (475,170) ;
\draw [shift={(451.25,139.26)}, rotate = 52.31] [fill={rgb, 255:red, 0; green, 0; blue, 0 }  ][line width=0.08]  [draw opacity=0] (7.14,-3.43) -- (0,0) -- (7.14,3.43) -- cycle    ;
%Straight Lines [id:da320460555452849] 
\draw [fill={rgb, 255:red, 255; green, 255; blue, 255 }  ,fill opacity=1 ][line width=0.75]    (373.75,97.67) -- (325,130) ;
\draw [shift={(376.25,96.01)}, rotate = 146.45] [fill={rgb, 255:red, 0; green, 0; blue, 0 }  ][line width=0.08]  [draw opacity=0] (7.14,-3.43) -- (0,0) -- (7.14,3.43) -- cycle    ;
%Straight Lines [id:da4257765400493081] 
\draw [fill={rgb, 255:red, 255; green, 255; blue, 255 }  ,fill opacity=1 ][line width=0.75]    (396.76,98.4) -- (445,130) ;
\draw [shift={(394.25,96.76)}, rotate = 33.23] [fill={rgb, 255:red, 0; green, 0; blue, 0 }  ][line width=0.08]  [draw opacity=0] (7.14,-3.43) -- (0,0) -- (7.14,3.43) -- cycle    ;
%Shape: Circle [id:dp1109096379469161] 
\draw  [fill={rgb, 255:red, 255; green, 255; blue, 255 }  ,fill opacity=1 ][line width=0.75] [general shadow={fill=black,shadow xshift=1.5pt,shadow yshift=-1.5pt}] (285,170) .. controls (285,164.48) and (289.48,160) .. (295,160) .. controls (300.52,160) and (305,164.48) .. (305,170) .. controls (305,175.52) and (300.52,180) .. (295,180) .. controls (289.48,180) and (285,175.52) .. (285,170) -- cycle ;
%Shape: Circle [id:dp6670429507855935] 
\draw  [fill={rgb, 255:red, 255; green, 255; blue, 255 }  ,fill opacity=1 ][line width=0.75] [general shadow={fill=black,shadow xshift=1.5pt,shadow yshift=-1.5pt}] (345,170) .. controls (345,164.48) and (349.48,160) .. (355,160) .. controls (360.52,160) and (365,164.48) .. (365,170) .. controls (365,175.52) and (360.52,180) .. (355,180) .. controls (349.48,180) and (345,175.52) .. (345,170) -- cycle ;
%Shape: Circle [id:dp9489763101292301] 
\draw  [fill={rgb, 255:red, 255; green, 255; blue, 255 }  ,fill opacity=1 ][line width=0.75] [general shadow={fill=black,shadow xshift=1.5pt,shadow yshift=-1.5pt}] (405,170) .. controls (405,164.48) and (409.48,160) .. (415,160) .. controls (420.52,160) and (425,164.48) .. (425,170) .. controls (425,175.52) and (420.52,180) .. (415,180) .. controls (409.48,180) and (405,175.52) .. (405,170) -- cycle ;
%Shape: Circle [id:dp17009001398474677] 
\draw  [fill={rgb, 255:red, 255; green, 255; blue, 255 }  ,fill opacity=1 ][line width=0.75] [general shadow={fill=black,shadow xshift=1.5pt,shadow yshift=-1.5pt}] (465,170) .. controls (465,164.48) and (469.48,160) .. (475,160) .. controls (480.52,160) and (485,164.48) .. (485,170) .. controls (485,175.52) and (480.52,180) .. (475,180) .. controls (469.48,180) and (465,175.52) .. (465,170) -- cycle ;
%Shape: Circle [id:dp311641882516994] 
\draw  [fill={rgb, 255:red, 255; green, 255; blue, 255 }  ,fill opacity=1 ][line width=0.75] [general shadow={fill=black,shadow xshift=1.5pt,shadow yshift=-1.5pt}] (315,130) .. controls (315,124.48) and (319.48,120) .. (325,120) .. controls (330.52,120) and (335,124.48) .. (335,130) .. controls (335,135.52) and (330.52,140) .. (325,140) .. controls (319.48,140) and (315,135.52) .. (315,130) -- cycle ;
%Shape: Circle [id:dp32731070719881616] 
\draw  [fill={rgb, 255:red, 255; green, 255; blue, 255 }  ,fill opacity=1 ][line width=0.75] [general shadow={fill=black,shadow xshift=1.5pt,shadow yshift=-1.5pt}] (435,130) .. controls (435,124.48) and (439.48,120) .. (445,120) .. controls (450.52,120) and (455,124.48) .. (455,130) .. controls (455,135.52) and (450.52,140) .. (445,140) .. controls (439.48,140) and (435,135.52) .. (435,130) -- cycle ;
%Shape: Circle [id:dp6495027217472407] 
\draw  [fill={rgb, 255:red, 255; green, 255; blue, 255 }  ,fill opacity=1 ][line width=0.75] [general shadow={fill=black,shadow xshift=1.5pt,shadow yshift=-1.5pt}] (375,90) .. controls (375,84.48) and (379.48,80) .. (385,80) .. controls (390.52,80) and (395,84.48) .. (395,90) .. controls (395,95.52) and (390.52,100) .. (385,100) .. controls (379.48,100) and (375,95.52) .. (375,90) -- cycle ;
%Shape: Circle [id:dp7240544767999253] 
\draw  [fill={rgb, 255:red, 255; green, 255; blue, 255 }  ,fill opacity=1 ][line width=0.75] [general shadow={fill=black,shadow xshift=1.5pt,shadow yshift=-1.5pt}] (240,50) .. controls (240,44.48) and (244.48,40) .. (250,40) .. controls (255.52,40) and (260,44.48) .. (260,50) .. controls (260,55.52) and (255.52,60) .. (250,60) .. controls (244.48,60) and (240,55.52) .. (240,50) -- cycle ;
%Shape: Rectangle [id:dp8463728262042536] 
\draw   (5,200) -- (45,200) -- (45,215) -- (5,215) -- cycle ;
%Shape: Rectangle [id:dp37573295720289646] 
\draw   (65,200) -- (105,200) -- (105,215) -- (65,215) -- cycle ;
%Shape: Rectangle [id:dp39722658574356395] 
\draw  [line width=0.75]  (125,200) -- (165,200) -- (165,215) -- (125,215) -- cycle ;
%Shape: Rectangle [id:dp7495732290873869] 
\draw  [fill={rgb, 255:red, 155; green, 155; blue, 155 }  ,fill opacity=1 ] (185,200) -- (225,200) -- (225,215) -- (185,215) -- cycle ;
%Shape: Rectangle [id:dp5653373151711263] 
\draw  [fill={rgb, 255:red, 155; green, 155; blue, 155 }  ,fill opacity=1 ] (275,200) -- (315,200) -- (315,215) -- (275,215) -- cycle ;
%Shape: Rectangle [id:dp25965689342656884] 
\draw   (335,200) -- (375,200) -- (375,215) -- (335,215) -- cycle ;
%Shape: Rectangle [id:dp14199578106093447] 
\draw   (395,200) -- (435,200) -- (435,215) -- (395,215) -- cycle ;
%Shape: Rectangle [id:dp8809655361412605] 
\draw   (455,200) -- (495,200) -- (495,215) -- (455,215) -- cycle ;
%Straight Lines [id:da10877585246462518] 
\draw [fill={rgb, 255:red, 255; green, 255; blue, 255 }  ,fill opacity=1 ][line width=0.75]    (25,200) -- (25,183) ;
\draw [shift={(25,180)}, rotate = 90] [fill={rgb, 255:red, 0; green, 0; blue, 0 }  ][line width=0.08]  [draw opacity=0] (7.14,-3.43) -- (0,0) -- (7.14,3.43) -- cycle    ;
%Straight Lines [id:da44045096101998715] 
\draw [fill={rgb, 255:red, 255; green, 255; blue, 255 }  ,fill opacity=1 ][line width=0.75]    (85,200) -- (85,183) ;
\draw [shift={(85,180)}, rotate = 90] [fill={rgb, 255:red, 0; green, 0; blue, 0 }  ][line width=0.08]  [draw opacity=0] (7.14,-3.43) -- (0,0) -- (7.14,3.43) -- cycle    ;
%Straight Lines [id:da342077588771736] 
\draw [fill={rgb, 255:red, 255; green, 255; blue, 255 }  ,fill opacity=1 ][line width=0.75]    (145,200) -- (145,183) ;
\draw [shift={(145,180)}, rotate = 90] [fill={rgb, 255:red, 0; green, 0; blue, 0 }  ][line width=0.08]  [draw opacity=0] (7.14,-3.43) -- (0,0) -- (7.14,3.43) -- cycle    ;
%Straight Lines [id:da5894662403845001] 
\draw [fill={rgb, 255:red, 255; green, 255; blue, 255 }  ,fill opacity=1 ][line width=0.75]    (205,200) -- (205,183) ;
\draw [shift={(205,180)}, rotate = 90] [fill={rgb, 255:red, 0; green, 0; blue, 0 }  ][line width=0.08]  [draw opacity=0] (7.14,-3.43) -- (0,0) -- (7.14,3.43) -- cycle    ;
%Straight Lines [id:da6075272503855431] 
\draw [fill={rgb, 255:red, 255; green, 255; blue, 255 }  ,fill opacity=1 ][line width=0.75]    (295,200) -- (295,183) ;
\draw [shift={(295,180)}, rotate = 90] [fill={rgb, 255:red, 0; green, 0; blue, 0 }  ][line width=0.08]  [draw opacity=0] (7.14,-3.43) -- (0,0) -- (7.14,3.43) -- cycle    ;
%Straight Lines [id:da7118849250215102] 
\draw [fill={rgb, 255:red, 255; green, 255; blue, 255 }  ,fill opacity=1 ][line width=0.75]    (355,200) -- (355,183) ;
\draw [shift={(355,180)}, rotate = 90] [fill={rgb, 255:red, 0; green, 0; blue, 0 }  ][line width=0.08]  [draw opacity=0] (7.14,-3.43) -- (0,0) -- (7.14,3.43) -- cycle    ;
%Straight Lines [id:da7937224127475853] 
\draw [fill={rgb, 255:red, 255; green, 255; blue, 255 }  ,fill opacity=1 ][line width=0.75]    (415,200) -- (415,183) ;
\draw [shift={(415,180)}, rotate = 90] [fill={rgb, 255:red, 0; green, 0; blue, 0 }  ][line width=0.08]  [draw opacity=0] (7.14,-3.43) -- (0,0) -- (7.14,3.43) -- cycle    ;
%Straight Lines [id:da9640722903578183] 
\draw [fill={rgb, 255:red, 255; green, 255; blue, 255 }  ,fill opacity=1 ][line width=0.75]    (475,200) -- (475,183) ;
\draw [shift={(475,180)}, rotate = 90] [fill={rgb, 255:red, 0; green, 0; blue, 0 }  ][line width=0.08]  [draw opacity=0] (7.14,-3.43) -- (0,0) -- (7.14,3.43) -- cycle    ;
%Straight Lines [id:da7095996565493987] 
\draw [fill={rgb, 255:red, 255; green, 255; blue, 255 }  ,fill opacity=1 ][line width=0.75]    (250,40) -- (250,13) ;
\draw [shift={(250,10)}, rotate = 90] [fill={rgb, 255:red, 0; green, 0; blue, 0 }  ][line width=0.08]  [draw opacity=0] (7.14,-3.43) -- (0,0) -- (7.14,3.43) -- cycle    ;

% Text Node
\draw (115,90) node    {$h$};
% Text Node
\draw (250,50) node    {$h$};
% Text Node
\draw (55,130) node    {$h$};
% Text Node
\draw (25,170) node    {$h$};
% Text Node
\draw (85,170) node    {$h$};
% Text Node
\draw (145,170) node    {$h$};
% Text Node
\draw (205,170) node    {$h$};
% Text Node
\draw (295,170) node    {$h$};
% Text Node
\draw (355,170) node    {$h$};
% Text Node
\draw (415,170) node    {$h$};
% Text Node
\draw (475,170) node    {$h$};
% Text Node
\draw (175,130) node    {$h$};
% Text Node
\draw (325,130) node    {$h$};
% Text Node
\draw (445,130) node    {$h$};
% Text Node
\draw (385,90) node    {$h$};
% Text Node
\draw (25,207.5) node    {$x_{1}$};
% Text Node
\draw (85,207.5) node    {$x_{2}$};
% Text Node
\draw (145,207.5) node    {$x_{3}$};
% Text Node
\draw (205,207.5) node    {$x_{4}$};
% Text Node
\draw (295,207.5) node    {$x_{5}$};
% Text Node
\draw (355,207.5) node    {$x_{6}$};
% Text Node
\draw (415,207.5) node    {$x_{7}$};
% Text Node
\draw (475,207.5) node    {$x_{8}$};
% Text Node
\draw (37,152.73) node [anchor=south east] [inner sep=0.75pt]    {$y_{1}$};
% Text Node
\draw (80.38,114.6) node [anchor=south east] [inner sep=0.75pt]    {\colorbox{gray!50}{$y_{9}$}};
% Text Node
\draw (170.13,70.48) node [anchor=south east] [inner sep=0.75pt]    {$y_{13}$};
% Text Node
\draw (156.75,155.85) node [anchor=south east] [inner sep=0.75pt]    {\colorbox{gray!50}{$y_{3}$}};
% Text Node
\draw (306.88,152.85) node [anchor=south east] [inner sep=0.75pt]    {$y_{5}$};
% Text Node
\draw (426.75,152.85) node [anchor=south east] [inner sep=0.75pt]    {$y_{7}$};
% Text Node
\draw (350.63,111.6) node [anchor=south east] [inner sep=0.75pt]    {$y_{11}$};
% Text Node
\draw (73.25,152.85) node [anchor=south west] [inner sep=0.75pt]    {$y_{2}$};
% Text Node
\draw (193,152.98) node [anchor=south west] [inner sep=0.75pt]    {$y_{4}$};
% Text Node
\draw (342.88,155.98) node [anchor=south west] [inner sep=0.75pt]    {\colorbox{gray!50}{$y_{6}$}};
% Text Node
\draw (463.13,153.23) node [anchor=south west] [inner sep=0.75pt]    {$y_{8}$};
% Text Node
\draw (149.38,112.1) node [anchor=south west] [inner sep=0.75pt]    {$y_{10}$};
% Text Node
\draw (419.63,114.98) node [anchor=south west] [inner sep=0.75pt]    {\colorbox{gray!50}{$y_{12}$}};
% Text Node
\draw (330,70.1) node [anchor=south west] [inner sep=0.75pt]    {$y_{14}$};
% Text Node
\draw (252,27) node [anchor=west] [inner sep=0.75pt]    {$y_{15}$};
% Text Node
\draw (55,207.5) node    {$< $};
% Text Node
\draw (115,207.5) node    {$< $};
% Text Node
\draw (175,207.5) node    {$< $};
% Text Node
\draw (235,207.5) node    {$< $};
% Text Node
\draw (325,207.5) node    {$< $};
% Text Node
\draw (385,207.5) node    {$< $};
% Text Node
\draw (445,207.5) node    {$< $};
% Text Node
\draw (265,207.5) node    {$< $};
% Text Node
\draw (250,207.5) node    {$x$};


\end{tikzpicture}
	\caption{排序树哈希。阴影的元素为 $x$ 提供非成员证明。}
	\label{fig:8-13}
\end{figure}

现在,给定某个 $x\notin T$,我们想要给出一个 $x$ 不在 $T$ 中的证明。验证者只持有排序 Merkle 树哈希 $y$。为了生成这样的证明,证明者首先找到 $T$ 中的两个相邻的叶子结点 $x_i$ 和 $x_{i+1}$,它们能够包裹 $x$,即 $x_i<x<x_{i+1}$。简单起见,我们假设 $x_1<x<x_n$,因此所需的 $x_i$ 和 $x_{i+1}$ 总是存在的。接下来,证明者提供一个 Merkle 证明,证明 $x_i$ 是 $T$ 中的第 $i$ 项,$x_{i+1}$ 是 $T$ 中的第 $i+1$ 项。验证者可以检查并确定,这两个叶子结点确实是相邻的,并且 $x_i<x<x_{i+1}$,这就证明了 $x$ 不在 $T$ 中。这是因为,如果 $x$ 在 $T$ 中,它必须占据 $x_i$ 和 $x_{i+1}$ 之间的一个叶子结点,但由于 $x_i$ 和 $x_{i+1}$ 是相邻的叶子结点,所以这是不可能的。

图 \ref{fig:8-13} 展示了一个例子,证明区间 $(x_4,x_5)$ 中的一个值 $x$ 不在 $T$ 中,相应的 Merkle 证明是一组哈希值 $(y_3,y_6,y_9,y_{12})$ 和数据项 $x_4,x_5$。验证者检查 Merkle 证明,确认 $x_4$ 和 $x_5$ 都在 $T$ 中,并且它们在树中是相邻的。然后它检查 $x_4<x<x_5$,这就证明 $x$ 不在树中。我们可以看到,在最坏的情况下,一个非成员证明包含 $2\log_2({n}/{2})$ 个 $\mathcal{Y}$ 上的元素,再加上两个 $\mathcal{X}$ 中的数据项。

该方案的安全性将在下一节讨论。它表明,当底层哈希函数 $h$ 是抗碰撞的时候,对手就无法欺骗验证者相信一个 $x\in T$ 不是 $T$ 的成员。如果用上面的信用卡的例子来说明,就是一个恶意的缓存服务器不能欺骗商家,使他接受一张已被撤销的信用卡。
\end{snote}

\subsection{认证数据结构}\label{subsec:8-9-1}

Merkle 树是一种更加抽象的概念的一个特例,后者被称为认证数据结构。认证数据结构是一种被用来计算一个序列 $T:=(x_1,\dots,x_n)$ 的短哈希的工具,这个短哈希可以在之后被用来证明 $T$ 的属性。Merkle 树让我们能够提供成员证明和非成员证明。也有支持其他操作的认证数据结构,比如有效插入和删除,我们将在下面讨论。

我们首先定义一个用于集合成员证明的认证数据结构,并讨论其安全属性。

\begin{definition}\label{def:8-3}
一个定义在 $(\mathcal{X}^n,\mathcal{Y})$ 上的\textbf{认证数据结构方案 (authenticated data structure scheme)} $\mathcal{D}=(H,P,V)$ 是一个由三个有效确定性算法构成的元组,其中:
\begin{itemize}
	\item $H$ 是一个算法,其调用方式为 $y\leftarrow H(T)$,其中 $T:=(x_1,\dots,x_n)\in\mathcal{X}^n$,$y\in\mathcal{Y}$。
	\item $P$ 是一个算法,其调用方式为 $\pi\leftarrow P(i,x,T)$,其中$x\in\mathcal{X}$,$1\leq i\leq n$。算法输出一个证明 $\pi$,以证明 $x=x_i$,其中 $T:=(x_1,\dots,x_n)$。
	\item $V$ 是一个算法,其调用方式为 $V(i,x,y,\pi)$,输出 $\mathsf{accept}$ 或 $\mathsf{reject}$。
	\item 我们要求,对于所有的 $T:=(x_1,\dots,x_n)\in\mathcal{X}^n$ 和所有的 $1\leq i\leq n$,我们都有:
	\[
	V\big(i,\,x_i,\,H(T),\,P(i,x_i,T)\big)=\mathsf{accept}
	\]
\end{itemize}
\end{definition}

上一小节介绍的 Merkle 树很显然也符合上述定义。

我们接下来定义安全性。如果对手能输出一个哈希值 $y\in\mathcal{Y}$,并且能够欺骗验证者相信两个不同元素 $x$ 和 $x'$ 都是 $\mathcal{X}$ 中索引为 $i$ 的值,我们就称该对手攻破了该方案。

\begin{game}[认证数据结构安全性]\label{game:8-2}
对于一个定义在 $(\mathcal{X}^n,\mathcal{Y})$ 上的认证数据结构 $\mathcal{D}=(H,P,V)$ 和一个给定对手 $\mathcal{A}$,攻击游戏运行如下:
\begin{quote}
对手 $\mathcal{A}$ 输出一个 $y\in\mathcal{Y}$,一个位置 $i\in\{1,\dots,n\}$ 以及两个数对 $(x,\pi)$ 和 $(x',\pi')$,其中 $x,x'\in\mathcal{X}$。
\end{quote}
如果 $x\neq x'$ 且 $V(i,x,y,\pi)=V(i,x',y,\pi')=\mathsf{accept}$,我们就称 $\mathcal{A}$ 赢得了该攻击游戏。我们将 $\mathcal{A}$ 就 $\mathcal{D}$ 的优势记为 ${\rm ADS}\mathsf{adv}[\mathcal{A},\mathcal{D}]$,其值为 $\mathcal{A}$ 赢得该游戏的概率。
\end{game}

\begin{definition}\label{def:8-4}
如果对于所有的有效对手 $\mathcal{A}$,${\rm ADS}\mathsf{adv}[\mathcal{A},\mathcal{D}]$ 的值都可忽略不计,我们就称认证数据结构方案 $\mathcal{D}$ 是安全的。
\end{definition}

\begin{theorem}\label{theo:8-8}
假设底层的哈希函数 $h$ 是抗碰撞的,那么 Merkle 树哈希方案就是一个安全的认证数据结构方案。
\end{theorem}

\begin{proof}
该证明与练习 \ref{exer:8-9} 的证明基本相同。
\end{proof}

对于为一个哈希数据集合提供非成员证明的需求,我们也同样可以给出一个安全定义。我们把给出这个安全定义作为一个启发式的练习留给读者。读者也可以证明,假设底层哈希函数是抗碰撞的,则排序 Merkle 树是一个安全的非成员证明方案。

\begin{snote}[可更新的 Merkle 数据结构。]
令 $T$ 是一个大小为 $n$ 的数据集。排序 Merkle 树哈希的一个缺点是,哪怕数据集 $T$ 中只有一个元素被改变,我们就需要重新调整所有元素的顺序,因而就需要重新计算整棵哈希树。这可能需要进行 $O(n)$ 次哈希计算。有些数据结构提供了与 Merkle 树相同的功能,但支持有效的更新,且最多只需要进行 $O(\log n)$ 次哈希计算。一个例子是基于 2-3 树的方案,另一个是基于跳表的方案。一些加密货币系统中还使用了基于 \emph{Patricia 树}数据结构的哈希树。
\end{snote}