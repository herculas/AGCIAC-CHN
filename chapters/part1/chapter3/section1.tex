\section{伪随机生成器}\label{sec:3-1}

回顾一下,在一次性密码本中,密钥、消息和密文都是 $L$ 比特的序列。然而,我们可能想要使用一个短得多的密钥。我们的想法是,使用一个短的、$\ell$ 比特的``种子" $s$ 作为加密密钥,其中 $\ell$ 比 $L$ 要小得多,然后将这个种子``拉伸"成一个较长的、$L$ 比特的序列,用来掩盖消息(以及解除对密文的掩盖)。拉伸种子 $s$ 需要使用某个有效的确定性算法 $G$,该算法能够将任意 $\ell$ 比特的序列映射为 $L$ 比特的序列。因此,这种修改后的一次性密码本的密钥空间为 $\{0,1\}^\ell$,而消息空间与密文空间仍为 $\{0,1\}^L$。对于 $s\in\{0,1\}^\ell$ 和 $m,c\in\{0,1\}^L$,加密和解密的定义如下:
\[
E(s,m):=G(s)\oplus m,\quad
D(s,c):=G(s)\oplus c
\]
这种修改后的一次性密码本被称作\textbf{流密码(stream cipher)},而函数 $G$ 被称为\textbf{伪随机生成器 (pseudo-random generator)}。

如果 $\ell<L$,那么根据香农定理,这种流密码肯定不能提供完美安全性。但是如果 $G$ 满足适当的安全属性,这种密码仍然可以被视作是语义安全的。假设 $s$ 是一个随机的 $\ell$ 比特序列,$r$ 是一个随机的 $L$ 比特序列。直观地说,如果一个对手无法有效地区分 $G(s)$ 与 $r$,它应该也就无法分辨出这个流密码与一次性密码本之间的区别;此外,既然后者是语义安全的,前者自然也是如此。为了使上述推理更加严谨,我们需要严格定义对手不能``有效区分 $G(s)$ 与 $r$"这一概念。

一种用于区分伪随机序列 $G(s)$ 和真随机序列 $r$ 的算法被称为\textbf{统计检验(statistical test)}。它将一个序列作为输入,并输出 $0$ 或 $1$。如果它在一个伪随机输入上输出 $1$ 的概率和在一个真随机输入上输出 $1$ 的概率之间存在显著的差异,这样的检验就被称作是\textbf{有效的}。即便是相对较小的概率差异,比如说 $1\%$,我们也认为是显著的;事实上,即使只有 $1\%$ 的差异,如果我们能够获得几百个互相独立的样本,而且这些样本要么都是伪随机的,要么都是真随机的,我们就能很有把握地推断出,我们所看到的序列到底是伪随机的,还是真随机的。然而,一个非零但可忽略不计的概率差异,比如 $2^{-100}$,就是没有帮助的。

我们该如何去设计一个有效的统计检验呢?一个基本的方法是:给定一个 $L$ 比特序列,计算某个统计量,然后看看在假设该序列是真随机序列的情况下,这个统计量是否与我们的预期存在较大的差异。

比如说,一个非常简单的、很容易计算的统计量就是二进制序列中比特 $1$ 出现的次数 $k$。对于一个真随机序列,我们期望 $k\approx L/2$。如果 PRG $G$ 对比特 $1$ 或者比特 $0$ 存在某种偏向性,我们就可以通过一个统计检验有效地侦测到这样的偏差。比如说,如果 $|k-0.5L|<0.01L$,我们就输出 $1$,否则输出 $0$。如果 PRG $G$ 确实对 $0$ 或 $1$ 有一些明显的偏向,这个统计检验是就会相当有效。

上面例子中的检验还可以进一步加强,我们不仅可以考虑单个比特,还能够考虑成对的比特。我们可以将 $L$ 比特的序列分解成 $\approx L/2$ 个比特对,并统计数对 $00$ 出现的次数 $k_{00}$,数对 $01$ 出现的次数 $k_{01}$,数对 $10$ 出现的次数 $k_{10}$,以及数对 $11$ 出现的次数 $k_{11}$。对于一个真随机序列,我们期望上面的这四个统计量都 $\approx L/2\cdot 1/4=1/8$。因此,一个自然而然的统计检验就是检验这些统计量与 $L/8$ 之间的偏差是否小于某个特定界限。或者,我们也可以将这些偏差的平方相加,并检验这个和是否小于某个特定界限——这就是统计学中经典的 $\chi$ 方检验。很显然,我们可以从比特对出发,将这个想法进一步推广到任何长度的元组。

我们还可以检查其他的许多简单的统计量。然而,这些简单的检验并不倾向于利用算法 $G$ 的更深层次的数学特性,而后者很可能被恶意对手利用,以设计专门针对 $G$ 的攻击。比如说,对于一些 PRG,上面介绍的两种简单的检验对其完全无效,但当给定足够多的输出比特时,其模式是完全可以预测的;也就是说,给定 $G(s)$ 的一个足够长的前缀,对手就可以计算出 $G(s)$ 的所有剩余比特,甚至能够计算出种子 $s$ 本身。

我们对 PRG 的安全性的定义正式地确定了一个概念,即不应该存在有效的(并且可有效计算的)统计检验。

\subsection{伪随机生成器的定义}\label{subsec:3-1-1}

\textbf{伪随机生成器(pseudo-random generator)},简称\textbf{PRG},是一种有效的确定性算法 $G$,它以一个种子 $s$ 为输入,计算并输出一个值 $r$。种子 $s$ 来自一个有限的\textbf{种子空间} $\mathcal{S}$,而输出 $r$ 属于一个有限的\textbf{输出空间} $\mathcal{R}$。通常情况下,$\mathcal{S}$ 和 $\mathcal{R}$ 是两个比特序列的集合,它们的长度都是被提前规定的(例如,在上面的讨论中,我们有 $\mathcal{S}=\{0,1\}^\ell$ 和 $\mathcal{R}=\{0,1\}^L$)。我们称 $G$ 是一个定义在 $(\mathcal{S},\mathcal{R})$ 上的 PRG。

我们对 PRG 的安全定义抓住了这样一个直观的概念:如果 $s$ 是从 $\mathcal{S}$ 中随机选出的,而 $r$ 是从 $\mathcal{R}$ 中随机选出的,那么任何有效对手都无法有效地分辨 $G(s)$ 和 $r$ 之间的差别:两者是\textbf{计算上不可区分(computationally indistinguishable)}的。我们将该定义表述为一个攻击游戏。

\begin{game}[伪随机生成器]\label{game:3-1}
对于一个定义在 $(\mathcal{S},\mathcal{R})$ 上的给定 PRG $G$ 和一个给定对手 $\mathcal{A}$,我们定义两个实验:实验 $0$ 和实验 $1$。对于 $b=0,1$,我们定义:

\vspace{5pt}

\noindent\textbf{实验 $b$:}
\begin{itemize}
	\item 挑战者按如下方式计算 $r\in\mathcal{R}$:
	\begin{itemize}
		\item 如果 $b=0$:$s\overset{\rm R}\leftarrow\mathcal{S}$,$r\leftarrow G(s)$;
		\item 如果 $b=1$:$r\overset{\rm R}\leftarrow\mathcal{R}$。
	\end{itemize}
	然后将 $r$ 发送给对手。
	\item 给定$r$,对手计算并输出一个比特 $\hat{b}\in\{0,1\}$。
\end{itemize}

对于 $b=0,1$,定义 $W_b$ 是 $\mathcal{A}$ 在实验 $b$ 中输出 $1$ 的事件。我们将 $\mathcal{A}$ 就 $G$ 的\textbf{优势}定义为:
\[
{\rm PRG\mathsf{adv}}[\mathcal{A},G]
:=
\big\lvert
\Pr[W_0]-\Pr[W_1]
\big\rvert
\]
\end{game}

该攻击游戏如图 \ref{fig:3-1} 所示。

\begin{figure}
	\centering
	

\tikzset{every picture/.style={line width=0.75pt}} %set default line width to 0.75pt        

\begin{tikzpicture}[x=0.75pt,y=0.75pt,yscale=-0.9,xscale=0.9]
%uncomment if require: \path (0,561); %set diagram left start at 0, and has height of 561

%Shape: Rectangle [id:dp9584963727181126] 
\draw  [line width=1.2]  (0,0) -- (180,0) -- (180,250) -- (0,250) -- cycle ;
%Shape: Rectangle [id:dp7975396883149009] 
\draw  [line width=1.2]  (270,0) -- (390,0) -- (390,250) -- (270,250) -- cycle ;
%Straight Lines [id:da23162186238918725] 
\draw    (180,160) -- (267,160) ;
\draw [shift={(270,160)}, rotate = 180] [fill={rgb, 255:red, 0; green, 0; blue, 0 }  ][line width=0.08]  [draw opacity=0] (7.14,-3.43) -- (0,0) -- (7.14,3.43) -- cycle    ;
%Straight Lines [id:da4308032893165852] 
\draw    (270,190) -- (183,190) ;
\draw [shift={(180,190)}, rotate = 360] [fill={rgb, 255:red, 0; green, 0; blue, 0 }  ][line width=0.08]  [draw opacity=0] (7.14,-3.43) -- (0,0) -- (7.14,3.43) -- cycle    ;
%Shape: Rectangle [id:dp834093932982575] 
\draw  [line width=1.2]  (0,300) -- (180,300) -- (180,550) -- (0,550) -- cycle ;
%Shape: Rectangle [id:dp817311654311772] 
\draw  [line width=1.2]  (270,300) -- (390,300) -- (390,550) -- (270,550) -- cycle ;
%Straight Lines [id:da25353554094146236] 
\draw    (180,460) -- (267,460) ;
\draw [shift={(270,460)}, rotate = 180] [fill={rgb, 255:red, 0; green, 0; blue, 0 }  ][line width=0.08]  [draw opacity=0] (7.14,-3.43) -- (0,0) -- (7.14,3.43) -- cycle    ;
%Straight Lines [id:da6047218840781998] 
\draw    (270,490) -- (183,490) ;
\draw [shift={(180,490)}, rotate = 360] [fill={rgb, 255:red, 0; green, 0; blue, 0 }  ][line width=0.08]  [draw opacity=0] (7.14,-3.43) -- (0,0) -- (7.14,3.43) -- cycle    ;

% Text Node
\draw (90,20) node   [align=left] {挑战者};
% Text Node
\draw (90,45) node   [align=left] {(实验 $\displaystyle 0$)};
% Text Node
\draw (60,110) node [anchor=west] [inner sep=0.75pt]    {$s\overset{\mathrm{R}}{\leftarrow }\mathcal{S}$};
% Text Node
\draw (60,140) node [anchor=west] [inner sep=0.75pt]    {$r\overset{\mathrm{R}}{\leftarrow } G( s)$};
% Text Node
\draw (225,150) node    {$r$};
% Text Node
\draw (225,175) node    {$\hat{b} \in \{0,1\}$};
% Text Node
\draw (330,20) node    {$\mathcal{A}$};
% Text Node
\draw (90,320) node   [align=left] {挑战者};
% Text Node
\draw (90,345) node   [align=left] {(实验 $\displaystyle 1$)};
% Text Node
\draw (225,450) node    {$r$};
% Text Node
\draw (225,475) node    {$\hat{b} \in \{0,1\}$};
% Text Node
\draw (330,320) node    {$\mathcal{A}$};
% Text Node
\draw (67,440) node [anchor=west] [inner sep=0.75pt]    {$r\overset{\mathrm{R}}{\leftarrow }\mathcal{R}$};


\end{tikzpicture}
	\caption{攻击游戏 \ref{game:3-1} 的实验$0$和实验$1$}
	\label{fig:3-1}
\end{figure}

\begin{definition}[安全的 PRG]\label{def:3-1}
如果对于所有有效对手 $\mathcal{A}$,${\rm PRG\mathsf{adv}}[\mathcal{A},G]$ 的值都可忽略不计,我们就称 PRG $G$ 是\textbf{安全的}。

\end{definition}

正如 \ref{subsec:2-2-5} 小节所讨论的,攻击游戏 \ref{game:3-1} 可以被重构为一个``比特猜测"游戏。此时,挑战者不再进行两个独立的实验,而是随机选择一个 $b\in\{0,1\}$,然后针对对手 $\mathcal{A}$ 运行实验 $b$。在这个游戏中,我们记 $\mathcal{A}$ 的\emph{比特猜测优势} ${\rm PRG\mathsf{adv}}^*[\mathcal{A},G]$ 为 $|\Pr[\hat{b}=b]-1/2|$。\ref{subsec:2-2-5} 小节的推广结论(即式 \ref{eq:2-11})在此也适用:
\begin{equation}
\mathrm{PRG}\mathsf{adv}[\mathcal{A},G]
=2\cdot
\mathrm{PRG}\mathsf{adv}^*[\mathcal{A},G]
\end{equation}

我们还注意到,只有当种子空间的\emph{势 (cardinality)}是超多项式的时候,PRG 才是安全的(见练习 \ref{exer:3-5})。

\subsection{数学细节}\label{subsec:3-1-2}

和 \ref{sec:2-3} 节一样,我们在这里给出更多与 PRG 有关的数学细节。读者在初读这一小节时可以安全地跳过,而不用担心在理解上有什么损失。

首先,我们使用定义 \ref{def:2-9} 中提供的术语来精确地定义 PRG。

\begin{definition}[伪随机生成器]\label{def:3-2}
一个伪随机生成器包含一个算法 $G$,以及两个具有系统参数化 $P$ 的空间族:
\[
\mathbf{S}=\{\mathcal{S}_{\lambda,\Lambda}\}_{\lambda,\Lambda},\quad
\mathbf{R}=\{\mathcal{R}_{\lambda,\Lambda}\}_{\lambda,\Lambda}
\]
满足:
\begin{enumerate}
	\item $\mathbf{S}$ 和 $\mathbf{R}$ 是可有效识别和采样的。
	\item 算法 $G$ 是一个有效的确定性算法,对于输入 $\lambda,\Lambda,s$,其中 $\lambda\in\mathbb{Z}_{\geq1}$,$\Lambda\in{\rm Supp}(P(\lambda))$,$s\in\mathcal{S}_{\lambda,\Lambda}$,它输出 $\mathcal{R}_{\lambda,\Lambda}$ 中的一个元素。
\end{enumerate}
\end{definition}

接下来,我们需要对定义 \ref{def:3-1} 进行正确的解释。首先,在攻击游戏 \ref{game:3-1} 中,读者需要理解,对于安全参数 $\lambda$ 的每个可能取值,我们都能得到一个不同的概率空间,该空间由挑战者的随机选择和对手的随机选择共同决定。其次,挑战者会产生一个系统参数 $\Lambda$,并在游戏一开始时将其发送给对手。第三,优势 ${\rm PRG\mathsf{adv}}[\mathcal{A},G]$ 是安全参数 $\lambda$ 的一个函数,而安全性意味着,该函数是一个可忽略不计函数。