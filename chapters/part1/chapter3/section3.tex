\section{流密码的局限性:对一次性密码本的攻击}

尽管流密码是语义安全的,但它非常脆弱,如果使用不当就会完全丧失安全性。

\subsection{两次密码本是不安全的}

一个流密码可以很好地加密从 Alice 到 Bob 的\emph{单一}消息。然而,Alice 可能希望向 Bob 发送多条消息。简单起见,假设 Alice 希望对两条消息 $m_1$ 和 $m_2$ 进行加密。她天真的解决方案是使用相同的流密码密钥 $s$ 对这两条消息进行加密,即:
\begin{equation}\label{eq:3-8}
c_1\leftarrow m_1\oplus G(s),\quad
c_2\leftarrow m_2\oplus G(s)
\end{equation}
稍微思考一下就会发现,这种结构根本就是不安全的。对手如果能够截获 $c_1$ 和 $c_2$,就能计算出:
\[
\Delta:=c_1\oplus c_2=
\big(m_1\oplus G(s)\big)\oplus\big(m_2\oplus G(s)\big)
=m_1\oplus m_2
\]
并得到 $m_1$ 和 $m_2$ 的异或。考虑到英文文本中往往会包含足够多的冗余,因此给定 $\Delta=m_1\oplus m_2$,对手可以轻易地恢复出 $m_1$ 和 $m_2$。因此,只要能够获得足够长的密文,式 \ref{eq:3-8} 中的构造就会泄露明文。

式 \ref{eq:3-8} 中的构造被戏称为\textbf{两次密码本(two-time pad)}。我们上面论证了两次密码本是完全不安全的,特别是,\textbf{一个流密码的密钥不应该被用来加密多条消息}。在这本书中,我们将看到很多例子,在这些例子中,使用一个一次性密码本就足够满足要求。例如像在 \ref{subsec:5-4-1} 小节中,我们为每个消息选择一个新的随机密钥。然而,在单一密钥可能会被多次使用的环境中,绝不能直接使用流密码。我们会在第\ref{chap:5}章构建可以多次使用的密码。

不正确地重复使用流密码密钥是部署系统中的一个常见错误。例如,一个叫做 PPTP 的协议使 $A$ 和 $B$ 两方能够互相发送加密的消息。微软在 Windows NT 中使用了一种叫做 RC4 的流密码来实现 PPTP 协议。在微软最初的实现中,从 $B$ 到 $A$ 的加密密钥与从 $A$ 到 $B$ 的加密密钥是完全相同的。因此,通过窃听两个相反方向的加密信息,攻击者就可以恢复这两条消息的明文。

另一个关于两次密码本的有趣故事来自 Klehr 的转述,他非常详细地描述了二战期间在美国的俄罗斯间谍是如何使用一次性密码本把消息送回莫斯科的。正如 Klehr 所解释的,该系统由一个致命的缺陷:
\begin{quote}
在二战期间,苏联无法生产足够多的一次性密码本...来满足庞大的密码需求...。因此,他们把一些一次性密码本用了两次,还认为这不会影响到他们的系统。二战期间,美国的反间谍部门收集了所有进出的国际电报。从 1946 年开始,在英国人的帮助下,他们开始大力破解苏联的信息,由于...苏联把一些一次性密码本当作两次密码本使用的错误,在接下来的 25 年里,它们破获了大约 2900 条消息,包含 1941 年至 1946 年(苏联在此期间改用另外一套系统)之间发送的 5000 多页共计数十万条消息。
\end{quote}
解密工作的代号为维诺那(Venona)计划。维诺那计划因揭发了朱利叶斯·罗森堡和艾瑟尔·罗森堡夫妇而声名鹊起,他们参与了苏联间谍集团的证据被维诺那计划完整地揭露出来。从 1995 年开始,所有 3000 多份由维诺那计划解密的消息都被公开了。

\subsection{一次性密码本是易被控制的}

尽管语义安全性能够确保对手无法读取明文,但它没有提供完整性的保证。在使用流密码时,对手可以改变密文,而这种修改永远不会被解密者发现。更糟糕的是,我们将表明,通过改变密文,攻击者甚至可以控制解密后的明文。

假设攻击者截获了一条密文$c:=E(s,m)=m\oplus G(s)$。攻击者将 $c$ 改为 $c':=c\oplus\Delta$,其中的 $\Delta$ 是攻击者选取的某个值。因此,解密者收到修改后的消息为:
\[
D(s,c')=c'\oplus G(s)=(c\oplus\Delta)\oplus G(s)=m\oplus\Delta
\]
因此,即使攻击者不知道 $m$ 或 $s$,它也能将解密后的消息变成 $m\oplus\Delta$,而 $\Delta$ 是由攻击者选择的。因此我们称流密码是\textbf{易被控制的(malleable)},因为攻击者可以对明文造成可预测的变化。我们将在第\ref{chap:9}章中构建能够同时满足机密性和完整性的密码。

一个简单的例子是,易被控制的特性可以帮助攻击者攻击加密文件系统。为了具体说明这一问题,假设 Bob 是一位教授,Alice 和 Molly 是两个学生。Bob 的学生通过电子邮件提交他们的作业,然后 Bob 将这些电子邮件存储在一个用流密码加密的磁盘上。一封电子邮件总是以一个标准头开始。简化一下,我们可以假设一封来自 Alice 的电子邮件总是以 \texttt{From:Alice} 为开头。

现在,假设 Molly 能够进入 Bob 的磁盘,并找到了 Alice 发出的包含她的作业的电子邮件的密文。于是 Molly 可以窃取 Alice 的作业,方法如下。她只需将适当的五个字符的字符串异或到密码文本的第 6 至 10 位,以便将邮件首部 \texttt{From:Alice} 改为 \texttt{From:Molly}。Molly 只需要对密文进行操作,并且不需要知道 Bob 的私钥。Bob 永远不会知道文件被修改过,当他给 Alice 的作业打分时,还以为这份作业是 Molly 交的,因此 Molly 就窃取了 Alice 的成果。

当然,为了使这种攻击有效,Molly 必须能以某种方式在 Bob 的加密磁盘上找到 Alice 的电子邮件。然而,在一些加密文件系统的实现中,文件的元数据(如文件名、修改时间等)是不加密的。有了这些元数据,Molly 就可以直接找到 Alice 的加密邮件并进行攻击了。