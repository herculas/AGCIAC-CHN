\section{一个有趣的应用:抛掷硬币与比特承诺}\label{sec:3-12}

Alice 和 Bob 要出去约会了。Alice 想看某一部电影,而 Bob 想看另一部。他们决定抛一枚硬币来随机选择电影。如果抛硬币的结果是正面,他们就去看 Alice 选择的电影,否则就去看 Bob 选择的电影。当 Alice 和 Bob 离得很近的时候,这很容易操作:他们中的一个人,比如说 Bob,抛出一枚硬币,他们都能验证这个结果。但当他们相距甚远,并且在电话中交谈时,这就比较难了。Bob 可以在他那边掷硬币,然后告诉 Alice 结果,但是 Alice 却没有理由相信这个结果。Bob 可以简单地声称抛硬币的结果是反面,而 Alice 则没有办法证实这一点。这不是一个开始约会的好方法。

解决这个问题的一个简单方法是利用一种叫做\textbf{比特承诺(bit commitment)}的密码学原语。它能够让 Bob 承诺一个他选择的比特 $b\in\{0,1\}$。稍后,Bob 可以打开承诺,让 Alice 相信 $b$ 是 Bob 所承诺的值。承诺一个比特 $b$ 的结果是一个\textbf{承诺序列(commitment string)} $c$,它和一个\textbf{打开序列(opening string)} 被一起发送给 Alice,由 Bob 在以后打开承诺时使用。如果一个承诺方案满足以下两个特性,它就是安全的:
\begin{itemize}
	\item \textbf{隐藏(hiding)}:承诺序列 $c$ 不会透露关于承诺比特 $b$ 的任何信息。更确切地说,当承诺比特为 $0$ 时$c$ 的分布与承诺比特为 $1$ 时 $c$ 的分布是不可区分的。在我们提出的比特承诺方案中,隐藏属性取决于特定的 PRG $G$ 的安全性。
	\item \textbf{绑定(binding)}:令 $c$ 是 Bob 输出的承诺序列。如果 Bob 可以将该承诺打开为某个 $b\in\{0,1\}$,那么他就不能将其打开为 $\bar{b}$。这就保证了一旦 Bob 承诺了一个比特 $b$,他就只可以把它作为 $b$ 打开,而不能是其他的值。在我们提出的承诺方案中,绑定属性无条件成立。
\end{itemize}

\begin{snote}[抛掷硬币。]
使用一个承诺方案,Alice 和 Bob 就可以生成一个随机比特 $b\in\{0,1\}$,这样,假设协议成功终止,任何一方都不能使结果偏向他们的倾向。这样的协议被称为\textbf{抛掷硬币协议(coin flipping protocols)}。结果比特 $b$ 将决定他们去看什么电影。

Alice 和 Bob 使用下面这个简单的抛掷硬币协议:

\vspace*{5pt}

\hspace*{5pt} 步骤 1:Bob 随机选择一个比特 $b_0\overset{\rm R}\leftarrow\{0,1\}$。\\
\hspace*{50pt} Alice 和 Bob 执行承诺协议。\\
\hspace*{50pt} 通过该协议,Alice 获得对 $b_0$ 的承诺 $c$,Bob 获得一个打开序列 $s$。\\
\hspace*{26pt} 步骤 2:Alice 随机选择一个比特 $b_1\overset{\rm R}\leftarrow\{0,1\}$,并将其明文发送给 Bob。\\
\hspace*{26pt} 步骤 3:Bob 通过向 Alice 揭露 $b_0$ 和 $s$ 来打开承诺。\\
\hspace*{50pt} Alice 验证 $c$ 是否是对 $b_0$ 的承诺,如果验证失败则终止。

\vspace*{3pt}

\hspace*{5pt} 输出:结果比特 $b:=b_0\oplus b_1$。

\vspace*{5pt}

\noindent
我们声称,如果协议成功终止,并且一方诚实地遵守协议,那么另一方就不能使结果朝它的倾向偏移。根据隐藏属性,Alice 在步骤 $1$ 结束时不会得到任何关于 $b_0$ 的信息,因此她对 $b_1$ 的选择与 $b_0$ 的值无关。根据绑定属性,Bob 只能在步骤 $3$ 中对他在步骤 $1$ 中选择的 $b_0$ 打开承诺 $c$。而由于他在 Alice 选择 $b_1$ 之前就选择了 $b_0$,所以 Bob 对 $b_0$ 的选择与 $b_1$ 无关。我们可以得出结论,输出比特 $b$ 是两个独立比特的异或。因此,如果一方诚实地遵守协议,另一方就不可能使产生的比特有所偏向。

这个协议的一个问题是,Bob 在步骤 $2$ 结束时,在 Alice 获取比特之前就知道了生成的比特。原则上,如果结果不是 Bob 想要的,他也可以在步骤 $2$ 结束时中止协议,并试图重新启动协议,希望下一次运行能得到他想要的结果。更复杂的抛掷硬币协议能够避免这个问题,但代价是需要更多回合的交互。
\end{snote}

\begin{snote}[来自安全PRG的比特承诺。]
剩下的工作就是构建一个让 Bob 承诺他的比特 $b_0\in\{0,1\}$ 的安全比特承诺方案。我们使用Naor提出的一个优雅的构造来完成这个工作。

令 $G:\mathcal{S}\to\mathcal{R}$ 是一个安全的 PRG,其中 $|\mathcal{R}|\geq|\mathcal{S}|^3$ 和 $\mathcal{R}=\{0,1\}^n$对某个 $n$ 成立。为了承诺比特 $b_0$,Alice 和 Bob 参与以下协议:

\vspace*{5pt}

\hspace*{5pt} Bob 承诺比特 $b0\in\{0,1\}$:\\
\hspace*{50pt} 步骤 1:Alice 随机选择一个 $r\in\mathcal{R}$,并将 $r$ 发送给 Bob。\\
\hspace*{50pt} 步骤 2:Bob 随机选择一个 $s\in\mathcal{S}$ 并计算 $c\leftarrow\mathrm{com}(s,r,b_0)$,其中 $\mathrm{com}(s,r,b_0)$ 是以下函数:\\
\[
c=\mathrm{com}(s,r,b_0):=
\left\{
\begin{array}{ll}
G(s) & b_0=0\\
G(s)\oplus r & b_0=1
\end{array}	
\right.
\]

\hspace*{5pt} Bob 输出 $c$ 作为承诺序列,并将 $s$ 作为打开序列。

\vspace*{5pt}

\noindent
当到了打开承诺的时候,Bob 向 Alice 发送 $(b_0,s)$。如果 $c=\mathrm{com}(s,r,b_0)$,Alice 就接受打开值,否则就拒绝它。

隐藏属性直接来自 PRG 的安全性:因为输出 $G(s)$ 与 $\mathcal{R}$ 上的均匀随机序列在计算上是不可区分的,所以 $G(s)\oplus r$ 与 $\mathcal{R}$ 上的均匀随机序列在计算上也是不可区分的。因此,无论是 $b_0=0$ 还是 $b_0=1$,承诺序列$c$ 与 $\mathcal{R}$ 上的均匀序列在计算上是不可区分的,与要求一样。

只要 $1/|\mathcal{S}|$ 可忽略不计,绑定属性就无条件成立。Bob 可以将一个承诺 $c\in\mathcal{R}$ 打开为 $0$ 和 $1$ 的唯一办法是找到两个种子 $s_0,s_1\in\mathcal{S}$ 使得 $c=G(s_0)=G(s_1)\oplus r$,这意味着 $G(s_0)\oplus G(s_1)=r$。如果存在种子$s_0,s_1\in\mathcal{S}$ 使得 $G(s_0)\oplus G(s_1)=r$,我们就称这样的 $r\in\mathcal{R}$ 是``坏的"。种子对 $(s_0,s_1)$ 的数量是 $|\mathcal{S}|^2$,因此坏的 $r$ 的数量最多也是 $|\mathcal{S}|^2$。由此可见,Alice 选到坏的 $r$ 的概率最多为 $|\mathcal{S}|^2/|\mathcal{R}| < |\mathcal{S}|^2/|\mathcal{S}|^3 = 1/|\mathcal{S}|$,该值可忽略不计。因此,Bob 能把承诺 $c$ 打开为 $0$ 和 $1$ 的概率是可忽略不计的。
\end{snote}

这样,我们就完成了对比特承诺方案的介绍。我们将在 \ref{sec:8-12} 节看到一个更有效的承诺方案以及更多与承诺有关的应用,但在那之前,我们还需要引入更多的密码学工具。