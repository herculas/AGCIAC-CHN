\section{下一比特测试}\label{sec:3-5}

令 $G$ 是一个定义在 $(\{0,1\}^\ell,\{0,1\}^L)$ 上的 PRG,它能将 $\ell$ 比特序列拉伸到 $L$ 比特长。有许多方法可以让对手区分 $G$ 的伪随机输出和真正的随机比特序列。事实上,当给定 $G$ 的输出的前 $L-1$ 比特时,假设一个有效对手能够计算出其输出的最后一笔特。直观上看,这样一个对手的存在就预示着 $G$ 是不安全的,因为给定一个 $L$ 比特的真随机序列的前 $L-1$ 比特,任何人都最多只有一半的机会猜中最后一个比特。事实上,该结论的一个逆命题也是真的。

我们下面会正式定义 PRG 的\textbf{不可预测性 (unpredictability)}。它在本质上就是说,给定 $G$ 的输出的前 $i$ 个比特(这里,$i$ 是一个对手选择的索引),以显著高于 $1/2$ 的概率预测出下一个比特(即第 $i+1$ 比特)是困难的。之后,我们将证明,不可预测性与安全性是等价的。安全性能够导出不可预测性这一事实是很直观的:如果能够有效地预测伪随机序列的下一比特,我们就能立即提供一个有效的统计检验。相反,由不可预测性能够导出安全性这一结论是相当有趣的(需要花一些精力来证明),这实际上是指,如果存在某个有效的统计检验能够打破 PRG 的安全性,就必然存在一个能够有效地预测伪随机序列的下一比特的方法。

\begin{game}[不可预测的 PRG]\label{game:3-2}
对于一个定义在 $(\mathcal{S},\{0,1\}^L)$ 上的给定 PRG $G$ 和一个给定对手 $\mathcal{A}$,攻击游戏的过程如下:
\begin{itemize}
	\item 对手向挑战者发送一个索引 $i$,其中 $0\leq i\leq L-1$。
	\item 挑战者计算:
	\[
	s\overset{\rm R}\leftarrow\mathcal{S},
	\quad
	r\leftarrow G(s)
	\]
	并将 $r[0\dots i-1]$ 发送给对手。
	\item 对手输出 $g\in\{0,1\}$。
\end{itemize}
如果 $r[i]=g$,我们就称 $\mathcal{A}$ \textbf{获胜}。我们将 $\mathcal{A}$ 的\textbf{优势} $\mathrm{Pred}\mathsf{adv}[\mathcal{A},G]$ 定义为 $|\Pr[\mathcal{A}\text{ wins}]-1/2|$。
\end{game}

\begin{definition}[不可预测的 PRG]\label{def:3-3}
如果对于所有有效对手 $\mathcal{A}$,$\mathrm{Pred}\mathsf{adv}[\mathcal{A},G]$ 的值都可忽略不计,我们就称 PRG $G$ 是\textbf{不可预测的 (unpredictable)}。
\end{definition}

我们下面首先证明,安全性能够导出不可预测性。

\begin{theorem}\label{theo:3-4}
令 $G$ 是一个定义在 $(\mathcal{S},\{0,1\}^L)$ 上的 PRG。如果 $G$ 是安全的,$G$ 就是不可预测的。
\begin{quote}
特别地,对于每个如攻击游戏 \ref{game:3-2} 中那样攻破 $G$ 的不可预测性的对手 $\mathcal{A}$,都存在一个如攻击游戏 \ref{game:3-1} 中那样攻破 $G$ 的安全性的对手 $\mathcal{B}$,其中 $\mathcal{B}$ 是一个围绕 $\mathcal{A}$ 的基本包装器,满足:
\end{quote}
\[
\mathrm{Pred}\mathsf{adv}[\mathcal{A},G]
=
\mathrm{PRG}\mathsf{adv}[\mathcal{B},G]
\]
\end{theorem}

\begin{proof}
令 $\mathcal{A}$ 是一个攻破 $G$ 的不可预测性的对手,令 $i$ 表示 $\mathcal{A}$ 所选择的索引。此外,假设 $\mathcal{A}$ 能以 $1/2+\epsilon$ 的概率赢得攻击游戏 \ref{game:3-2},则有 $\mathrm{Pred}\mathsf{adv}[\mathcal{A},G]=|\epsilon|$。

我们下面以 $\mathcal{A}$ 为子程序,建立一个攻击 $G$ 的安全性的对手 $\mathcal{B}$,其运行方式如下:

\vspace*{10pt}

\hspace*{5pt} 当从它的挑战者处收到 $r\in\{0,1\}^L$ 时,$\mathcal B$ 进行以下操作:

\vspace*{5pt}

\hspace*{28.5pt} $\mathcal{B}$ 将 $r[0\dots i-1]$ 发送 $\mathcal{A}$,获得 $\mathcal{A}$ 的输出 $g\in\{0,1\}$;\\
\hspace*{50pt} 如果 $r[i]=g$,$\mathcal{B}$ 输出 $1$,否则就输出 $0$。

\vspace*{10pt}

对于 $b=0,1$,令 $W_b$ 为$\mathcal{B}$在攻击游戏 \ref{game:3-1} 的实验 $b$ 中输出 $1$ 的事件。在实验 $0$ 中,$r$ 是 $G$ 的一个伪随机输出,并且当且仅当 $r[i]=g$ 成立时,$W_0$ 才会发生,因此根据定义,我们有:
\[
\Pr[W_0]=1/2+\epsilon
\]
在实验 $1$ 中,$r$ 是一个真随机的比特序列。但同样地,当且仅当 $r[i]=g$ 成立时,$W_1$ 才会发生;然而,在这种情况下,随机变量 $r[i]$ 和 $g$ 的值相互独立,因此我们有:
\[
\Pr[W_1]=1/2
\]
于是有:
\[
\mathrm{PRG}\mathsf{adv}[\mathcal{B},G]
=\big\lvert\Pr[W_1]-\Pr[W_0]\big\rvert
=|\epsilon|
=\mathrm{Pred}\mathsf{adv}[\mathcal{A},G]
\qedhere
\]
\end{proof}

相比之下,证明不可预测性能够导出安全性,是一个更有趣,也更有挑战性的任务。在详细介绍具体的证明之前,我们先勾勒出整体思路。

首先,我们将采用一个混合论证,这将使我们能够论证,如果 $\mathcal{A}$ 是一个能够有效区分 $L$ 比特的伪随机序列和真随机序列的有效对手,我们就必然可以构造一个有效对手 $\mathcal{B}$,它能够有效地区分:
\[
x_1\cdots x_j\;r
\]
和:
\[
x_1\cdots x_j\;x_{j+1}
\]
其中 $j$ 是一个随机选出的索引,$x_1,\dots,x_L$ 是伪随机输出,而 $r$ 是一个随机比特。因此,对手 $\mathcal{B}$ 可以在给定 $x_1,\dots,x_j$ 这个``侧信息"的情况下有效区分伪随机比特 $x_{j+1}$ 和真随机比特 $r$。

我们想把 $\mathcal{B}$ 的区分优势变成一种预测优势。大致的想法是这样的:给定 $x_1,\dots,x_j$,我们向 $\mathcal{B}$ 提供一个序列 $x_1\cdots x_j\;r$,其中的 $r$ 是一个随机选出的比特;如果 $\mathcal{B}$ 输出 $1$,我们对 $x_{j+1}$ 的预测值就是 $r$;否则,我们对 $x_{j+1}$ 的预测值就是 $\bar{r}$($r$ 的补码)。

下面的一般结论能够证明上述预测策略的有效性,我们称之为\emph{区分者/预测者引理}。基本设置是这样的:
\begin{itemize}
	\item 一个随机变量 $\mathsf{X}$,它对应于上面的``侧信息" $x_1,\dots,x_j$,以及对手 $\mathcal{B}$ 所使用的任何随机硬币;
	\item 一个取值为 $0$ 或 $1$ 的随机变量 $\mathsf{B}$,它对应于上面的 $x_{j+1}$,并可能与 $\mathsf{X}$ 相关;
	\item 一个取值为 $0$ 或 $1$ 的随机变量 $\mathsf{R}$,它对应于上面的 $r$,并且与 $(\mathsf{X},\mathsf{B})$ 无关;
	\item 一个函数 $d$,它对应于使得 $\mathcal{B}$ 的区分优势等于 $|\epsilon|$ 的 $\mathcal{B}$ 的策略,其中 $\epsilon=\Pr[d(\mathsf{X},\mathsf{B})=1]-\Pr[d(\mathsf{X},\mathsf{R})=1]$。
\end{itemize}
该引理表明,如果我们使用上面介绍的预测策略来定义 $\mathsf{B}'$,即当 $d(\mathsf{X},\mathsf{R})=1$ 时,令 $\mathsf{B}'=\mathsf{R}$,否则令 $\mathsf{B}'=\mathsf{\overline R}$,那么预测值 $\mathsf{B}'$ 等于实际值 $\mathsf{B}$ 的概率正好就是 $1/2+\epsilon$。下面是该引理的精确陈述:

\begin{lemma}[区分者/预测者引理]\label{lemma:3-5}
令 $\mathsf{X}$ 是一个在某个集合 $S$ 中取值的随机变量,令 $\mathsf{B}$ 和 $\mathsf{R}$ 是取值为 $0$ 或 $1$ 的随机变量,其中 $\mathsf{R}$ 均匀分布在 $\{0,1\}$ 上,且与 $(\mathsf{X},\mathsf{B})$ 无关。令 $d:S\times\{0,1\}\to\{0,1\}$ 是一个任意的函数,并令:
\[
\epsilon
:=
\Pr[d(\mathsf{X},\mathsf{B})=1]
-
\Pr[d(\mathsf{X},\mathsf{R})=1]
\]
按如下方式定义随机变量 $\mathsf{B}'$:
\[
\mathsf{B}':=
\left\{
\begin{array}{ll}
\mathsf{R}, & d(\mathsf{X},\mathsf{R})=1\\
\mathsf{\overline R}, & \text{otherwise}
\end{array}
\right.
\]
则有:
\[
\Pr[\mathsf{B}'=\mathsf{B}]={1}/{2}+\epsilon
\]
\end{lemma}

\begin{proof}
以事件 $\mathsf{B}=\mathsf{R}$ 和 $\mathsf{B}=\mathsf{\overline R}$ 为条件,我们计算 $\Pr[\mathsf{B}'=\mathsf{B}]$:
\[
\begin{aligned}
\Pr[\mathsf{B}'=\mathsf{B}]
&=\Pr[\mathsf{B}'=\mathsf{B}\,|\,\mathsf{B}=\mathsf{R}]\cdot\Pr[\mathsf{B}=\mathsf{R}]+\Pr[\mathsf{B}'=\mathsf{B}\,|\,\mathsf{B}=\mathsf{\overline R}]\cdot\Pr[\mathsf{B}=\mathsf{\overline R}]\\
&=\Pr[d(\mathsf{X},\mathsf{R})=1\,|\,\mathsf{B}=\mathsf{R}]\cdot\frac{1}{2}+\Pr[d(\mathsf{X},\mathsf{R})=0\,|\,\mathsf{B}=\mathsf{\overline R}]\cdot\frac{1}{2}\\
&=\frac{1}{2}
\Big(\Pr[d(\mathsf{X},\mathsf{R})=1\,|\,\mathsf{B}=\mathsf{R}]+
\big(
1-\Pr[d(\mathsf{X},\mathsf{R})=1\,|\,\mathsf{B}=\mathsf{\overline R}]
\big)
\Big)\\
&=\frac{1}{2}+\frac{1}{2}(\alpha-\beta)
\end{aligned}
\]
其中:
\[
\alpha:=\Pr[d(\mathsf{X},\mathsf{R})=1\,|\,\mathsf{B}=\mathsf{R}],
\quad
\beta:=\Pr[d(\mathsf{X},\mathsf{R})=1\,|\,\mathsf{B}=\mathsf{\overline R}]
\]
根据独立性,我们有:
\[
\alpha=\Pr[d(\mathsf{X},\mathsf{R})=1\,|\,\mathsf{B}=\mathsf{R}]=\Pr[d(\mathsf{X},\mathsf{B})=1\,|\,\mathsf{B}=\mathsf{R}]=\Pr[d(\mathsf{X},\mathsf{B})=1]
\]
想要知道最后一个等式为什么成立,可以参考练习 \ref{exer:3-25} 的结论。

因此,我们可以计算出:
\[
\begin{aligned}
\epsilon
&=\Pr[d(\mathsf{X},\mathsf{B})=1]-\Pr[d(\mathsf{X},\mathsf{R})=1]\\
&=\alpha-
\Big(\Pr[d(\mathsf{X},\mathsf{R})=1\,|\,\mathsf{B}=\mathsf{R}]\cdot\Pr[\mathsf{B}=\mathsf{R}]+\Pr[d(\mathsf{X},\mathsf{R})=1\,|\,\mathsf{B}=\mathsf{\overline R}]\cdot\Pr[\mathsf{B}=\mathsf{\overline R}]
\Big)\\
&=\alpha-\frac{1}{2}(\alpha+\beta)\\
&=\frac{1}{2}(\alpha-\beta)
\end{aligned}
\]
这就证明了该引理。
\end{proof}

\begin{theorem}\label{theo:3-6}
令 $G$ 是一个定义在 $(\mathcal{S},\{0,1\}^L)$ 上的 PRG。如果 $G$ 是不可预测的,$G$ 就是安全的。
\begin{quote}
特别地,对于每个如攻击游戏 \ref{game:3-1} 中那样攻破 $G$ 的安全性的对手 $\mathcal{A}$,都存在一个如攻击游戏 \ref{game:3-2} 中那样攻破 $G$ 的不可预测性的对手 $\mathcal{B}$,其中 $\mathcal{B}$ 是一个围绕 $\mathcal{A}$ 的基本包装器,满足:
\end{quote}
\[
\mathrm{PRG}\mathsf{adv}[\mathcal{A},G]
=L\cdot
\mathrm{Pred}\mathsf{adv}[\mathcal{B},G]
\]
\end{theorem}

\begin{proof}
令 $\mathcal{A}$ 像攻击游戏 \ref{game:3-1} 中那样攻击 $G$。利用 $\mathcal{A}$,我们建立一个预测器 $\mathcal{B}$,它像攻击游戏 \ref{game:3-2} 中那样攻击 $G$,工作方式如下:
\begin{itemize}
	\item 随机选出 $\omega\in\{1,\dots,L\}$。
	\item 向挑战者发送 $L-\omega$,得到一个序列 $x\in\{0,1\}^{L-\omega}$。
	\item 随机生成 $\omega$ 个比特 $r_1,\dots,r_\omega$,并将 $L$ 比特序列 $x\,\Vert\,r_1\cdots r_\omega$ 发送给 $\mathcal{A}$。
	\item 如果 $\mathcal{A}$ 输出 $1$,$\mathcal{B}$ 就输出 $r_1$;否则,$\mathcal{B}$ 输出 $\overline r_1$。
\end{itemize}

为了分析 $\mathcal{B}$,我们考虑 $L+1$ 个混合游戏,称为混合 $0$,混合 $1$,$\dots$,混合 $L$。对于 $j=0,\dots,L$,我们将混合 $j$ 定义为 $\mathcal{A}$ 和挑战者之间进行的游戏,挑战者生成一个由 $L-j$ 个伪随机比特后缀 $j$ 个真随机比特组成的序列 $r$;也就是说,挑战者随机选择 $s\in\mathcal{S}$ 和 $t\in\{0,1\}^j$,并向 $\mathcal{A}$ 发送比特序列:
\[
r=G(s)[0\dots L-j-1]\,\Vert\,t
\]
同之前一样,$\mathcal{A}$ 在游戏结束时输出 $0$ 或 $1$,我们记 $p_j$ 为 $\mathcal{A}$ 在混合 $j$ 中输出 $1$ 的概率。请注意,$p_0$ 就等于 $\mathcal{A}$ 在攻击游戏 \ref{game:3-1} 的实验 $0$ 中输出 $1$ 的概率,而 $p_L$ 就等于 $\mathcal{A}$ 在攻击游戏 \ref{game:3-1} 的实验 $1$ 中输出 $1$ 的概率。

令 $W$ 为 $\mathcal{B}$ 在攻击游戏 \ref{game:3-2} 中获胜的事件(即它正确地预测了下一个比特),那么我们有:
\[
\begin{aligned}
\Pr[W]
&=\sum_{j=1}^L\Pr[W\,|\,\omega=j]\cdot\Pr[\omega=j]\\
&=\frac{1}{L}\sum_{j=1}^L\Pr[W\,|\,\omega=j]\\
&=\frac{1}{L}\sum_{j=1}^L\Big(\frac{1}{2}+p_{j-1}-p_j\Big)
\quad\text{\emph{(根据引理} \ref{lemma:3-5}\emph{\,)}}\\
&=\frac{1}{2}+\frac{1}{L}(p_0-p_L)
\end{aligned}
\]
这就证明了该定理。
\end{proof}