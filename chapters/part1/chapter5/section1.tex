\section{引言}\label{sec:5-1}

在第\ref{chap:2}章中,我们重点讨论了对单一消息进行加密的问题。现在,我们想要考虑对多条消息进行加密的问题。为了使讨论更加具体,假如 Alice 想加密她存放在某个文件服务器上的文件,同时把她的密钥安全地保存在她的 U 盘上。

一种可能的方法是,Alice 使用不同的密钥分别对每个文件进行加密。对于这种方法,Alice 需要在文件服务器中存储每个文件的密文,然后在 U 盘中存储相应的密钥。正如我们将在 \ref{sec:5-2} 节中详细探讨的,只要 Alice 所使用的密码是语义安全的,这种方法就确实能够为她提供一定程度的安全性。虽然一个文件可能有几兆字节那么大,但是任何实用的密码的密钥都只有几个字节长。但问题是,如果 Alice 需要管理成千上万的文件,她就需要在 U 盘上存储成千上万个密钥,这时,密钥的存储空间就成了一个不可忽略的问题。

正如我们所看到的,上述方法虽然安全,但空间利用效率不高,因为每个文件都需要一个密钥。面对这个问题,Alice 可能会简单地决定用同一个密钥来加密所有文件。虽然这样效率更高,但这种方法可能是不安全的。事实上,如果 Alice 使用一个只提供语义安全性的密码(如定义 \ref{def:2-2}),她就无法得到任何有意义的安全保证,甚至很可能被暴露在现实的攻击之下。

比如说,假如 Alice 使用 \ref{sec:3-2} 节所讨论的流密码 $\mathcal{E}$。在这里,Alice 的密钥是一个 PRG $G$ 的种子 $s$。如果将文件 $m$ 视作一个比特序列,那么她会通过计算 $c:=m\oplus\Delta$ 来加密 $m$,其中 $\Delta$ 是由``密钥流" $G(s)$ 的前 $|m|$ 比特组成的。但是,如果 Alice 使用相同的种子 $s$ 加密多个文件,那么对手很容易就能发动攻击。例如,如果对手知道一个文件的某些比特,它就可以直接计算出密钥流的相应比特,从而获得任何其他文件的相应比特。你可能会问,对手怎么可能知道某个文件的某些比特呢?事实上,很多文件格式,比如电子邮件,都包含标准的头部信息(见例 \ref{exmp:2-6})。因此,如果对手知道一个给定的密文是一个电子邮件的加密,它可以得到对应于这个标准头部位置的密钥流的比特。为了发动更具破坏性的攻击,对手可能会尝试更狡猾的方法:它可以简单地给 Alice 发送一封大的电子邮件,例如 1 MB 长。假设 Alice 的软件会自动将这封电子邮件的加密存储在她的服务器上,那么当对手窥探她的文件服务器时,它就可以恢复相应的 1 MB 的密钥流,然后它就可以解密任何存储在 Alice 的服务器上的长度在 1 MB 以内的文件!这封邮件甚至可能被 Alice 的垃圾邮件过滤器捕捉到,所以尽管她的加密软件可能非常勤奋地将这封邮件和其他所有东西一起加密,她也有可能压根看不到这封邮件。这种类型的攻击被称为\emph{选择明文攻击},因为对手迫使 Alice 在攻击过程中为自己加密若干个由它所选择的明文。

显然,上面的流密码不足以胜任这项工作。事实上,流密码以及\emph{任何其它的确定性密码},都不应该被用来用同一个密钥加密多个文件。这是为什么呢?因为任何确定性密码如果被用来用同一个密钥对多个文件进行加密,都会有一个固有的弱点:对手总是能够分辨出两个文件是否相同。事实上,对于一个确定性的密码,如果相同的密钥被用来加密相同的信息,所产生的密文总是相同的(反之,对于\emph{任何}密码,如果相同的密钥被用来加密两个不同的信息,所产生的密文必然是不同的)。虽然这种类型的攻击肯定不像上面讨论的那些对手可以任意读取 Alice 的文件的攻击那样引人注目,但它仍然是一个严重的漏洞。比如说,虽然从技术上说,我们在 \ref{subsec:4-1-4} 小节中所介绍的 ECB 模式讨论的是加密由许多分组构成的单一消息,但它其实仍然适用于这里的用同一密钥加密多个单分组消息的问题。

事实上,Alice \emph{确实}可以使用一个密码和一个单一短密钥安全地加密她所有的文件,但是她需要使用一种更适合这项任务的密码。特别是,由于任何确定性密码都有上述的固有弱点,她将不得不使用一个\emph{概率性密码},即一个使用概率性加密算法的密码。这样,在同一密钥下对同一明文的不同加密将(通常)产生不同的密文。对于 Alice 的任务,她希望有一个密码能够达到比语义安全更强的安全水平。对于这种需求,适当的安全概念被称为\emph{针对选择明文攻击的语义安全性}。在 \ref{sec:5-3} 节及随后的几节,我们将正式定义这一概念,然后介绍一些基于语义安全密码、PRF 和分组密码的构造,并且考察几个``现实世界"中的实际案例。

虽然我们以``文件加密"这一问题为例来引出本章的主题,但我们也可以考虑``安全网络通信"的问题。在这种场景下,我们可以考虑这样的情况:Alice 和 Bob 共享一个(或多个)密钥,Alice 想通过一个不安全的网络秘密地将一些消息传送给 Bob。如果 Alice 可以方便地将她所有的消息都串接成一个长消息,她就可以使用流密码对整个消息进行加密。然而,由于各种技术原因,这可能是不可行的,比如说在一些场景中,Alice 希望能够以任意的顺序和时间发送消息。这时,她就面临着一个与``文件加密"问题非常相似的需求。同样地,如果 Alice 和 Bob 想使用一个单一的短密钥,他们就需要一个语义安全的密码,以防止选择明文攻击。

我们再次强调,就像第\ref{chap:2}章一样,本章所涉及的技术\emph{并不}提供任何\emph{数据完整性},也不解决双方如何在开始共享一个密钥的问题。我们会在更后面的章节中讨论这些问题。