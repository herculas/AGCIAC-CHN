\section{针对选择明文攻击的语义安全性}\label{sec:5-3}

现在,我们考虑一下 Alice 在本章引言中面临的问题,即她想用一个单一的,并且最好是短的密钥来加密她系统中所有的文件。对于这项任务,正确的安全概念是\textbf{针对选择明文攻击的语义安全性(semantic security against chosen plaintext attack)},简称为 \textbf{CPA 安全性}。

\begin{game}[CPA安全性]\label{game:5-2}
对于一个定义在 $(\mathcal{K},\mathcal{M},\mathcal{C})$ 上的给定密码 $\mathcal{E}=(E,D)$ 和一个给定对手 $\mathcal{A}$,我们定义两个实验:实验$0$和实验$1$。对于$b=0,1$,我们定义:

\noindent\textbf{实验$b$:}
\begin{itemize}
	\item 挑战者选取 $k\overset{\rm R}\leftarrow\mathcal{K}$。
	\item 对手向挑战者发起一连串的查询。\\
    对于 $i = 1,2,\dots$,第 $i$ 个查询是一对相同长度的消息 $m_{i0},m_{i1}\in\mathcal{M}$。\\
    挑战者选取 $c_i\overset{\rm R}\leftarrow E(k,m_{ib})$,然后将 $c_i$ 发送给对手。
    \item 对手输出一个比特 $\hat{b}\in\{0,1\}$。
\end{itemize}

对于 $b=0,1$,令 $W_b$ 是 $\mathcal{A}$ 在实验 $b$ 中输出 $1$ 的事件。我们定义 $\mathcal{A}$ 就 $\mathcal{E}$ 的\textbf{优势}为:
\[
{\rm CPA}\mathsf{adv}[\mathcal{A}, \mathcal{E}]:=|\Pr[W_0]-\Pr[W_1]|
\]
\end{game}

CPA攻击游戏 \ref{game:5-2} 和MSS攻击游戏 \ref{game:5-1} 之间的唯一区别是,在CPA攻击游戏中,所有加密都使用相同的密钥,而在MSS攻击游戏中,每次加密都选择不同的密钥。在CPA攻击游戏中,对手的查询可以是自适应的,就像在MSS攻击游戏中一样。

\begin{definition}[CPA 安全性]\label{def:5-2}
如果对于所有有效对手 $\mathcal{A}$,${\rm CPA}\mathsf{adv}[\mathcal{A}, \mathcal{E}]$ 的值都可忽略不计,我们就称密码 $\mathcal{E}$ \textbf{对选择明文攻击是语义安全 (semantically secure against chosen plaintext attack)} 的,简称为\textbf{CPA 安全 (CPA secure)}的。
\end{definition}

和 \ref{subsec:2-2-5} 小节一样,攻击游戏 \ref{game:5-2} 可以被重构为一个``比特猜测"游戏,在这个游戏中,挑战者不再有两个独立的实验,而是随机选择一个 $b\in\{0,1\}$,然后对对手 $\mathcal{A}$ 运行实验 $b$。我们将 $\mathcal{A}$ 的\emph{比特猜测优势}记为${\rm CPA}\mathsf{adv}^*[\mathcal{A}, \mathcal{E}]:=|\Pr[\hat{b}=b]-{1}/{2}|$,那么与之前一样(根据式 \ref{eq:2-11}),我们有:
\begin{equation}\label{eq:5-4}
{\rm CPA}\mathsf{adv}[\mathcal{A}, \mathcal{E}]=2\cdot{\rm CPA}\mathsf{adv}^*[\mathcal{A}, \mathcal{E}]
\end{equation}

我们再次回到本章引言中讨论的``文件加密"问题。定义 \ref{def:5-2} 说的是,如果 Alice 只用一个单一的密钥用 CPA 安全密码对她的所有文件进行加密,那么即使对手能够得到存储在文件服务器上的密文,它也无法获取关于 Alice 文件明文的有效信息(可能除了一些关于长度的信息)。请注意,即使对手能够主动地影响 Alice 加密的明文,也无法破坏加密的安全性。

\begin{example}\label{exmp:5-1}
为了让读者熟悉上面的定义,我们先表明,没有任何确定性密码能够满足 CPA 安全性的定义。假设 $\mathcal{E}=(E,D)$ 是一个确定性密码。我们用如下方式构建一个 CPA 对手 $\mathcal{A}$。假设 $m,m'$ 是 $\mathcal{E}$ 的消息空间中的任意两条不同的消息。对手 $\mathcal{A}$ 向其挑战者发出两个查询:第一个查询是 $(m,m')$,第二个查询是 $(m,m)$。假设挑战者对这两个查询的应答分别为 $c_1$ 和 $c_2$。如果 $c_1=c_2$,对手 $\mathcal{A}$ 输出 $1$,否则就输出 $0$。

我们下面计算 ${\rm CPA}\mathsf{adv}[\mathcal{A}, \mathcal{E}]$。一方面,在攻击游戏 \ref{game:5-2} 的实验 $0$ 中,挑战者在回应两个查询时都加密了 $m$,因此 $c_1=c_2$。所以,$\mathcal{A}$ 在这个实验中输出 $1$ 的概率为 $1$(这正是我们需要 $\mathcal{E}$ 的确定性假设的地方)。另一方面,在实验 $1$ 中,挑战者加密了 $m'$ 和 $m$,所以 $c_1\neq c_2$;因此,$\mathcal{A}$ 在这个实验中输出 $1$ 的概率为 $0$。由此可知,${\rm CPA}\mathsf{adv}[\mathcal{A}, \mathcal{E}]=1$。

将这个例子中的攻击进一步推广,CPA 安全的密码不仅必须是概率性的,还要保证对于同一消息,两次加密所产生的密文相同的可能性尽可能小,见练习 \ref{exer:5-12}。
\end{example}

\begin{remark}\label{remark:5-1}
与定理 \ref{theo:5-1} 类似,可以直接得到,如果一个密码本身是 CPA 安全的,那么它在多密钥的情况下也是 CPA 安全的,见练习 \ref{exer:5-2}。
\end{remark}