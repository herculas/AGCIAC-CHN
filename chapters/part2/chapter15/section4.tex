\section{基于配对的密码学}\label{sec:15-4}

到目前为止,我们将椭圆曲线作为一个有效的群,它上面的离散对数问题以及其变体 CDH 和 DDH 问题都被认为是困难的。这种群能够让我们有效地实例化前面章节中所介绍的许多方案。我们现在声称,某些特定的椭圆曲线拥有一种额外的结构,称为\textbf{配对 (Paring)}。这种结构开创了一个全新的密码学世界,如果没有这种特殊的结构,许多来自离散对数群的方案都无从谈起。由此产生的方案构成了一个全新的领域,称为\textbf{基于配对的密码学 (pairing based cryptography)}。

为了介绍基于配对的方案,我们将把重点放在配对带来的新功能上,并抽象出椭圆曲线群的细枝末节。同前几章一样,我们把群操作用乘法形式表示。注意,这与本章的前几节不同。此前,为了与传统的椭圆曲线密码学的数学符号保持一致,我们一直在使用加法形式表示群操作。

\begin{definition}\label{def:15-2}
令 $\mathbb{G}_0,\mathbb{G}_1,\mathbb{G}_\mathrm{T}$ 是三个 $q$ 阶循环群,其中 $q$ 为素数,$g_0\in\mathbb{G}_0$ 和 $g_1\in\mathbb{G}_1$ 是生成元。一个\textbf{配对 (pairing)}是满足以下性质的一个可有效计算函数 $e:\mathbb{G}_0\times\mathbb{G}_1\to\mathbb{G}_\mathrm{T}$:
\begin{enumerate}
	\item 双线性 (bilinear):对于所有的 $u,u'\in\mathbb{G}_0$ 和 $v,v'\in\mathbb{G}_1$,我们有:
	\[
		e(u\cdot u',\,v)=e(u,v)\cdot e(u',v)
		\quad\text{and}\quad
		e(u,\,v\cdot v')=e(u,v)\cdot e(u,v')
	\]
	\item 非退化性 (non-degenerate):$g_\mathrm{T}:=e(g_0,g_1)$ 是 $\mathbb{G}_\mathrm{T}$ 的一个生成元。
\end{enumerate}
当 $\mathbb{G}_0=\mathbb{G}_1$ 时,我们称该配对为一个\textbf{对称配对 (symmetric pairing)}。我们将 $\mathbb{G}_0$ 和 $\mathbb{G}_1$ 称为\textbf{配对群 (pairing groups)} 或原群 (source groups),将 $\mathbb{G}_\mathrm{T}$ 称为\textbf{目标群 (target group)}。
\end{definition}

双线性意味着配对有下述的核心属性,它能被应用于我们所有的构造中:对于所有的 $\alpha,\beta\in\mathbb{Z}_q$,我们都有:
\begin{equation}\label{eq:15-5}
e(g_0^\alpha,g_1^\beta)=e(g_0,g_1)^{\alpha\cdot\beta}=e(g_0^\beta,g_1^\alpha)
\end{equation}
该等式是由 $e(g_0^\alpha,g_1^\beta)=e(g_0,g_1^\beta)^{\alpha}=e(g_0^\alpha,g_1)^\beta$这个有用的等式推出的,而后者本身就是双线性的直接结果。我们注意到,在 $\mathbb{G}_0$ 和 $\mathbb{G}_1$ 这样的循环群中,式 \ref{eq:15-5} 也等价于定义 \ref{def:15-2} 中所述的双线性性质。

定义 \ref{def:15-2} 中的非退化性要求确保配对 $e$ 对于所有输入并不总是输出 $1\in\mathbb{G}_\mathrm{T}$。这样的配对确实是双线性的,但是意义不大。此外,我们希望离散对数问题在群 $\mathbb{G}_0$ 和 $\mathbb{G}_1$ 上都是困难的。很多构造还要求其他的一些问题是困难的,比如 CDH 和 DDH 的某些变体。

\begin{snote}[直接的后果。]
配对 $e:\mathbb{G}_0\times\mathbb{G}_1\to\mathbb{G}_\mathrm{T}$ 会给群 $\mathbb{G}_0$ 和 $\mathbb{G}_1$ 带来一些额外的后果。首先,当 $\mathbb{G}_0=\mathbb{G}_1$ 时,$\mathbb{G}_0$ 上的确定性 Diffie-Hellman (DDH) 问题是很容易的。想要知道为什么,注意到,给定一个三元组 $(u,v,w)=(g_0^\alpha,g_0^\beta,g_0^\gamma)\in\mathbb{G}^3$,我们可以通过检查:
\[
e(u,v)=e(g_0,w)
\]
是否成立来确认$\gamma=\alpha\cdot\beta$ 是否在 $\mathbb{Z}_q$ 上成立。根据式 \ref{eq:15-5},当且仅当$e(g_0,g_0)^{\alpha\beta}=e(g_0,g_0)^\gamma$时,上述等式才成立,而前者当且仅当 $\gamma=\alpha\cdot\beta$ 时成立。正是此时,$(u,v,w)$ 才是一个 DH 三元组。因此,DDH 是容易的。

上述结论的一个结果是,乘性 ElGamal 加密方案在对称配对群中会丧失语义安全性,因此不应该在这样的群中被实例化。同样地,\ref{sec:11-5} 节中的完整 ElGamal 加密方案 $\mathcal{E}_{\rm EG}$ 也不能用基于 DDH 的分析来证明其安全性。然而,$\mathcal{E}_{\rm EG}$ 仍然可以通过 \ref{subsec:11-5-1} 中的分析,在随机预言机模型中基于 CDH 证明其安全性。

对于\textbf{非对称配对(asymmetric pairing)},即 $\mathbb{G}_0\neq\mathbb{G}_1$ 的情况,DDH 假设仍有可能在 $\mathbb{G}_0$ 和 $\mathbb{G}_1$ 上都成立。事实上,对于来自椭圆曲线的最有效的非对称配对,DDH 假设被认为在 $\mathbb{G}_0$ 和 $\mathbb{G}_1$ 上都仍然成立。

\vspace{5pt}

其次,在 $\mathbb{G}_0$ 或 $\mathbb{G}_1$ 上计算离散对数不会比在目标群 $\mathbb{G}_\mathrm{T}$ 上更难。想知道为什么,假设我们得到了一个 $u_0=g_0^\alpha\in\mathbb{G}_0$,并被要求计算其离散对数 $\alpha\in\mathbb{Z}_q$。为此,我们计算 $u:=e(u_0,g_1)$ 并置 $g_\mathrm{T}:=e(g_0,g_1)$。可以观察到 $u\in\mathbb{G}_\mathrm{T}$ 满足 $u=(g_\mathrm{T})^\alpha$,因此我们可以通过计算在 $\mathbb{G}_\mathrm{T}$ 上的以 $g_\mathrm{T}$ 为基底的 $u$ 的离散对数来找到 $\alpha$。因此,如果 $\mathbb{G}_\mathrm{T}$ 上的离散对数问题很容易,那么 $\mathbb{G}_0$ 上的离散对数问题也是很容易的。类似的论证也适用于 $\mathbb{G}_1$。因此,为了使离散对数在 $\mathbb{G}_0$ 和 $\mathbb{G}_1$ 上都是困难的,我们必须谨慎选择群,以使得目标群 $\mathbb{G}_\mathrm{T}$ 上的离散对数也是困难的。
\end{snote}

\begin{snote}[从椭圆曲线构建配对。]
一个配对 $e:\mathbb{G}_0\times\mathbb{G}_1\to\mathbb{G}_\mathrm{T}$ 通常是由椭圆曲线 $E/\mathbb{F}_p$ 构造而来的。来自椭圆曲线的最自然和有效的配对是不对称的,即 $\mathbb{G}_0\neq\mathbb{G}_1$。出于这个原因,我们将使用非对称配对来描述本章中所有基于配对的系统。

一个由椭圆曲线 $E/\mathbb{F}_p$ 构造的配对 $e:\mathbb{G}_0\times\mathbb{G}_1\to\mathbb{G}_\mathrm{T}$,其中的群 $\mathbb{G}_0$,$\mathbb{G}_1$ 和 $\mathbb{G}_\mathrm{T}$ 具有以下特性:
\begin{itemize}
	\item $\mathbb{G}_0$ 是 $E(\mathbb{F}_{p})$ 的一个 $q$ 阶子群,其中 $q$ 为某个素数,
	\item $\mathbb{G}_1$ 是 $E(\mathbb{F}_{p^d})$ 的一个 $q$ 阶子群,其中 $d>0$ 且 $\mathbb{G}_1\cap\mathbb{G}_0=\{\mathcal{O}\}$,
	\item $\mathbb{G}_\mathrm{T}$ 是有限域 $\mathbb{F}_{p^d}$ 的一个 $q$ 阶乘法子群。
\end{itemize}
整数 $d>0$ 被称为曲线的\textbf{嵌入度 (embedding degree)}。显然 $d$ 需要很小,因为只有这样,$\mathbb{G}_1$ 和 $\mathbb{G}_\mathrm{T}$ 中的元素才能被有效表示。如果 $d$ 很大,我们就无法表示 $\mathbb{G}_1$ 和 $\mathbb{G}_\mathrm{T}$ 中的元素。$d$ 相对小的椭圆曲线,比如 $d\leq16$ 的曲线,被称为\textbf{配对友好的椭圆曲线 (pairing friendly elliptic curve)}。

因为 $\mathbb{G}_0$ 定义在基域 $\mathbb{F}_p$ 上,而 $\mathbb{G}_1$ 定义在更大的域 $\mathbb{F}_{p^d}$ 上,所以 $\mathbb{G}_0$ 中的元素通常比 $\mathbb{G}_1$ 中的元素短得多。这在选择每个群所扮演的角色时会起到重要的作用。例如,为了最大限度地减少密文的长度,我们倾向于把密文中出现的群元素放在 $\mathbb{G}_0$ 中。

椭圆曲线上的配对函数 $e$ 来自于一个称为 \textbf{Weil 配对}的代数配对。这种配对可以用 Victor Miller 的算法来有效评估。在实践中,我们通常使用 Weil 配对的变体,称为 Tate 配对和 Ate 配对,对于这些配对,Victor Miller 算法更加有效。对于这些配对的定义和如何评估它们的更多信息,可以参考 [63]。

实践中最常用的曲线被称为 bn256 和 bls381。这两条曲线的嵌入度都是 $d=12$,群的阶 $q$ 是 $256$ 比特素数。两条曲线都定义在素数域上:bn256 定义在一个 $256$ 比特的素数域上,bls381 定义在一个 $381$ 比特的素数域上。曲线 bls381 被认为能够提供更好的安全性,因为 $\mathbb{G}_\mathrm{T}$ 上的离散对数是更难的,但由于素数更大,bls381 中的算术运算也因而更慢。
\end{snote}

\begin{snote}[配对的性能。]
令 $e:\mathbb{G}_0 \times \mathbb{G}_1 \to \mathbb{G}_\mathrm{T}$ 是一个由配对友好的椭圆曲线 $E/\mathbb{F}_p$ 派生而来的配对,所有三个群都有素阶 $q$。计算配对需要进行 $O(\log q)$ 次 $\mathbb{F}_p$ 上的算术运算,类似于 $\mathbb{G}_0$ 和 $\mathbb{G}_1$ 上的指数运算。然而,在实践中,计算配对所需的时间往往比指数运算更长(确切的开销取决于具体的实现)。因此,我们希望在设计中尽量减少计算配对的次数。为了了解相对的计算开销,表 \ref{tab:15-1} 展示了在曲线 bn256 上进行指数和配对运算的运行时间\footnote{使用 Miracl 库在 2.4 GHz 的 Intel i5 520M CPU 上进行基准测试。数字来自 Miracl 的文档。}。可以看出,$\mathbb{G}_0$ 上的配对运算比指数运算慢 10 倍左右。

\begin{table}[]
\centering
\begin{tabular}{|ll|c|}
\hline
\multicolumn{2}{|l|}{}               & 时间(毫秒) \\ \hline
指数运算: & 在 $\mathbb{G}_0$ 上      & 0.22      \\
         & 在 $\mathbb{G}_1$ 上      & 0.44      \\
         & 在 $\mathbb{G}_\mathrm{T}$ 上      & 0.95      \\ \hline
配对运算: & $\mathbb{G}_0\times\mathbb{G}_1\to\mathbb{G}_\mathrm{T}$ & 2.32      \\ \hline
\end{tabular}
\caption{曲线bn256上配对的基准测试。}
\label{tab:15-1}
\end{table}

在实现配对时,我们可以进行许多优化:
\begin{itemize}
	\item 首先,当其中一个配对的输入是固定的时候,使用预处理可以大大加快配对的计算速度。具体来说,假设我们需要对 $i=1,2,\dots,n$计算许多配对 $e(u,v_i)$,其中$u\in\mathbb{G}_0$ 是固定的。那么,我们就可能建立一个只取决于 $u$ 的表。一旦构建了这个表,计算每个配对就比没有预计算表的情况要快得多。这种加速类似于附录 \ref{subsec:A-2-4} 小节中讨论的固定基数的指数运算的预计算加速方法。
	\item 其次,配对的乘积 $\prod_{i=1}^ne(u_i,v_i)\in\mathbb{G}_\mathrm{T}$ 的计算速度可以比逐个计算 $n$ 个配对还要快。例如,配对计算的最后一步是将计算值用一个固定大数进行升幂。这个最后的指数运算可以在乘积的所有配对中汇总,对整个乘积只做一次。
\end{itemize}
通常,把这两种优化结合在一起会带来显著的速度提升。
\end{snote}