\section{椭圆曲线密码学}\label{sec:15-3}

令 $E/\mathbb{F}_p$ 是一条椭圆曲线,并令 $E(\mathbb{F}_{p^e})$ 表示该曲线上的点群。现在,我们有了一个有限群,我们就可以考察这个群上的离散对数、计算性 Diffie-Hellman (CDH) 和确定性 Diffie-Hellman (CDH) 等问题的复杂性。

令 $P$ 是 $E(\mathbb{F}_{p^e})$ 上的一个点,且后者的阶为素数 $q$,则 $qP=\mathcal{O}$。那么,$E(\mathbb{F}_{p^e})$ 上的离散对数问题就是给定一对点 $P$ 和 $\alpha P$ 作为输入,计算 $\mathbb{Z}_q$ 中的一个随机值 $\alpha$ 的问题。正如本章开始时所讨论的那样,对于大多数椭圆曲线来说,针对这个问题的最优已知算法也需要 $\Omega(\sqrt{q})$ 级别的时间。然而,也存在少数几种例外情况,在这些情况下,离散对数的计算要容易得多。下面是两个例子:
\begin{itemize}
	\item 当 $|E(\mathbb{F}_p)|=p$ 时,$E(\mathbb{F}_p)$ 上的离散对数问题可以在多项式时间内解出。
	\item 假设有一个小整数 $\tau>0$ 能使得 $|E(\mathbb{F}_p)|$ 整除 $p^{\tau}-1$。则 $E(\mathbb{F}_p)$ 上的离散对数问题就可以归约到有限域 $\mathbb{F}_{p^\tau}$ 上的离散对数问题,如 \ref{sec:15-4} 节将要介绍的。$\mathbb{F}_{p^\tau}$ 上的离散对数问题可以使用普通数域筛选法 (GNFS) 的离散对数算法变体解决。例如,如果 $\tau$ 很小,比如说 $\tau=2$,并且 $p$ 是一个 $256$ 比特的素数,那么 $E(\mathbb{F}_p)$ 上的离散对数问题就可以被有效地解决:首先,将给定的离散对数问题归约到 $\mathbb{F}_{p^2}$ 上,然后在 $\mathbb{F}_{p^2}$ 上应用 GNFS。因此,这迫使我们确保 $p^\tau$ 足够大,以至于使得 $\mathbb{F}_{p^\tau}$ 上的 GNFS 是完全不可行的。
\end{itemize}
为了避免这些隐患,许多实现都使用固定的曲线集。人们通常认为,这种做法比随机选取一个素数$p$,然后在 $\mathbb{F}_{p}$ 上随机生成一条椭圆曲线要更加安全。两种最广泛使用的曲线被称为 P256 和 Curve25519,我们将在下面分别介绍它们。

一旦我们确保了群 $E(\mathbb{F}_{p})$ 上离散对数、CDH 和 DDH 问题的难度,我们就可以用这个群来实例化我们在前面几章中涉及到的所有构造。由此产生的系统就被称为\textbf{椭圆曲线密码学系统 (elliptic curve cryptosystems)}。

\subsection{P256曲线}

1999年,美国国家标准研究所 (NIST) 公布了一份供联邦政府使用的椭圆曲线清单。这些曲线中最流行的是 \textbf{secp256r1},或者简称 \textbf{P256}。所有 TLS 1.3 的实现都需要支持这种曲线以进行 Diffie-Hellman 密钥交换。它是 TLS 标准(见 \ref{sec:21-10} 节)中唯一的强制性曲线。

P256 曲线定义在素数 $p:=2^{256}-2^{224}+2^{192}+2^{96}-1$ 上。$p$ 的特殊结构可以用来提高模 $p$ 算术运算的性能。该曲线具有标准的 Weierstrass 形式 $y^2=x^3-3x+b$,其中的 $b$ 用十六进制表示为:
\[
b:=\texttt{5ac635d8 aa3a93e7 b3ebbd55 769886bc 651d06b0 cc53b0f6 3bce3c3e 27d2604b}
\]
这条曲线上的点的数量是一个素数 $q$。标准还规定了一个可以生成整个群的点 $G$。

因为素数 $p$ 接近 $2^{256}$,所以点的数量 $q$ 也接近 $2^{256}$。于是,如果假设没有任何捷径的话,在这条曲线上计算离散对数大约就需要 $\sqrt{q}$ 次群运算,也就是大约 $2^{128}$ 次。设计意图是,在这条曲线上计算离散对数(以及 CDH 和 DDH)至少应该和破解 AES-128 一样困难。因此,如果 AES-128 能够被用于加密明文数据,则 P256 也可以被用于 Diffie-Hellman 密钥交换、公钥加密和数字签名。

一些高安全性的应用使用 AES-256 来加密明文数据。在这些情况下,我们应该使用具有更高安全参数的椭圆曲线。一个选择是名为 \textbf{secp521r1} 的曲线,其大小约为 $2^{521}$。它定义在 Mersenne 素数 $p=2^{521}-1$ 上。在这条曲线上计算离散对数被认为至少需要进行 $2^{256}$ 次群运算。

\begin{snote}[参数的选择。]
P256 中那个看起来很奇怪的参数 $b$ 是如何被选出来的?答案是,我们并不清楚。该标准列出了一个无法解释的常数,称为\textbf{种子} $S$。这个种子作为输入参数被提供给一个公开的确定性算法中以产生参数 $b$。这个过程被设计为以伪随机方式选择一条能抵御所有已知的离散对数攻击的曲线。

我们并不清楚种子 $S$ 是如何被选出来的。这可能会让想要使用 P256 的外国政府担心。他们可能会忧虑种子是以对抗性的方式选择的,这样,产生种子的组织就可以有效地计算出所产生的曲线的离散对数。但目前我们尚不知道这样的种子是怎么被选出来的,所以这种担心只是一种耐人寻味的猜测。就我们目前所知,P256 仍是一个可以使用的优良曲线,它仍然在实践中被广泛地使用。
\end{snote}

\subsection{Curve25519}

令 $E/\mathbb{F}_p$ 是一条椭圆曲线,并令 $n:=|E(\mathbb{F}_{p})|=p+1-t$。我们将在 \ref{subsec:17-2-4} 小节说明,$E(\mathbb{F}_{p})$ 上的离散对数的计算难度只与 $n$ 的最大素因子相当。具体地说,存在一种离散对数算法,其运行时间为 $\sqrt{q}$,而 $q$ 就是 $n$ 的最大素因子。如果 $n$ 的最大素因子很小,那么 $E(\mathbb{F}_{p})$ 上的离散对数问题也就很简单。出于这个原因,我们始终坚持 $n$ 是一个素数,或者是一个素数的小倍数。我们用 $q$ 来表示 $n$ 的最大素因子。这个 $q$ 也是 $E(\mathbb{F}_{p})$ 的最大素阶子群的大小。

\begin{snote}[扭变安全性。]
每条椭圆曲线 $E/\mathbb{F}_p$ 都有一条与之相关的曲线 $\tilde E/\mathbb{F}_p$,称为 $E$ 的\textbf{扭变 (twist)}。令 $w\in\mathbb{F}_p$ 是 $\mathbb{F}_p$ 中的某个二次非剩余。如果 $E$ 是曲线 $y^2=x^3+ax+b$,那么它的扭变 $\tilde E$ 就是 $wy^2=x^3+ax+b$。假设 $|E(\mathbb{F}_{p})|$ 是奇数。那么我们可以证明,每个 $x\in\mathbb{F}_p$ 要不是 $E(\mathbb{F}_{p})$ 上点的横坐标,要不就是 $\tilde E(\mathbb{F}_{p})$ 上点的横坐标,但不能同时是。由此可以推断,$\tilde E(\mathbb{F}_{p})$ 上点的数量是 $\tilde n:=p+1+t$。回顾一下,$E/\mathbb{F}_p$ 上点的数量是 $p+1-t$。

如果离散对数在 $E(\mathbb{F}_{p})$ 和 $\tilde E(\mathbb{F}_{p})$ 上都是难以解决的,我们就说曲线 $E/\mathbb{F}_p$ 是\textbf{扭变安全 (twist secure)} 的。要使 $E/\mathbb{F}_p$ 是扭变安全的,我们至少需要 $n=|E(\mathbb{F}_{p})|$ 和 $\tilde n= |\tilde E(\mathbb{F}_{p})|$ 都是素数,或者都是大素数的小倍数。

为什么我们需要扭变安全性?考虑这样的一个系统,其中 Bob 有一个私钥 $α\in\mathbb{Z}_q$。在正常的操作下,任何人都可以向 Bob 发送一个点 $P\in E(\mathbb{F}_{p})$,而 Bob 会以点 $\alpha P$ 作为应答。\ref{subsec:11-7-3} 小节中的不经意 PRF 就是这样的一个系统。在应答之前,Bob 最好检查给定的点 $P$ 是否在 $E(\mathbb{F}_{p})$ 上;否则,Bob 发回的应答可能会泄露他的秘钥 $\alpha$,正如练习 15.1 中所讨论的那样(也可以参见备注 12.1,其中出现了一个类似的问题)。检查一个点 $P=(x1,y1)$ 是否满足曲线方程是非常简单和高效的。然而,有些实现采用了练习 15.2 和练习 15.4 中的优化方法,Bob 只会被给予 $P$ 的横坐标,纵坐标既不需要,也不会被发送给他。在这种情况下,检查给定的 $x_1∈\mathbb{F}_p$ 是否有效,就需要对其进行指数运算,以确认 $x^3_1+ax_1+b$ 是否是 $\mathbb{F}_p$ 的二次剩余(见附录 \ref{subsec:a-2-3} 小节)。假设 Bob 跳过这个很耗费资源的检查,那么攻击者就可以向 Bob 发送一个 $x_1∈\mathbb{F}_p$,它是扭变 $\tilde E(\mathbb{F}_{p})$ 上的一个点 $\tilde P$ 的横坐标。然后 Bob 就会应答一个 $\tilde E(\mathbb{F}_{p})$ 上的点 $\alpha\tilde P$ 的横坐标。如果 $\tilde E(\mathbb{F}_{p})$ 上的离散对数是容易的,这个应答就会暴露 Bob 的私钥 $\alpha$。因此,如果 Bob 跳过这个群元素检查,我们必须至少确保 $\tilde E(\mathbb{F}_{p})$ 中的离散对数是难的,这样 $\alpha\tilde P$ 就不会暴露 $\alpha$。扭变安全性正是为了确保这一点。

曲线 P256 的设计并不是扭变安全的。其扭变的大小可以被 $34905=3×5×13×179$ 整除。因此,在扭变上的离散对数比在 P256 上要容易 $\sqrt{34905}\approx187$ 倍(见 \ref{subsec:17-2-4} 小节)。这个事实不容忽视,但是也不应当因此就放弃使用 P256 曲线。
\end{snote}

\begin{snote}[Curve25519。]
Curve25519 的设计能够支持一种优化的群操作,而且是扭变安全的。这条曲线定义在素数 $p:=2^{255}-19$ 上,这就是其名字的由来。这个 $p$ 是小于 $2^{255}$ 的最大素数,这使得 $\mathbb{F}_p$ 上的算术运算速度极快。

将 Curve25519 描述为一条 Montgomery 曲线是非常容易的,即一条形如 $E:By^2=x^3+Ax^2+x$ 的曲线,其中 $A,B\in\mathbb{F}_p$,且 $p>3$。练习 15.4 表明,这种曲线支持一种快速的乘法算法,能够从 $P$ 计算出 $\alpha P$,其中 $P\in E(\mathbb{F}_{p})$,且 $\alpha\in\mathbb{Z}$。我们在前面已经介绍过,一条 Montgomery 曲线上的点的数量 $|E(\mathbb{F}_{p})|$ 总是 $4$ 的倍数。

用 Montgomery 形式描述的 Curve25519 的方程为:
\[
y^2=x^3+486662\cdot x^2+x
\]
这条曲线上的点的数量是一个素数的 $8$ 倍,因此,我们称该曲线的\textbf{协因子 (cofactor)} 为 $8$。该曲线由一个点 $P=(x_1,y_1)$ 生成,其中 $x_1=9$。完整起见,我们注意到,Curve25519 也可以表示为一条 Edwards 曲线 $x^2+y^2=1+(121665/121666)x^2y^2$。
\end{snote}

\begin{snote}[为什么常数是 486662?]
在定义一条 Montgomery 曲线时,$A$ 越小,群运算就越快,正如我们将在练习 15.4 中说明的。为了获得最佳性能,我们需要让 ${(A-2)}/{4}$ 尽量小。设计这条曲线的 Dan Bernstein 选择了尽可能小的 $A$,这样曲线就能安全地抵御已知的离散对数攻击。他还确保该曲线与其扭变的阶都是某个素数的 $4$ 倍或 $8$ 倍。Bernstein 表示:
\begin{quote}
$A$ 的最小正整数选择有 $358990$、$464586$ 和 $486662$。我没有使用 $A=358990$,因为它的一个素因子比 $2^{252}$ 略小,这就导致了一个问题,即标准和实现应该如何处理用户的私钥与该素数相匹配的理论上的可能性;讨论这个问题比换一个 $A$ 还要困难。我不使用 $464586$ 的原因也是如此。因此最后,我选择了 $A=486662$。
\end{quote}
这个说明比 P256 中的那个缺乏解释的常数更能令人信服。
\end{snote}