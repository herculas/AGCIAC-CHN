\chapter*{前言}

密码学是用于保护计算机系统信息的一个不可或缺的工具。它无处不在,每天被全世界数十亿人用于保护静态或动态的数据。密码学系统是很多标准协议的重要组成部分,比如说非常知名的传输层安全 (TLS) 协议,它们使得将强大的密码学技术纳入到各式各样的应用当中变成一件相当容易的事情。

尽管密码学非常有用,它往往也是非常脆弱的。最安全的密码学系统也会因为一个规范上或编程上的错误而完全丧失安全性。而且,无论我们进行多少次单元测试,都无法发现密码学系统的安全漏洞。相反,为了论证一个密码学系统是安全的,我们得依靠数学建模和证明来说明某个系统满足指定的安全属性。我们经常需要引入某些合理的假设来推动我们的安全论证。

本书恰是关注这个话题:构建实用的密码学系统,并在合理的假设下论证其安全性。本书涵盖了许多密码学构造,它们分别针对的是不同的密码学任务。对于每一项任务,我们都会定义一个精确的安全目标,然后提出实现所需目标的构造。为了分析这些构造,我们会发展一套统一的框架来进行密码学证明。掌握了这个框架的读者将有能力将其应用于书中可能没有提到的其他新构造。

在整本书中,我们会通过展示案例来考察已经实际部署的系统是如何运作的。我们会强调读者需要避免的常见错误,以及针对现实世界中系统的常见攻击方式,并借此说明密码学中严谨的重要性。在每一章的结尾,我们都会给出一个有趣的应用,它们会以某种意想不到的方式践行本章的观点。

\section*{目标受众以及本书的使用}

本书旨在自成一体。我们在附录中提供了一些包含概率论和代数学的基本知识作为补充材料。本书主要分为三部分:
\begin{itemize}
	\item 第一部分将介绍\emph{对称加密(symmetric encryption)},它会试图解释,当 Alice 和 Bob 两方共享一个攻击者未知的密钥时,他们该如何安全地交换信息。我们会讨论数据机密性、数据完整性以及认证加密的重要概念。
	\item 第二部分将介绍\emph{公钥加密(public-key encryption)}和\emph{数字签名(digital signature)}的概念,这些技术使得 Alice 和 Bob 可以在没有预先共享密钥的情况下安全地进行通信。
	\item 第三部分关于\emph{密码学协议(cryptographic protocols)},比如关于用户身份识别、密钥交换、零知识和安全计算的协议。
\end{itemize}

初学者可以通过阅读本书来了解密码学系统是如何工作的以及为什么它们是安全的。书中的每一个安全定理后面都附有一个证明思路,它们会从高层次上解释了为什么该方案是安全的。初读时,为了不丢失连贯性,读者可以跳过详细的证明。那些探讨某些定义的细微差别的数学细节章节也可以在第一次阅读时跳过。

进阶读者可能会喜欢阅读详细的证明,以学习如何在密码学中进行安全证明。我们在每一章的结尾都提供了很多练习,它们会探讨该章所涉及的内容的其他方面。有些练习是对已学知识的复习,但更多的练习是对正文内容的扩展,或是讨论本章未涉及的话题。

\section*{本书的状态}

目前的草稿包含第一部分和第二、三部分的大部分内容。最后两章即将问世。我们希望你喜欢这本书。如果你发现错别字或内容舛误,请务必向我们提出意见。

\begin{snote}[引用。]
目前的草稿虽然已基本完成,但本书参考的许多已有工作和参考文献尚未被全部纳入。这些内容很快就会被添加进来,我们会在每章末尾的``笔记"一节列举该章所涉及的参考文献。
\end{snote}

\vspace{50pt}

\noindent Dan Boneh 和 Victor Shoup\\
\noindent 2020年1月