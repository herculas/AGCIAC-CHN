\tikzset{every picture/.style={line width=0.75pt}}     

\begin{tikzpicture}[x=0.75pt,y=0.75pt,yscale=-0.9,xscale=0.9]

\draw  [fill={rgb, 255:red, 255; green, 255; blue, 255 }  ,fill opacity=1 ][line width=1.2] [general shadow={fill=black,shadow xshift=2.25pt,shadow yshift=-2.25pt}] (85,20) -- (375,20) -- (375,120) -- (85,120) -- cycle ;
\draw  [fill={rgb, 255:red, 155; green, 155; blue, 155 }  ,fill opacity=0.6 ][line width=1.2]  (100,70) .. controls (100,50.67) and (122.39,35) .. (150,35) .. controls (177.61,35) and (200,50.67) .. (200,70) .. controls (200,89.33) and (177.61,105) .. (150,105) .. controls (122.39,105) and (100,89.33) .. (100,70) -- cycle ;
\draw  [line width=1.2]  (260,40) -- (350,40) -- (350,100) -- (260,100) -- cycle ;

\draw  [->]  (0,70) -- (100,70) ;
\draw  [->]  (0,60) -- (100,60) ;
\draw  [->]  (0,80) -- (100,80) ;
\draw  [->]  (200,70) -- (260,70) ;
\draw  [->]  (350,70) -- (460,70) ;

\draw (150,70) node   [align=left] {内部状态};
\draw (305,70) node   [align=left] {生成函数};
\draw (420,65) node [anchor=south] [inner sep=0.75pt]   [align=left] {伪随机输出};
\draw (2,57) node [anchor=south west] [inner sep=0.75pt]   [align=left] {熵};


\end{tikzpicture}